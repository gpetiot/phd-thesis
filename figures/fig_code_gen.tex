\begin{figure}[bt]\centering
  \begin{tikzpicture}[font=\scriptsize]
    \tikzstyle{ins}=[rounded corners,inner sep=.5mm,thick,node distance=.6cm]
    \node[ins,draw=blue] at (-2.5,2) (ins1) {
      \lstinline{Z_t v1; Z_init(v1); Z_set(v1,x);}
      \insertball[fill=blue!40]{1}};
    \node[left of=ins1,node distance=3.5cm] (l1) {$l$,};
    \node[ins,draw=red,below of=ins1] (ins2) {
      \lstinline{Z_t v2; Z_init(v2); Z_set(v2,y);}
      \insertball[fill=red!40]{2}};
    \node[left of=ins2,node distance=3.5cm] (l2) {$l$,};
    \node[ins,inner sep=0,below of=ins2] (ins3a) {
      \lstinline{int v3 = Z_lt(v1, v2);}};
    \node[inner sep=0,below of=ins3a,node distance=.3cm] (ins3b) {
      \lstinline{Z_clear(v1); Z_clear(v2);}
      \insertball[fill=greenv!40]{3}};
    \node[ins,draw=greenv,fit=(ins3a)(ins3b)] (ins3) {};
    \node[left of=ins3a,node distance=3.5cm] (l3) {$l$,};
    \node[ins,draw=gray,below of=ins3] (ins4) {
      \lstinline{Z_t v4; Z_init(v4); Z_set(v4,x);}
      \insertball[fill=gray!40]{5}};
    \node[ins,draw=gray,below of=ins4] (ins5) {
      \lstinline{Z_t v5; Z_init(v5); Z_set(v5,y);}
      \insertball[fill=gray!40]{6}};
    \node[ins,inner sep=0,below of=ins5] (ins6a) {
      \lstinline{int v6 = Z_lt(v4, v5);}};
    \node[inner sep=0,below of=ins6a,node distance=.3cm] (ins6b) {
      \lstinline{Z_clear(v4); Z_clear(v5);}
      \insertball[fill=gray!40]{7}};
    \node[ins,draw=gray,fit=(ins6a)(ins6b)] (ins6) {};
    \node[left of=ins4,node distance=3.95cm] (l5) {$\mathit{EndIter_l}$,};
    \node[left of=ins5,node distance=3.95cm] (l6) {$\mathit{EndIter_l}$,};
    \node[left of=ins6a,node distance=3.95cm] (l7) {$\mathit{EndIter_l}$,};
    \node[ins,draw=orange,below of=ins6] (ins7) {
      \lstinline{fassert(v3);}\insertball[fill=orange!40]{4}};
    \node[ins,draw=gray,below of=ins7] (ins8) {
      \lstinline{fassert(v6);}\insertball[fill=gray!40]{8}};
    \node[left of=ins7,node distance=3.5cm] (l4) {$l$,};
    \node[left of=ins8,node distance=3.95cm] (l8) {$\mathit{EndIter_l}$,};
  \end{tikzpicture}
  \caption{Génération d'insertions de code à partir d'un invariant de boucle}
  \label{fig:code-gen}
\end{figure}
