
\chapter{Combinaisons de méthodes statiques et dynamiques dans Frama-C}

\begin{definition}{Hard Testing}\hfill \\
  Utilisation d'un générateur de tests afin de forcer l'exécution des chemins
  menant à une erreur. Le programme à tester est instrumenté de manière à
  générer ces chemins d'erreur. Pour une assertion \texttt{assert X} ou une
  post-condition \texttt{ensures X}, on génère
  une conditionnelle \texttt{if(X)}, afin de forcer l'exploration des branches
  \texttt{X = true} et \texttt{X = false}. C'est ce qu'on appellera ``forcer
  l'erreur par le test''. S'il existe un contre-exemple pour une propriété,
  nous serons en mesure de trouver des entrées concrètes permettant d'emprunter
  un chemin d'exécution menant à cette erreur. Ainsi, si aucun contre-exemple
  n'est trouvé pour une propriété donnée (sous réserve d'un parcours sans erreur
  de tous les chemins), alors il n'en existe pas et la propriété peut etre
  considérée valide.
\end{definition}

\begin{definition}{Soft Testing}\hfill \\
  Utilisation d'un générateur de tests afin de simplement constater l'erreur,
  et non la forcer, par opposition au Hard Testing.
  On ne génère pas de chemins supplémentaires pour forcer le parcours
  des chemins d'erreur, à la place, l'instrumentation reporte les vérifications
  dans l'oracle de la fonction sous test. Le fait de reporter ces vérifications
  distingue cette approche du monitoring. Cette approche est moins couteuse que
  le hard testing en raison du nombre plus faible de chemins à explorer,
  néanmoins nous n'avons pas la certitude de trouver des contre-exemples aux
  propriétés s'ils existent, et donc, ne pouvons pas valider les propriétés
  pour lesquelles aucun contre-exemple n'a été trouvé.
\end{definition}
