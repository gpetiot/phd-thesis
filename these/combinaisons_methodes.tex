
\chapter{Combinaisons de méthodes statiques et dynamiques dans Frama-C}

\begin{definition}{Hard Testing}\hfill \\
  Utilisation d'un générateur de tests afin de forcer l'exécution des chemins
  menant à une erreur. Le programme à tester est instrumenté de manière à
  générer ces chemins d'erreur. Pour une assertion \texttt{assert X} ou une
  post-condition \texttt{ensures X}, on génère
  une conditionnelle \texttt{if(X)}, afin de forcer l'exploration des branches
  \texttt{X = true} et \texttt{X = false}. C'est ce qu'on appellera ``forcer
  l'erreur par le test''. S'il existe un contre-exemple pour une propriété,
  nous serons en mesure de trouver des entrées concrètes permettant d'emprunter
  un chemin d'exécution menant à cette erreur. Ainsi, si aucun contre-exemple
  n'est trouvé pour une propriété donnée (sous réserve d'un parcours sans erreur
  de tous les chemins), alors il n'en existe pas et la propriété peut etre
  considérée valide.
\end{definition}

\begin{definition}{Soft Testing}\hfill \\
  Utilisation d'un générateur de tests afin de simplement constater l'erreur,
  et non la forcer, par opposition au Hard Testing.
  On ne génère pas de chemins supplémentaires pour forcer le parcours
  des chemins d'erreur, à la place, l'instrumentation reporte les vérifications
  dans l'oracle de la fonction sous test. Le fait de reporter ces vérifications
  distingue cette approche du monitoring. Cette approche est moins couteuse que
  le hard testing en raison du nombre plus faible de chemins à explorer,
  néanmoins nous n'avons pas la certitude de trouver des contre-exemples aux
  propriétés s'ils existent, et donc, ne pouvons pas valider les propriétés
  pour lesquelles aucun contre-exemple n'a été trouvé. Le {\em Soft Testing}
  n'est pas complet et n'offre donc pas une garantie égale au
  {\em Hard Testing}.
\end{definition}

\begin{definition}{Preuve partielle}\hfill \\
  Preuve pour laquelle on a renforcé la pré-condition par une clause
  \texttt{typically} (instanciation des variables d'entrée ou réduction de leur
  domaine de définition).
\end{definition}

\begin{definition}{Preuve totale}\hfill \\
  Preuve pour laquelle on ne renforce pas la pré-condition par la clause
  \texttt{typically}. Si de telles clauses sont présentes, on les ignore. Une
  telle preuve offre une garantie plus forte qu'une preuve partielle.
\end{definition}

\begin{definition}{Partial Hard Testing}\hfill \\
  Utilisation du {\em Hard Testing} avec renforcement de la pré-condition afin
  d'obtenir une preuve partielle.
\end{definition}

\begin{definition}{Total Hard Testing}\hfill \\
  Utilisation du {\em Hard Testing} sans renforcement de la pré-condition afin
  d'obtenir une preuve totale.
\end{definition}

On ne fait pas de distinction total/partiel pour le {\em Soft Testing} puisqu'il
n'offre aucune garantie en terme de validité des propriétés, que la
pré-condition soit renforcée ou non.



\section{Obtenir la meme garantie avec le test et la vérification déductive}

\subsection{Tentative de Preuve totale $\rightarrow$ Total Hard Testing}

En cas d'échec d'une preuve totale avec un greffon de preuve, nous pouvons
utiliser un générateur de tests en mode {\em Hard Testing}. Si, pour une
propriété donnée, aucun contre-exemple n'a été trouvé et tous les chemins
d'exécution ont été couverts sans qu'aucune erreur ne se soit produite, alors
cette propriété peut etre validée. La validation de cette propriété par le
générateur de tests offre la meme garantie qu'un greffon de vérification
déductive : on obtient une preuve totale pour cette propriété.
Cette combinaison permet d'obtenir des contre-exemples ou une preuve totale par
le test après un échec de preuve totale par vérification déductive.

\subsection{Tentative de Preuve partielle $\rightarrow$ Partial Hard Testing}

Après un échec d'une preuve partielle, nous pouvons utiliser le {\em Hard
Testing}. Si aucun contre-exemple n'a été trouvé pour une propriété donnée et
si tous les chemins d'exécution ont été couverts sans qu'aucune erreur ne se
soit produite, alors on obtient une preuve partielle pour cette propriété.
Cette combinaison permet d'obtenir des contre-exemples ou une preuve partielle
par le test après un échec de preuve partielle par vérification déductive.




\section{Obtenir une garantie plus forte avec le test}

\subsection{Tentative de Preuve partielle $\rightarrow$ Total Hard Testing}

Après un échec de preuve pour laquelle la pré-condition a été renforcée, il est
possible d'utiliser le {\em Hard Testing} sans renforcement de la pré-condition.
Ainsi, si aucun contre-exemple n'a été trouvé et si tous les chemins d'exécution
ont été couverts sans qu'aucune erreur ne se soit produite, alors on obtient une
preuve totale pour les propriétés concernées. Cette combinaison permet d'obtenir
des contre-exemples ou une preuve totale par le test après un échec de preuve
partielle par vérification déductive.




\section{Obtenir une garantie plus faible avec le test}

\subsection{Tentative de Preuve totale $\rightarrow$ Partial Hard Testing}

Après un échec de preuve pour laquelle la pré-condition n'a pas été renforcée,
il est possible d'utiliser le {\em Hard Testing} avec un renforcement de la
pré-condition. Ainsi, si aucun contre-exemple n'a été trouvé et si tous les
chemins d'exécution ont été couverts sans qu'aucune erreur ne se soit produite,
alors on obtient une preuve partielle pour les propriétés concernées. Cette
combinaison permet d'obtenir des contre-exemples ou une preuve partielle par le
test après un échec de preuve totale par vérification déductive.




\section{Obtenir une garantie nulle avec le test}

\subsection{Tentative de Preuve totale $\rightarrow$ Soft Testing}

Si nous utilisons un générateur de tests en mode {\em Soft Testing} après un
échec de preuve par un greffon de vérification déductive, nous aurons moins de
chance d'activer les chemins menant à des contre-exemples des propriétés à
prouver. Et si aucun contre-exemple n'a été trouvé et tous les chemins
d'exécution ont été couverts sans qu'aucune erreur ne se soit produite, nous
ne pouvons pas considérer la propriété comme valide. Cette combinaison permet
uniquement de trouver des contre-exemples après un échec de preuve totale.

\subsection{Tentative de Preuve partielle $\rightarrow$ Soft Testing}

Après un échec d'une preuve partielle, nous pouvons utiliser le {\em Soft
Testing}. Si aucun contre-exemple n'a été trouvé pour une propriété donnée et si
tous les chemins d'exécution ont été couverts sans qu'aucune erreur ne se soit
produite, nous ne pouvons rien conclure sur cette propriété.
Cette combinaison permet uniquement de trouver des contre-exemples après un
échec de preuve partielle.
