
\chapterintro

Dans ce chapitre nous présentons une solution pour le monitoring mémoire des
programmes C, dévelopée pour la vérification d'assertions à l'exécution
({\em runtime assertion checking} \cite{Clarke/06}) dans \framac
\cite{\citeframac}.
Cette solution inclut un langage de spécification exécutable, \eacsl,
et un traducteur, \eacsltoc \cite{\citeeacsltoc}.
Notre objectif est de pouvoir exécuter des annotations écrites avec le langage
de spécification \eacsl.
Nous avons développé une bibliothèque C permettant d'enregistrer et de récupérer
les informations de validité et d'initialisation d'un programme. Les appels aux
fonctions de cette bibiliothèque C sont automatiquement ajoutés aux emplacements
adéquats dans le code source du programme par \eacsltoc durant la
traduction de la spécification \eacsl en C.

\eacsltoc est un greffon de \framac qui traduit automatiquement
un programme C annoté en un autre programme dont l'exécution échouera si une
annotation n'est pas valide. Si aucune annotation n'est violée, le comportement
du nouveau programme est exactement le même que celui du programme d'origine.


\section{Implémentation}


\subsection{Patricia tries}

Le code instrumenté pouvant accéder et modifier fréquemment les données du
$store$, une implémentation efficace du requiert une structure de données
offrant une bonne complexité en temps et en espace. Cette structure doit être
triée : on peut avoir besoin d'accéder directement à un bloc à partir de son
adresse de base, mais aussi à partir de n'importe quelle adresse contenue dans
le bloc (donc accéder au bloc précédent). Par exemple, la fonction
\lstinline{__base_addr(p)} utilisée pour le traitement de la construction
\acsl \lstinline{\base_addr(p)} cherche l'adresse de base la plus
proche et inférieure à \lstinline{p} (et enfin vérifie les bornes du bloc).
Cette contrainte ne nous permet pas d'utiliser une table de hachage. Les listes
chaînées ne sont pas assez efficaces à cause de la complexité linéaire au pire
cas. Les arbres binaires de recherche non équilibrés ont aussi une complexité
linéaire au pire cas quand les données sont insérées dans un ordre strictement
monotone, ce qui est souvent le cas. Enfin, le coût du rééquilibrage de l'arbre
(pour un arbre binaire de recherche équilibré) serait amorti dans le cas où les
modifications de la structure de l'arbre sont moins nombreuses que les accès
simples; ce qui n'est pas nécessairement vrai sur les exemples de code que nous
avons instrumentés avec \eacsltoc.

Notre implémentation du $store$ se base sur un {\em Patricia trie}
\cite{Szpankowski/90} (appelé aussi {\em radix tree} ou ``arbre à préfixe
compact'', cette structure est efficace même si l'abre n'est pas équilibré.
Les clés sont les adresses de base des blocs (c'est-à-dire des mots de 32 ou 64
bits) ou des préfixes d'adresses. Chaque feuille contient les données relatives
à un bloc en mémoire (voir section précédente pour le détail des informations
stockées). Le routage de la racine jusqu'à une feuille particulière se fait
grâce aux n\oe{}uds internes, chacun d'eux contient le plus grand préfixe commun
de l'adresse de base de ses deux fils. La Fig.~\ref{fig:insertion-Patricia-trie}
a illustre un Patricia trie (sur des adresses 8-bits pour des raisons de
simplicité).

Il contient 3 blocs dans ses feuilles (seules les adresses de base apparaissent
sur le schéma), et 2 préfixes stockés dans les n\oe{}uds internes. Le symbole
``{\tt *}'' signifie que la valeur du bit à cette position n'a pas d'importance.

La complexité théorique au pire cas d'un accès dans un Patricia trie dans notre
cas est en {\em O(k)} où $k$ est la longueur d'un mot (c'est-à-dire 32 ou 64
bits). En pratique, un programme ne pouvant allouer des blocs que dans un
espace mémoire limité, la profondeur de l'arbre est inférieure à cette borne.
De plus, contrairement aux chaînes de caractères (la première application des
Patricia tries), la comparaison des mots peut être implémentée très
efficacement par des opérations bit-à-bit.

Les données de chaque bloc n'occupent que quelques octets en mémoire, exception
faite des données d'initialisation du bloc. Le statut d'initialisation de chaque
octet est monitoré séparément
(les champs de bits ne sont pas encore supportés). Dans le pire cas (bloc
partiellement initialisé), chaque octet utilise un bit supplémentaire portant
l'information sur son initialisation. Dans le cas où tous les octets (ou aucun)
sont initialisés (on utilise un compteur d'octets initialisés), le tableau
censé contenir les bits portant l'information d'initialisation est libéré, et
cette information est donc portée uniquement par le compteur. De plus, nous
utilisons une fonction spécifique dans le cas où tous les octets d'un bloc sont
initialisés d'un coup, au lieu d'invoquer une fonction d'initialisation sur
chaque octet du bloc.


\subsection{Calcul du plus grand préfixe commun}


\begin{figure}[h]
\begin{lstlisting}
typedef unsigned char byte;
// index            0    1    2    3    4    5    6    7    8
byte  masks[] = {0x00,0x80,0xC0,0xE0,0xF0,0xF8,0xFC,0xFE,0xFF};
int longer [] = {   0,  -1,   3,  -3,   6,  -5,   7,   8,  -8};
int shorter[] = {   0,   0,   1,  -2,   2,  -4,   5,  -6,  -7};
byte gtCommonPrefixMask(byte a, byte b) {
  byte nxor = ~(a ^ b);   // a bit = 1 iff this bit is equal in a and b
  int i = 4;              // search starts in the middle of the word
  while(i > 0)            // we stop when i<=0
    if (nxor >= masks[i]) // first i bits equal,
      i = longer[i];      // try a longer prefix 
    else i = shorter[i];  // otherwise, try a shorter prefix 
  return masks[-i];       // when i<=0, masks[-i] is the answer
}
\end{lstlisting}
\caption{Calcul du plus grand préfixe commun}
\label{fig:prefix}
\end{figure}


Appelons ``masque du plus grand préfixe commun'' M de A et B. M est composé
d'une suite de $n$ ``1'', suivie d'une suite de ``0'', où $n$ est le nombre de
bits communs entre A et B. Par exemple, le plus grand préfixe commun de
A = \ppleaf{\texttt{0110\,0111}} et B = \ppleaf{\texttt{0111\,1111}} est P = 
\ppnode{\texttt{011*\,****}}
et le masque du plus grand préfixe commun est M = \texttt{1110\,0000}.

Les calculs et comparaisons des préfixes ont été optimisés par un usage
intensif des opérations bit-à-bit. Le calcul du plus grand préfixe commun a lui
aussi été reconçu pour de meilleures performances. L'implémentation naïve
initiale consistait en un parcours linéaire des mots mémoire de gauche à droite
jusqu'à trouver des bits différents, de la même manière qu'on pourrait le faire 
sur des chaînes de caractères.

La version optimisée de ce calcul consiste
maintenant en une recherche dichotomique dans un tableau pré-calculé qui
contient tous les préfixes possibles. Les transitions entre les étapes de la
recherche se font en utilisant des indices pré-calculés, de manière à obtenir
le prochain masque à essayer. La Fig.~\ref{fig:prefix} illustre cette
implémentation du calcul sur des mots de 8 bits. Les masques sont
stockés dans le tableau de la ligne 3. Les indices à utiliser pour tester un
masque plus long (resp. plus court) sont stockés dans le tableau ligne 4 (resp.
ligne 5). Par exemple, pour A et B définis plus haut, nxor = \texttt{1110\,0111}
et la fonction essaie i = 4, puis i = shorter[4] = 2, puis i = longer[2] = 3,
puis i = longer[3] = -3, pour finalement retourner mask[3] = \texttt{0xE0}, qui
est précisément \texttt{1110\,0000}.



\subsection{Recherche}



\subsubsection*{Recherche exacte}

Cet algorithme retourne le block $B$ tel que l'adresse de base de $B$ soit égale
au pointeur passé en paramètre. On suppose que l'algorithme n'est utilisé que
lorsqu'un tel bloc existe.

Tant qu'on n'est pas sur une feuille -- on est donc sur un n\oe{}ud ayant deux
fils
-- on se dirige vers le fils ayant le plus grand préfixe commun avac l'adresse
passée en paramètre de l'algorithme. Quand on arrive sur une feuille, cette
dernière contient donc forcément l'adresse que l'on cherchait.

\subsubsection*{Recheche du ``contenant''}

Cet algorithme retourne le block $B$ contenant l'adresse $ptr$ passée en
paramètre, tel que : $beginAddr_B \le ptr < beginAddr_B + size_B$. Si un tel
bloc n'existe pas, \lstinline{NULL} est renvoyé. Cet algorithme est utilisé pour
des requêtes du type \lstinline{\valid} ou \lstinline{\initialized} où
l'adresse passée en paramètre ne correspond pas forcément à l'adresse de base
d'un bloc.

Cet algorithme est similaire à celui de la recherche exacte, au détail près
qu'il faut vérifier la formule de l'encadrement de $ptr$ par $B$ quand on arrive
sur une feuille. On utilise une pile contenant les n\oe{}uds à partir desquels
il
faut reprendre le parcours si la recherche n'aboutit pas. Quand on explore le
fils droit, on empile le fils gauche. Quand on arrive sur une feuille, on
vérifie l'encadrement, s'il est vérifié on a trouvé le bloc qu'on cherchait,
sinon on utilise le dernier n\oe{}ud empilé (et on le dépile) s'il existe, sinon
on retourne \lstinline{NULL}.

\subsection{Ajout}


\begin{figure}[h]
  \begin{center}
    \begin{tabular}{ccc}
      \begin{tikzpicture}[grow=down,sibling distance=18mm,level distance=6mm,
          style={font=\scriptsize}]
        \node[pnode] {\texttt{0010\,****}}
        child { node[pleaf] {\texttt{0010\,0110}} }
        child { node[pnode] {\texttt{0010\,1***}}
          child { node[pleaf] {\texttt{0010\,1001}} }
          child { node[pleaf] {\texttt{0010\,1101}} }
        };
        \node at (-1.7,0) {\textbf{a)}};
      \end{tikzpicture}
& 
      \hspace{1mm} 
&
      \begin{tikzpicture}[grow=down,level 2/.style={sibling distance=17mm},
          sibling distance=35mm,level distance=6mm,style={font=\scriptsize}]
        \node[pnode] {\texttt{0010\,****}}
        child { node[pnode] {\texttt{0010\,011*}}
          child { node[pleaf] {\texttt{0010\,0110}} }
          child { node[pleaf] {\texttt{0010\,0111}} }
        }
        child { node[pnode] {\texttt{0010\,1***}}
          child { node[pleaf] {\texttt{0010\,1001}} }
          child { node[pleaf] {\texttt{0010\,1101}} }
        };
        \node at (-3.4,0) {\textbf{b)}};
     \end{tikzpicture}   
    \end{tabular}
  \end{center}
  \vspace{-3mm}
  \caption{Exemple de Patricia trie \textbf{a)} avant,
    et \textbf{b)} après insertion de \texttt{0010\,0111}}
  \vspace{-3mm}
  \label{fig:insertion-Patricia-trie}
\end{figure}


L'ajout d'un nouvel élément à l'arbre se déroule comme suit. Si l'arbre est
vide, l'élément est ajouté à la racine. Sinon, on recherche le n\oe{}ud $x$ le
plus
similaire au n\oe{}ud à insérer. Celui-ci sera son frère dans la nouvelle
configuration de l'arbre. Nous créons ensuite le n\oe{}ud correspondant au père.
Le
plus grand préfixe commun des deux fils est calculé pour le père, et les fils
sont ordonnés en fonction de leur adresse (le plus petit à gauche). Puis le
père est inséré dans l'arbre à l'ancien emplacement de $x$. Les champs de $x$ et
de l'ancien père de $x$ (s'il existe) sont également mis à jour pour maintenir
la cohérence de l'arbre. La Fig.~\ref{fig:insertion-Patricia-trie} illustre
l'insertion de l'adresse \ppleaf{\texttt{0010\,0111}} dans un arbre.
L'algorithme a déterminé que le n\oe{}ud le plus similaire serait
\ppleaf{\texttt{0010\,0110}}, et crée
le père correspondant : \ppnode{\texttt{0010\,011*}}.


\subsubsection*{Recherche du n\oe{}ud le plus similaire}

Dès qu'on arrive sur une feuille on la renvoie. Sinon on calcule le plus grand
préfixe commun de l'adresse passée en paramètre avec le fils gauche, de même
avec le fils droit. Si l'un des préfixes est strictement supérieur à l'autre
on se dirige vers la branche correspondante, sinon on renvoie le n\oe{}ud
courant.



\subsection{Suppression}


\begin{figure}[h]
  \begin{center}
    \begin{tabular}{ccc}
      \begin{tikzpicture}[grow=down,level 2/.style={sibling distance=17mm},
          sibling distance=35mm,level distance=6mm,style={font=\scriptsize}]
        \node[pnode] {\texttt{0010\,****}}
        child { node[pnode] {\texttt{0010\,011*}}
          child { node[pleaf] {\texttt{0010\,0110}} }
          child { node[pleaf] {\texttt{0010\,0111}} }
        }
        child { node[pnode] {\texttt{0010\,1***}}
          child { node[pleaf] {\texttt{0010\,1001}} }
          child { node[pleaf] {\texttt{0010\,1101}} }
        };
        \node at (-3.4,0) {\textbf{a)}};
     \end{tikzpicture}   
& 
      \hspace{1mm} 
&
       \begin{tikzpicture}[grow=down,sibling distance=18mm,level distance=6mm,
          style={font=\scriptsize}]
        \node[pnode] {\texttt{0010\,****}}
        child { node[pleaf] {\texttt{0010\,0110}} }
        child { node[pnode] {\texttt{0010\,1***}}
          child { node[pleaf] {\texttt{0010\,1001}} }
          child { node[pleaf] {\texttt{0010\,1101}} }
        };
        \node at (-1.7,0) {\textbf{b)}};
      \end{tikzpicture}
    \end{tabular}
  \end{center}
  \vspace{-3mm}
  \caption{Exemple de Patricia trie \textbf{a)} avant,
    et \textbf{b)} après suppression de \texttt{0010\,0111}}
  \vspace{-3mm}
  \label{fig:suppression-Patricia-trie}
\end{figure}

La suppression d'une feuille $x$ du Patricia trie se déroule comme suit. Si cet
élément est la racine, l'arbre devient vide. Sinon, Cette feuille a un frère $y$
et un père. $x$ et son père sont supprimés, et $y$ remonte d'un niveau (et prend
donc la place du père). Enfin, les champs du nouveau père de $y$ sont mis à jour
pour maintenir la cohérence de l'arbre (le plus grand préfixe commun est
recalculé). La Fig.~\ref{fig:suppression-Patricia-trie} illustre la suppression
de l'adresse \ppleaf{\texttt{0010\,0111}} dans un Patricia trie. Ce n\oe{}ud
ainsi que son père \ppnode{\texttt{0010\,011*}} sont supprimés, et
\ppleaf{\texttt{0010\,0110}} est remonté d'un niveau.


\section{Expérimentations}


Pour évaluer notre solution, nous avons effectué plus de 300 exécutions sur
plus de 30 programmes, obtenus à partir d'une dizaine d'exemples. Nous avons
volontairement gardés des exemples plutôt courts (moins de 200 lignes de code)
car ils ont dû être annotés en \acsl manuellement.

Nous avons mesuré le temps d'exécution du programme d'origine et du code
instrumenté par \eacsltoc avec différentes options, afin d'évaluer les
performances des différentes implémentations et optimisations. Des indicateurs
comme le nombre de variables, d'allocations mémoires, d'enregistrements et de
requêtes a également été enregistré.



\begin{description}

\item[Implémentation du store] \hfill \\
Pour déterminer quelle implémentation du $store$ est la plus appropriée, nous
avons comparé des implémentations utilisant : des Patricia tries, des listes
chaînées, des arbres binaires de recherche non équilibrés et des Splay trees.

Notre implémentation utilisant les Patricia tries est en moyenne 2500 fois plus
rapide que l'implémentation à base de listes chaînées, 200 fois plus rapide que
celle utilisant les arbres binaires de recherche, et 27 fois plus rapide que
celle se basant sur les Splay trees.

La version utilisant les Splay trees offre
des performances comparables (ou légèrement meilleures, jusqu'à 3 fois) sur les
exemples contenant de fréquents accès mémoire consécutifs au même bloc dans le
$store$. En revanche, sur des examples où les accès méoire consécutifs ne se
font pas sur le même bloc (une multiplication de matrices dans notre exemple),
les performances sont beaucoup moins bonnes (jusqu'à 500 fois). Ceci est dû à
la nature des Splay treees : le dernier élément accédé est remonté à la racine
de l'arbre.

\item[Calcul du plus grand préfixe commun] \hfill \\
Nous avons comparé deux implémentations de ce calcul. La première utilise un
parcours linéaire de l'adresse (bit-à-bit, de gauche à droite). La seconde
est une recherche dichotomique du meilleur préfixe dans un tableau dont le
contenu et les indices (indiquant le prochain élément à tester) sont
pré-calculés. Cette seconde implémentation s'est révélée en moyenne 2.7 fois
plus rapide que la première sur nos exemples.

\item[Capacité de détection d'erreurs] \hfill \\
Nous avons utilisé le ``test mutationnel'' pour évaluer la capacité de détection
d'erreurs en utilisant la vérification d'assertion à l'exécution avec
\framac. Nous avons considéré 5 exemples annotés et généré leurs
{\em mutants} (en appliquant une {\em mutation} sur leur code source) et leur
avons appliqué la vérification à l'exécution. Les mutations incluent :
modifications d'opérateur arithmétique numérique, modifications d'opérateur
arithmétique sur les pointeurs, modifications d'opérateur de comparaison et
modifications d'opérateur logique ($land$ et $lor$).

L'outil de génération de test \pathcrawler \cite{\citepathcrawler} a été
utilisé pour produire les cas de test. Chaque mutant a été instrumenté par
\eacsltoc et exécuté sur chaque cas de test pour vérifier que la
spécification était satisfaite à l'exécution. Les programmes d'origine passent
toutes les vérifications à l'exécution. Lorsqu'une violation d'une annotation a
été reportée pour au moins un cas de testn le mutant est considéré comme étant
{\em tué}. La Table~\ref{tab:mutation-exp} illustre les résultats. Exception
faite des mutants équivalents (lorsque la mutation produit un programme
équivalent au programme d'origine), tous les mutants erronés ont été tués.

\end{description}

La Table~\ref{tab:mmodel-exp} contient les résultats des expérimentations
comparant les différentes implémentations du $store$ et du calcul du plus grand
préfixe commun, la Fig.~\ref{fig:mmodel-exp} représente graphiquement ces
données. bS$_{10000}$ est une recherche binaire dans un tableaux de 10000
éléments. iS$_{10000}$ est un tri par insertion d'un tableau de 10000 éléments.
mM$_{n^2}$ est une multiplication de matrices $n \times n$. mI$_{n^2}$ contient
des calculs matriciels (dont inversion et multiplication) sur des  matrices
$n \times n$. qS$_n$ est un tri rapide sur un tableau de $n$ éléments.
bbS$_{10000}$ est un tri à bulles sur un tableau à 10000 éléments. m$_{30000}$ est
une fusion de deux listes chaînées de 10000 et 20000 éléments. Rbt$_{10000}$ est
une insertion/suppression de 10000 éléments dans un arbre rouge et noir. mS$_n$
est un tri fusion d'une liste chaînée de $n$ éléments. La ligne supplémentaire
``+ RTE'' de chaque exemple correspond à une application préalable du greffon
\rte qui génère des assertions qui sont vraies si le programme ne
contient pas d'erreur à l'exécution.

Les colonnes ont la signification suivante : \danger contient le nombre
d'alarmes du programme,  $\emptyset$ contient le temps d'exécution du programme
original, bst correspond à l'implémentation par arbres binaires de recherche,
mask est le nombre de fois qu'est effectué le calcul du plus grand préfixe
commun, sb est le nombre d'insertion dans le $store$, Pt correspond à
l'implémentation par Patricia tries, St correspond à l'implémentation par Splay
trees. L'exposant $^1$ (respectivement $^2$) correspond aux expérimentations
sans (resp. avec) application d'une analyse statique $Dataflow$ permettant de
n'instrumenter que ce qui est nécessaire (section 6 de \cite{\citeeacsltoc}).
L'indice $_1$ (resp. $_2$)
correspond à l'implémentation non optimisée (resp. optimisée) du calcul du
plus grand préfixe commun pour l'implémentation utilisant les Patricia tries.
Le temps d'analyse du programme avec \valgrind \cite{\citevalgrind} est
indiqué dans la dernière colonne.

Nous remarquons que le temps d'exécution de \valgrind n'est pas
comparable avec celui de notre solution, cela s'explique simplement par le fait
que celui-ci ne prend pas en compte la spécification \acsl, et se
contente de vérifier des propriétés comme l'absence d'erreur de segmentation ou
l'absence de fuite de mémoire. En revanche, notre démarche vise à supporter au
maximum les annotations \acsl, ce qui nécessite un monitoring plus
lourd.

Nos expérimentations, présentées dans la Fig.~\ref{fig:mmodel-exp}, confirment
nos hypothèses, à savoir :
\begin{itemize}
\item le Patricia trie est la structure de données la plus appropriée pour
  l'implémentation du $store$;
\item notre optimisation du calcul du plus grand préfixe commun par recherche
  dichotomique et utilisation d'indices pré-calculés entraîne un vrai gain de
  performance;
\item l'utilisation d'une analyse statique visant à réduire l'instrumentation
  du programme permet de réduire le temps d'exécution de manière efficace.
\end{itemize}


\begin{landscape}
  \begin{table}[h]
    \centering
    \begin{footnotesize}
    \begin{tabular}{l|c|c|c|c|c|c|c|HHc|c|HHc|c|c|c}
  & $\danger$ & $\emptyset$ & list & list-AS & bst & bst-AS & Pt & mask$^1$ & sb$^1$ & Pt-dicho & Pt-AS & mask$^2$ & sb$^2$ & Pt-dicho-AS & St & St-AS & valgrind \\
  \hline
  bS$_{10000}$ &22 &\multirow{2}{*}{.01} &1.10 &0.50 &1.14 &0.64 &1.55 &99 &16 &1.55 &0.57 &0 &1 &0.55 &1.39 &0.61 &\multirow{2}{*}{0.27}\\
  + RTE &41 &&1.10 &0.51 &1.14 &0.62 &1.59 &109 &16 &1.59 &0.53 &0 &1 &0.53 &1.39 &0.64 &\\
  \hline
  iS$_{10000}$ &5 &\multirow{2}{*}{.12} &2.29 &0.12 &1.83 &0.12 &2.89 &170k &20k &2.91 &0.12 &0 &0 &0.12 &2.46 &0.12 &\multirow{2}{*}{2.81}\\
  + RTE &24 &&3.52 &1.27 &2.89 &1.26 &3.99 &170k &20k &3.86 &1.25 &0 &0 &1.25 &3.46 &1.30 &\\
  \hline
  mM$_{100^2}$ &0 &\multirow{2}{*}{.01} &2.29 &0.73 &2.94 &1.15 &0.14 &17k &1k &0.14 &0.10 &5k &612 &0.09 &1.07 &0.98 &\multirow{2}{*}{0.34}\\
  + RTE &82 &&13.17 &10.66 &21.92 &17.39 &2.78 &18k &1k &3.00 &2.64 &5k &612 &2.82 &75.97 &73.62 &\\
  \hline
  mM$_{150^2}$ &0 &\multirow{2}{*}{.01} &13.23 &3.96 &15.65 &6.20 &0.54 &21k &2k &0.51 &0.36 &9k &912 &0.35 &5.86 &5.64 &\multirow{2}{*}{0.48}\\
  + RTE &82 &&72.36 &58.48 &110.70 &90.43 &10.77 &24k &2k &9.01 &8.57 &9k &912 &8.75 &403.50 &398.60 &\\
  \hline
  mI$_{100^2}$ &2 &\multirow{2}{*}{.01} &22.51 &0.10 &7.74 &0.13 &0.09 &68k &5k &0.08 &0.01 &7k &609 &0.01 &0.19 &0.10 &\multirow{2}{*}{0.35}\\
  + RTE &155 &&28.96 &4.22 &13.67 &5.48 &0.54 &73k &5k &0.55 &0.53 &7k &611 &0.47 &26.37 &26.16 &\\
  \hline
  mI$_{150^2}$ &2 &\multirow{2}{*}{.02} &130.04 &0.34 &40.35 &0.45 &0.28 &99k &8k &0.27 &0.02 &12k &909 &0.02 &0.68 &0.34 &\multirow{2}{*}{0.47}\\
  + RTE &155 &&153.30 &21.54 &73.55 &29.94 &2.00 &105k &8k &1.90 &1.42 &12k &911 &1.53 &146.15 &145.80 &\\
  \hline
  qS$_{1000}$ &15 &\multirow{2}{*}{.01} &12.70 &2.08 &1.76 &0.59 &0.33 &1M &92k &0.06 &0.13 &683k &39k &0.02 &0.02 &0.01 &\multirow{2}{*}{0.27}\\
  + RTE &32 &&12.38 &2.13 &1.64 &0.56 &0.38 &1M &92k &0.12 &0.14 &727k &39k &0.04 &0.03 &0.02 &\\
  \hline
  qS$_{2000}$ &15 &\multirow{2}{*}{.01} &85.99 &11.31 &8.39 &2.78 &0.71 &3M &198k &0.14 &0.28 &1M &84k &0.05 &0.03 &0.02 &\multirow{2}{*}{0.27}\\
  + RTE &32 &&81.65 &11.15 &7.72 &2.67 &1.13 &4M &198k &0.48 &0.36 &1M &84k &0.13 &0.05 &0.02 &\\
  \hline
  bbS$_{10000}$ &4 &\multirow{2}{*}{.22} &13.78 &1.02 &16.84 &1.67 &117.47 &499M &49M &22.36 &1.57 &30 &7 &1.54 &8.80 &1.67 &\multirow{2}{*}{3.36}\\
  + RTE &16 &&23.08 &4.64 &30.69 &7.16 &107.05 &599M &49M &32.58 &7.26 &29 &7 &6.90 &17.29 &7.21 &\\
  \hline
  m$_{30000}$ &2 &\multirow{2}{*}{.01} &412.10 &11.38 &176.35 &11.01 &1.11 &5M &420k &0.26 &0.30 &1M &60k &0.06 &0.08 &0.01 &\multirow{2}{*}{0.45}\\
  + RTE &49 &&451.58 &101.33 &219.12 &94.80 &1.15 &5M &420k &0.29 &0.47 &2M &130k &0.14 &0.10 &0.05 &\\
  \hline
  Rbt$_{10000}$ &0 &\multirow{2}{*}{.01} &47.39 &0.28 &48.44 &0.27 &0.32 &1M &159k &0.09 &0.03 &151k &10k &0.01 &0.59 &0.01 &\multirow{2}{*}{0.51}\\
  + RTE &270 &&120.02 &101.69 &165.77 &145.20 &0.47 &1M &159k &0.30 &0.39 &979k &119k &0.27 &18.82 &19.59 &\\
  \hline
  mS$_{1000}$ &7 &\multirow{2}{*}{.01} &6.45 &0.34 &6.32 &0.11 &0.32 &1M &95k &0.07 &0.06 &331k &18k &0.01 &0.02 &0.01 &\multirow{2}{*}{0.27}\\
  + RTE &45 &&6.82 &1.35 &7.98 &0.38 &0.34 &1M &95k &0.10 &0.13 &701k &38k &0.04 &0.02 &0.01 &\\
  \hline
  mS$_{5000}$ &7 &\multirow{2}{*}{.01} &362.87 &11.00 &218.01 &3.43 &2.28 &11M &562k &0.76 &0.43 &2M &106k &0.09 &0.14 &0.03 &\multirow{2}{*}{0.27}\\
  + RTE &45 &&371.40 &50.94 &290.88 &10.34 &2.46 &11M &562k &0.80 &0.83 &4M &218k &0.22 &0.16 &0.08 &\\
  \hline
  mS$_{10000}$ &7 &\multirow{2}{*}{.01} &3624.01 &47.94 &1673.00 &16.10 &6.46 &23M &1M &2.75 &1.00 &5M &223k &0.21 &0.31 &0.08 &\multirow{2}{*}{0.27}\\
  + RTE &45 &&3406.43 &257.18 &2086.32 &46.22 &6.30 &23M &1M &2.66 &1.83 &9M &457k &0.51 &0.35 &0.18 &\\
  \hline
  mS$_{50000}$ &7 &\multirow{2}{*}{.01} &$\infty$ &3847.72 &$\infty$ &1100.93 &135.54 &146M &6M &111.22 &6.90 &33M &1M &1.65 &2.08 &0.58 &\multirow{2}{*}{0.63}\\
  + RTE &45 &&$\infty$ &25554.08 &$\infty$ &2781.90 &118.86 &145M &6M &95.74 &11.64 &54M &2M &3.37 &2.18 &1.15 &\\
  \hline
  mS$_{100000}$ &7 &\multirow{2}{*}{.01} &$\infty$ &$\infty$ &$\infty$ &$\infty$ &631.41 &296M &14M &559.93 &13.55 &70M &2M &3.35 &4.03 &1.15 &\multirow{2}{*}{0.27}\\
  + RTE &45 &&$\infty$ &$\infty$ &$\infty$ &$\infty$ &573.47 &308M &14M &513.85 &25.02 &116M &5M &7.63 &4.68 &2.50 &\\
\end{tabular}

    \end{footnotesize}
    \label{tab:mmodel-exp}
    \caption{Comparaison des différentes implémentations du $store$}
  \end{table}
\end{landscape}


\begin{figure}[bt]
  \begin{tikzpicture}
    \begin{axis}[axis y line=left,width=\textwidth,height=\textwidth,ymode=log]
      \pgfplotstableread{data/table_eacsl_experiments_merge_sort.dat}
      \loadedtable;
      \foreach \i in {
        list,list-DFA,bst,bst-DFA,Pt,Pt-opti,Pt-DFA,Pt-opti-DFA,St,St-DFA} {
        \addplot table [x=N, y=\i] {\loadedtable};
      }
      \legend{list,list-DFA,bst,bst-DFA,Pt,Pt-opti,Pt-DFA,Pt-opti-DFA,St,St-DFA}
    \end{axis}
  \end{tikzpicture}
  \label{fig:mmodel-exp}
  \caption{Comparaison des différentes implémentations du $store$}
\end{figure}


\begin{table}[tb]
  \centering
  \begin{tabular}{c|c|c|c|c|c}
    & alarmes & mutants & équivalents & tués & \% erronés tués \\
    \hline
    fibonacci & 19  & 27 & 2 & 25 & 100\% \\
    \hline
    bubbleSort & 15  & 44 & 2 & 42 & 100\% \\
    \hline
    insertionSort & 10  & 39 & 3 & 36 & 100\% \\
    \hline
    binarySearch & 7 & 38 & 1 & 37 & 100\% \\
    \hline
    merge & 5 & 92 & 5 & 87 & 100\% \\
  \end{tabular}
  \label{tab:mutation-exp}
  \caption{Capacité de détection d'erreurs}
\end{table}
