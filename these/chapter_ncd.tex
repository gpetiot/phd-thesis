
\chapter{Découverte de Non-Conformités}
\label{sec:ncd}

\chapterintro


Il est souvent difficile de déterminer la raison de l'échec de la vérification
déductive d'un programme C annoté en \eacsl.
Dans ce chapitre et le suivant, on suppose que l'échec de vérification ne vient
pas des limitations du prouveur.
L'échec vient donc de problèmes dans le code et/ou dans sa spécification.
On distingue deux sortes de problèmes, de nature très différente : la
non-conformité entre un code et sa spécification, approfondie dans ce chapitre,
et la faiblesse de spécification, étudiée dans le chapitre suivant.
La non-conformité entre un code et sa spécification est une notion de correction
relative entre code et spécification.
Ce n'est que lorsqu'une spécification et un code sont corrects, c'est-à-dire
conformes, qu'il est pertinent d'étudier le second problème, celui de la
précision suffisante de certaines annotations pour permettre d'en démontrer
d'autres.
C'est pourquoi cette seconde question -- de complétude de la spécification --
est étudiée après la question de non-conformité étudiée ici.

Il n'est pas assez précis de parler d'erreur dans le code et/ou dans sa
spécification.
En effet, on ne s'intéresse pas ici directement à la détection dans le code
d'erreurs dites ``à l'exécution'' (\textit{runtime error}), telles que les
divisions par zéro ou les accès à un tableau hors de ses bornes.
On ne s'y intéresse qu'indirectement, par vérification déductive d'assertions
précédant immédiatement un code susceptible de générer une telle erreur.
Une telle assertion doit être fausse lorsque les conditions de déclenchement de
l'erreur sont réunies.
On ne considère donc que des erreurs concernant le code \textbf{et} sa
spécification.
De plus, parmi ces erreurs, on ne considère que celles que nous pouvons
détecter par analyse dynamique d'une instrumentation du programme annoté,
c'est-à-dire qu'elles correspondent à la violation d'une annotation du fragment
d'\eacsl défini au chapitre~\ref{sec:lang}.
Ce sont ces erreurs que nous appelons ``non-conformités''.

Nous proposons plusieurs scénarios de détection de non-conformités entre un code
et sa spécification, par instrumentation du programme annoté et génération de
tests structurels pour cette instrumentation.
La partie~\ref{sec:ncd-def} définit la notion de non-conformité.
La partie~\ref{sec:ncd-ex} présente l'exemple que nous utilisons tout au long
de ce chapitre pour illustrer nos propos.
La partie~\ref{sec:ncd-scenarios} illustre la détection de non-conformités dans
deux scénarios construits sur l'exemple présenté en partie~\ref{sec:ncd-ex}.


\section{Définition de la non-conformité}
\label{sec:ncd-def}


Soit $P$ un programme C annoté avec \eacsl, et $f$ la fonction sous vérification
de $P$.
La fonction $f$ est supposée non-récursive.
Elle peut appeler d'autres fonctions, notons $g$ l'une d'entre elles.
Un \emph{cas de test} $V$ pour $f$ est un vecteur de valeurs pour chaque
variable d'entrée de $f$.
Le \emph{chemin de programme} activé par un cas de test $V$, noté $\pi_V$, est
la séquence d'instructions exécutées par le programme $P$ sur le cas de test
$V$.
Nous utilisons le terme générique de \emph{contrat} pour désigner l'ensemble des
annotations \eacsl décrivant une boucle ou une fonction. 
Un contrat de fonction est composé de pré- et postconditions incluant les
clauses \lstinline{requires}, \lstinline{assigns} et \lstinline{ensures}.
Un contrat de boucle est composé des clauses \lstinline{loop invariant},
\lstinline{loop variant} et \lstinline{loop assigns}.

Le chapitre~\ref{sec:traduction} a présenté la transformation d'un programme
$P$ annoté avec \eacsl en un programme instrumenté (dénoté $P^\NC$ dans la
suite) sur lequel nous pouvons appliquer la génération de tests pour produire
des données de test violant des annotations à l'exécution.
$P^\NC$ vérifie toutes les annotations de $P$ aux endroits correspondants dans
le programme et reporte toute violation d'annotation éventuelle.
Par exemple, la postcondition $\textit{Post}_f$ de $f$ est évaluée par le code
suivant, inséré à la fin de la fonction $f$ dans $P^\NC$ :

\begin{equation}
\mbox{\lstinline[mathescape]'int post_f; $~\mathit{Spec2Code}$($\textit{Post}_f$, post_f); fassert(post_f);'}
\end{equation}

Pour un prédicat \eacsl \lstinline[mathescape]'$\textit{P}$',
nous notons par \lstinline[mathescape]'$\mathit{Spec2Code}$($\textit{P}$, b)'
le code C généré qui évalue le prédicat \lstinline[mathescape]'$\textit{P}$'
et donne sa valeur de vérité à la variable booléenne \lstinline[mathescape]'b'.
De manière similaire, les préconditions et postconditions d'une fonction appelée
$g$ sont évaluées respectivement avant et après l'exécution de la fonction $g$.
Un invariant de boucle est vérifié avant la boucle (on dit qu'il est établi) et
après chaque pas de la boucle (on dit qu'il est préservé par l'itération).
Une assertion est vérifiée à sa position courante.
Afin de produire uniquement des données de test qui respectent la précondition
$\textit{Pre}_f$ de $f$, cette précondition est testée au début de $f$ de sorte
à admettre la précondition :

\begin{equation}
\mbox{\lstinline[mathescape]'int pre_f; $~\mathit{Spec2Code}$($\textit{Pre}_f$, pre_f); fassume(pre_f);'}
\end{equation}

\begin{definition}[Non-conformité] 
  \label{def:NC}
  Il y a une \emph{non-conformité} entre le code d'une fonction $f$ et sa
  spécification dans le programme $P$ s'il existe un cas de test $V$ pour $f$
  respectant sa précondition, tel que $P^{\NC}$ produit une violation
  d'annotation à partir de $V$.
  On dit que $V$ est un \emph{contre-exemple de non-conformité} (ou \NCCE pour
  \textit{non-compliance counter-example}).
\end{definition}

La génération de tests peut être utilisée sur le programme instrumenté $P^{\NC}$
pour générer des \NCCE{}s.
Nous appelons cette technique \emph{Détection de Non-Conformité} (ou \NCD pour
\textit{Non-Compliance Detection}).
Dans nos travaux, nous utilisons le générateur de tests \pathcrawler qui essaie
de couvrir tous les chemins d'exécution faisables du programme.
Étant donné que l'étape de traduction a ajouté des branches dans le programme
afin d'expliciter la violation des annotations, \pathcrawler essaie de couvrir
au moins un chemin d'exécution où l'annotation n'est pas vérifiée.

L'étape \NCD peut produire trois résultats.
Elle retourne un triplet (\nc,\,$V$,\,$a$) si un \NCCE $V$ a été trouvé,
indiquant que le chemin de programme $\pi_V$ activé par $V$ sur $P^{\NC}$
provoque une violation de l'annotation $a$.
Si \NCD a exploré tous les chemins du programme sans trouver de \NCCE, elle
retourne ``\no''.
Sinon, si la couverture des chemins du programme par \NCD est partielle (si
un {\em timeout} a été atteint par exemple), elle retourne ``\textsf{?}'', ce
qui signifie qu'elle n'a pas été en mesure de conclure.


\section{Exemple illustratif}
\label{sec:ncd-ex}


\begin{figure}[t]
  \lstinputlisting{listings/deleteSubstrTrous.c}
  \vspace{-3mm}
  \caption{Fonction \lstinline{delete_substr} non spécifiée, appelant la
    fonction de la figure~\,\ref{fig:findSubstr}}
  \vspace{-3mm}
  \label{fig:deleteSubstrTrous}
\end{figure}

\begin{figure}[t]
  \lstinputlisting{listings/findSubstr.c}
  \vspace{-3mm}
  \caption{Contrat \eacsl de la fonction vérifiée \lstinline{find_substr}}
  \vspace{-3mm}
  \label{fig:findSubstr}
\end{figure}

Supposons qu'Alice est une ingénieure validation en charge de la spécification
et de la vérification déductive de la fonction \lstinline{delete_substr} de la
figure~\ref{fig:deleteSubstrTrous}.
Dans ce chapitre, nous suivons Alice pendant le processus de validation afin
d'illustrer les problèmes et les différents choix possibles que rencontre un
ingénieur validation en charge d'une preuve de programme.

Le comportement attendu de la fonction \lstinline{delete_substr} est de
supprimer une occurrence de la sous-chaîne \lstinline{substr} de longueur
\lstinline{sublen} au sein d'une chaîne \lstinline{str} de longueur
\lstinline{strlen}.
Le résultat de cette opération est affecté à la chaîne \lstinline{dest} de
longueur \lstinline'strlen', allouée au préalable.
Les chaînes \lstinline{str} et \lstinline{substr} ne doivent pas être modifiées.
Pour des raisons de simplicité, nous utilisons des tableaux plutôt que des
chaînes de caractères ``à la C'', terminées par un 0.
La fonction \lstinline{delete_substr} retourne $1$ si une occurrence de la
sous-chaîne a été trouvée et supprimée, et $0$ sinon.
La figure~\ref{fig:findSubstr} présente le contrat de la fonction
\lstinline{find_substr}, appelée par la fonction \lstinline{delete_substr}, qui
retourne l'index d'une occurrence de \lstinline{substr} dans \lstinline{str} si
la sous-chaîne est présente, et $-1$ sinon.
Nous supposons qu'Alice a déjà prouvé avec succès la correction de la fonction
\lstinline{find_substr} vis-à-vis de son contrat.


\section{Scénarios de détection de non-conformités}
\label{sec:ncd-scenarios}


Pendant le processus de spécification et de vérification déductive, la
génération de tests peut fournir automatiquement à l'ingénieur validation un
retour rapide et pertinent afin de l'aider dans son travail.
En effet, la génération de tests peut trouver un contre-exemple mettant en
évidence le non respect de la spécification par le code ou l'inverse.
Ceci est applicable dès le début du processus de spécification, lorsque la
vérification déductive ne peut pas aboutir car le programme à vérifier n'est
pas entièrement spécifié (il manque par exemple des contrats de boucle et de
fonctions appelées).
La génération de tests peut également être utile dans le cas d'un échec de
preuve concernant une propriété de programme, lorsque l'ingénieur validation
n'a pas d'autre alternative que d'analyser manuellement le programme afin de
découvrir la raison de cet échec.

L'absence de \NCCE après une exploration partielle (ou, si possible, complète)
des chemins du programme donne à l'ingénieur validation une confiance accrue
(respectivement une garantie) quant à la conformité du code vis-à-vis de la
spécification.
Cette information pousse l'ingénieur à penser que l'échec de la preuve est dû 
soit à un manque dans la spécification, soit à une annotation trop faible,
plutôt qu'à une erreur, ce qui l'incite à annoter davantage le programme.

Dans le cas contraire, un contre-exemple met immédiatement en évidence le
non-respect de la spécification par le programme (ou l'inverse), signifiant pour
l'ingénieur validation que cette non-conformité doit être corrigée avant de
poursuivre la preuve.
Remarquons que l'objectif n'est certainement pas de faire correspondre à tout
prix la spécification à un code (potentiellement erroné), mais d'aider
l'ingénieur validation à identifier le problème (dans la spécification ou dans
le code).

Nous illustrons maintenant nos propos sur des scénarios de vérification
concrets que l'ingénieure validation Alice pourrait rencontrer.


\subsection{Validation des contrats de la fonction sous vérification}
\label{sec:ncd-early}


Supposons qu'Alice écrive tout d'abord la précondition suivante (ajoutée avant
la ligne 1 de la figure~\ref{fig:deleteSubstrTrous}) :

\begin{pretty-codeACSL}
requires 0 < sublen <= strlen;
requires \valid(str+(0..strlen-1));
requires \valid(dest+(0..strlen-1));
requires \valid(substr+(0..sublen-1));
requires \separated(dest+(0..strlen-1), substr+(0..sublen-1));
requires \separated(dest+(0..strlen-1), str+(0..strlen-1));
typically strlen <= 5;
\end{pretty-codeACSL}

Nous rappelons que la clause \lstinline'typically p;' étend \eacsl et définit
une précondition \lstinline'p' qui n'est prise en compte que par la génération
de tests.
Cette clause permet à Alice de renforcer la précondition si elle souhaite
restreindre l'exploration des chemins du programme, en diminuant leur nombre.
La couverture des chemins du programme est alors partielle et contrôlée par
l'utilisateur.
Ici, la clause \lstinline'typically strlen <= 5' contraint la génération de
tests à ne considérer que les chemins faisables pour lesquels la chaîne
\lstinline'str' contient au plus cinq caractères.
Étant ignorée par la vérification déductive, cette clause n'impacte pas la
preuve du programme.

Puis Alice spécifie que la fonction peut uniquement modifier le contenu du
tableau \lstinline'dest', et définit la postcondition dans le cas où la
sous-chaîne n'est pas présente dans la chaîne \lstinline'str'.
Elle ajoute les clauses (erronées) suivantes au contrat de la fonction, après
la précondition :

\begin{pretty-codeACSL}
assigns dest[0..strlen-1];
behavior not_present:
 assumes !(\exists integer i; 0 <= i < strlen-sublen && 
  (\forall integer j; 0 <= j < sublen ==> str[i+j] != substr[j]));
 ensures \forall integer k; 0 <= k < strlen ==> \old(str[k]) == dest[k];
 ensures \result == 0;
\end{pretty-codeACSL}

Afin de valider cette postcondition avant d'aller plus loin, Alice applique
la méthode \NCD.
La génération de tests reporte que les deux clauses \lstinline{ensures} sont
invalidées par le contre-exemple suivant :
\lstinline{strlen = 2}, \lstinline{sublen = 1}, \lstinline{str[0] = 'A'},
\lstinline{str[1] = 'B'}, \lstinline{substr[0] = 'A'}, 
\lstinline{dest[0] = 'B'} and \lstinline{\result = 0}.
Alice remarque que dans ce cas, la chaîne \lstinline{substr} est présente dans
la chaîne \lstinline'str'.
Le comportement \lstinline'not_present' ne devrait donc pas s'appliquer, ce qui
indique que l'erreur vient de la clause \lstinline'assumes'.
Ceci aide Alice à corriger sa spécification en corrigeant cette clause en
remplaçant \lstinline[style=pretty-c]{!=} par \lstinline{==}, pour obtenir :

\begin{pretty-codeACSL}
  assumes !(\exists integer i; 0 <= i < strlen-sublen && 
   (\forall integer j; 0 <= j < sublen ==> str[i+j] == substr[j]));
\end{pretty-codeACSL}

Exécuter \NCD une seconde fois informe Alice que tous les chemins faisables pour
lesquels \lstinline{strlen <= 5} ont été couverts (en 3.4 secondes), que 9442
cas de tests ont été générés et qu'aucune violation d'annotation n'a été
observée.
Alice a maintenant confiance que le comportement \lstinline'not_present' est
correctement défini.

Concernant le comportement complémentaire, \lstinline{present}, Alice copie le
comportement \lstinline{not_present} et le modifie (avec erreur).
Elle écrit par exemple ceci :

\begin{pretty-codeACSL}
behavior present:
 assumes \exists integer i; 0 <= i < strlen-sublen && 
  (\forall integer j; 0 <= j < sublen ==> str[i+j] == substr[j]);
 ensures \exists integer i; 0 <= i < strlen-sublen &&
  (\forall integer j; 0 <= j < sublen ==> \old(str[i+j]) == \old(substr[j])) &&
  (\forall integer k; 0 <= k < i ==> \old(str[k]) == dest[k]) &&
  (\forall integer l; i <= l < strlen ==> \old(str[l+sublen]) == dest[l]);
 ensures \result == 1;
\end{pretty-codeACSL}

La détection de non-conformité rapporte une erreur à l'exécution lors de l'accès
à l'élément de la chaîne \lstinline{str} à l'indice \lstinline{l+sublen} dans
l'avant-dernière clause \lstinline{ensures}.
Cette information permet à Alice de comprendre que la borne supérieure de
l'indice \lstinline'l' devrait être \lstinline{strlen-sublen} au lieu de
\lstinline{strlen}.
Elle corrige cette erreur et ré-applique la \NCD.
La génération de tests rapporte que 13448 cas de tests couvrent sans erreur les
chemins faisables pour lesquels \lstinline{strlen <= 5}.
Alice est maintenant satisfaite avec les comportements du programme ainsi
définis.


\subsection{Validation des spécifications de boucles}
\label{sec:ncd-incr}


\begin{figure}[bt]
  \lstinputlisting{listings/deleteSubstr.c}
  \vspace{-3mm}
  \caption{Fonction \lstinline{delete_substr} spécifiée
    et prouvée automatiquement
  }
  \vspace{-3mm}
  \label{lst:delete-substr-full}
\end{figure}

Alice spécifie maintenant la première boucle \lstinline'for' de la ligne 4 de la
figure~\ref{fig:deleteSubstrTrous} :

\begin{pretty-codeACSL}
loop invariant \forall integer m; 0 <= m < k ==> dest[m] == \at(str[m],Pre);
loop assigns k, dest[0..strlen-1];
loop variant strlen-k;
\end{pretty-codeACSL}

La vérification déductive du programme échoue à prouver la postcondition de la
fonction \lstinline{delete_substr}, notamment parce que les deux autres boucles
du programme aux lignes 7 et 8 ne sont pas spécifiées.
En revanche, Alice s'attend à ce que la vérification déductive de ce premier
contrat de boucle réussisse.
Ce n'est malheureusement pas le cas et Alice n'arrive pas à déterminer si cet
échec de la preuve est dû à une non-conformité du code de la boucle vis-à-vis de
son contrat, ou parce que le contrat de boucle n'est pas assez précis pour être
vérifié automatiquement.

\NCD ne détecte aucune erreur dans le contrat de boucle et la postcondition,
couvrant le programme avec 15635 cas de tests.
Ces résultats poussent Alice à penser que la spécification de la boucle est
valide mais incomplète, ce qui la conduit à ajouter un invariant de boucle
supplémentaire, définissant les bornes de $k$ : 

\begin{pretty-codeACSL}
loop invariant 0 <= k < strlen;
\end{pretty-codeACSL}

Alice réessaie à présent de prouver la boucle, et la vérification déductive de
\Wp échoue une fois encore.
Elle ré-applique \NCD et cette fois le nouvel invariant de boucle est invalidé.
Après analyse de cet échec sur le contre-exemple produit, Alice comprend que
l'invariant de boucle \lstinline{k < strlen} n'est pas correct.
En effet, \lstinline{k} est égal à \lstinline{strlen} après la dernière
itération de la boucle.
L'invariant de boucle devrait alors être \lstinline{k <= strlen}.
Après correction de l'erreur, le contrat de boucle est prouvé avec succès par
\Wp.
De manière similaire, Alice spécifie et vérifie de manière itérative les deux
autres boucles du programme.

La fonction \lstinline{delete_substr} étant maintenant entièrement spécifiée
(figure~\ref{lst:delete-substr-full}), elle peut être prouvée par \Wp.
Cependant le {\em timeout} par défaut du prouveur (10 secondes par propriété)
doit être étendu de manière significative (par exemple 50 secondes par
propriété).
Lorsque la preuve du programme n'aboutit pas car un {\em timeout} a été atteint,
l'ingénieur validation doit décider s'il souhaite retenter la preuve avec
un {\em timeout} plus élevé ou s'il analyse manuellement le programme afin de
trouver une non-conformité ou une faiblesse de contrat.
Le premier choix peut causer une perte de temps importante à l'ingénieur qui
peut être évitée si \NCD parvient à produire un contre-exemple exhibant une
non-conformité.
Dans ce cas, \NCD peut aider l'ingénieur validation à se décider.
Le fait que la génération de tests parvienne (en 4 secondes) à couvrir une
portion significative des chemins du programme (restreinte par la clause
\lstinline'typically') sans qu'aucune non-conformité ne soit rapportée convainc
Alice d'augmenter le {\em timeout}.
Attention toutefois, car le risque que le contrat soit trop faible pour la
vérification déductive du programme n'a pas été écarté par \NCD.


\section*{Conclusion du chapitre}


Dans ce chapitre, nous avons défini la notion de non-conformité entre un code
et sa spécification.
Nous avons montré l'intérêt de la détection de non-conformité pour effectuer
des preuves de programmes.
Nous avons montré que la détection de non-conformité par génération de tests
rendue possible par la traduction des annotations permettait d'aborder la preuve
de manière incrémentale et itérative.
Nous avons illustré la démarche sur l'exemple de la
figure~\ref{fig:deleteSubstrTrous}.
Cette démarche a abouti au code et à la spécification de la
figure~\ref{lst:delete-substr-full}
qui est prouvée automatiquement avec le greffon \Wp de \framac
\footnote{nécessite l'option \texttt{-wp-split}, \framac Sodium, \altergo 0.95.2
  et \cvcthree 2.4.1}.
En effet, il est possible de spécifier partiellement un programme puis de
détecter si le code est conforme à cette partie avant de poursuivre le travail
de spécification.
La preuve peut être décomposée en une succession d'étapes de ce type.

En revanche, ce chapitre n'aborde pas le cas où un contrat d'une boucle ou
d'une fonction appelée (que nous appelerons sous-contrat) est trop faible pour
prouver entièrement le programme.
Cette difficulté du processus de spécification et de vérification est traité au
chapitre~\ref{sec:swd}, qui présente la détection de faiblesses de
sous-contrats.
Le chapitre~\ref{sec:method} montrera que la combinaison de ces deux méthodes
peut améliorer l'aide au diagnostic des échecs de preuve lors de la vérification
déductive de programmes.
