
\chapter{Bilan et perspectives}
\label{sec:end}


Dans ce chapitre, nous rappelons les objectifs de la thèse en
partie~\ref{sec:obj} puis nous présentons le bilan des travaux réalisés en
partie~\ref{sec:bilan} ainsi que les perspectives envisagées en
partie~\ref{sec:after}.


\section{Rappel des objectifs}
\label{sec:obj}


Notre objectif principal était de fournir une méthode de diagnostic automatique
des échecs de preuve afin de décider si la cause d'un échec de preuve est une
non-conformité entre le code et la spécification, ou la faiblesse d'un ou
plusieurs sous-contrats de la spécification, ou enfin une incapacité de l'outil
de vérification déductive.
Nos travaux ont donc pour but d'améliorer et de faciliter le processus de
spécification et de preuve des programmes qui est long et difficile et nécessite
une connaissance des outils de preuve de programmes.
En effet, il est souvent difficile pour l'utilisateur de décider laquelle de ces
trois raisons est la cause de l'échec de la preuve car cette information n'est
pas (ou rarement) donnée par le prouveur et requiert donc une revue approfondie
du code et de la spécification.

Notre objectif était également de fournir une implémentation efficace de cette
méthode de diagnostic des échecs de preuve, et d'évaluer les capacités de cet
outil sur différents programmes.


\section{Bilan}
\label{sec:bilan}


Nous proposons une méthode originale d'aide à la vérification déductive par
génération de tests structurels sur une version instrumentée du programme
d'origine.
Cette méthode consiste en une première étape de détection de non-conformités
puis en une seconde étape de détection de faiblesses de sous-contrats (contrats
de boucle ou de fonction appelée).

Nous avons dans un premier temps conçu une méthode de détection des
non-conformités du code par rapport à la spécification dans les programmes C.
Cette méthode nécessite une traduction en C des annotations \acsl afin
d'expliciter les violations potentielles d'annotations par la création de
nouvelles branches dans le programme.
La préservation de la sémantique par cette traduction nous assure de trouver
les erreurs dans le code et dans la spécification par génération de tests
structurels à condition de couvrir tous les chemins d'exécution faisables du
programme.
Nous avons formalisé cette traduction afin de justifier sa correction.
Puis nous avons implémenté cette méthode dans \stady et expérimenté son
efficacité par des mutations.
Il est apparu que sur des programmes mutants, nous étions capables de détecter
les erreurs introduites.
En revanche lorsqu'aucune erreur n'a été détectée, le programme mutant peut
être équivalent au programme d'origine, ou bien on peut avoir introduit une
faiblesse de sous-contrat.
Une telle faiblesse ne rend pas le programme invalide, mais empêche la
vérification déductive du programme.

Nous avons alors décidé dans un second temps d'adapter notre méthode de
détection des non-conformités afin de pouvoir détecter les faiblesses de
sous-contrats.
Cette nouvelle méthode nécessite une traduction en C des annotations \acsl
différente de la première, car nous voulons pouvoir ``exécuter'' certains
contrats au lieu du code qu'ils spécifient.
Nous avons intégré cette méthode à \stady et expérimenté son efficacité par des
mutations dans le cadre du diagnostic des échecs de preuve.
Nos expérimentations ont permis de mettre en évidence le gain apporté par cette
dernière méthode par rapport à une détection des non-conformités seule.

Nous avons ensuite conçu une méthode combinant la détection des non-conformités
et la détection des faiblesses de sous-contrats afin d'aider au diagnostic des
échecs de preuve.
Cette méthode combinée essaie d'abord d'exhiber des non-conformités.
Si aucune non-conformité n'est détectée, elle recherche des faiblesses de
sous-contrats.
Si aucune de ces deux phases n'a fourni de raison quant à l'échec de la preuve
et si la génération de tests a couvert tous les chemins d'exécution faisables
du programme, alors l'échec de la preuve est dû à une incapacité du prouveur,
sinon le problème reste non résolu.

Les différentes composantes de la méthode sont donc implémentées dans le
greffon \stady, mais la méthode de diagnostic elle-même ne l'est pas.
D'autre part, l'implémentation souffre de quelques limitations dues au support
incomplet du langage de spécification \acsl.
Ces deux points sont laissés en perspectives.


\subsection{Publications associées à la thèse}


Dans le cadre de nos travaux, nous avons publié quatre \commentGP{ou cinq}
articles de recherche lors de conférences internationales :

L'article~\cite{Kosmatov/RV13} présente notre méthode de validation à
l'exécution des annotations \acsl liées au modèle mémoire et l'implémentation
d'une bibliothèque permettant de valider ces annotations à l'exécution.

L'article~\cite{Petiot/TAP14} détaille l'aspect méthodologique de notre méthode
de détection des non-conformités entre le code et sa spécification.

L'article~\cite{Petiot/SCAM14} présente la traduction des annotations \acsl en
C pour la détection de non-conformités entre le code et la spécification.

L'article~\cite{Genestier/TAP15} recense plusieurs programmes C dont la
vérification déductive a été facilitée par \stady.

L'article~\cite{Petiot/ISSRE15} présente une méthode globale de diagnostic des
échecs de preuve combinant la détection des non-conformités entre le code et la
spécification et la détection des faiblesses de sous-contrats.


\section{Perspectives}
\label{sec:after}


Nous suggérons maintenant quelques perspectives faisant suite à nos travaux dans
le cadre des combinaisons d'analyses statiques et dynamiques et de l'aide à la
vérification déductive.


\subsection{Formalisation de la méthode}


Compte tenu de l'importance de la traduction des annotations \acsl en C dans
notre approche, aussi bien pour la détection des non-conformités que pour la
détection des faiblesses de sous-contrats, la correction de cette traduction
doit être assurée.
Nous avons donné une première justification de cette correction dans le
chapitre~\ref{sec:traduction}, mais il est possible d'aller plus loin dans cette
direction et de formaliser cette traduction à l'aide d'un outil comme \coq
\cite{\citecoq} ou \whythree \cite{\citewhythree}, d'une manière similaire à
\cite{Herms/VSTTE12}.

Il serait également utile d'appliquer ce principe au générateur de tests.
Ceci peut se traduire par la formalisation en \coq de certaines parties de
\pathcrawler et/ou par l'utilisation d'un solveur de contraintes certifié comme
celui formalisé dans~\cite{Carlier/FM12}.

Ces deux pistes permettraient de s'assurer formellement de la correction
des résultats produits par nos outils.


\subsection{Extension de la prise en charge d'acsl}


Dans nos travaux nous avons pris en charge la majorité des constructions du
langage \eacsl (sous-ensemble ``exécutable'' d'\acsl).
Il serait peut-être possible de traiter des constructions du langage \acsl qui
sont en dehors du langage \eacsl, comme les prédicats inductifs.
\cite{Tollitte/CPP12} propose une méthode d'extraction de code fonctionnel
(pour \coq) à partir de spécifications inductives.
Peut-être que ces travaux peuvent servir de base à une méthode de traduction
des spécifications inductives en code impératif (en C notamment) ce qui
permettrait de prendre en charge les prédicats inductifs dans \stady.

Notre implémentation ne traite pas non plus les annotations faisant intervenir
les nombres réels mathématiques et les nombres flottants du C.
De telles annotations permettent notamment d'exprimer les erreurs d'arrondi
se produisant lors des calculs sur les nombres flottants
\cite{Goubault/VMCAI11}.
Leur intégration au sein de \stady nécessiterait de formaliser leur traduction
en C et de pouvoir les exécuter symboliquement et concrètement lors de la
génération de tests, d'une manière similaire à ce que nous avons réalisé pour
les entiers mathématiques d'\eacsl.


\subsection{Automatisation de la méthode}


La méthode de détection des non-conformités et la méthode de détection des
faiblesses de sous-contrats sont implémentées dans \stady, mais la méthode les
combinant ne l'est pas.
En effet, l'utilisation actuelle de \stady est la suivante.
\stady doit être lancé une première fois en mode
``détection de non-conformités''' (son mode par défaut).
Puis, si aucune non-conformité n'a été détectée, \stady peut être lancé un
nombre quelconque de fois en mode ``détection de faiblesses de sous-contrats''.
Dans ce second mode, l'utilisateur indique à \stady pour quels contrats une
faiblesse doit être recherchée.
Il revient donc à l'utilisateur de réduire ou d'augmenter progressivement
l'ensemble des contrats au fur et à mesure des exécutions de \stady.

Ce processus pourrait être automatisé au cours d'une seule exécution de \stady
qui lancerait d'abord une détection de non-conformités puis une ou plusieurs
détections de faiblesses de sous-contrats.
Les contrats pour lesquels une faiblesse peut être recherchée pourraient être
déterminée en analysant les dépendances entre les annotations du programme.


\subsection{Expérimentations avec des utilisateurs}


Au cours de cette thèse nous avons expérimenté l'efficacité de notre outil en
terme de capacité de diagnostic et de temps d'exécution.
Néanmoins, nous n'avons pas pu mesurer l'impact réel de l'outil du point de vue
d'un utilisateur qui a pour tâche de spécifier et vérifier un programme.

Nous pourrions par exemple former deux groupes d'utilisateurs débutants dans le
domaine de la vérification déductive, avec le niveau le plus homogène possible.
Les utilisateurs du premier groupe auraient uniquement à leur disposition
l'outil de preuve, tandis que ceux du second groupe pourraient également
utiliser \stady.
Les deux groupes auraient à spécifier et prouver formellement des programmes
et nous pourrions ainsi comparer les résultats produits par chacun des groupes.

Une telle expérimentation permettrait de quantifier et qualifier l'aide au
diagnostic des échecs de preuve fourni par \stady et donc l'apport d'un tel
outil pour la spécification et la vérification déductive des programmes.
