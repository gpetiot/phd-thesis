
\chapter[Vérification liée au modèle mémoire]{Vérification à l'exécution des annotations liées au modèle mémoire}
\label{sec:runtime}

\chapterintro

Dans ce chapitre nous abordons la vérification dynamique des annotations liées
au modèle mémoire.
Ceci requiert de pouvoir simuler un modèle mémoire bas niveau afin de pouvoir
surveiller les opérations bas niveau des programmes C (allocations,
initialisations, etc.).
En revanche, le modèle mémoire utilisé par le générateur de tests structurels
que nous utilisons n'est pas assez bas niveau et ne permet donc pas l'exécution
symbolique de ces constructions.
En conséquence, nous changeons d'approche pour ces propriétés.
En effet, contrairement au chapitre~\ref{sec:traduction}, ces annotations ne
seront pas vérifiées par test structurel mais par validation à l'exécution
({\em runtime assertion checking}).

Dans un premier temps nous présentons les différentes annotations que nous ne
pouvons pas traiter par génération de tests et que nous qualifions d'opérations
de bas niveau (partie~\ref{sec:mem-annots}).
Puis nous présentons quelques détails de conception de notre modèle mémoire
(partie~\ref{sec:mem-model}).
Enfin nous abordons les principes de l'instrumentation nécessaire pour la
vérification de ces annotations (partie~\ref{sec:mem-instru}).


\section{Annotations liées au modèle mémoire}
\label{sec:mem-annots}


Définissons d'abord quelques notions.
Les objets de la mémoire sont des blocs.
Chaque bloc est caractérisé par une adresse de base, une taille (en nombre
d'octets) et un contenu (qui peut être initialisé ou non).
Tout ceci constitue le modèle mémoire.

Les annotations \eacsl que nous considérons sont décrites dans la
figure~\ref{fig:gram-ext}.
Cette figure est une extension de la grammaire des termes et des prédicats
\eacsl définie au chapitre~\ref{sec:lang}.


\begin{figure}
  \centering
  \begin{tabular}{cc}
    \begin{tabular}{lrl}
      \textit{term} ::= & ... \\
      & $\mid$
      & \underline{\lstinline'\\base_addr'} \underline{\lstinline'('} \textit{term} \underline{\lstinline')'} \\
      & $\mid$
      & \underline{\lstinline'\\block_length'} \underline{\lstinline'('} \textit{term} \underline{\lstinline')'} \\
      & $\mid$
      & \underline{\lstinline'\\offset'} \underline{\lstinline'('} \textit{term} \underline{\lstinline')'} \\
    \end{tabular}
    &
    \begin{tabular}{lrl}
      \textit{pred} ::= & ... \\
      & $\mid$
      & \underline{\lstinline'\\valid'} \underline{\lstinline'('} \textit{term} \underline{\lstinline')'} \\
      & $\mid$
      & \underline{\lstinline'\\valid_read'} \underline{\lstinline'('} \textit{term} \underline{\lstinline')'} \\
      & $\mid$
      & \underline{\lstinline'\\initialized'} \underline{\lstinline'('} \textit{term} \underline{\lstinline')'} \\
    \end{tabular}
  \end{tabular}
  \caption{Extension de la grammaire des termes et des prédicats
    \label{fig:gram-ext}}
\end{figure}



\begin{center}
  \begin{tikzpicture}[>=latex,font=\sffamily]
    \node [memcell] at (0,0) (A) {};
    \node [anchor=west,memcell,full] at (A.east) (B) {$bloc_1$};
    \node [anchor=west,memcell,full,minimum width=4cm] at (B.east) (C) {
      \begin{tikzpicture}
        \node[memcell,init] (p0) {\lstinline't[0]'};
        \node[memcell,init,anchor=west] at (p0.east) (p1) {\lstinline't[1]'};
        \node[memcell,uninit,anchor=west] at (p1.east) (p2){\lstinline't[2]'};
      \end{tikzpicture}
    };
    \node [anchor=west,memcell] at (C.east) (D) {};
    \node [anchor=west,memcell,full] at (D.east) (E) {$bloc_3$};
    \node [anchor=west,memcell] at (E.east) (F) {};
    \node [below left=-3mm and -6.5mm of C,node distance=0mm] (p0-bot) {};
    \node [below left=-3mm and -16.5mm of C,node distance=0mm] (p1-bot) {};
    \node [below left=-3mm and -26.5mm of C,node distance=0mm] (p2-bot) {};
    \node [below left=-3mm and -36.5mm of C,node distance=0mm] (p3-bot) {};
    \node [above left=1mm and -6.5mm of C,node distance=0mm] (p0-top) {};
    \node [above left=1mm and -16.5mm of C,node distance=0mm] (p1-top) {};
    \node [above left=1mm and -26.5mm of C,node distance=0mm] (p2-top) {};
    \node [above left=1mm and -36.5mm of C,node distance=0mm] (p3-top) {};
    \node [below of=p1-bot,node distance=1.5cm] (p) {\lstinline{p}};
    \draw [->,line width=1.2pt] (p) -- (p1-bot);
    \node [below of=p0-bot,node distance=1.5cm] (t) {\lstinline{t}};
    \draw [->,line width=1.2pt] (t) -- (p0-bot);
    \draw [|<->|,line width=1.2pt] (p0-bot.south) -- (p1-bot.south)
    node [below=.6cm,fill=white,anchor=center] {\lstinline{\offset(p)}};
    \draw [|<->|,line width=1.2pt] (p0-top.south) -- (p3-top.south)
    node [above] {\lstinline{\block_length(p)}};
    \node [above of=p0-top,node distance=.7cm] (base-p)
          {\lstinline{\base_addr(p)}};
          \draw [->,line width=1.2pt] (base-p) -- (p0-top.south);
  \end{tikzpicture}
\end{center}



La figure~\ref{fig:mem-annots} illustre les annotations \lstinline{\base_addr},
\lstinline{\block_length} et \lstinline{\offset}.
On y voit un modèle mémoire simplifié contenant 3 blocs.
Le deuxième bloc est constitué de trois cases dénotées \lstinline't[0]',
\lstinline't[1]' et \lstinline't[2]', pouvant contenir chacune un entier.
Le pointeur \lstinline'p' pointe vers la deuxième case du deuxième bloc.
L'adresse de base de \lstinline'p' (\lstinline'\base_addr(p)') désigne le
début du bloc.
La longueur du bloc de \lstinline'p' (\lstinline'\block_length(p)') est la
taille du bloc à partir de son adresse de base (trois fois la taille d'un entier
sur cet exemple).
L'offset de \lstinline'p' (\lstinline'\offset(p)') est le décalage de
\lstinline'p' par rapport à son adresse de base (la taille d'un entier sur cet
exemple).


Donnons la sémantique informelle de chacune de ces annotations.
\baseaddrt retourne l'adresse de base du bloc alloué dans lequel pointe le
pointeur $t$.
\blocklengtht retourne la longueur (en octets) du bloc alloué dans lequel
pointe le pointeur $t$.
\offsett retourne le décalage (en octets) entre $t$ et l'adresse de
base du bloc dans lequel pointe $t$.
\validt, respectivement \validreadt, est vrai si le déréférencement de
$t$ est autorisé en lecture et en écriture, respectivement au moins en
lecture.
\validt implique \validreadt mais l'inverse n'est pas vrai.
\initializedt est un prédicat prenant en paramètre un pointeur $t$ sur une
left-value $lv$, il est vrai si $lv$ est initialisée.
Soit une mémoire $\mem = (\env, \store)$ en accord avec la formalisation des
mémoires du chapitre~\ref{sec:lang}.
\initializedt est vrai si il existe une left-value $lv$ et une adresse $\loc$
telles que l'association $lv \mapsto \loc \in \env$ et le pointeur $t$ a pour
valeur $\loc$ et $\store(\loc) \neq \bot$.


\lstinputlisting[caption={Annotations mémoire -- Exemple en C},
   label=lst:mem-annots,firstline=4,firstnumber=1]{listings/mem_annots.c}

\begin{figure}[h!]
\lstinputlisting[caption={Version instrumentée du programme du listing~\ref{lst:mem-annots}},
   label=lst:mem-annots-instru]{listings/mem_annots_instru.c}
\end{figure}


Le Listing~\ref{lst:mem-annots} illustre l'utilisation de toutes ces annotations
sur un programme C.
Un tableau \lstinline't' de trois entiers est alloué ligné 2.
Les deux premières cases du tableau sont initialisées à 1 aux lignes 3--4.
Un pointeur \lstinline'p' est ensuite initialisé à \lstinline't+1' ligne 5,
c'est-à-dire pointe la deuxième case du tableau.
L'assertion ligne 6 affirme que \lstinline'p' et \lstinline't' ont la même
adresse de base.
L'assertion ligne 7 affirme que le bloc contenant \lstinline'p' a une taille de
trois entiers.
L'assertion ligne 8 affirme que \lstinline'p' a un offset de la taille d'un
entier dans son bloc.
L'assertion ligne 9 affirme que \lstinline'p' peut être déréférencé sur les
offsets 0 et 1 (\lstinline'*(p+0)' et \lstinline'*(p+1)').
L'assertion ligne 10 affirme que \lstinline'p' est initialisé mais pas
\lstinline'p+1'.
Les annotations du programme sont satisfaites : leur vérification à l'exécution
ne produit pas d'erreur.

Le listing~\ref{lst:mem-annots-instru} présente le programme du
listing~\ref{lst:mem-annots} instrumenté afin de vérifier les annotations à
l'exécution.
Les lignes 15--17 du listing~\ref{lst:mem-annots-instru} permettent d'exécuter
l'assertion ligne 6 du listing~\ref{lst:mem-annots}.
Les lignes 19--22 du listing~\ref{lst:mem-annots-instru} permettent d'exécuter
l'assertion ligne 7 du listing~\ref{lst:mem-annots}.
Les lignes 24--26 du listing~\ref{lst:mem-annots-instru} permettent d'exécuter
l'assertion ligne 8 du listing~\ref{lst:mem-annots}.
Les lignes 28--37 du listing~\ref{lst:mem-annots-instru} permettent d'exécuter
l'assertion ligne 9 du listing~\ref{lst:mem-annots}.
Les lignes 39--48 du listing~\ref{lst:mem-annots-instru} permettent d'exécuter
l'assertion ligne 10 du listing~\ref{lst:mem-annots}.


\section{Modèle mémoire pour la vérification à l'exécution}
\label{sec:mem-model}


Dans cette partie, nous présentons les détails de conception du modèle mémoire
proposé.
Nous justifions nos choix d'implémentation puis présentons les algorithmes de
calcul du plus grand préfixe commun, de recherche, d'ajout et de suppression
d'un bloc dans le modèle mémoire.


\subsection{Structure de données ``store''}


Au chapitre~\ref{sec:lang}, nous avons défini un {\em store} comme étant une
fonction associant une valeur à une adresse.
Dans ce chapitre, ce terme fait référence à une structure de données qui
implémente efficacement cette fonction, donc contenant des adresses et des
valeurs associées.

Le code instrumenté pouvant accéder et modifier fréquemment les données du
\textit{store}, une implémentation efficace requiert une structure de données
offrant une bonne complexité en temps et en espace.
Cette structure de données doit être triée : on peut avoir besoin d'accéder
directement à un bloc à partir de son adresse de base, mais aussi à partir de
n'importe quelle adresse contenue dans le bloc (c'est-à-dire entre l'adresse de
base et l'adresse de fin du bloc).
Par exemple, la fonction \lstinline'__base_addr' utilisée pour le traitement
de la construction \baseaddrt cherche l'adresse de base la plus proche et
inférieure à $t$.
Cette contrainte ne nous permet pas d'utiliser une table de hachage.
Les listes chaînées ne sont pas assez efficaces à cause de la complexité
linéaire dans le pire cas.
Les arbres binaires de recherche non équilibrés ont aussi une complexité
linéaire dans le pire des cas quand les données sont insérées dans un ordre
strictement croissant ou strictement décroissant.
Certes, le coût du rééquilibrage de l'arbre (pour un arbre binaire de recherche
équilibré) serait amorti dans le cas où les modifications de la structure de
l'arbre sont moins nombreuses que les accès simples; mais ce n'est pas
nécessairement vrai sur les exemples de code que nous avons instrumentés comme
nous le verrons au chapitre~\ref{sec:eacsl}.

Notre choix s'est donc porté sur la structure de {\em Patricia trie}
\cite{Szpankowski/90} (appelé aussi {\em radix tree} ou ``arbre à préfixe
compact''.
Cette structure est efficace même si l'arbre n'est pas équilibré.
Les clés sont les adresses de base des blocs (c'est-à-dire des mots de 32 ou 64
bits) ou des préfixes d'adresses.
Chaque feuille contient les données relatives à un bloc en mémoire (voir
partie~\ref{sec:mem-annots} pour le détail des informations stockées).
Le routage de la racine jusqu'à une feuille particulière se fait grâce aux
n\oe{}uds internes.
Chacun d'eux contient le plus grand préfixe de l'adresse de base commun entre
ses deux fils.

La figure~\ref{fig:insertion-Patricia-trie-0} illustre un Patricia
trie sur des adresses 8 bits, limite choisie pour des raisons de simplicité.
Il contient trois blocs représentés par trois feuilles.
Seules les adresses de base apparaissent sur le schéma.
Il contient deux préfixes stockés dans les n\oe{}uds qui ne sont pas des
feuilles.
Le symbole ``{\tt *}'' signifie que la valeur du bit à cette position n'a pas
d'importance.

\begin{figure}[h]
  
  \begin{center}
    \begin{tabular}{ccc}
      \begin{tikzpicture}[grow=down,sibling distance=18mm,level distance=6mm,
          style={font=\scriptsize}]
        \node[pnode] {\texttt{0010\,****}}
        child { node[pleaf] {\texttt{0010\,0110}} }
        child { node[pnode] {\texttt{0010\,1***}}
          child { node[pleaf] {\texttt{0010\,1001}} }
          child { node[pleaf] {\texttt{0010\,1101}} }
        };
        \node at (-1.7,0) {\textbf{a)}};
      \end{tikzpicture}
& 
      \hspace{1mm} 
&
      \begin{tikzpicture}[grow=down,level 2/.style={sibling distance=17mm},
          sibling distance=35mm,level distance=6mm,style={font=\scriptsize}]
        \node[pnode] {\texttt{0010\,****}}
        child { node[pnode] {\texttt{0010\,011*}}
          child { node[pleaf] {\texttt{0010\,0110}} }
          child { node[pleaf] {\texttt{0010\,0111}} }
        }
        child { node[pnode] {\texttt{0010\,1***}}
          child { node[pleaf] {\texttt{0010\,1001}} }
          child { node[pleaf] {\texttt{0010\,1101}} }
        };
        \node at (-3.4,0) {\textbf{b)}};
     \end{tikzpicture}   
    \end{tabular}
  \end{center}
  \vspace{-3mm}
  \caption{Exemple de Patricia trie avant \textbf{(a)}
    et après \textbf{(b)} insertion de \texttt{0010\,0111}}
  \vspace{-3mm}

  \label{fig:insertion-Patricia-trie-0}
\end{figure}

La complexité théorique dans le pire des cas d'un accès dans un Patricia trie
dans notre cas est en $O(k)$ où $k$ est la longueur d'un mot (c'est-à-dire
32 ou 64 bits).
En pratique, un programme ne pouvant allouer des blocs que dans un
espace mémoire limité, la profondeur de l'arbre est inférieure à cette borne.
De plus, contrairement aux chaînes de caractères (la première application des
Patricia tries), la comparaison des mots peut être implémentée très
efficacement par des opérations bit-à-bit.

Les données de chaque bloc n'occupent que quelques octets en mémoire, exception
faite des données d'initialisation du bloc.
Le statut d'initialisation de chaque octet est monitoré séparément (les champs
de bits ne sont pas encore supportés).
Dans le pire cas (bloc partiellement initialisé), chaque octet utilise un bit
supplémentaire portant l'information sur son initialisation.
Nous conservons le nombre d'octets initialisés de chaque bloc dans un
compteur.
Dans le cas où tous les octets (ou aucun) sont initialisés, le tableau censé
contenir les bits portant l'information d'initialisation est libéré et nous
utilisons alors la valeur du compteur pour avoir cette information.
De plus, nous utilisons une fonction spécifique dans le cas où tous les octets
d'un bloc sont initialisés d'un coup, au lieu d'invoquer une fonction
d'initialisation sur chaque octet du bloc.


\subsection{Calcul du masque du plus grand préfixe commun (MPGPC)}
\label{sec:mpgpc}

%\lstinputlisting[caption={Calcul du plus grand préfixe commun},
%  label=lst:prefix]{listings/common_prefix_mask.c}

Définissons les types {\em word} et {\em masque} utilisés dans ce chapitre.
Un {\em word} de taille $n$ est une suite de $n$ bits ($0$ ou $1$).
La taille d'un word est généralement un multiple de $2$ et dépend de
l'architecture matérielle.
Nous utiliserons la constante \lstinline'NULL' pour dénoter l'adresse
nulle et nous noterons $word$ le type des words.
Nous utilisons une constante $WLEN$ pour faire référence à la taille des words.
Un {\em masque} est un word dont les $k \le WLEN$ premiers bits sont à $1$ et
les $WLEN-k$ bits suivants sont à $0$.
Nous noterons $Mask$ le type des masques.
Les types de données word et Mask seront utilisés dans les algorithmes présentés
dans ce chapitre.
Nous définissons un ordre lexicographique sur les masques.
Soient $m$ et $m'$ deux masques, $m <= m'$ si et seulement si $m'$ contient au
moins autant de $1$ que $m$.

Le {\em masque du plus grand préfixe commun} $m$ de deux adresses $a$ et $b$ est
le plus grand des masques (suivant l'ordre lexicographique) tel que, si $n$ est
le nombre de $1$ de $m$, alors les $n$ premiers bits de $a$ et $b$ sont
identiques.
On dira aussi que $n$ est la taille du préfixe commun à $a$ et $b$.

Par exemple, le masque du plus grand préfixe commun de $a$ =
\ppleaf{\texttt{0110\,0111}} et $b$ = \ppleaf{\texttt{0111\,1111}} est
$m$ = \texttt{1110\,0000}.
En effet, seuls les trois premiers bits de $a$ sont identiques au trois premiers
bits de $b$, les words divergent à partir du quatrième bit.
On notera $p$ = \ppnode{\texttt{011*\,****}} le plus grand préfixe commun à $a$
et $b$, où le symbole ``{\tt *}'' signifie que la valeur du bit à cette position
n'a pas d'importance.


\begin{algorithm}
\begin{algorithmic}[1]
\Statex $array[ 0 .. \mathit{WLEN} ]$ of Mask $\mathit{masks}$
\Comment{masques dans l'ordre lexicographique}
\Statex $array[ 0 .. \mathit{WLEN} ]$ of int $\mathit{longer}$
\Comment{indices des masques plus longs à tester}
\Statex $array[ 0 .. \mathit{WLEN} ]$ of int $\mathit{shorter}$
\Comment{indices des masques plus courts à tester}
\Statex
\Ensure $(a \Band \mresult) == (b \Band \mresult)$
\Ensure $\forall i. (0 \le i \le \mathit{WLEN} \Rightarrow (\mathit{masks}[i] \Band a) == (\mathit{masks}[i] \Band b) \Rightarrow \mresult \ge \mathit{masks}[i])$
\Ensure $\exists i. (0 \le i \le \mathit{WLEN} \land \mresult == \mathit{masks}[i])$
\Statex
\Function{MPGPC}{word $a$, $b$}
\State int $i$, word $nxor$
\Let{$\mathit{nxor}$}{$~\sim(a ~\string^{}~ b)$}
\Comment{$\string^{}$ : ``ou'' exclusif bit-à-bit, $\sim$ : ``non'' bit-à-bit}
\State $i \gets \mathit{WLEN}/2$
\While{$i > 0$}
  \If{$\mathit{nxor} \ge \mathit{masks}[i]$}
    \State $i \gets \mathit{longer}[i]$ 
  \Else
    \State $i \gets \mathit{shorter}[i]$
  \EndIf
\EndWhile\\
\Return $masks[-i]$
\EndFunction
\end{algorithmic}
\caption{Recherche du masque du plus grand préfixe commun de $a$ et $b$
  \label{algo:prefix}}
\end{algorithm}


Nous présentons maintenant un algorithme efficace de recherche du plus grand
préfixe commun de deux adresses.
Une version naîve et inefficace de cet algorithme consiste en un parcours
linéaire des mots mémoire de gauche à droite jusqu'à trouver des bits
différents, de la même manière qu'on pourrait le faire sur des chaînes de
caractères.

La version optimisée de cet algorithme consiste maintenant en une recherche
dichotomique du masque dans un tableau pré-calculé qui contient tous les masques
de préfixes possibles.
Les transitions entre les étapes de la recherche se font en utilisant des
indices pré-calculés, de manière à obtenir le prochain masque à essayer.
L'algorithme~\ref{algo:prefix} présente cette méthode de recherche.
Le tableau $masks$ contient tous les $WLEN+1$ masques possibles sur
$WLEN$ bits.
Le tableau $longer$ contient les indices à essayer en cas de recherche d'un
masque plus long : si $nxor$ est plus grand que le masque courant à l'indice $i$
($masks[i]$), le prochain masque à essayer sera celui à l'indice $longer[i]$
($masks[longer[i]]$).
De manière similaire, $shorter$ contient les indices à essayer en cas de
recherche d'un masque plus court : si $nxor$ est plus petit que le masque
courant, le prochain masque à essayer sera à l'indice $shorter[i]$.
Les tableaux $longer$ et $shorter$ ont été remplis de telle manière que si un
des éléments $k$ de ces tableaux est négatif, alors $- k$ est l'indice du masque
du plus grand préfixe commun.

Sur des words de longueur 16 bits, les tableaux $masks$, $longer$ et $shorter$
sont initialisés de cette manière :

\begin{lstlisting}
masks = {
  0x0,0x8000,0xc000,0xe000,0xf000,0xf800,0xfc00,0xfe00,0xff00,0xff80,
  0xffc0,0xffe0,0xfff0,0xfff8,0xfffc,0xfffe,0xffff };
longer = { 0,-1,3,-3,6,-5,7,-7,12,-9,11,-11,14,-13,15,16,-16 };
shorter = { 0,0,1,-2,2,-4,5,-6,4,-8,9,-10,10,-12,13,-14,-15 };
\end{lstlisting}

Reprenons notre exemple sur des words de 8 bits, avec $a$ =
\ppleaf{\texttt{0110\,0111}} et $b$ = \ppleaf{\texttt{0111\,1111}}.
$nxor$ prend la valeur \texttt{1110\,0111}.
L'algorithme essaie $i = 4$ d'abord, puis $i = shorter[4] = 2$, puis
$i = longer[2] = 3$, puis $i = longer[3] = -3$, pour finalement retourner
$masks[3] = 0xE0$, qui est \texttt{1110\,0000}, soit précisément le masque du
plus grand préfixe commun de $a$ et $b$.


\subsection{Recherche}


Présentons maintenant les algorithmes de recherche d'un bloc dans le
\textit{store} à partir d'une adresse $a$.
Ils sont de deux sortes.
La recherche exacte cherche un bloc dont l'adresse de base est égale à $a$, elle
est notamment utilisée dans l'instrumentation des fonctions \lstinline'free' et
\lstinline'realloc'.
La recherche du contenant cherche un bloc qui contient $a$, c'est-à-dire que
$a$ est incluse entre le début (adresse de base) et la fin du bloc.
Elle est notamment utilisée dans l'instrumentation des annotations
\lstinline'\valid' et \lstinline'\base_addr'.


Dans ces algorithmes, nous noterons $Ptree$ le type de données des Patricia
tries et $Block$ le type des blocs.
Un objet de type $Block$ est caractérisé par un word et une taille.
Un objet de type $Ptree$ est quant à lui caractérisé par un booléen indiquant
si l'arbre est une feuille, un word encodant l'adresse du n\oe{}ud, le masque
du plus grand préfixe commun à ses deux fils, le sous-arbre gauche, le
sous-arbre droit, le n\oe{}ud ``père'' et le bloc porté par le n\oe{}ud si c'est
une feuille.
Voici une définition simplifiée de ces types :

\begin{lstlisting}
Block = { word ptr; int size; };
Ptree = {
  bool isLeaf;
  word addr;
  Mask mask;
  Ptree left, right, father;
  Block leaf;
};
\end{lstlisting}
Nous utiliserons la constante $emptyPtree$ pour dénoter un Patricia trie vide.



\subsubsection*{Recherche exacte}


\begin{algorithm}
\begin{algorithmic}[1]
\Statex Ptree $root$
\Statex
\Require $root \neq emptyPtree \land a \neq \mnull$
\Ensure $\mresult \neq noBlock \Rightarrow \mresult.ptr == a$
\Statex
\Function{Find-exact}{word $a$}
\State Ptree $x$
\Let{$x$}{$root$}
\While{$\lnot \mathit{x.isLeaf}$}
  \If{$\mathit{x.addr} \Band \mathit{x.mask} \neq a \Band \mathit{x.mask}$}
    \Return $noBlock$
  \EndIf
  \If{$\mathit{x.right.addr} \Band \mathit{x.right.mask} == a \Band \mathit{x.right.mask}$}
    \State $x \gets \mathit{x.right}$
  \ElsIf{$\mathit{x.left.addr} \Band \mathit{x.left.mask} == a \Band \mathit{x.left.mask}$}
    \State $x \gets \mathit{x.left}$
  \Else{} \Return $noBlock$
  \EndIf
\EndWhile
\If{$a \neq \mathit{x.leaf.ptr}$}
  \Return $noBlock$
\EndIf{}
\Return $\mathit{x.leaf}$
\EndFunction
\end{algorithmic}
\caption{Recherche d'une adresse exacte $a$, retourne Block $\cup$ \{noBlock\}
  \label{algo:get-exact}}
\end{algorithm}


L'algorithme~\ref{algo:get-exact} retourne le block $B$ dont l'adresse de base
est égale à l'adresse $a$ passée en paramètre si un tel bloc est dans le store,
ou la valeur $noBlock$ sinon.
Nous notons $root$ le {\em store} global représenté par un Patricia trie, dans
lequel tous les blocs sont enregistrés.

L'algorithme procède comme suit : tant que $x$ n'est pas une feuille -- c'est
donc un n\oe{}ud ayant deux fils -- on se dirige vers le fils ayant le plus
grand préfixe commun avec l'adresse $a$ passée en paramètre de l'algorithme.
Quand on arrive sur une feuille, cette dernière contient l'adresse que l'on
cherchait.


\subsubsection*{Recherche du ``contenant''}


\begin{algorithm}[h!]
\begin{algorithmic}[1]
\Statex Ptree $root$
\Statex
\Function{Find-cont}{word $a$}
\Statex Ptree $x$, Stack $stack$
\If{$(\mathit{root} == emptyPtree) \lor (a == \mnull)$}
  \Return $noBlock$
\Else
  \Let{$x$}{$root$}
  \Let{$stack$}{$\mathit{emptyStack}$}
  \While{$\mathit{true}$}
    \If{$x.isLeaf$}
      \If{$\mathit{x.addr} > a$}
        \If{$stack = \mathit{emptyStack}$}
          \Return $noBlock$
        \Else
          \State $x \gets \mathit{pop(stack)}$
          \State $\mathit{continue}$
        \EndIf
      \ElsIf{$a < \mathit{x.leaf.size} + \mathit{x.addr}$}
        \Return $\mathit{x.leaf}$
      \ElsIf{$\mathit{x.leaf.size} == 0 \land a == \mathit{x.leaf.ptr}$}
        \Return $\mathit{x.leaf}$
      \ElsIf{$\mathit{stack} = \mathit{emptyStack}$}
        \Return $noBlock$
      \Else
        \State $x \gets \mathit{pop(stack)}$
        \State $\mathit{continue}$
      \EndIf
    \EndIf
    \If{$\mathit{x.right.addr} \Band \mathit{x.right.mask} \le a \Band \mathit{x.right.mask}$}
      \State $\mathit{stack} \gets \mathit{push(x.left)}$
      \State $x \gets \mathit{x.right}$
    \ElsIf{$\mathit{x.left.addr} \Band \mathit{x.left.mask} \le a \Band \mathit{x.left.mask}$}
      \State $x \gets \mathit{x.left}$
    \ElsIf{$\mathit{stack} == \mathit{emptyStack}$}
      \Return $noBlock$
    \Else
      \State $x \gets \mathit{pop(stack)}$
    \EndIf
  \EndWhile
\EndIf
\EndFunction
\end{algorithmic}
\caption{Recherche du bloc contenant une adresse $a$, retourne Block $\cup$
  \{noBlock\}
  \label{algo:get-cont}}
\end{algorithm}

L'algorithme~\ref{algo:get-cont} retourne le bloc $B$ contenant
l'adresse $a$ passée en paramètre, c'est-à-dire que $a$ est plus grande que
l'adresse de base de $B$ et plus petite que l'adresse de fin de $B$ (adresse
de base du bloc + taille du bloc).
Si un tel bloc n'existe pas, la valeur $noBlock$ est renvoyée.
Cet algorithme est utilisé pour des requêtes du type \lstinline{\valid} ou
\lstinline{\initialized} où l'adresse passée en paramètre ne correspond pas
nécessairement à l'adresse de base d'un bloc.

Cet algorithme est similaire à celui de la recherche exacte, au détail près
qu'il faut vérifier que $a$ est comprise entre l'adresse de base et l'adresse
de fin de $B$ quand on arrive sur une feuille.
On utilise une pile $stack$ contenant les n\oe{}uds à partir desquels il faut
reprendre le parcours si la recherche n'aboutit pas.
Nous notons $Stack$ le type de la pile, $emptyStack$ une pile vide, $push$ la
fonction d'empilement et $pop$ la fonction de dépilement.
Quand on explore le fils droit, on empile le fils gauche.
Quand on arrive sur une feuille, on vérifie que $a$ est comprise entre l'adresse
de base et l'adresse de fin du bloc.
Si c'est le cas alors a trouvé le bloc que l'on cherchait.
Sinon on utilise le dernier n\oe{}ud empilé (et on le dépile) s'il existe,
sinon on retourne $noBlock$.
L'instruction $continue$ aux lignes 12 et 18 de l'algorithme~\ref{algo:get-cont}
indique que l'exécution ignore les instructions suivantes et reprend au début de
la boucle à la ligne 6.


\subsection{Ajout d'un bloc}


La figure~\ref{fig:insertion-Patricia-trie} illustre l'insertion de l'adresse
\ppleaf{\texttt{0010\,0111}} dans un arbre.
L'algorithme~\ref{algo:add-block} a déterminé que le n\oe{}ud le plus similaire
est \ppleaf{\texttt{0010\,0110}}, et crée le père correspondant :
\ppnode{\texttt{0010\,011*}}.

\begin{figure}[h]
  
  \begin{center}
    \begin{tabular}{ccc}
      \begin{tikzpicture}[grow=down,sibling distance=18mm,level distance=6mm,
          style={font=\scriptsize}]
        \node[pnode] {\texttt{0010\,****}}
        child { node[pleaf] {\texttt{0010\,0110}} }
        child { node[pnode] {\texttt{0010\,1***}}
          child { node[pleaf] {\texttt{0010\,1001}} }
          child { node[pleaf] {\texttt{0010\,1101}} }
        };
        \node at (-1.7,0) {\textbf{a)}};
      \end{tikzpicture}
& 
      \hspace{1mm} 
&
      \begin{tikzpicture}[grow=down,level 2/.style={sibling distance=17mm},
          sibling distance=35mm,level distance=6mm,style={font=\scriptsize}]
        \node[pnode] {\texttt{0010\,****}}
        child { node[pnode] {\texttt{0010\,011*}}
          child { node[pleaf] {\texttt{0010\,0110}} }
          child { node[pleaf] {\texttt{0010\,0111}} }
        }
        child { node[pnode] {\texttt{0010\,1***}}
          child { node[pleaf] {\texttt{0010\,1001}} }
          child { node[pleaf] {\texttt{0010\,1101}} }
        };
        \node at (-3.4,0) {\textbf{b)}};
     \end{tikzpicture}   
    \end{tabular}
  \end{center}
  \vspace{-3mm}
  \caption{Exemple de Patricia trie avant \textbf{(a)}
    et après \textbf{(b)} insertion de \texttt{0010\,0111}}
  \vspace{-3mm}

  \label{fig:insertion-Patricia-trie}
\end{figure}

\begin{algorithm}[h!]
\begin{algorithmic}[1]
\Statex Ptree $root$
\Statex
\Procedure{Add}{Block $a$}
\Statex Ptree $x$, $b$, $f$; Mask $lprefix$, $rprefix$
\State $x.addr \gets a.ptr$, $x.mask \gets masks[WLEN]$, $x.left \gets emptyPtree$,
\State $x.right \gets emptyPtree$, $x.father \gets emptyPtree$, $x.leaf \gets a$
\If{$root == emptyPtree$}
  \Let{$root$}{$x$}
\Else
  \Let{$b$}{$root$}
  \While{$\mathit{true}$}
    \If{$b.isLeaf$}
      \State $\mathit{break}$
    \EndIf
    \Let{$lprefix$}{$MPGPC(\mathit{b.left.addr} \Band \mathit{b.left.mask},\mathit{a.ptr})$}
    \Let{$rprefix$}{$MPGPC(\mathit{b.right.addr} \Band \mathit{b.right.mask},\mathit{a.ptr})$}
    \If{$lprefix > rprefix$}
      \Let{$b$}{$b.left$}
    \ElsIf{$rprefix > lprefix$}
      \Let{$b$}{$b.right$}
    \Else
      \State $\mathit{break}$
    \EndIf
  \EndWhile
  \Let{$f.addr$}{$b.addr \Band x.addr$}
  \Let{$f.leaf$}{$noBlock$}
  \Let{$f.left$}{$(\mathit{x.addr} \le \mathit{b.addr})~?~x~:~b$}
  \Let{$f.right$}{$(\mathit{x.addr} \le \mathit{b.addr})~?~b~:~x$}
  \Let{$x.father$}{$f$}
  \If{$b == root$}
    \Let{$f.father$}{$emptyPtree$}
    \Let{$f.mask$}{$MPGPC(b.addr \Band b.mask, a.ptr)$}
    \Let{$root$}{$f$}
  \Else
    \If{$b.father.left == b$}
      \Let{$b.father.left$}{$f$}
    \Else
      \Let{$b.father.right$}{$f$}
    \EndIf
    \Let{$f.father$}{$b.father$}
    \Let{$f.mask$}{$MPGPC(f.left.addr \Band f.left.mask, f.right.addr \Band f.right.mask)$}
  \EndIf
  \Let{$b.father$}{$f$}
  \If{$\lnot b.isLeaf$}
    \Let{$b.mask$}{$MPGPC(b.left.addr \Band b.left.mask, b.right.addr \Band b.right.mask)$}
  \EndIf
\EndIf
\EndProcedure
\end{algorithmic}
\caption{Ajout d'un bloc $a$, modifie le Ptree $root$
  \label{algo:add-block}}
\end{algorithm}

L'algorithme~\ref{algo:add-block} présente l'ajout d'un nouvel élément $a$ à
l'arbre du {\em store}.
Si l'arbre est vide, l'élément est ajouté à la racine.
Sinon, on recherche le n\oe{}ud $b$ (qui n'est pas nécessairement une feuille)
le plus similaire au n\oe{}ud à insérer.
Celui-ci sera son frère dans la nouvelle configuration de l'arbre.
Nous créons ensuite le n\oe{}ud correspondant au père.
Le plus grand préfixe commun des deux fils est calculé pour le père, et les fils
sont ordonnés en fonction de leur adresse (le plus petit à gauche).
Puis le père est inséré dans l'arbre à l'ancien emplacement de $b$.
Les champs de $b$ et de l'ancien père de $b$ (s'il existe) sont également mis à
jour pour maintenir la cohérence de l'arbre.


\subsection{Suppression d'un bloc}


La figure~\ref{fig:suppression-Patricia-trie} illustre la suppression de
l'adresse \ppleaf{\texttt{0010\,0111}} dans un Patricia trie.
Ce n\oe{}ud ainsi que son père \ppnode{\texttt{0010\,011*}} sont supprimés, et
\ppleaf{\texttt{0010\,0110}} est remonté d'un niveau.

\begin{figure}[h]
  \begin{center}
    \begin{tabular}{ccc}
      \begin{tikzpicture}[grow=down,level 2/.style={sibling distance=17mm},
          sibling distance=35mm,level distance=6mm,style={font=\scriptsize}]
        \node[pnode] {\texttt{0010\,****}}
        child { node[pnode] {\texttt{0010\,011*}}
          child { node[pleaf] {\texttt{0010\,0110}} }
          child { node[pleaf] {\texttt{0010\,0111}} }
        }
        child { node[pnode] {\texttt{0010\,1***}}
          child { node[pleaf] {\texttt{0010\,1001}} }
          child { node[pleaf] {\texttt{0010\,1101}} }
        };
        \node at (-3.4,0) {\textbf{a)}};
     \end{tikzpicture}   
& 
      \hspace{1mm} 
&
       \begin{tikzpicture}[grow=down,sibling distance=18mm,level distance=6mm,
          style={font=\scriptsize}]
        \node[pnode] {\texttt{0010\,****}}
        child { node[pleaf] {\texttt{0010\,0110}} }
        child { node[pnode] {\texttt{0010\,1***}}
          child { node[pleaf] {\texttt{0010\,1001}} }
          child { node[pleaf] {\texttt{0010\,1101}} }
        };
        \node at (-1.7,0) {\textbf{b)}};
      \end{tikzpicture}
    \end{tabular}
  \end{center}
  \vspace{-3mm}
  \caption{Exemple de Patricia trie avant \textbf{(a)}
    et après \textbf{(b)} suppression de \texttt{0010\,0111}}
  \vspace{-3mm}
  \label{fig:suppression-Patricia-trie}
\end{figure}



\begin{algorithm}[h!]
\begin{algorithmic}[1]
\Require $root \neq emptyPtree \land B \neq noBlock$
\Statex
\Procedure{Rem}{Block $B$}
\Statex Ptree $x$, $b$, $f$, $g$
\Let{$x$}{$root$}
\While{$\lnot x.isLeaf$}
  \If{$\mathit{x.right.addr} \Band \mathit{x.right.mask} == \mathit{B.ptr} \Band \mathit{x.right.mask}$}
    \Let{$x$}{$x.right$}
  \Else
    \Let{$x$}{$x.left$}
  \EndIf
\EndWhile
\Let{$f$}{$x.father$}
\If{$f == emptyPtree$}
  \Let{$root$}{$emptyPtree$}
\Else
  \Let{$b$}{$(x == \mathit{f.left}) ~?~ \mathit{f.right} ~:~ \mathit{f.left}$}
  \Let{$f$}{$b$}
  \If{$\lnot b.isLeaf$}
    \Let{$b.left.father$}{$f$}
    \Let{$b.right.father$}{$f$}
  \EndIf
  \Let{$g$}{$f.father$}
  \If{$g \neq emptyPtree$}
    \Let{$g.mask$}{$MPGPC(\mathit{g.left.addr} \Band \mathit{g.left.mask}, \mathit{g.right.addr} \Band \mathit{g.right.mask})$}
  \EndIf
\EndIf
\EndProcedure
\end{algorithmic}
\caption{Suppression d'un bloc $B$, modifie le Ptree $root$
  \label{algo:rem-block}}
\end{algorithm}

La procédure de suppression d'un bloc $B$ porté par une feuille du Patricia trie
est présentée par l'algorithme~\ref{algo:rem-block}.
L'algorithme commence par rechercher la feuille $x$ qui porte le bloc $B$.
Si cet élément est la racine, l'arbre devient vide.
Sinon, Cette feuille a un frère $b$ et un père $f$.
$b$ et son $f$ sont supprimés, et $b$ remonte d'un niveau (et prend donc la
place de $f$).
Enfin, les champs du nouveau père de $b$ ($g$) sont mis à jour pour
maintenir la cohérence de l'arbre (le masque du plus grand préfixe commun est
recalculé).


\section{Instrumentation pour la vérification à l'exécution}
\label{sec:mem-instru}


Nous présentons l'instrumentation d'un programme C annoté dans une optique de
vérification à l'exécution des annotations liées au modèle mémoire.
Certaines instructions C doivent être instrumentées afin d'enregistrer à
l'exécution des informations sur la mémoire, et les annotations sont
instrumentées afin de pouvoir utiliser ces informations.


\subsection{Instrumentation des allocations et affectations}


Dans cette partie, nous définissons de manière informelle l'instrumentation à
opérer sur le programme annoté afin de pouvoir observer les allocations et
les affectations en mémoire.

Comme nous l'avons vu, pour traiter ces annotations \eacsl, nous devons
conserver pour chaque bloc les informations suivantes : l'adresse de base, le
nombre d'octets occupés en mémoire, l'initialisation de chaque octet et un
booléen indiquant si le bloc est en lecture seule (par exemple si c'est une
chaîne littérale).

Afin d'enregistrer ces informations nous définissons les fonctions C suivantes.
La fonction \lstinline'__store_block', qui prend en paramètre un pointeur et une
taille en octets, ajoute un bloc dans le \textit{store} (présenté en
partie~\ref{sec:mem-model}).
La fonction \lstinline'__store_block' est invoquée sur chaque variable locale
et paramètre formel de fonction (au début de leur portée), ainsi que sur chaque
variable globale (au début de la fonction \lstinline'main').
Elle utilise l'algorithme~\ref{algo:add-block}.
La fonction \lstinline'__delete_block' permet d'enlever du \textit{store} un
bloc reçu en paramètre.
Elle est appelée en fin de portée d'une variable et utilise
l'algorithme~\ref{algo:rem-block}.

Nous définissons également des ``surcouches'' aux fonctions de la bibliothèque
standard du C : \lstinline'malloc', \lstinline'realloc', \lstinline'calloc' et
\lstinline'free'.
L'instrumentation va substituer les appels à ces fonctions par un appel à nos
fonctions : \lstinline'__malloc', \lstinline'__realloc', \lstinline'__calloc'
et \lstinline'__free'.
Ces fonctions permettent d'ajouter ou de supprimer automatiquement les blocs
au \textit{store}, ainsi que de transférer les informations d'initialisation
d'un bloc à un autre en cas d'application de la fonction \lstinline'realloc'.

Enfin, nous définissons une fonction \lstinline'__initialize' qui permet de
marquer un certain nombre d'octets d'un bloc comme étant initialisés.
L'instrumentation ajoute des appels à cette fonction pour chaque affectation du
programme original, ainsi que pour chaque variable globale (initialisée à 0 par
défaut en C).


\subsection{Instrumentation des annotations}


\begin{figure*}[b!]
  \scriptsize{
    {\myinference{$\tau$-base-addr}
      {(l, t:ptr) \trule (I, e) }
      {(l, \mbox{\lstinline'\\base_addr('}t\mbox{\lstinline')'}) \trule
        (I \concat
        (l, \mbox{\lstinline'void* var_'}n~
        \mbox{\lstinline'= __base_addr('}e\mbox{\lstinline');'}),
        \mbox{\lstinline'var_'}n)
      }{}
    }

    {\myinference{$\tau$-block-length}
      {(l, t:ptr) \trule (I, e) }
      {(l, \mbox{\lstinline'\\block_length('}t\mbox{\lstinline')'}) \trule
        (I \concat
        (l, \mbox{\lstinline'unsigned long var_'}n~
        \mbox{\lstinline'= __block_length('}e\mbox{\lstinline');'}),
        \mbox{\lstinline'var_'}n)
      }{}
    }

    {\myinference{$\tau$-offset}
      {(l, t:ptr) \trule (I, e) }
      {(l, \mbox{\lstinline'\\offset('}t\mbox{\lstinline')'}) \trule
        (I \concat
        (l, \mbox{\lstinline'int var_'}n~
        \mbox{\lstinline'= __offset('}e\mbox{\lstinline');'}),
        \mbox{\lstinline'var_'}n)
      }{}
    }

    {\myinference{$\pi$-valid}
      {(l, t:ptr) \trule (I, e) }
      {(l, \mbox{\lstinline'\\valid('}t\mbox{\lstinline')'}) \prule
        (I \concat
        (l, \mbox{\lstinline'int var_'}n~
        \mbox{\lstinline'= __valid('}e,\mbox{\lstinline'sizeof(typeof('}e
        \mbox{\lstinline')));'}),
        \mbox{\lstinline'var_'}n)
      }{}
    }

    {\myinference{$\pi$-valid-read}
      {(l, t:ptr) \trule (I, e) }
      {(l, \mbox{\lstinline'\\valid_read('}t\mbox{\lstinline')'}) \prule
        (I \concat
        (l, \mbox{\lstinline'int var_'}n~
        \mbox{\lstinline'= __valid_read('}e,\mbox{\lstinline'sizeof(typeof('}
        \mbox{\lstinline'e)));'}),
        \mbox{\lstinline'var_'}n)
      }{}
    }

    {\myinference{$\pi$-initialized}
      {(l, t:ptr) \trule (I, e) }
      {(l, \mbox{\lstinline'\\initialized('}t\mbox{\lstinline')'}) \prule
        (I \concat
        (l, \mbox{\lstinline'int var_'}n~
        \mbox{\lstinline'= __initialized('}e,\mbox{\lstinline'sizeof(typeof('}
        \mbox{\lstinline'e)));'}),
        \mbox{\lstinline'var_'}n)
      }{}
    }
  }
  \caption{Règles de traduction pour les annotations liées au modèle mémoire}
  \label{fig:mem-annots-rules}
\end{figure*}


Afin de pouvoir exploiter les informations sur les blocs stockés dans le
\textit{store}, nous définissons les fonctions \lstinline'__base_addr',
\lstinline'__block_length', \lstinline'__offset', \lstinline'__valid',
\lstinline'__valid_read' et \lstinline'__initialized'.
La sémantique de ces fonctions est conforme à la sémantique des annotations
\eacsl donnée en partie~\ref{sec:mem-annots}.
La fonction \lstinline'__base_addr' retourne l'adresse de début d'un bloc dans
lequel pointe le pointeur qu'elle reçoit en argument.
La fonction \lstinline'__block_length' retourne le nombre d'octets occupés par
le bloc dans lequel pointe le pointeur qu'elle reçoit en argument.
La fonction \lstinline'__offset' retourne le décalage d'un pointeur par rapport
à son adresse de base dans le bloc dans lequel pointe le pointeur qu'elle reçoit
en argument.
Les fonctions \lstinline'__valid' et \lstinline'__valid_read'
retournent le statut de validité à partir d'un pointeur et d'un nombre d'octets.
La fonction \lstinline'__initialized' retourne le statut d'initialisation
d'un nombre d'octets d'un pointeur en argument.
Ces fonctions utilisent les algorithmes~\ref{algo:get-exact}
et~\ref{algo:get-cont} afin de récupérer les informations du {\em store}.

La figure~\ref{fig:mem-annots-rules} présente les règles de traduction des
annotations en utilisant le même formalisme qu'au chapitre~\ref{sec:traduction}.
Nous rappelons que $\tau$ est la fonction de traduction des termes \eacsl et
$\pi$ est la fonction de traduction des prédicats \eacsl.
Dans les règles \textsc{$\pi$-valid}, \textsc{$\pi$-valid-read} et
\textsc{$\pi$-initialized}, \lstinline'typeof(e)' désigne le type de
l'expression \lstinline'e', qui est un type C tel que \lstinline'int' ou
\lstinline'char'.
\lstinline'sizeof(typeof(e))' est le nombre d'octets occupés par un objet
de ce type en mémoire.


\section*{Conclusion du chapitre}


Nous avons présenté dans ce chapitre une manière de vérifier dynamiquement les
annotations \eacsl qui concernent la mémoire : \lstinline'\base_addr',
\lstinline'\block_length', \lstinline'\offset', \lstinline'\valid',
\lstinline'\valid_read' et \lstinline'\initialized'.
Nous avons présenté ces annotations, un modèle mémoire permettant de décider de
leur satisfaction ou non à l'exécution, ainsi qu'une manière d'instrumenter le
programme afin de les vérifier à l'exécution.

Rappelons que nous avons choisi de vérifier ces annotations à l'exécution plutôt
que d'utiliser la génération de tests structurels car le modèle mémoire utilisé
par le générateur de tests structurels que nous utilisons n'est pas assez bas
niveau et ne permet donc pas l'exécution symbolique de ces constructions.

Notons que nous avons choisi d'implémenter la notion de {\em store} par des
Patricia tries pour des raisons d'efficacité.
Le chapitre~\ref{sec:eacsl} donne plus de détails concernant cette
implémentation du modèle mémoire ainsi qu'une évaluation de ses performances.
