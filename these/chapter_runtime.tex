
\chapter{Vérification à l'exécution des annotations liées au modèle mémoire}
\label{sec:runtime}

\chapterintro

Dans ce chapitre nous abordons la vérification dynamique des annotations liées
au modèle mémoire.
Ceci requiert de pouvoir simuler un modèle mémoire bas niveau afin de pouvoir
surveiller les opérations bas niveau des programmes C (allocations,
initialisations, etc.).
Le support de ces opérations par un générateur de tests structurels nécessite
de pouvoir les interpréter symboliquement, ce qui n'est pas toujours le cas.
Nous avons donc décidé de changer d'approche pour ces propriétés.
En effet, contrairement au chapitre~\ref{sec:traduction}, ces annotations ne
seront pas vérifiées par test structurel mais par validation à l'exécution
({\em runtime assertion checking}).

Dans un premier temps nous présentons les différentes annotations que nous ne
pouvons pas traiter par génération de tests et que nous qualifions d'opérations
de bas niveau (partie~\ref{sec:mem-annots}).
Puis nous abordons les principes de l'instrumentation nécessaire pour leur
vérification (partie~\ref{sec:mem-instru}).
Enfin, nous présentons quelques détails de conception de notre modèle mémoire
(partie~\ref{sec:mem-model}).


\section{Annotations liées au modèle mémoire}
\label{sec:mem-annots}


Définissons d'abord quelques notions.
Les objets de la mémoire sont des blocs.
Chaque bloc est caractérisé par une adresse de base, une taille (en octets) et
un contenu (qui peut être initialisé ou non).
Tout ceci constitue le modèle mémoire.

Soit \textit{t} un terme \eacsl de type pointeur.
Les annotations \eacsl que nous considérons sont les suivantes :

\begin{center}
\begin{tabular}{lrl}
  \textit{term} ::= & ... \\
  & $\mid$
  & \underline{\lstinline'\\base_addr('} \textit{t} \underline{\lstinline')'} \\
  & $\mid$
  & \underline{\lstinline'\\block_length('} \textit{t} \underline{\lstinline')'} \\
  & $\mid$
  & \underline{\lstinline'\\offset('} \textit{t} \underline{\lstinline')'} \\
  \textit{pred} ::= & ... \\
  & $\mid$
  & \underline{\lstinline'\\valid('} \textit{t} \underline{\lstinline')'} \\
  & $\mid$
  & \underline{\lstinline'\\valid_read('} \textit{t} \underline{\lstinline')'} \\
  & $\mid$
  & \underline{\lstinline'\\initialized('} \textit{t} \underline{\lstinline')'} \\
\end{tabular}
\end{center}


Donnons la sémantique informelle de chacune de ces annotations.
\baseaddrt retourne l'adresse de base du bloc alloué qui contient le pointeur
\textit{t}.
\blocklengtht retourne la longueur (en octets) du bloc alloué qui contient, le
pointeur \textit{t}.
\offsett retourne le décalage (en octets) entre \textit{t} et son adresse de
base.
\validt (respectivement \validreadt) est vrai si le déréférencement de
\textit{t} est autorisé en lecture et en écriture (respectivement au moins en
lecture).
\validt implique \validreadt mais l'inverse n'est pas vrai.
\initializedt est un prédicat prenant en paramètre un pointeur \textit{t} sur
une left-value, ce prédicat est vrai si la left-value en question est
initialisée.



\begin{center}
  \begin{tikzpicture}[>=latex,font=\sffamily]
    \node [memcell] at (0,0) (A) {};
    \node [anchor=west,memcell,full] at (A.east) (B) {$bloc_1$};
    \node [anchor=west,memcell,full,minimum width=4cm] at (B.east) (C) {
      \begin{tikzpicture}
        \node[memcell,init] (p0) {\lstinline't[0]'};
        \node[memcell,init,anchor=west] at (p0.east) (p1) {\lstinline't[1]'};
        \node[memcell,uninit,anchor=west] at (p1.east) (p2){\lstinline't[2]'};
      \end{tikzpicture}
    };
    \node [anchor=west,memcell] at (C.east) (D) {};
    \node [anchor=west,memcell,full] at (D.east) (E) {$bloc_3$};
    \node [anchor=west,memcell] at (E.east) (F) {};
    \node [below left=-3mm and -6.5mm of C,node distance=0mm] (p0-bot) {};
    \node [below left=-3mm and -16.5mm of C,node distance=0mm] (p1-bot) {};
    \node [below left=-3mm and -26.5mm of C,node distance=0mm] (p2-bot) {};
    \node [below left=-3mm and -36.5mm of C,node distance=0mm] (p3-bot) {};
    \node [above left=1mm and -6.5mm of C,node distance=0mm] (p0-top) {};
    \node [above left=1mm and -16.5mm of C,node distance=0mm] (p1-top) {};
    \node [above left=1mm and -26.5mm of C,node distance=0mm] (p2-top) {};
    \node [above left=1mm and -36.5mm of C,node distance=0mm] (p3-top) {};
    \node [below of=p1-bot,node distance=1.5cm] (p) {\lstinline{p}};
    \draw [->,line width=1.2pt] (p) -- (p1-bot);
    \node [below of=p0-bot,node distance=1.5cm] (t) {\lstinline{t}};
    \draw [->,line width=1.2pt] (t) -- (p0-bot);
    \draw [|<->|,line width=1.2pt] (p0-bot.south) -- (p1-bot.south)
    node [below=.6cm,fill=white,anchor=center] {\lstinline{\offset(p)}};
    \draw [|<->|,line width=1.2pt] (p0-top.south) -- (p3-top.south)
    node [above] {\lstinline{\block_length(p)}};
    \node [above of=p0-top,node distance=.7cm] (base-p)
          {\lstinline{\base_addr(p)}};
          \draw [->,line width=1.2pt] (base-p) -- (p0-top.south);
  \end{tikzpicture}
\end{center}



La figure~\ref{fig:mem-annots} illustre les annotations \lstinline{\base_addr},
\lstinline{\block_length} et \lstinline{\offset}.
On y voit un modèle mémoire simplifié contenant 3 blocs.
Le pointeur \lstinline'p' pointe vers la deuxième case du second block.
L'adresse de base de \lstinline'p' (\lstinline'\base_addr(p)') désigne le
début du bloc.
La longueur du bloc de \lstinline'p' (\lstinline'\block_length(p)') est la
taille du bloc à partir de son adresse de base.
L'offset de \lstinline'p' (\lstinline'\offset(p)') est le décalage de
\lstinline'p' par rapport à son adresse de base.
Sur l'exemple de la figure~\ref{fig:mem-annots}, l'offset de \lstinline'p' est
1.


\lstinputlisting[caption={Annotations mémoire -- Exemple en C},
   label=lst:mem-annots,firstline=4,firstnumber=1]{listings/mem_annots.c}


Le Listing~\ref{lst:mem-annots} illustre toutes ces annotations sur un programme
C.
Un tableau \lstinline't' de trois entiers est alloué ligné 2.
Les deux premières cases du tableau sont initialisées à 1 aux lignes 3--4.
Un pointeur \lstinline'p' est ensuite initialisé à \lstinline't+1' ligne 5,
c'est-à-dire pointe la deuxième case du tableau.
Les annotations du programme sont correctes : leur vérification à l'exécution
ne produit pas d'erreur.
L'assertion ligne 6 affirme que \lstinline'p' et \lstinline't' ont la même
adresse de base.
L'assertion ligne 7 affirme que le bloc contenant \lstinline'p' a une taille de
trois entiers.
L'assertion ligne 8 affirme que \lstinline'p' a un offset de 1 dans son bloc.
L'assertion ligne 9 affirme que \lstinline'p' peut être déréférencé sur les
offsets 0 et 1 (\lstinline'*(p+0)' et \lstinline'*(p+1)').
L'assertion ligne 10 affirme que \lstinline'p' est initialisé mais pas
\lstinline'p+1'.


\section{Modèle mémoire pour la Vérification à l'exécution}
\label{sec:mem-model}


\subsection{Structure de données ``store''}

Le code instrumenté pouvant accéder et modifier fréquemment les données du
\textit{store}, une implémentation efficace requiert une structure de données
offrant une bonne complexité en temps et en espace.
Cette structure doit être triée : on peut avoir besoin d'accéder directement à
un bloc à partir de son adresse de base, mais aussi à partir de n'importe quelle
adresse contenue dans le bloc (donc accéder au bloc précédent).
Par exemple, la fonction \lstinline'__base_addr' utilisée pour le traitement
de la construction \baseaddrt cherche l'adresse de base la
plus proche et inférieure à \textit{t} (et enfin vérifie les bornes du bloc).
Cette contrainte ne nous permet pas d'utiliser une table de hachage.
Les listes chaînées ne sont pas assez efficaces à cause de la complexité
linéaire dans le pire cas.
Les arbres binaires de recherche non équilibrés ont aussi une complexité
linéaire dans le pire des cas quand les données sont insérées dans un ordre
strictement croissant ou strictement décroissant.
Enfin, le coût du rééquilibrage de l'arbre (pour un arbre binaire de recherche
équilibré) serait amorti dans le cas où les modifications de la structure de
l'arbre sont moins nombreuses que les accès simples; ce qui n'est pas
nécessairement vrai sur les exemples de code que nous avons instrumentés comme
nous le verrons au chapitre~\ref{sec:eacsl}.

Notre choix s'est donc porté sur la structure de {\em Patricia trie}
\cite{Szpankowski/90} (appelé aussi {\em radix tree} ou ``arbre à préfixe
compact'', cette structure est efficace même si l'arbre n'est pas équilibré.
Les clés sont les adresses de base des blocs (c'est-à-dire des mots de 32 ou 64
bits) ou des préfixes d'adresses.
Chaque feuille contient les données relatives à un bloc en mémoire (voir
section précédente pour le détail des informations stockées).
Le routage de la racine jusqu'à une feuille particulière se fait grâce aux
n\oe{}uds internes, chacun d'eux contient le plus grand préfixe commun de
l'adresse de base de ses deux fils.

La figure~\ref{fig:insertion-Patricia-trie} \textbf{a)} illustre un Patricia
trie sur des adresses 8 bits, limite choisie pour des raisons de simplicité.
Il contient trois blocs représentés par trois feuilles.
Seules les adresses de base apparaissent sur le schéma.
Il contient deux préfixes stockés dans les n\oe{}uds qui ne sont pas des
feuilles.
Le symbole ``{\tt *}'' signifie que la valeur du bit à cette position n'a pas
d'importance.

La complexité théorique dans le pire des cas d'un accès dans un Patricia trie
dans notre cas est en {\em O(k)} où $k$ est la longueur d'un mot (c'est-à-dire
32 ou 64 bits).
En pratique, un programme ne pouvant allouer des blocs que dans un
espace mémoire limité, la profondeur de l'arbre est inférieure à cette borne.
De plus, contrairement aux chaînes de caractères (la première application des
Patricia tries), la comparaison des mots peut être implémentée très
efficacement par des opérations bit-à-bit.

Les données de chaque bloc n'occupent que quelques octets en mémoire, exception
faite des données d'initialisation du bloc.
Le statut d'initialisation de chaque octet est monitoré séparément (les champs
de bits ne sont pas encore supportés).
Dans le pire cas (bloc partiellement initialisé), chaque octet utilise un bit
supplémentaire portant l'information sur son initialisation.
Dans le cas où tous les octets (ou aucun) sont initialisés (on utilise un
compteur d'octets initialisés), le tableau censé contenir les bits portant
l'information d'initialisation est libéré, et cette information est donc portée
uniquement par le compteur.
De plus, nous utilisons une fonction spécifique dans le cas où tous les octets
d'un bloc sont initialisés d'un coup, au lieu d'invoquer une fonction
d'initialisation sur chaque octet du bloc.


\subsection{Calcul du plus grand préfixe commun}


%\lstinputlisting[caption={Calcul du plus grand préfixe commun},
%  label=lst:prefix]{listings/common_prefix_mask.c}

\newcommand{\Band}{\land_{\mathbb{B}}}
\newcommand{\Bor}{\lor_{\mathbb{B}}}
\newcommand{\Bnot}{\lnot_{\mathbb{B}}}
\newcommand{\Bxor}{\oplus_{\mathbb{B}}}
\newcommand{\mresult}{\mbox{\lstinline'\\result'}}

\begin{algorithm}
\begin{algorithmic}
\State $masks[ 0 .. WORDBITS ] \gets$ masques par préfixes croissants
\State $longer[ 0 .. WORDBITS ] \gets$ indices des masques plus longs à tester
\State $shorter[ 0 .. WORDBITS ] \gets$ indices des masques plus courts à tester
\State $nxor \gets \Bnot(A \Bxor B)$
\State $i \gets WORDBITS/2$
\While{$i > 0$}
  \If{$nxor \ge masks[i]$}
    \State i $\gets longer[i]$ 
  \Else
    \State i $\gets shorter[i]$
  \EndIf
\EndWhile\\
\Return $masks[-i]$
\Ensure $\forall i. (0 \le i \le WORDBITS \implies (masks[i] \Band a) == (masks[i] \Band B) \implies \mresult \ge masks[i])$
\Ensure $(A \Band \mresult) == (B \Band \mresult)$
\Ensure $\exists i. (0 \le i \le WORDBITS \land \mresult == masks[i])$
\end{algorithmic}
\caption{Calcul du plus grand préfixe commun de $a$ et $b$
  \label{algo:prefix}}
\end{algorithm}


Appelons ``masque du plus grand préfixe commun'' M de A et B. M est composé
d'une suite de $n$ ``1'', suivie d'une suite de ``0'', où $n$ est le nombre de
bits communs entre A et B.
Par exemple, le plus grand préfixe commun de A = \ppleaf{\texttt{0110\,0111}} et
B = \ppleaf{\texttt{0111\,1111}} est P =  \ppnode{\texttt{011*\,****}} et le
masque du plus grand préfixe commun est M = \texttt{1110\,0000}.

Les calculs et comparaisons des préfixes ont été optimisés par un usage
intensif des opérations bit-à-bit.
Le calcul du plus grand préfixe commun a lui aussi été reconçu pour de
meilleures performances.
L'implémentation naïve initiale consistait en un parcours linéaire des mots
mémoire de gauche à droite jusqu'à trouver des bits différents, de la même
manière qu'on pourrait le faire sur des chaînes de caractères.

La version optimisée de ce calcul consiste maintenant en une recherche
dichotomique dans un tableau pré-calculé qui contient tous les préfixes
possibles.
Les transitions entre les étapes de la recherche se font en utilisant des
indices pré-calculés, de manière à obtenir le prochain masque à essayer.
%Le listing~\ref{lst:prefix} illustre cette implémentation du calcul sur des
%mots de 8 bits.
%Les masques sont stockés dans le tableau de la ligne 3.
%Les indices à utiliser pour tester un masque plus long (respectivement plus
%court) sont stockés dans le tableau ligne 4 (respectivement ligne 5).
L'algorithme~\ref{algo:prefix} présente la méthode de calcul du plus grand
préfixe commun de deux adresses A et B.
Le tableau $masks$ contient tous les $WORDBITS+1$ masques possibles sur
$WORDBITS$ bits.
Les indices à utiliser pour tester un masque plus long (respectivement plus
court) sont stockés dans le tableau $longer$ (respectivement $shorter$).
Un indice négatif signifie la fin de la recherche par dichotomie.
Par exemple, pour A et B définis plus haut, nxor = \texttt{1110\,0111}
et la fonction essaie \lstinline'i = 4', puis \lstinline'i = shorter[4] = 2',
puis \lstinline'i = longer[2] = 3', puis \lstinline'i = longer[3] = -3', pour
finalement retourner \lstinline'mask[3] = 0xE0', qui
est précisément \texttt{1110\,0000}.


\subsection{Recherche}


\subsubsection*{Recherche exacte}

\commentGP{TODO: algo}

Cet algorithme retourne le block $B$ tel que l'adresse de base de $B$ soit égale
au pointeur passé en paramètre.
On suppose que l'algorithme n'est utilisé que lorsqu'un tel bloc existe.

Tant qu'on n'est pas sur une feuille -- on est donc sur un n\oe{}ud ayant deux
fils -- on se dirige vers le fils ayant le plus grand préfixe commun avec
l'adresse passée en paramètre de l'algorithme.
Quand on arrive sur une feuille, cette dernière contient donc forcément
l'adresse que l'on cherchait.

\subsubsection*{Recheche du ``contenant''}

\commentGP{TODO: algo}

Cet algorithme retourne le block $B$ contenant l'adresse $ptr$ passée en
paramètre, tel que : $beginAddr_B \le ptr < beginAddr_B + size_B$.
Si un tel bloc n'existe pas, \lstinline{NULL} est renvoyé.
Cet algorithme est utilisé pour des requêtes du type \lstinline{\valid} ou
\lstinline{\initialized} où l'adresse passée en paramètre ne correspond pas
nécessairement à l'adresse de base d'un bloc.

Cet algorithme est similaire à celui de la recherche exacte, au détail près
qu'il faut vérifier la formule de l'encadrement de $ptr$ par $B$ quand on arrive
sur une feuille.
On utilise une pile contenant les n\oe{}uds à partir desquels il faut reprendre
le parcours si la recherche n'aboutit pas.
Quand on explore le fils droit, on empile le fils gauche.
Quand on arrive sur une feuille, on vérifie l'encadrement, s'il est vérifié on
a trouvé le bloc qu'on cherchait, sinon on utilise le dernier n\oe{}ud empilé
(et on le dépile) s'il existe, sinon on retourne \lstinline{NULL}.

\subsection{Ajout}


\commentGP{TODO: algo}


  \begin{center}
    \begin{tabular}{ccc}
      \begin{tikzpicture}[grow=down,sibling distance=18mm,level distance=6mm,
          style={font=\scriptsize}]
        \node[pnode] {\texttt{0010\,****}}
        child { node[pleaf] {\texttt{0010\,0110}} }
        child { node[pnode] {\texttt{0010\,1***}}
          child { node[pleaf] {\texttt{0010\,1001}} }
          child { node[pleaf] {\texttt{0010\,1101}} }
        };
        \node at (-1.7,0) {\textbf{a)}};
      \end{tikzpicture}
& 
      \hspace{1mm} 
&
      \begin{tikzpicture}[grow=down,level 2/.style={sibling distance=17mm},
          sibling distance=35mm,level distance=6mm,style={font=\scriptsize}]
        \node[pnode] {\texttt{0010\,****}}
        child { node[pnode] {\texttt{0010\,011*}}
          child { node[pleaf] {\texttt{0010\,0110}} }
          child { node[pleaf] {\texttt{0010\,0111}} }
        }
        child { node[pnode] {\texttt{0010\,1***}}
          child { node[pleaf] {\texttt{0010\,1001}} }
          child { node[pleaf] {\texttt{0010\,1101}} }
        };
        \node at (-3.4,0) {\textbf{b)}};
     \end{tikzpicture}   
    \end{tabular}
  \end{center}
  \vspace{-3mm}
  \caption{Exemple de Patricia trie avant \textbf{(a)}
    et après \textbf{(b)} insertion de \texttt{0010\,0111}}
  \vspace{-3mm}



L'ajout d'un nouvel élément à l'arbre se déroule comme suit. Si l'arbre est
vide, l'élément est ajouté à la racine.
Sinon, on recherche le n\oe{}ud $x$ le plus similaire au n\oe{}ud à insérer.
Celui-ci sera son frère dans la nouvelle configuration de l'arbre.
Nous créons ensuite le n\oe{}ud correspondant au père.
Le plus grand préfixe commun des deux fils est calculé pour le père, et les fils
sont ordonnés en fonction de leur adresse (le plus petit à gauche).
Puis le père est inséré dans l'arbre à l'ancien emplacement de $x$.
Les champs de $x$ et de l'ancien père de $x$ (s'il existe) sont également mis à
jour pour maintenir la cohérence de l'arbre.
La figure~\ref{fig:insertion-Patricia-trie} illustre l'insertion de l'adresse
\ppleaf{\texttt{0010\,0111}} dans un arbre.
L'algorithme a déterminé que le n\oe{}ud le plus similaire serait
\ppleaf{\texttt{0010\,0110}}, et crée le père correspondant :
\ppnode{\texttt{0010\,011*}}.


\subsubsection*{Recherche du n\oe{}ud le plus similaire}

\commentGP{TODO: algo}

Dès qu'on arrive sur une feuille on la renvoie.
Sinon on calcule le plus grand préfixe commun de l'adresse passée en paramètre
avec le fils gauche, de même avec le fils droit.
Si l'un des préfixes est strictement supérieur à l'autre on se dirige vers la
branche correspondante, sinon on renvoie le n\oe{}ud courant.



\subsection{Suppression}

\commentGP{TODO: algo}

\begin{figure}[h]
  \begin{center}
    \begin{tabular}{ccc}
      \begin{tikzpicture}[grow=down,level 2/.style={sibling distance=17mm},
          sibling distance=35mm,level distance=6mm,style={font=\scriptsize}]
        \node[pnode] {\texttt{0010\,****}}
        child { node[pnode] {\texttt{0010\,011*}}
          child { node[pleaf] {\texttt{0010\,0110}} }
          child { node[pleaf] {\texttt{0010\,0111}} }
        }
        child { node[pnode] {\texttt{0010\,1***}}
          child { node[pleaf] {\texttt{0010\,1001}} }
          child { node[pleaf] {\texttt{0010\,1101}} }
        };
        \node at (-3.4,0) {\textbf{a)}};
     \end{tikzpicture}   
& 
      \hspace{1mm} 
&
       \begin{tikzpicture}[grow=down,sibling distance=18mm,level distance=6mm,
          style={font=\scriptsize}]
        \node[pnode] {\texttt{0010\,****}}
        child { node[pleaf] {\texttt{0010\,0110}} }
        child { node[pnode] {\texttt{0010\,1***}}
          child { node[pleaf] {\texttt{0010\,1001}} }
          child { node[pleaf] {\texttt{0010\,1101}} }
        };
        \node at (-1.7,0) {\textbf{b)}};
      \end{tikzpicture}
    \end{tabular}
  \end{center}
  \vspace{-3mm}
  \caption{Exemple de Patricia trie avant \textbf{(a)}
    et après \textbf{(b)} suppression de \texttt{0010\,0111}}
  \vspace{-3mm}
  \label{fig:suppression-Patricia-trie}
\end{figure}



La suppression d'une feuille $x$ du Patricia trie se déroule comme suit.
Si cet élément est la racine, l'arbre devient vide.
Sinon, Cette feuille a un frère $y$ et un père.
$x$ et son père sont supprimés, et $y$ remonte d'un niveau (et prend donc la
place du père).
Enfin, les champs du nouveau père de $y$ sont mis à jour pour maintenir la
cohérence de l'arbre (le plus grand préfixe commun est recalculé).
La figure~\ref{fig:suppression-Patricia-trie} illustre la suppression de
l'adresse \ppleaf{\texttt{0010\,0111}} dans un Patricia trie.
Ce n\oe{}ud ainsi que son père \ppnode{\texttt{0010\,011*}} sont supprimés, et
\ppleaf{\texttt{0010\,0110}} est remonté d'un niveau.


\section{Instrumentation pour la Vérification à l'exécution}
\label{sec:mem-instru}


Nous présentons l'instrumentation d'un programme C annoté dans une optique de
vérification à l'exécution des annotations liées au modèle mémoire.
Certaines instructions C doivent être instrumentées afin d'enregistrer
des informations sur la mémoire à l'exécution, et les annotations sont
instrumentées afin de pouvoir utiliser ces informations.


\subsection{Instrumentation des allocations et affectations}


Dans cette partie, nous définissons de manière informelle l'instrumentation à
opérer sur le programme annoté afin de pouvoir observer les allocations et
les affectations en mémoire.

Comme nous l'avons vu, pour traiter ces annotations \eacsl, nous devons
conserver pour chaque bloc les informations suivantes : l'adresse de base, le
nombre d'octets occupés en mémoire, l'initialisation de chaque octet et un
booléen indiquant si le bloc est en lecture seule (par exemple si c'est une
chaîne littérale).

Afin d'enregistrer ces informations nous définissons les fonctions C suivantes.
La fonction \lstinline'__store_block', qui prend en paramètre un pointeur et une
taille en octets, ajoute un bloc dans ce que nous appelerons le \textit{store}.
Le \textit{store} est une structure de donnée qui recense tous les blocs en
mémoire que nous souhaitons surveiller.
Nous donnons plus de détails sur le \textit{store} en
partie~\ref{sec:mem-model}.
La fonction \lstinline'__store_block' est invoquée sur chaque variable locale
et paramètre formel de fonction (au début de leur portée), ainsi que sur chaque
variable globale (au début de la fonction \lstinline'main').
La fonction \lstinline'__delete_block' permet d'enlever du \textit{store} un
bloc reçu en paramètre.
Elle est appelée en fin de portée d'une variable.

Nous définissons également des ``surcouches'' aux fonctions de la bibliothèque
standard du C : \lstinline'malloc', \lstinline'realloc', \lstinline'calloc' et
\lstinline'free'.
L'instrumentation va substituer les appels à ces fonctions par un appel à nos
fonctions : \lstinline'__malloc', \lstinline'__realloc', \lstinline'__calloc'
et \lstinline'__free'.
Ces fonctions permettent d'ajouter ou de supprimer automatiquement les blocs
au \textit{store}, ainsi que de transférer les informations d'initialisations
d'un bloc à un autre en cas d'application de la fonction \lstinline'realloc'.

Enfin, nous définissons une fonction \lstinline'__initialize' qui permet de
marquer un certain nombre d'octets d'un bloc comme étant initialisés.
L'instrumentation ajoute des appels à cette fonction pour chaque affectation du
programme original, ainsi que pour chaque variable globale (initialisée à 0 par
défaut en C).


\subsection{Instrumentation des annotations}


\begin{figure*}[bt]
  \scriptsize{
    {\myinference{$\tau$-base-addr}
      {(l, \mbox{\lstinline't'}:ptr) \rulearrow (I, \mbox{\lstinline'e'}) }
      {(l, \mbox{\lstinline'\\base_addr(t)'}) \rulearrow
        (I \concat
        (l, \mbox{\lstinline'void* var_n = __base_addr(e);'}),
        \mbox{\lstinline'var_n'})
      }{}
    }

    {\myinference{$\tau$-block-length}
      {(l, \mbox{\lstinline't'}:ptr) \rulearrow (I, \mbox{\lstinline'e'}) }
      {(l, \mbox{\lstinline'\\block_length(t)'}) \rulearrow
        (I \concat
        (l, \mbox{\lstinline'unsigned long var_n = __block_length(e);'}),
        \mbox{\lstinline'var_n'})
      }{}
    }

    {\myinference{$\tau$-offset}
      {(l, \mbox{\lstinline't'}:ptr) \rulearrow (I, \mbox{\lstinline'e'}) }
      {(l, \mbox{\lstinline'\\offset(t)'}) \rulearrow
        (I \concat
        (l, \mbox{\lstinline'int var_n = __offset(e);'}),
        \mbox{\lstinline'var_n'})
      }{}
    }

    {\myinference{$\pi$-valid}
      {(l, \mbox{\lstinline't'}:ptr) \rulearrow (I, \mbox{\lstinline'e'}) }
      {(l, \mbox{\lstinline'\\valid(t)'}) \rulearrow
        (I \concat
        (l, \mbox{\lstinline'int var_n = __valid(e,sizeof(typeof(e)));'}),
        \mbox{\lstinline'var_n'})
      }{}
    }

    {\myinference{$\pi$-valid-read}
      {(l, \mbox{\lstinline't'}:ptr) \rulearrow (I, \mbox{\lstinline'e'}) }
      {(l, \mbox{\lstinline'\\valid_read(t)'}) \rulearrow
        (I \concat
        (l, \mbox{\lstinline'int var_n = __valid_read(e,sizeof(typeof(e)));'}),
        \mbox{\lstinline'var_n'})
      }{}
    }

    {\myinference{$\pi$-initialized}
      {(l, \mbox{\lstinline't'}:ptr) \rulearrow (I, \mbox{\lstinline'e'}) }
      {(l, \mbox{\lstinline'\\initialized(t)'}) \rulearrow
        (I \concat
        (l, \mbox{\lstinline'int var_n = __initialized(e,sizeof(typeof(e)));'}),
        \mbox{\lstinline'var_n'})
      }{}
    }
  }
  \caption{Règles de traduction pour les annotations liées au modèle mémoire}
  \label{fig:mem-annots-rules}
\end{figure*}


Afin de pouvoir exploiter les informations sur les blocs stockés dans le
\textit{store}, nous définissons les fonctions \lstinline'__base_addr',
\lstinline'__block_length', \lstinline'__offset', \lstinline'__valid',
\lstinline'__valid_read' et \lstinline'__initialized'.
La sémantique de ces fonctions est calquée sur la sémantique des annotations
\eacsl donnée en partie~\ref{sec:mem-annots}.
La fonction \lstinline'__base_addr' retourne l'adresse de début d'un bloc à
partir du pointeur qu'elle reçoit en argument.
La fonction \lstinline'__block_length' retourne le nombre d'octets occupés par
le bloc.
La fonction \lstinline'__offset' retourne le décalage d'un pointeur par rapport
à son adresse de base dans le bloc.
Les fonctions \lstinline'__valid' et \lstinline'__valid_read'
retournent le statut de validité à partir d'un pointeur et d'un nombre d'octets.
La fonction \lstinline'__initialized' retourne le statut d'initialisation
d'un nombre d'octets d'un pointeur en argument.
La figure~\ref{fig:mem-annots-rules} présente les règles de traduction des
annotations en utilisant le même formalisme qu'au chapitre~\ref{sec:traduction}.
Dans les règles \textsc{$\pi$-valid-read} et \textsc{$\pi$-initialized},
\lstinline'typeof(e)' désigne le type de l'expression \lstinline'e', qui est un
type C tel \lstinline'int' ou \lstinline'char'.
\lstinline'sizeof(typeof(e))' correspond au nombre d'octets occupés par un objet
de ce type en mémoire.


\section*{Conclusion du chapitre}


Nous avons présenté dans ce chapitre une manière de vérifier dynamiquement les
annotations \eacsl qui concernent la mémoire : \lstinline'\base_addr',
\lstinline'\block_length', \lstinline'\offset', \lstinline'\valid',
\lstinline'\valid_read' et \lstinline'\initialized'.
Nous avons présenté ces annotations, un modèle mémoire permettant de décider de
leur validité à l'exécution, ainsi qu'une manière d'instrumenter le programme
afin de les vérifier à l'exécution.
Le chapitre~\ref{sec:eacsl} donne plus de détails concernant l'implémentation
du modèle mémoire ainsi qu'une évaluation de ses performances.
