
\chapter{Analyse de l'existant}


Le domaine de la vérification et de la validation regroupe un ensemble de
techniques du cycle de développement des logiciels qui ont pour objectif de
s'assurer de leur correction et de leur sûreté. Ces deux notions sont apparues
dans les années soixante-dix avec les travaux de \bsc{Dijkstra} \cite{Dijkstra},
\bsc{Floyd} \cite{Floyd} et \bsc{Hoare} \cite{Hoare}.

La correction d'un logiciel représente le respect de l'implémentation par
rapport aux spécifications. La sûreté d'un logiciel est liée à son absence
d'erreurs à l'exécution.


\section{Model-checking}
\label{sec:model-checking}

Le model-checking \cite{model-checking} permet de vérifier algorithmiquement si
un modèle donné (le système ou une abstraction de ce système) satisfait une
spécification, formulée en termes de logique temporelle \cite{LTL}. Un modèle
est un ensemble d'états, de propriétés que vérifie chaque état, et de
transitions entre ces états qui décrivent l'évolution du système.\\

Le model-checking couvre l'ensemble des états du système et des transitions afin
d'analyser toutes les exécutions possibles du système. Sur de grands systèmes,
cette méthode est pénalisée par l'explosion combinatoire du nombre des états (et
la complexité en temps ou en espace qui en résulte). Il est néanmoins possible
de modéliser des algorithmes asynchrones répartis et des automates non
déterministes, comme le fait notamment l'outil \textsc{Spin} \cite{SPIN}.\\

La plateforme \textsc{Frama-C} du LSL intègre \textsc{Aoraï}, un greffon
permettant d'annoter automatiquement le code source d'un programme C d'après
une formule de logique temporelle linéaire, de sorte que les annotations sont
vérifiées si le programme repecte la formule.


\section{Analyse statique}
\label{sec:AS}

L'analyse statique \cite{static-analysis} examine le code source du programme
sans l'exécuter. Elle raisonne sur tous les comportements qui pourraient
survenir lors de l'exécution et permet donc de déduire des propriétés devant
être vérifiées pour toutes ces exécutions, dans le but de prouver la correction
du programme.\\

En revanche, la vérification de programme étant en général indécidable
\cite{undecidability}, il est souvent nécessaire d'utiliser des
sur-approximations, ce qui implique que les résultats peuvent être moins précis
que ce que l'on souhaite mais ils sont garantis pour toutes les exécutions.
Ainsi, on peut établir des propriétés de sûreté ({\em safety}), où l’on cherche
des invariants sur les valeurs des variables du programme (une plage de valeurs
par exemple), afin d'exclure certains risques d'erreurs à l'exécution.\\

Parmi les méthodes statiques sont distinguées : l'interprétation abstraite
(section~\ref{sec:interpretation-abstraite}), l'abstraction à partir de
prédicats (section~\ref{sec:abstraction-predicats}) et la preuve de théorèmes
(section~\ref{sec:preuve}).


\subsection{Interprétation abstraite}
\label{sec:interpretation-abstraite}

L'interprétation abstraite \cite{abstract-interpretation} s'appuie sur les
théories du point-fixe et des domaines pour introduire des sur-approximations
des comportements d'un programme. Elle consiste à abstraire les domaines des
variables par des domaines finis et beaucoup plus petits. Par exemple, le
domaine des entiers pourrait être abstrait par un domaine de trois valeurs :
$(-, 0, +)$. Appliquée à l’analyse de valeurs, elle consiste à calculer à
chaque ligne du code une sur-approximation de l’ensemble des valeurs prises par
chaque variable en cette ligne lors de toutes les exécutions du programme,
permettant ainsi de détecter certaines erreurs comme les divisions par zéro ou
les accès en dehors des bornes des tableaux.\\

Pour contourner le problème d’indécidabilité, la théorie de l’interprétation
abstraite construit une méthode qui, à la même question, répondra ``oui'',
``non'' ou ``peut-être''. Si la méthode répond ``peut-être'', c’est qu’on n’a pu
prouver ni l’un ni l’autre des deux premiers cas. C’est ce qu’on appelle une
alarme : il est possible qu’une des exécutions du programme produise une erreur
donnée, mais nous n’avons été capable ni de le confirmer ni de l’infirmer.
L’erreur signalée par une alarme peut ne jamais apparaître à l’exécution, dans
ce cas on l’appelle fausse alarme. On ne calcule donc pas la propriété exacte
mais une abstraction de cette propriété, en imposant la contrainte de sûreté
suivante : ``la propriété abstraite calculée ne doit oublier aucune exécution
concrète''. L'abstraction est effectuée à partir de prédicats atomiques
définissant des abstractions des domaines des variables.\\

\textsc{PolySpace C Verifier} \cite{PolySpace} a été le premier outil commercial
utilisant l'interprétation abstraite pour détecter les erreurs à l'exécution
dans les programmes en C, C++ et Ada mais signale beaucoup de fausses alarmes.
L'ENS a développé \textsc{Astrée} \cite{ASTREE}, spécifique au langage C et aux
logiciels critiques. Le CEA LIST a développé \textsc{Fluctuat} \cite{Fluctuat},
qui mesure précisément les approximations faites à l'exécution d'un programme C.
\textsc{Frama-C} intègre un greffon d'interprétation abstraite : \textsc{Value}
\cite{Value}.


\subsection{Abstraction à partir de prédicats}
\label{sec:abstraction-predicats}

L'abstraction à partir de prédicats \cite{predicate-abstraction} est une
technique permettant de générer automatiquement des abstractions de systèmes au
nombre d'états infini. Pour un programme $P$ au nombre d'états infini, un
ensemble fini de prédicats $E = \{f_1, ..., f_n\}$ est défini, ces prédicats
sont des expressions booléennes sur les variables de $P$ et les constantes du
langage.\\

Chaque état concret de $P$ est mis en correspondance avec un état abstrait de
l'abstraction de $P$, après évaluation par les prédicats de $E$. Un état
abstrait est un $n$-upplet de valeurs booléennes correspondant à la
satisfaisabilité des $n$ prédicats (au moyen d'un solveur SMT).\\

Par exemple, si 3 prédicats $f_1, f_2, f_3$ sont définis et que la
satisfaisabilité de ces prédicats à l'état concret $e$ est évaluée
respectivement à $false, true, true$, alors dans l'abstraction générée, l'état
$e$ correspond à l'état abstrait $(\lnot f_1, f_2, f_3)$.\\

L'abstraction générée comporte un nombre fini d'états (au plus $2^n$) car il n'y
a qu'un nombre fini de prédicats, le model-checking peut donc être appliqué à
cette abstraction. Si une propriété de sûreté est vérifiée dans l'abstraction,
elle l'est également dans le système concret.

Cette technique est notamment utilisée par l'outil \textsc{Slam} \cite{SLAM} de
Microsoft.


\subsection{Preuve de théorèmes}
\label{sec:preuve}

La preuve de théorèmes utilise des fondements mathématiques et logiques
\cite{Hoare} pour prouver des propriétés de programmes. Tout d'abord, le système
est décrit par un ensemble d'axiomes et de règles d'inférence. Puis, le calcul
de la plus faible précondition \cite{Dijkstra} est utilisé pour générer des
formules appelées obligations de preuve, qui sont finalement soumises à un
prouveur de théorèmes, qui applique différentes techniques de résolution.\\

Contrairement au model-checking, la preuve de théorèmes a l'avantage d'être
indépendante de la taille de l'espace des états, et peut donc s'appliquer sur
des systèmes de grande taille. En contre-partie, cette technique requiert une
expertise de l'utilisateur pour adapter le programme à la preuve (en l'annotant
par exemple) et guider le prouveur si nécessaire.\\

Il existe des prouveurs automatiques tels que \textsc{Simplify}
\cite{Simplify}, \textsc{Ergo} \cite{Ergo} et \textsc{Z3} \cite{Z3}; et des
prouveurs interactifs, où la preuve est guidée par l'utilisateur, tels que
\textsc{Coq} \cite{Coq}, \textsc{Isabelle} \cite{Isabelle}, et \textsc{Hol}
\cite{HOL}. Certains de ces prouveurs ou assistants de preuve sont intégrés à
d'autres outils tels Boogie \cite{Boogie} ou ESC/Java \cite{ESC/Java}.
\textsc{Frama-C} intègre deux greffons de preuve, \textsc{Jessie} et
\textsc{Wp}, qui traitent des programmes dont le code contient des annotations
\textsc{Acsl} \cite{ACSL}.


\section{Analyse dynamique}
\label{sec:AD}

L’analyse dynamique est basée sur des techniques d’exécution du programme, de
simulation \cite{simulation} d’un modèle ou d'exécution symbolique
\cite{symbolic-execution}, regroupées sous le terme générique ``test''.\\

Les tests peuvent s’appliquer tout au long du cycle de développement d’un
logiciel. Les tests unitaires vérifient le bon fonctionnement des différentes
entités d’un système, indépendamment les unes des autres. Les tests
d'intégration vérifient la bonne communication entre ces entités. Les tests de
validation s'assurent que les fonctionnalités correspondent au besoin de
l’utilisateur final. Enfin, les tests de non-régression vérifient que l'ajout de
nouvelles fonctionnalités ne détériore pas les anciennes fonctionnalités.\\

En général, les techniques de test ne sont pas exhaustives et n'explorent qu'un
sous-ensemble des chemins d'exécutions du programme, en conséquence, l’absence
d’échecs lors du passage des tests n’est pas une garantie de bon fonctionnement
du système. Néanmoins, selon les critères utilisés pour la génération des
tests, et selon la couverture des chemins d'exécution fournie par les tests, un
système ainsi validé peut acquérir un certain niveau de confiance.\\

Les méthodes de test peuvent être classées en trois catégories : le test
aléatoire, le test structurel (section~\ref{sec:test-structurel}) et le test
fonctionnel (section~\ref{sec:test-fonctionnel}). Comme son nom l'indique, le
test aléatoire consiste à générer des valeurs d'entrée du programme au hasard et
ne sera pas détaillé davantage ici.


\subsection{Test structurel}
\label{sec:test-structurel}

Le test structurel, ou test ``boîte blanche'', est une technique de test qui
fonde la détermination des différents cas de test sur une analyse de la
structure du code source du programme étudié. On distingue deux types de tests
structurels : le test orienté flot de contrôle et le test orienté flot de
données.\\

Le test orienté flot de données cherche à couvrir certaines relations entre la
définition d’une variable et son utilisation, par exemple, on peut souhaiter
couvrir toutes les lectures d'une variable suivant une écriture.\\

Le test orienté flot de contrôle s’intéresse quant à lui à la structure du
programme : l'ordre dans lequel les instructions sont exécutées. Il se base sur
le graphe de flot de contrôle du programme : un graphe connexe orienté avec un
unique n\oe{}ud d’entrée et un unique n\oe{}ud de sortie, dont les n\oe{}uds
sont les blocs de base du programme et les arcs représentent les branchements
(conditions). Une couverture structurelle de ce graphe est recherchée, selon un
critère qui peut être par exemple ``toutes les instructions'',
``toutes les branches'' (toutes les décisions), ``tous les chemins'' ou
``tous les $k$-chemins'' (détaillé section~\ref{sec:PathCrawler}).\\

L’exécution symbolique dynamique, ou exécution ``concolique'', associe
l’exécution concrète du programme et l’exécution symbolique afin d’explorer les
chemins du programme. L’exécution concrète sert à confirmer que le chemin
parcouru est bien celui pour lequel le cas de test exécuté a été généré, comme
présenté section~\ref{sec:PathCrawler}.\\

Plusieurs outils se basent sur l'exécution concolique pour explorer un programme
sous test, dont \textsc{Smart} \cite{SMART}, \textsc{Pex} \cite{Pex} et
\textsc{Sage} \cite{Sage} développés par Microsoft, \textsc{Cute} \cite{CUTE},
\textsc{Klee} \cite{KLEE}, \textsc{Exe} \cite{EXE} et enfin
\textsc{PathCrawler}, \cite{PathCrawler} développé au CEA LIST. Ces outils
utilisent des solveurs de contraintes pour générer des cas de test permettant
d'aboutir à une couverture souhaitée des exécutions du programme par les tests.


\subsection{Test fonctionnel}
\label{sec:test-fonctionnel}

Le test fonctionnel, ou test ``boîte noire'', génère des jeux de test en
fonction du comportement attendu du programme : un cas de test sera choisi pour
chaque comportement particulier. Le test fonctionnel est utilisé pour vérifier
la conformité des réactions du logiciel avec les attentes de l'utilisateur, sans
connaissance du code source. Il existe de nombreuses techniques qui se
différencient par la manière de choisir les données de test, parmi lesquelles :

\begin{description}
\item[le test de partition] \hfill \\
les valeurs d’entrées du logiciel sont regroupées en classes d’équivalence, sur
lesquelles le logiciel doit avoir le même comportement ({\em domain splitting}),
une seule valeur aléatoire est choisie dans chaque classe de la partition;
\item[le test aux limites] \hfill \\
les données de test sont choisies aux bornes des domaines de définition des
variables.
\end{description}

Le CEA LIST développe \textsc{GATeL} \cite{GATEL}, un outil de génération de
tests fonctionnels qui se base sur une représentation symbolique des états du
système : le programme, les invariants et les contraintes décrivant l'objectif
de test sont exprimés dans le langage \textsc{Lustre} \cite{Lustre}. Cet outil
offre la possibilité de réaliser des partitions de domaines.


\section{Simplification syntaxique}

La simplification syntaxique, ou {\em slicing} \cite{slicing}, est une technique
de transformation qui permet de simplifier un programme tout en préservant les
comportements définis par un critère de {\em slicing}. Il s’agit d’une
modification purement structurelle fondée sur l'analyse du flot de contrôle et
du flot de données du programme. Toutes les instructions et les branches du
programme original qui n’ont pas d’intérêt par rapport au critère considéré
n’apparaissent pas dans le programme simplifié ({\em slice}). Cette technique
permet d’isoler les instructions influant sur une ou plusieurs variables à un
point donné.\\

Le calcul d’une {\em slice} consiste à construire récursivement l’ensemble des
n\oe{}uds et des arcs pertinents en remontant le graphe de flot de contrôle.
Cette technique peut être appliquée de manière intra-procédurale en exploitant
le {\em Program Dependence Graph} \cite{PDG} ou de manière inter-procédurale en
exploitant le {\em System Dependence Graph} \cite{SDG}.\\

\textsc{Frama-C} intègre \textsc{Slicing}, un greffon produisant un programme
composé d'un sous-ensemble des instructions (l'ordre est conservé) du programme
analysé. Les instructions sont choisies d'après un critère de {\em slicing}
spécifié par l'utilisateur. Le résultat est garanti d'être un programme C
correct ayant le même comportement que le programme d'origine selon le critère
de {\em slicing} choisi.


\section{Combinaison d'analyses statiques et dynamiques}
\label{sec:combinaison}

Les méthodes statiques et les méthodes dynamiques ont des avantages et des
inconvénients complémentaires : l'analyse statique étant complète mais
imprécise, l'analyse dynamique étant précise mais incomplète. L’idée de les
combiner pour associer leurs avantages et combattre leurs inconvénients
\cite{duality} est une voie de recherche active et fructueuse dans le domaine de
la vérification de programmes.\\

Une première combinaison de méthodes est le raffinement d'abstraction guidé par
des contre-exemples \cite{CEGAR}, qui associe l’abstraction par prédicats et le
model-checking : une abstraction du programme est générée à partir d’un ensemble
de prédicats et invariants. Si le {\em model-checking} ne peut pas établir
qu'une propriété est satisfaite, des contre-exemples sont générés, permettant de
raffiner l’abstraction en ajoutant de nouveaux prédicats, et ainsi de suite. Ce
processus peut ne pas terminer. Plusieurs outils de vérification des programmes
C se basent sur cette méthode, parmi lesquels \textsc{Blast} \cite{BLAST},
\textsc{Yogi} \cite{YOGI} et \textsc{Magic} \cite{MAGIC}.\\

L'outil \textsc{DSD-Crasher} \cite{DSD-Crasher} utilise une approche différente
et combine une génération d'invariants par génération de tests et des techniques
d'apprentissage \cite{discover-invariants} et une analyse statique signalant des
alarmes qui seront confirmées par un générateur de tests \cite{JCrasher}. Les
invariants détectés permettent de guider la classification d'alarmes et la
génération de tests, ce qui rend l'outil très dépendant de leur qualité.\\

La méthode \textsc{Sante} \cite{TheseOmar, SANTE}, mise en \oe{}uvre au sein de
\textsc{Frama-C}, combine l'interprétation abstraite, le {\em slicing} et la
génération de tests structurels avec \textsc{PathCrawler}. L’analyse statique
signale les instructions risquant de provoquer des erreurs à l’exécution par des
alarmes, dont certaines peuvent être de fausses alarmes, puis l’analyse
dynamique génère des tests confirmant ou infirmant ces alarmes. Sur des
programmes de grande taille, l'analyse dynamique peut manquer de temps pour
classer toutes ces alarmes (à cause de l'explosion combinatoire des exécutions
possibles). Le {\em slicing} est utilisé pour réduire la taille des programmes
testés et donc le temps nécessaire à leur analyse.


\subsection{DyTa}
Code Contracts static checker + Pex

\subsection{Collaborative verification and testing with explicit assumptions}
\cite{collaborative-verification}\\
\cite{Dafny}\\
\cite{Pex}

\subsection{Checking properties described by state machines : on synergy of
  instrumentation, slicing and symbolic execution}
\cite{checking-prop-state-machines}\\
\textsc{Klee} \cite{KLEE} pour l'exécution symbolique
