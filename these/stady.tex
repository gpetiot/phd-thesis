
\chapter{Méthode combinant Tests et Preuves}

\begin{itemize}
\item Introduction
\item Présentation de la méthode,
  Présentation de scénarios types illustrés par des exemples
\item Règles d'instrumentation des programmes C à partir des spécifications
  E-ACSL
\item Preuve de correction de l'instrumentation
\item Choix d'instrumentation spécifiques de l'objectif Test
\item Difficultés et limites de l'instrumentation
\end{itemize}

\chapter{Implémentation: \textsc{StaDy}}

\begin{itemize}
\item Introduction
\item Rapprochement Frama-C et PathCrawler
\item Instrumentation de E-ACSL
\item Efficacité et amélioration
\end{itemize}

\chapter{Expérimentations et évaluation de la méthode}

\begin{itemize}
\item évaluer la capacité à trouver des erreurs dans le code par détection de
  contre-exemples. On prend des codes faux engendrés par mutation, une
  spécification juste et on mesure pour combien de codes faux on trouve des
  contr-exemples.
\item même chose en mutant les spécifications
\item détection de sous-spécification (par mutation)
\end{itemize}
