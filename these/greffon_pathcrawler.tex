
\chapter{Intégration de \textsc{PathCrawler} dans \textsc{Frama-C}}

Ce travail est nécessaire afin de poursuivre l'intégration de
\textsc{PathCrawler} \cite{PathCrawler} dans \textsc{Frama-C} \cite{Frama-C}.
Cela évite à l'utilisateur de devoir spécifier son programme avec deux
préconditions différentes et facilite la mise en \oe{}uvre des différentes
combinaisons d'analyse statique et dynamique au sein de la plate-forme.


On considère un programme C contenant des annotations ACSL insérées par l'utilisateur ou par un greffon (Value, RTE, ...). Les différentes étapes sont :
\begin{enumerate}
  \item traduction des annotations ACSL en C, tel qu'une fonction particulière
    (par défaut une assertion) est appelée si l'annotation correspondante n'est
    pas valide à l'exécution;
  \item seconde étape de traduction : modifications mineures du code C généré
    pour que celui-ci soit traitable par PathCrawler. En particulier, il faut
    mettre la pré-condition dans une fonction séparée ayant un nom spécifique,
    idem pour la post-condition qui doit en plus se trouver dans un fichier
    séparé ayant lui aussi un nom particulier. PathCrawler utilise aussi son
    propre jeu de macros d'asserts, qu'on utilisera à la place des asserts
    générés par E-ACSL. On supprimera les appels aux fonctions de
    monitoring des accès mémoire, car on ne veut pas que PathCrawler essaie de
    couvrir tous les chemins pour ces fonctions. Enfin, les appels à
    $\backslash$valid seront remplacés par des appels à pathcrawler\_dimension,
    mais les restrictions actuelles de PathCrawler font que cela ne fonctionnera
    que si le pointeur passé en argument est une adresse de base et un paramètre
    de la fonction;
  \item appel de PathCrawler avec la stratégie de converture ``tous les
    chemins'', un socket est ouvert entre PathCrawler et le greffon. Les
    informations transmises sont : verdict des cas de test, identifiants des
    assertions violées, cas de test violant ces assertions, nombre de cas de
    test générés. Ces informations permettent au greffon de mettre à jour le
    statut de chaque annotation dans Frama-C;
  \item la mise à jour de ces statuts peut permettre de faire avancer une
    analyse effectuée par un autre greffon (WP ou Value).
\end{enumerate}~\\

Les constructions suivantes ne sont pas encore prises en charge par E-ACSL :
\begin{itemize}
\item loop invariant
\item loop variant
\item loop assigns
\item assigns
\item notation ``range'' : X..Y
\end{itemize}

\section{Transformations de code pour prise en charge pré- et post-condition}

\subsection{Transformations de code pour la pré-condition}
\begin{itemize}
\item type de retour de la fonction devient ${\tt int}$
\item remplacer les ${\tt e\_acsl\_assert(X)}$ par 
  ${\tt if(!X) return 0;}$ avant l'appel à la fonction sous test
\item supprimer l'appel à la fonction sous test
\item supprimer tous les statements différents du return après l'appel à la
  fonction sous test
\item ${\tt return 1;}$ à la fin
\end{itemize}

\subsection{Transformations de code pour la post-condition}
\begin{itemize}
\item type de retour de la fonction devient ${\tt void}$
\item dupliquer les variables formelles de type pointeur ou tableau :
  ${\tt ..., int* Pre\_x, int* x, ...}$
\item ajouter une formelle du type de valeur de retour de la fonction sous test,
  nommée ${\tt \_\_retres}$
\item supprimer les ${\tt e\_acsl\_assert}$ avant appel à la fonction sous test
\item supprimer appel à la fonction sous test
\item remplacer les ${\tt e\_acsl\_assert(X)}$ par
  ${\tt if(!X) \{pathcrawler\_verdict\_failure(); return;\}}$ après l'appel à la
  fonction sous test
\item ajouter ${\tt pathcrawler\_verdict\_success(); return;}$ à la fin
\end{itemize}

