%% \newpage
%% \commentGP{mettre toute la preuve en annexe ?}
%% \commentGP{juste mettre la preuve des invariants en annexe ?}
%% \section{Annexe : Preuve de la Traduction des Quantifications Universelles}

%% {\myinference[$\pi$-forall]
%%   {
%%     (l, \mbox{\lstinline't1'} : \mathbb{Z}) \rulearrow
%%     (I_1, \mbox{\lstinline'e1'}) \\
%%     (l, \mbox{\lstinline't2'} : \mathbb{Z}) \rulearrow
%%     (I_2, \mbox{\lstinline'e2'}) \\
%%     (l, \mbox{\lstinline'p'}) \rulearrow (I_3, \mbox{\lstinline'e3'})
%%   }
%%   {
%%     \splitfrac {
%%       (l, \mbox{\lstinline'\\forall integer k; t1 <= k <= t2 ==> p'})
%%       \rulearrow
%%     } {
%%       \splitfrac {
%%         (I_1 \cdot I_2
%%         \cdot (l, \mbox{\lstinline'int e = 1;'})
%%         \cdot
%%           (l, \Zinit \underline{\mbox{\lstinline'k = e1;'}} )
%%       } {
%%         \cdot
%%           (l, \mbox{\lstinline'while('}
%%           \underline{\mbox{\lstinline'k <= e2'}}~ \mbox{\lstinline'&& e)'}
%%           \bopen
%%           I_3
%%           \cdot (l, \mbox{\lstinline'e = e3;'})
%%           \cdot (l, \underline{\mbox{\lstinline'k++'}} \semicolon)
%%           \bclose )
%%         \cdot
%%           (l, \underline{\mbox{\lstinline'k'}} \Zclear \semicolon )
%%         \cdot
%%           (l, \underline{\mbox{\lstinline'e1'}} \Zclear \semicolon )
%%         \cdot
%%           (l, \underline{\mbox{\lstinline'e2'}} \Zclear \semicolon ),
%%         \mbox{\lstinline'e'})
%%       }
%%     }
%%   }
%% }~\\


%% Prouvons que pour n'importe quel environnement $\env$ :

%% \eval{\lstinline'e'}{(\compi{$I$}{$\env$})} =
%% \eval{\lstinline'\\forall integer k; t1 <= k <= t2 ==> p'}{$\env$}

%% et

%% $\env$ \subenv \compi{$I$}{$\env$}

%% et

%% \compi{$I$}{$\env$} $\neq$ $\errormem$.


%% \begin{proof}
%%   ~\\
%%   Hypothèses : \lstinline't1', \lstinline't2' et \lstinline'p' sont correctement
%%   traduits.
%%   Pour tout environnement $\env$ :

%%   \begin{tabular}{rclr}
%%     \eval{\lstinline'e1'}{(\compi{$I_1$}{$\env$})}
%%     & = & \eval{\lstinline't1'}{$\env$} & \eqlabel{h1} \\
%%     \eval{\lstinline'e2'}{(\compi{$I_2$}{$\env$})}
%%     & = & \eval{\lstinline't2'}{$\env$} & \eqlabel{h2} \\
%%     \eval{\lstinline'e3'}{(\comp{$I_3$}{$\env$})}
%%     & = & \eval{\lstinline'p'}{$\env$} & \eqlabel{h3} \\
%%     $\env$ & \subenv & \compi{$I_1$}{$\env$} & \eqlabel{h4} \\
%%     $\env$ & \subenv & \compi{$I_2$}{$\env$} & \eqlabel{h5} \\
%%     $\env$ & \subenv & \compi{$I_3$}{$\env$} & \eqlabel{h6} \\
%%     \compi{$I_1$}{$\env$} & $\neq$ & $\errormem$ & \eqlabel{h7} \\
%%     \compi{$I_2$}{$\env$} & $\neq$ & $\errormem$ & \eqlabel{h8} \\
%%     \compi{$I_3$}{$\env$} & $\neq$ & $\errormem$ & \eqlabel{h9} \\
%%   \end{tabular}

%%   Développons \compi{$I$}{$\env$} :
  
%%   \begin{tabular}{p{0cm} p{.5cm} p{14cm}}
%%     \multicolumn{3}{l}{\compi{$I$}{$\env$}} \\
%%     &=&
%%     \compi{$I_1 \cdot I_2 \cdot (l, \mbox{\lstinline'int e = 1;'}) \cdot
%%       (l, \Zinit \underline{\mbox{\lstinline'k = e1;'}} ) \cdot
%%       (l, \mbox{\lstinline'while('}
%%       \underline{\mbox{\lstinline'k <= e2'}}~ \mbox{\lstinline'&& e)'}
%%       \bopen I_3 \cdot (l, \mbox{\lstinline'e = e3;'}) \cdot
%%       (l, \underline{\mbox{\lstinline'k++'}} \semicolon) \bclose ) \cdot
%%       (l, \underline{\mbox{\lstinline'k'}} \Zclear \semicolon ) \cdot
%%       (l, \underline{\mbox{\lstinline'e1'}} \Zclear \semicolon ) \cdot
%%       (l, \underline{\mbox{\lstinline'e2'}} \Zclear \semicolon )$}{$\env$}
%%     \\
%%     &$\eq{\eqlabel{I-split}}$&
%%     \compi{$I_2 \cdot (l, \mbox{\lstinline'int e = 1;'}) \cdot
%%       (l, \Zinit \underline{\mbox{\lstinline'k = e1;'}} ) \cdot
%%       (l, \mbox{\lstinline'while('}
%%       \underline{\mbox{\lstinline'k <= e2'}}~ \mbox{\lstinline'&& e)'}
%%       \bopen I_3 \cdot (l, \mbox{\lstinline'e = e3;'}) \cdot
%%       (l, \underline{\mbox{\lstinline'k++'}} \semicolon) \bclose ) \cdot
%%       (l, \underline{\mbox{\lstinline'k'}} \Zclear \semicolon ) \cdot
%%       (l, \underline{\mbox{\lstinline'e1'}} \Zclear \semicolon ) \cdot
%%       (l, \underline{\mbox{\lstinline'e2'}} \Zclear \semicolon )$}{
%%       (\compi{$I_1$}{$\env$})}
%%     \\
%%     &$\eq{\eqlabel{I-split}}$&
%%     \compi{$(l, \mbox{\lstinline'int e = 1;'}) \cdot
%%       (l, \Zinit \underline{\mbox{\lstinline'k = e1;'}} ) \cdot
%%       (l, \mbox{\lstinline'while('}
%%       \underline{\mbox{\lstinline'k <= e2'}}~ \mbox{\lstinline'&& e)'}
%%       \bopen I_3 \cdot (l, \mbox{\lstinline'e = e3;'}) \cdot
%%       (l, \underline{\mbox{\lstinline'k++'}} \semicolon) \bclose ) \cdot
%%       (l, \underline{\mbox{\lstinline'k'}} \Zclear \semicolon ) \cdot
%%       (l, \underline{\mbox{\lstinline'e1'}} \Zclear \semicolon ) \cdot
%%       (l, \underline{\mbox{\lstinline'e2'}} \Zclear \semicolon )$}{
%%       (\compi{$I_2$}{(\compi{$I_1$}{$\env$})})
%%     }
%%     \\
%%     &$\eq{\eqlabel{I-split}}$&
%%     \compi{$(l, \Zinit \underline{\mbox{\lstinline'k = e1;'}} ) \cdot
%%       (l, \mbox{\lstinline'while('}
%%       \underline{\mbox{\lstinline'k <= e2'}}~ \mbox{\lstinline'&& e)'}
%%       \bopen I_3 \cdot (l, \mbox{\lstinline'e = e3;'}) \cdot
%%       (l, \underline{\mbox{\lstinline'k++'}} \semicolon) \bclose ) \cdot
%%       (l, \underline{\mbox{\lstinline'k'}} \Zclear \semicolon ) \cdot
%%       (l, \underline{\mbox{\lstinline'e1'}} \Zclear \semicolon ) \cdot
%%       (l, \underline{\mbox{\lstinline'e2'}} \Zclear \semicolon )$}{
%%       (\compi{$(l, \mbox{\lstinline'int e = 1;'})$}{
%%         (\compi{$I_2$}{(\compi{$I_1$}{$\env$})})})
%%     }
%%     \\
%%     &$\eq{\eqlabel{I-instr}}$&
%%     \compi{$(l, \Zinit \underline{\mbox{\lstinline'k = e1;'}} ) \cdot
%%       (l, \mbox{\lstinline'while('}
%%       \underline{\mbox{\lstinline'k <= e2'}}~ \mbox{\lstinline'&& e)'}
%%       \bopen I_3 \cdot (l, \mbox{\lstinline'e = e3;'}) \cdot
%%       (l, \underline{\mbox{\lstinline'k++'}} \semicolon) \bclose ) \cdot
%%       (l, \underline{\mbox{\lstinline'k'}} \Zclear \semicolon ) \cdot
%%       (l, \underline{\mbox{\lstinline'e1'}} \Zclear \semicolon ) \cdot
%%       (l, \underline{\mbox{\lstinline'e2'}} \Zclear \semicolon )$}{
%%       (\comp{\lstinline'int e = 1;'}{(\compi{$I_2$}{(\compi{$I_1$}{$\env$})})})
%%     }
%%     \\
%%     &$\eq{\eqlabel{C-set}}$&
%%     \compi{$(l, \Zinit \underline{\mbox{\lstinline'k = e1;'}} ) \cdot
%%       (l, \mbox{\lstinline'while('}
%%       \underline{\mbox{\lstinline'k <= e2'}}~ \mbox{\lstinline'&& e)'}
%%       \bopen I_3 \cdot (l, \mbox{\lstinline'e = e3;'}) \cdot
%%       (l, \underline{\mbox{\lstinline'k++'}} \semicolon) \bclose ) \cdot
%%       (l, \underline{\mbox{\lstinline'k'}} \Zclear \semicolon ) \cdot
%%       (l, \underline{\mbox{\lstinline'e1'}} \Zclear \semicolon ) \cdot
%%       (l, \underline{\mbox{\lstinline'e2'}} \Zclear \semicolon )$}{
%%       ((\compi{$I_2$}{(\compi{$I_1$}{$\env$})})[\lstinline'e' $\mapsto$ 1])
%%     }
%%     \\
%%     &$\eq{\eqlabel{I-split}}$&
%%     \compi{$(l, \mbox{\lstinline'while('}
%%       \underline{\mbox{\lstinline'k <= e2'}}~ \mbox{\lstinline'&& e)'}
%%       \bopen I_3 \cdot (l, \mbox{\lstinline'e = e3;'}) \cdot
%%       (l, \underline{\mbox{\lstinline'k++'}} \semicolon) \bclose ) \cdot
%%       (l, \underline{\mbox{\lstinline'k'}} \Zclear \semicolon ) \cdot
%%       (l, \underline{\mbox{\lstinline'e1'}} \Zclear \semicolon ) \cdot
%%       (l, \underline{\mbox{\lstinline'e2'}} \Zclear \semicolon )$}{
%%       (\compi{$(l, \Zinit \underline{\mbox{\lstinline'k = e1;'}} )$}{
%%         ((\compi{$I_2$}{(\compi{$I_1$}{$\env$})})[\lstinline'e' $\mapsto$ 1])})
%%     }
%%     \\
%%     &$\eq{\eqlabel{I-instr}}$&
%%     \compi{$(l, \mbox{\lstinline'while('}
%%       \underline{\mbox{\lstinline'k <= e2'}}~ \mbox{\lstinline'&& e)'}
%%       \bopen I_3 \cdot (l, \mbox{\lstinline'e = e3;'}) \cdot
%%       (l, \underline{\mbox{\lstinline'k++'}} \semicolon) \bclose ) \cdot
%%       (l, \underline{\mbox{\lstinline'k'}} \Zclear \semicolon ) \cdot
%%       (l, \underline{\mbox{\lstinline'e1'}} \Zclear \semicolon ) \cdot
%%       (l, \underline{\mbox{\lstinline'e2'}} \Zclear \semicolon )$}{
%%       (\comp{$\Zinit \underline{\mbox{\lstinline'k = e1;'}}$}{
%%         ((\compi{$I_2$}{(\compi{$I_1$}{$\env$})})[\lstinline'e' $\mapsto$ 1])})
%%     }
%%     \\
%%     &$\eq{\eqlabel{C-Z-set}}$&
%%     \compi{$(l, \mbox{\lstinline'while('}
%%       \underline{\mbox{\lstinline'k <= e2'}}~ \mbox{\lstinline'&& e)'}
%%       \bopen I_3 \cdot (l, \mbox{\lstinline'e = e3;'}) \cdot
%%       (l, \underline{\mbox{\lstinline'k++'}} \semicolon) \bclose ) \cdot
%%       (l, \underline{\mbox{\lstinline'k'}} \Zclear \semicolon ) \cdot
%%       (l, \underline{\mbox{\lstinline'e1'}} \Zclear \semicolon ) \cdot
%%       (l, \underline{\mbox{\lstinline'e2'}} \Zclear \semicolon )$}{
%%       ((\compi{$I_2$}{(\compi{$I_1$}{$\env$})})[\lstinline'e' $\mapsto$ 1,
%%         \lstinline'k' $\mapsto$ \eval{\lstinline't1'}{$\env$}])
%%     }
%%     \\
%%     &$\eq{\eqlabel{I-split}}$&
%%     \compi{$(l, \underline{\mbox{\lstinline'k'}} \Zclear \semicolon ) \cdot
%%       (l, \underline{\mbox{\lstinline'e1'}} \Zclear \semicolon ) \cdot
%%       (l, \underline{\mbox{\lstinline'e2'}} \Zclear \semicolon )$}{
%%       (\compi{$(l, \mbox{\lstinline'while('}
%%         \underline{\mbox{\lstinline'k <= e2'}}~ \mbox{\lstinline'&& e)'}
%%         \bopen I_3 \cdot (l, \mbox{\lstinline'e = e3;'}) \cdot
%%         (l, \underline{\mbox{\lstinline'k++'}} \semicolon) \bclose )$}{
%%         ((\compi{$I_2$}{(\compi{$I_1$}{$\env$})})[\lstinline'e' $\mapsto$ 1,
%%           \lstinline'k' $\mapsto$ \eval{\lstinline't1'}{$\env$}])
%%       })
%%     }
%%     \\
%%     &$\eq{\eqlabel{I-instr}}$&
%%     \compi{$(l, \underline{\mbox{\lstinline'k'}} \Zclear \semicolon ) \cdot
%%       (l, \underline{\mbox{\lstinline'e1'}} \Zclear \semicolon ) \cdot
%%       (l, \underline{\mbox{\lstinline'e2'}} \Zclear \semicolon )$}{
%%       (\comp{$\mbox{\lstinline'while('}
%%         \underline{\mbox{\lstinline'k <= e2'}}~ \mbox{\lstinline'&& e)'}
%%         \bopen I_3 \cdot (l, \mbox{\lstinline'e = e3;'}) \cdot
%%         (l, \underline{\mbox{\lstinline'k++'}} \semicolon) \bclose$}{
%%         ((\compi{$I_2$}{(\compi{$I_1$}{$\env$})})[\lstinline'e' $\mapsto$ 1,
%%           \lstinline'k' $\mapsto$ \eval{\lstinline't1'}{$\env$}])
%%       })
%%     }
%%     \\
%%   \end{tabular}

%%   (1) Cas \eval{\lstinline't1'}{$\env$} $>$ \eval{\lstinline't2'}{$\env$}
%%   (nouvelle hypothèse \eqlabel{rel1}) :

%%   \begin{tabular}{p{0cm} p{.5cm} p{14cm}}
%%     &$\eq{\eqlabel{C-while}}$&
%%     \compi{$(l, \underline{\mbox{\lstinline'k'}} \Zclear \semicolon ) \cdot
%%       (l, \underline{\mbox{\lstinline'e1'}} \Zclear \semicolon ) \cdot
%%       (l, \underline{\mbox{\lstinline'e2'}} \Zclear \semicolon )$}{
%%       ((\compi{$I_2$}{(\compi{$I_1$}{$\env$})})[\lstinline'e' $\mapsto$ 1,
%%         \lstinline'k' $\mapsto$ \eval{\lstinline't1'}{$\env$}])
%%     }
%%     \\
%%     &$\eq{\eqlabel{I-split}}$&
%%     \compi{$(l, \underline{\mbox{\lstinline'e1'}} \Zclear \semicolon ) \cdot
%%       (l, \underline{\mbox{\lstinline'e2'}} \Zclear \semicolon )$}{
%%       (\compi{$(l, \underline{\mbox{\lstinline'k'}} \Zclear \semicolon )$}{
%%         ((\compi{$I_2$}{(\compi{$I_1$}{$\env$})})[\lstinline'e' $\mapsto$ 1,
%%           \lstinline'k' $\mapsto$ \eval{\lstinline't1'}{$\env$}])
%%       })
%%     }
%%     \\
%%     &$\eq{\eqlabel{I-instr}}$&
%%     \compi{$(l, \underline{\mbox{\lstinline'e1'}} \Zclear \semicolon ) \cdot
%%       (l, \underline{\mbox{\lstinline'e2'}} \Zclear \semicolon )$}{
%%       (\comp{$\underline{\mbox{\lstinline'k'}} \Zclear \semicolon$}{
%%         ((\compi{$I_2$}{(\compi{$I_1$}{$\env$})})[\lstinline'e' $\mapsto$ 1,
%%           \lstinline'k' $\mapsto$ \eval{\lstinline't1'}{$\env$}])
%%       })
%%     }
%%     \\
%%     &$\eq{\eqlabel{C-Z-unset}}$&
%%     \compi{$(l, \underline{\mbox{\lstinline'e1'}} \Zclear \semicolon ) \cdot
%%       (l, \underline{\mbox{\lstinline'e2'}} \Zclear \semicolon )$}{
%%       ((\compi{$I_2$}{(\compi{$I_1$}{$\env$})})[\lstinline'e' $\mapsto$ 1])
%%     }
%%     \\
%%     &$\eq{\eqlabel{I-split}}$&
%%     \compi{$(l, \underline{\mbox{\lstinline'e2'}} \Zclear \semicolon )$}{(
%%       \compi{$(l, \underline{\mbox{\lstinline'e1'}} \Zclear \semicolon )$}{
%%         ((\compi{$I_2$}{(\compi{$I_1$}{$\env$})})[\lstinline'e' $\mapsto$ 1])
%%       })
%%     }
%%     \\
%%     &$\eq{\eqlabel{I-instr}}$&
%%     \compi{$(l, \underline{\mbox{\lstinline'e2'}} \Zclear \semicolon )$}{(
%%       \comp{$\underline{\mbox{\lstinline'e1'}} \Zclear \semicolon$}{
%%         ((\compi{$I_2$}{(\compi{$I_1$}{$\env$})})[\lstinline'e' $\mapsto$ 1])
%%       })
%%     }
%%     \\
%%     &$\eq{\eqlabel{C-Z-unset}}$&
%%     \compi{$(l, \underline{\mbox{\lstinline'e2'}} \Zclear \semicolon )$}{(
%%       ((\compi{$I_2$}{(\compi{$I_1$}{$\env$})})[\lstinline'e' $\mapsto$ 1])
%%       $-~\{\mbox{\lstinline'e1'}\}$)
%%     }
%%     \\
%%     &$\eq{\eqlabel{I-instr}}$&
%%     \comp{$\underline{\mbox{\lstinline'e2'}} \Zclear \semicolon$}{(
%%       ((\compi{$I_2$}{(\compi{$I_1$}{$\env$})})[\lstinline'e' $\mapsto$ 1])
%%       $-~\{\mbox{\lstinline'e1'}\}$)
%%     }
%%     \\
%%     &$\eq{\eqlabel{C-Z-unset}}$&
%%       ((\compi{$I_2$}{(\compi{$I_1$}{$\env$})})[\lstinline'e' $\mapsto$ 1])
%%       $-~\{\mbox{\lstinline'e1',\lstinline'e2'}\}$
%%     \\
%%   \end{tabular}

%%   En évaluant \lstinline'e' dans ce nouvel environnement, on obtient :

%%   \begin{tabular}{rcl}
%%     \eval{\lstinline'e'}{
%%       (((\compi{$I_2$}{(\compi{$I_1$}{$\env$})})[\lstinline'e' $\mapsto$ 1])
%%       $-~\{\mbox{\lstinline'e1',\lstinline'e2'}\}$)}
%%     & $\eq{\eqlabel{E-lval}}$ & 1 \\
%%     & = & \lstinline'\forall integer k; \false ==> p' \\
%%     & $\eq{\eqlabel{rel1}}$
%%     & \lstinline'\forall integer k; t1 <= k <= t2 ==> p' \\
%%   \end{tabular}

%%   D'après les hypothèses \eqlabel{h4}, \eqlabel{h5} et le fait que les seules
%%   variables modifiées ont été générées par l'instrumentation (\lstinline'e',
%%   \lstinline'e1' et \lstinline'e2'), on a :

%%   $\env$ \subenv
%%   (((\compi{$I_2$}{(\compi{$I_1$}{$\env$})})[\lstinline'e' $\mapsto$ 1])
%%   $-~\{\mbox{\lstinline'e1',\lstinline'e2'}\}$).

%%   Les hypothèses \eqlabel{h7} et \eqlabel{h8} nous permettent de déduire que :

%%   (((\compi{$I_2$}{(\compi{$I_1$}{$\env$})})[\lstinline'e' $\mapsto$ 1])
%%   $-~\{\mbox{\lstinline'e1',\lstinline'e2'}\}$) $\neq$ $\errormem$.

%%   (2) Cas \eval{\lstinline't1'}{$\env$} $\le$ \eval{\lstinline't2'}{$\env$}
%%   (nouvelle hypothèse \eqlabel{rel2}) :

%%   Pour calculer le nouvel environnement résultant de l'exécution de la boucle
%%   nous avons besoin de deux invariants de boucle :

%%   \begin{tabular}{rclr}
%%     \multicolumn{3}{c}{
%%       \eval{\lstinline't1'}{$\env$} $\le$ \eval{\lstinline'k'}{$\env$} $\le$
%%       \eval{\lstinline't2'}{$\env$} + 1
%%     }
%%     & \eqlabel{inv-1} \\
%%     \eval{\lstinline'e'}{$\env$} &=&
%%     \eval{\lstinline'\\forall integer z; t1 <= z < k ==> p'}{$\env$}
%%     & \eqlabel{inv-2} \\
%%   \end{tabular}

%%   Supposons que ces propriétés ont été prouvées.
%%   \commentGP{Preuve longue. Pas nécessaire ?}
%%   Notons $\env$$'$ l'environnement après exécution de la boucle :

%%   $\env$$'$ = 
%%   \comp{$\mbox{\lstinline'while('}
%%     \underline{\mbox{\lstinline'k <= e2'}}~ \mbox{\lstinline'&& e)'}
%%     \bopen I_3 \cdot (l, \mbox{\lstinline'e = e3;'}) \cdot
%%     (l, \underline{\mbox{\lstinline'k++'}} \semicolon) \bclose$}{
%%     ((\compi{$I_2$}{(\compi{$I_1$}{$\env$})})[\lstinline'e' $\mapsto$ 1,
%%       \lstinline'k' $\mapsto$ \eval{\lstinline't1'}{$\env$}])
%%   }

%%   \begin{tabular}{p{3cm} p{.5cm} p{11.5cm}}
%%     \eval{\lstinline'e'}{\compi{$I$}{$\env$}}
%%     &=&
%%     \eval{\lstinline'e'}{
%%       (\compi{$(l, \underline{\mbox{\lstinline'k'}} \Zclear \semicolon ) \cdot
%%         (l, \underline{\mbox{\lstinline'e1'}} \Zclear \semicolon ) \cdot
%%         (l, \underline{\mbox{\lstinline'e2'}} \Zclear \semicolon )$}{
%%         (\compi{$(l, \mbox{\lstinline'while('}
%%           \underline{\mbox{\lstinline'k <= e2'}}~ \mbox{\lstinline'&& e)'}
%%           \bopen I_3 \cdot (l, \mbox{\lstinline'e = e3;'}) \cdot
%%           (l, \underline{\mbox{\lstinline'k++'}} \semicolon) \bclose )$}{
%%           ((\compi{$I_2$}{(\compi{$I_1$}{$\env$})})[\lstinline'e' $\mapsto$ 1,
%%             \lstinline'k' $\mapsto$ \eval{\lstinline't1'}{$\env$}])
%%         })
%%       })
%%     } \\
%%     &=&
%%     \eval{\lstinline'e'}{
%%         (\compi{$(l, \mbox{\lstinline'while('}
%%           \underline{\mbox{\lstinline'k <= e2'}}~ \mbox{\lstinline'&& e)'}
%%           \bopen I_3 \cdot (l, \mbox{\lstinline'e = e3;'}) \cdot
%%           (l, \underline{\mbox{\lstinline'k++'}} \semicolon) \bclose )$}{
%%           ((\compi{$I_2$}{(\compi{$I_1$}{$\env$})})[\lstinline'e' $\mapsto$ 1,
%%             \lstinline'k' $\mapsto$ \eval{\lstinline't1'}{$\env$}])
%%         })
%%     } \\
%%     &=& \eval{\lstinline'e'}{$\env$$'$} \\
%%     &=& \eval{\lstinline'\\forall integer z; t1 <= z < k ==> p'}{$\env$} \\
%%   \end{tabular}

%%   Rappelons que
%%   \eval{\lstinline't1'}{$\env$$'$} $\le$ \eval{\lstinline'k'}{$\env$$'$} $\le$
%%   \eval{\lstinline't2'}{$\env$$'$} + 1
%%   (premier invariant de boucle).

%%   On sait que la condition de boucle est fausse dans cet environnement $\env$$'$ :

%%   \eval{\lstinline'k <= e2 && e'}{$\env$$'$} = \textit{false}

%%   Il y a donc deux cas non exclusifs :
%%   \begin{itemize}
%%   \item[$\bullet$] \eval{\lstinline'k <= e2'}{$\env$$'$} = \textit{false}
%%   \item[$\bullet$] \eval{\lstinline'e'}{$\env$$'$} = \textit{false}
%%   \end{itemize}

%%   (2-a) Cas \eval{\lstinline'k <= e2'}{$\env$$'$} = \textit{false}, soit
%%   \eval{\lstinline'k > t2'}{$\env$$'$} = \textit{true} :
  
%%   On sait que \eval{\lstinline'k > t2'}{$\env$$'$} et
%%   \eval{\lstinline'k <= t2+1'}{$\env$$'$},
%%   donc on
%%   a \eval{\lstinline'k'}{$\env$$'$} = \eval{\lstinline't2+1'}{$\env$$'$}.
%%   Avec cette nouvelle connaissance, on a :

%%   \begin{tabular}{p{3cm} p{.5cm} p{11.5cm}}
%%     \eval{\lstinline'e'}{\compi{$I$}{$\env$}}
%%     &=& \eval{\lstinline'\\forall integer z; t1 <= z < k ==> p'}{$\env$} \\
%%     &=& \eval{\lstinline'\\forall integer z; t1 <= z < t2+1 ==> p'}{$\env$} \\
%%     &=& \eval{\lstinline'\\forall integer z; t1 <= z <= t2 ==> p'}{$\env$} \\
%%   \end{tabular}

%%   (2-b) Cas \eval{$e$}{$\env$$'$} = \textit{false} :

%%   \begin{tabular}{p{3cm} p{.5cm} p{11.5cm}}
%%     \eval{\lstinline'e'}{$\env$$'$} = \textit{false}
%%     &$\equiv$&
%%     \eval{\lstinline'\\forall integer z; t1 <= z < k ==> p'}{$\env$$'$}
%%     = \textit{false} \\
%%     &$\equiv$&
%%     \eval{\lstinline'\\forall integer z; t1 <= z < t2+1 ==> p'}{$\env$$'$}
%%     = \textit{false} \\
%%     \multicolumn{3}{c}{car \eval{\lstinline'k'}{$\env$$'$} $\le$
%%       \eval{\lstinline't2+1'}{$\env$$'$}} \\
%%     &$\equiv$&
%%     \eval{\lstinline'\\forall integer z; t1 <= z <= t2 ==> p'}{$\env$$'$}
%%     = \textit{false} \\
%%     &$\equiv$&
%%     \eval{\lstinline'\\forall integer z; t1 <= z <= t2 ==> p'}{$\env$}
%%     = \textit{false} \\
%%     \multicolumn{3}{c}{
%%       car (\eval{\lstinline't1'}{$\env$}) = (\eval{\lstinline't1'}{$\env$$'$}) et
%%       (\eval{\lstinline't2'}{$\env$}) = (\eval{\lstinline't2'}{$\env$$'$})
%%     } \\
%%   \end{tabular}

%%   Donc on a :
%%   \eval{\lstinline'e'}{$\env$$'$} $\equiv$
%%   \eval{\lstinline'\\forall integer z; t1 <= z <= t2 ==> p'}{$\env$}.

%%   On a donc aussi :

%%   \eval{\lstinline'e'}{\compi{$I$}{$\env$}} =
%%   \eval{\lstinline'\\forall integer z; t1 <= z <= t2 ==> p'}{$\env$}.

%%   Grâce aux hypothèses \eqlabel{h4}, \eqlabel{h5}, \eqlabel{h6} et le fait que
%%   les seules variables affectées sont \lstinline'e' et \lstinline'k', qui
%%   n'appartiennent pas à l'environnement initial $\env$, on peut établir :
%%   $\env$ \subenv \compi{$I$}{$\env$}.

%%   De manière similaire, grâce aux hypothèses \eqlabel{h7}, \eqlabel{h8},
%%   \eqlabel{h9}, et le fait que les fragments de code générés de provoquent pas
%%   d'erreur, le nouvel environnement est différent de $\errormem$.
%% \end{proof}



%% \newpage

%% \begin{proof}
%%   ~\\
%%   \commentGP{à mettre en annexe ou pas ?}
%%   \commentGP{je pense que la preuve n'est pas nécessaire}

%%   Montrons que \eqlabel{inv-1} et \eqlabel{inv-2} sont vrais pour tout
%%   environnement $\env$.
%%   Montrons tout d'abord que ces invariants sont établis avant la première
%%   itération.

%%   Développons
%%   \eval{$t_1$}{$\env$} $\le$ \eval{$k$}{$\env$} $\le$ \eval{$t_2$}{$\env$} + 1 :

%%   \begin{tabular}{rcl}
%%     \eval{$t_1$}{$\env$} $\le$ \eval{$k$}{$\env$} $\le$ \eval{$t_2$}{$\env$} + 1
%%     &=& \eval{$t_1$}{$\env$} $\le$ \eval{$k$}{$\env$}
%%     $\le$ \eval{$t_2$}{$\env$} + 1 \\
%%     &=& \eval{$t_1$}{$\env$} $\le$ \eval{$t_2$}{$\env$} + 1 \\
%%     &$\eq{rel2}$& \textit{true} \\
%%   \end{tabular}

%%   Développons le membre droit de

%%   \eval{$e$}{$\env$} =
%%   \eval{\lstinline'\\forall integer z; t_1 <= z < k ==> p'}{$\env$} :

%%   \begin{tabular}{rcl}
%%     \multicolumn{3}{l}{
%%       \eval{\lstinline'\\forall integer z; t_1 <= z < k ==> p'}{$\env$}
%%     } \\
%%     &=& \eval{\lstinline'\\forall integer z; t_1 <= z <= k-1 ==> p'}{$\env$} \\
%%     &=& \eval{\lstinline'\\forall integer z; t_1 <= z <= t_1-1 ==> p'}{$\env$} \\
%%     &=& \eval{\lstinline'\\forall integer z; \\false ==> p'}{$\env$} \\
%%     &=& \eval{\lstinline'\\true'}{$\env$} \\
%%     &=& 1 \\
%%     &=& \eval{$e$}{$\env$} \\
%%   \end{tabular}

%%   Ces deux invariants sont établis avant la première itération, montrons qu'ils
%%   sont maintenus après chaque itération.
%%   On suppose donc que ces invariants sont vrais après une itération quelconque
%%   et sont donc des hypothèses.
%%   Soit $\env$ l'environnement au début de l'itération.

%%   Prouvons que le premier invariant est maintenu après chaque itération :

%%   On a : \eval{$t_1$}{$\env$} $\le$ \eval{$k$}{$\env$} $\le$ \eval{$t_2$}{$\env$} + 1
  
%%   Ce qui peut être réécrit en :
%%   \eval{$t_1 \le k \le t_2 + 1$}{$\env$} = \textit{true}

%%   Montrons :

%%   \eval{$t_1 \le k \le t_2 + 1$}{\compi{
%%       $I_3 \cdot (l, \mbox{\lstinline'e = e3;'}) \cdot
%%       (l, \underline{\mbox{\lstinline'k++'}} \semicolon)$
%%     }{
%%       ($\env$[$e_1 \mapsto$ \eval{$t_1$}{$\env$},
%%         $e_2 \mapsto$ \eval{$t_2$}{$\env$},
%%         $e \mapsto$
%%         \eval{\lstinline'\\forall integer z; t_1 <= z < k ==> p'}{$\env$}])
%%     }} = \textit{true}

%%   Développons le membre gauche :

%%   \begin{tabular}{p{6cm} p{.5cm} p{9cm}}
%%     \eval{$t_1 \le k \le t_2 + 1$}{\comp{
%%       $I_3 \cdot \mbox{\lstinline'e = e_3;'}
%%       \cdot \underline{\mbox{\lstinline'k++'}} \semicolon$
%%     }{
%%       ($\env$[$e_1 \mapsto$ \eval{$t_1$}{$\env$},
%%         $e_2 \mapsto$ \eval{$t_2$}{$\env$},
%%         $e \mapsto$
%%         \eval{\lstinline'\\forall integer z; t_1 <= z < k ==> p'}{$\env$}])
%%     }}
%%     &=&
%%     \eval{$t_1 \le k \le t_2 + 1$}{
%%       $\env$[$e_1 \mapsto$ \eval{$t_1$}{$\env$},
%%         $e_2 \mapsto$ \eval{$t_2$}{$\env$},
%%         $e_3 \mapsto$ \eval{\lstinline'p'}{$\env$},
%%         $e \mapsto$ \eval{\lstinline'p'}{$\env$},
%%         $k \mapsto$ (\eval{\lstinline'k'}{$\env$})+1]
%%     } \\
%%     &=&
%%     \eval{$t_1 \le k+1 \le t_2 + 1$}{
%%       $\env$[$e_1 \mapsto$ \eval{$t_1$}{$\env$},
%%         $e_2 \mapsto$ \eval{$t_2$}{$\env$},
%%         $e_3 \mapsto$ \eval{\lstinline'p'}{$\env$},
%%         $e \mapsto$ \eval{\lstinline'p'}{$\env$},
%%         $k \mapsto$ (\eval{\lstinline'k'}{$\env$})]
%%     } \\
%%     &=& \eval{$t_1 \le k+1 \le t_2 + 1$}{$\env$} \\
%%     &=& \eval{$t_1 \le k+1$}{$\env$} $\land$ \eval{$k+1 \le t_2 + 1$}{$\env$} \\
%%     &=& \eval{$t_1 \le k+1$}{$\env$} $\land$ \eval{$k \le t_2$}{$\env$} \\
%%     &=& \textit{true} $\land$ \eval{$k \le t_2$}{$\env$} \\
%%     \multicolumn{3}{c}{par hypothèse de récurrence} \\
%%     &=& \eval{$k \le t_2$}{$\env$} \\
%%     &=& \eval{$k$}{$\env$} $\le$ \eval{$t_2$}{$\env$} \\
%%     &=& \eval{$k$}{$\env$} $\le$ \eval{$e_2$}{$\env$} \\
%%     &=& \eval{$k \le e_2$}{$\env$} \\
%%     &=& \textit{true} \\
%%     \multicolumn{3}{c}{la condition de boucle est vraie sinon on n'exécuterait
%%       pas le code de la boucle} \\
%%   \end{tabular}
  
%%   Le premier invariant est donc maintenu après chaque itération.
%%   Prouvons maintenant que le deuxième invariant est maintenu après chaque
%%   itération.
%%   Développons le membre gauche :

%%   \begin{tabular}{p{6cm} p{.5cm} p{9cm}}
%%     (\comp{
%%       $I_3 \cdot \mbox{\lstinline'e = e_3;'}
%%       \cdot \underline{\mbox{\lstinline'k++'}} \semicolon$
%%     }{
%%       ($\env$[$e_1 \mapsto$ \eval{$t_1$}{$\env$},
%%         $e_2 \mapsto$ \eval{$t_2$}{$\env$},
%%         $e \mapsto$
%%         \eval{\lstinline'\\forall integer z; t_1 <= z < k ==> p'}{$\env$}])
%%     })(e)
%%     &=&
%%     (\comp{$\underline{\mbox{\lstinline'k++'}} \semicolon$}{(
%%       \comp{\lstinline'e = e_3;'}{(
%%         \comp{$I_3$}{(
%%           $\env$[$e_1 \mapsto$ \eval{$t_1$}{$\env$},
%%             $e_2 \mapsto$ \eval{$t_2$}{$\env$},
%%             $e \mapsto$
%%             \eval{\lstinline'\\forall integer z; t_1 <= z < k ==> p'}{$\env$}]
%%           )}
%%         )}
%%       )}
%%     )(e) \\
%%     &=&
%%     (\comp{$\underline{\mbox{\lstinline'k++'}} \semicolon$}{(
%%       \comp{\lstinline'e = e_3;'}{(
%%         $\env$[$e_1 \mapsto$ \eval{$t_1$}{$\env$},
%%           $e_2 \mapsto$ \eval{$t_2$}{$\env$},
%%           $e \mapsto$
%%           \eval{\lstinline'\\forall integer z; t_1 <= z < k ==> p'}{$\env$},
%%           $e_3 \mapsto$ \eval{\lstinline'p'}{$\env$}]
%%         )}
%%       )}
%%     )(e) \\
%%     &=&
%%     (\comp{$\underline{\mbox{\lstinline'k++'}} \semicolon$}{(
%%       $\env$[$e_1 \mapsto$ \eval{$t_1$}{$\env$},
%%         $e_2 \mapsto$ \eval{$t_2$}{$\env$},
%%         $e_3 \mapsto$ \eval{\lstinline'p'}{$\env$},
%%         $e \mapsto$ \eval{\lstinline'p'}{$\env$}]
%%       )}
%%     )(e) \\
%%     &=&
%%     ($\env$[$e_1 \mapsto$ \eval{$t_1$}{$\env$},
%%       $e_2 \mapsto$ \eval{$t_2$}{$\env$},
%%       $e_3 \mapsto$ \eval{\lstinline'p'}{$\env$},
%%       $e \mapsto$ \eval{\lstinline'p'}{$\env$},
%%       $k \mapsto$ (\eval{\lstinline'k'}{$\env$})+1]
%%     )(e) \\
%%     &=& \eval{\lstinline'p'}{$\env$} \\
%%     &=& \eval{\lstinline'e'}{$\env$} $\land$ \eval{\lstinline'p'}{$\env$} \\
%%     \multicolumn{3}{c}{\eval{\lstinline'e'}{$\env$} est vrai sinon on
%%       n'exécuterait pas le code de la boucle} \\
%%   \end{tabular}

%%   Développons le membre droit :

%%   \begin{tabular}{p{5cm} p{.5cm} p{9cm}}
%%     \eval{\lstinline'\\forall integer z; t_1 <= z < k ==> p'}{
%%       (\comp{
%%         $I_3 \cdot \mbox{\lstinline'e = e_3;'}
%%         \cdot \underline{\mbox{\lstinline'k++'}} \semicolon$
%%       }{
%%         ($\env$[$e_1 \mapsto$ \eval{$t_1$}{$\env$},
%%           $e_2 \mapsto$ \eval{$t_2$}{$\env$},
%%           $e \mapsto$
%%           \eval{\lstinline'\\forall integer z; t_1 <= z < k ==> p'}{$\env$}])
%%       })
%%     }
%%     &=&
%%     \eval{\lstinline'\\forall integer z; t_1 <= z < k ==> p'}{
%%       ($\env$[$e_1 \mapsto$ \eval{$t_1$}{$\env$},
%%         $e_2 \mapsto$ \eval{$t_2$}{$\env$},
%%         $e_3 \mapsto$ \eval{\lstinline'p'}{$\env$},
%%         $e \mapsto$ \eval{\lstinline'p'}{$\env$},
%%         $k \mapsto$ (\eval{\lstinline'k'}{$\env$})+1])
%%     } \\
%%     &=& \eval{\lstinline'\\forall integer z; t_1 <= z < k+1 ==> p'}{$\env$} \\
%%     &=& \eval{\lstinline'\\forall integer z; t_1 <= z <= k ==> p'}{$\env$} \\
%%     &=&
%%     \eval{\lstinline'\\forall integer z; t_1 <= z < k ==> p'}{$\env$} \newline
%%     $\land$ \eval{\lstinline'\\forall integer z; z == k ==> p'}{$\env$} \\
%%     &=&
%%     \eval{\lstinline'e'}{$\env$}
%%     $\land$ \eval{\lstinline'\\forall integer z; z == k ==> p'}{$\env$} \\
%%     &=& \eval{\lstinline'e'}{$\env$} $\land$ \eval{\lstinline'p'}{$\env$} \\
%%     \multicolumn{3}{c}{car \eval{\lstinline'p'}{$\env$} est la valeur de $p$ pour
%%     le $k$ courant} \\
%%   \end{tabular}

%%   Le deuxième invariant est donc lui aussi maintenu après chaque itération.
%% \end{proof}
