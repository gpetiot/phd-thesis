
\chapter{Programmes C annotés avec E-ACSL}
\label{sec:lang}

\chapterintro


Ce chapitre présente le sous-ensemble du langage C et du langage de
spécification \eacsl que nous traitons.
Dans la partie~\ref{sec:lang-syntax} nous présentons la syntaxe du langage
illustrée par un exemple.
C'est la syntaxe normalisée de programmes C annotés par des clauses E-ACSL.
Dans la partie~\ref{sec:lang-semantics} nous définissons la notion de mémoire
ainsi que les fonctions permettant de décrire la sémantique dénotationnelle de
chaque élément du langage.


\section{Syntaxe des programmes annotés}
\label{sec:lang-syntax}


Le langage de spécification \eacsl est un sous-langage exécutable d'\acsl, qui
est un langage de spécification comportementale implémenté dans \framac.
Premièrement, étant un sous-ensemble d'\acsl, \eacsl préserve la sémantique
d'\acsl, ce qui implique que les greffons de \framac existants supportant \acsl
peuvent être utilisés avec \eacsl sans modification.
Deuxièmement, le langage \eacsl est \textit{exécutable}, ce qui veut dire que
toutes les annotations peuvent être traduites en C et leur traduction peut être
exécutée pour l'analyse dynamique et le \textit{monitoring} (surveillance
d'exécution).


\subsection{Exemple de fonction C normalisée}
\label{sec:translation-example}


Nous présentons dans le listing~\ref{lst:normalized} un exemple de fonction
normalisée en suivant la grammaire de la figure~\ref{fig:gram-c}.
Le terme \lstinline'\old' fait référence à la valeur du terme au début de la
fonction courante.
\lstinline{\result} fait référence à la valeur retournée par la fonction.
La clause \lstinline'ensures' aux lignes 4--5 indique qu'elle retourne une
valeur différente de zéro quand la valeur \lstinline'v' en paramètre est
présente dans le tableau \lstinline't' de longueur \lstinline'n' en paramètre,
et $0$ sinon.

La clause \lstinline'requires' de la ligne 1 indique que la taille \lstinline'n'
du tableau est positive ou nulle.
La clause \lstinline'requires' de la ligne 2 indique que \lstinline'(t+0)',
\ldots, \lstinline'(t+(n-1))' sont des pointeurs valides.
La clause \lstinline'typically' de la ligne 3 est une extension de la syntaxe
d'\eacsl définissant une précondition pour le test.
Elle renforce la précondition pour limiter l'explosion combinatoire du nombre de
chemins à explorer par la génération de tests.
Elle permet d'obtenir une couverture partielle des chemins : seuls les chemins
satisfaisant cette clause sont considérés.
Ici elle limite l'espace des valeurs de \lstinline'n' (et donc la taille de
\lstinline't') à $[0, 6]$.
Cette réduction de l'espace des états d'entrée peut être vue comme une
``finitization'' \cite{\citekorat}).

\begin{figure}
\lstinputlisting[style=c,escapechar=§,caption={Fonction C annotée décidant si
    \lstinline{v} est présent dans le tableau \lstinline{t} de taille
    \lstinline{n}},
  label=lst:normalized]{listings/is_present_normalized.c}
\end{figure}

L'invariant de boucle de la ligne 9 spécifie des bornes pour \lstinline'i'
sur l'ensemble des itérations de la boucle.
Un variant de boucle permet de s'assurer de la terminaison d'une boucle : le
terme du variant doit être positif ou nul avant chaque itération et doit
décroître strictement entre chaque itération.
Ainsi, le variant de boucle de la ligne 10 spécifie que la boucle pourra
s'exécuter au plus \lstinline'n-i' fois.
Notons que pour prouver formellement la postcondition, l'invariant de boucle
suivant est nécessaire, mais n'est pas inclus volontairement dans cet exemple
simplifié : \lstinline'\forall integer k; 0<=k<=i-1 ==> \old(t[k])!=v;'.
Les clauses \lstinline'assigns' et \lstinline'loop assigns' ont volontairement
été omises de cet exemple simplifié. \todo{assigns}


\subsection{Syntaxe des instructions et fonctions}

\begin{figure}[h!]
  \begin{tabular}{lrl}
    \textit{basic-type} & ::= & \underline{\lstinline'int'}
    $\mid$ \underline{\lstinline'char'}
    $\mid$ \underline{\lstinline'long'} \\
    \textit{type} & ::= & \textit{basic-type}
    $\mid$ \textit{basic-type} \underline{\lstinline'*'} \\
    \textit{left-value} & ::= & \textit{id}
    $\mid$  \textit{id} \underline{\lstinline'['} \textit{exp}
      \underline{\lstinline']'} \\
    \textit{instr} & ::= & \textit{type} \textit{id} \underline{\semicolon} \\
    & $\mid$ & \textit{left-value} \underline{\lstinline'='} \textit{exp}
    \underline{\semicolon} \\
    & $\mid$ & \textit{id} \underline{\lstinline'('} \textit{exp}$^{*}_,$
    \underline{\lstinline');'} \\
    & $\mid$ & \textit{left-value} \underline{\lstinline'='} \textit{id}
    \underline{\lstinline'('} \textit{exp}$^{*}_,$ \underline{\lstinline');'} \\
    & $\mid$ & \underline{\lstinline'if'} \underline{\lstinline'('} \textit{exp}
    \underline{\lstinline')'}
    \underline{$\bopen$} \textit{instr}$^{*}$ \underline{$\bclose$}
    \underline{\lstinline'else'} \underline{$\bopen$} \textit{instr}$^{*}$
    \underline{$\bclose$} \\
    & $\mid$ & \textit{label} \underline{\lstinline'\:'}
    \underline{\lstinline'/*@'} \underline{\lstinline'assert'} \textit{pred}
    \underline{\lstinline'\;'} \underline{\lstinline'*/'} \\
    & $\mid$ & \textit{label}$^{\textit{as}\;l}$ \underline{\lstinline'\:'}
    \underline{\lstinline'/*@'} \underline{\lstinline'loop invariant'} \textit{pred}
    \underline{\semicolon} \\
    &        & \underline{\lstinline'    loop assigns'} \textit{term}$^{+}_,$
    \underline{\semicolon} \\
    &        & \underline{\lstinline'    loop variant'} \textit{term}
    \underline{\lstinline'\;'} \underline{\lstinline'*/'} \\
    &        & \underline{\lstinline'while'} \underline{\lstinline'('} \textit{exp}
    \underline{\lstinline')'}
    \underline{$\bopen$}
    \textit{BegIter}$_l$ \underline{\lstinline'\:'} \underline{\semicolon}
    \textit{instr}$^{*}$
    \textit{EndIter}$_l$ \underline{\lstinline'\:'} \underline{\semicolon}
    \underline{$\bclose$} \\
    \textit{function} & ::= & \underline{\lstinline'/*@'} \underline{\lstinline'requires'}
    \textit{pred} \underline{\semicolon} \\
    &     & \underline{\lstinline'    typically'} \textit{pred}
    \underline{\semicolon} \\
    &     & \underline{\lstinline'    assigns'}
    \textit{term}$^{+}_,$ \underline{\semicolon} \\
    &     & \underline{\lstinline'    ensures'} \textit{pred}
    \underline{\lstinline'\;'} \underline{\lstinline'*/'} \\
    &     & \textit{type}$^{\textit{as}\;T}$ \textit{id}$^{\textit{as}\;f}$
    \underline{\lstinline'('} (\textit{type} \textit{id})$^{*}_,$
    \underline{\lstinline')'} \underline{$\bopen$}
    \textit{Beg}$_f$ \underline{\lstinline'\:'}
    $T$ \underline{\lstinline'res'$_f\semicolon$}
    \textit{instr}$^{*}$
    \textit{End}$_f$ \underline{\lstinline'\:'}
    \underline{\lstinline'return res'$_f\semicolon$}
    \underline{$\bclose$} \\
  \end{tabular}
  \caption{Grammaire des instructions et fonctions}
  \label{fig:gram-c}
\end{figure}


La figure~\ref{fig:gram-c} définit la grammaire des instructions et fonctions C
considérées ici.
Les symboles terminaux sont \underline{soulignés}.
Les symboles non-terminaux sont en \textit{italique}.

\textit{left-value} est toute expression pouvant se trouver en
partie gauche d'une affectation (variable simple ou déréférencée).
Les accès aux pointeurs et aux tableaux sont notés \lstinline't[i]' (ou
\lstinline'*(t+i)').
Pour simplifier, \textit{basic-type} se limite aux types C \lstinline'int',
\lstinline'char' et \lstinline'long', et \textit{type} se limite aux types
simples (non pointeur) et aux pointeurs sur types simples.
\textit{id} fait référence aux identificateurs de variable (et de fonctions)
autorisés dans le langage C : suite de caractères alphanumériques pouvant
contenir des ``\lstinline'_''' mais dont le premier caractère ne peut pas être
numérique.

Les instructions de la forme \textit{id} \underline{\lstinline'('}
\textit{exp}$^{*}$ \underline{\lstinline');'} (respectivement
\textit{left-value} \underline{\lstinline'='} \textit{id}
\underline{\lstinline'('} \textit{exp}$^{*}$ \underline{\lstinline');'})
correspondent respectivement aux appels de fonction sans (resp. avec)
affectation de la valeur de retour.
La notation $e^*_s$ signifie que le motif $e$ est répété 0, 1 ou plusieurs fois,
et les occurrences du motif sont séparées par le symbole $s$.
De manière similaire, la notation $e^+_s$ signifie que le motif $e$ est répété
une ou plusieurs fois. \todo{notations}
\textit{label} respecte les mêmes contraintes que \textit{id} mais fait
référence aux labels d'un programme C.
Pour des raisons de commodité, nous supposons qu'un label précède chaque
annotation (assertion ou contrat de boucle) à l'intérieur d'une fonction.
Nous considérons aussi qu'un label est présent au début et à la fin de chaque
corps de boucle.
Les assertions peuvent être associées à des instructions quelconques.
Un invariant de boucle, un variant et une clause \lstinline'loop assigns'
peuvent être associés aux boucles.

Nous supposons que les fonctions respectent la forme normale définie par
l'entité syntaxique \textit{function} de la grammaire définie dans la
figure~\ref{fig:gram-c}.
Dans la figure~\ref{fig:gram-c}, la notation ``${\textit{as}\;X}$'' en exposant
d'un non-terminal signifie que chaque occurrence de $X$ dans la règle courante
doit être remplacée par la chaîne à laquelle fait référence ce non-terminal.
Par exemple, si \texttt{foo} est le nom (\textit{id}) d'une fonction,
la règle \textit{function} indique que les labels $\texttt{Beg}_f$ et
$\texttt{End}_f$ dans le corps de cette fonction doivent être
$\texttt{Beg}_{\texttt{foo}}$ et $\texttt{End}_{\texttt{foo}}$, ce qui assure des
labels uniques au début et à la fin du corps de la fonction.
Nous supposons qu'il existe une unique instruction \lstinline'return' à la fin
de chaque fonction, qu'elle retourne une variable appelée \lstinline'res'$_f$ et
que cette variable n'entre pas en conflit avec une autre variable du programme.

Un contrat de fonction à la \eiffel~\cite{\citeeiffel} peut être associé à
chaque fonction pour spécifier les préconditions et postconditions.
La clause \lstinline'requires' d'un contrat de fonction désigne la précondition.
La clause \lstinline'typically' désigne la précondition pour le test.
Elle est spécifique au test et permet de limiter l'explosion combinatoire du
nombre de chemins.
La clause \lstinline'assigns' liste les termes que la fonction a le droit de
modifier.
La clause \lstinline'ensures' désigne la postcondition de la fonction.

Les symboles non-terminaux \textit{exp}, \textit{term} et \textit{pred} sont
définis dans les parties~\ref{sec:gram-term} et~\ref{sec:gram-pred}.


\subsection{Syntaxe des expressions C et des termes \eacsl}
\label{sec:gram-term}

\begin{figure}[h!]
  \begin{tabular}{lrl}
    \textit{unop} & ::= & \underline{\lstinline'-'}
    $\mid$ \underline{\lstinline'\~'} \\
    \textit{binop} & ::= & \underline{\lstinline'+'}
    $\mid$ \underline{\lstinline'-'}
    $\mid$ \underline{\lstinline'*'}
    $\mid$ \underline{\lstinline'/'}
    $\mid$ \underline{\lstinline'\%'} \\
    \textit{exp} & ::= & \textit{cst} \\
    & $\mid$ & \textit{left-value} \\
    & $\mid$ & \textit{unop} \textit{exp} \\
    & $\mid$ & \textit{exp} \textit{binop} \textit{exp} \\
    \textit{term} & ::= & \textit{cst} \\
    & $\mid$ & \textit{left-value} \\
    & $\mid$ & \textit{unop} \textit{term} \\
    & $\mid$ & \textit{term} \textit{binop} \textit{term} \\
    & $\mid$
    & \underline{\lstinline'\\abs'} \underline{\lstinline'('} \textit{term} \underline{\lstinline')'} \\
    & $\mid$
    & \underline{\lstinline'\\sum'} \underline{\lstinline'('} \textit{term}
    \underline{\lstinline','} \textit{term} \underline{\lstinline','}
    \underline{\lstinline'\\lambda integer'} \textit{id}
    \underline{\lstinline'\;'} \textit{term} \underline{\lstinline')'} \\
    & $\mid$ & \underline{\lstinline'\\product'} \underline{\lstinline'('} \textit{term}
    \underline{\lstinline','} \textit{term} \underline{\lstinline','}
    \underline{\lstinline'\\lambda integer'} \textit{id}
    \underline{\lstinline'\;'} \textit{term} \underline{\lstinline')'} \\
    & $\mid$ & \underline{\lstinline'\\numof'} \underline{\lstinline'('} \textit{term}
    \underline{\lstinline','} \textit{term} \underline{\lstinline','}
    \underline{\lstinline'\\lambda integer'} \textit{id}
    \underline{\lstinline'\;'} \textit{term} \underline{\lstinline')'} \\
    & $\mid$ & \textit{term} \underline{\texttt{?}} \textit{term}
    \underline{\lstinline'\:'} \textit{term}\\
    & $\mid$ & \underline{\lstinline'\\old'} \underline{\lstinline'('} \textit{term}
    \underline{\lstinline')'} \\
    & $\mid$ & \underline{\lstinline'\\null'} \\
    & $\mid$ & \underline{\lstinline'\\result'} \\
  \end{tabular}
  \caption{Grammaire des expressions C et termes \eacsl}
  \label{fig:gram-exp-term}
\end{figure}


La figure~\ref{fig:gram-exp-term} définit la grammaire des termes \eacsl et des
expressions C du langage.
Le symbole \lstinline'~' est l'opérateur de négation bit-à-bit.
Le non-terminal \textit{cst} est une constante entière ou une chaîne de
caractère littérale.
Les constantes flottantes ne sont pas considérées.
Les termes \eacsl peuvent être vus comme une extension des expressions C.
Ils peuvent être d'un type C (comme une expression C) ou d'un type logique (nous
nous limitons à \lstinline'integer' ($\mathbb{Z}$) dans nos travaux).
Les termes sont ``purs'', c'est-à-dire qu'ils sont sans effet de bord.
De même, nous nous limitons aux expressions C ``pures''.
Pour cette raison, nous ne considérons pas les expressions comme \lstinline'x++'
ou \lstinline'++x'.
Pour cette même raison, les appels de fonction -- pouvant produire des effets 
de bord -- sont considérés non pas comme des expressions mais comme des
instructions.
L'hypothèse d'avoir des expressions pures n'est pas restrictive puisqu'une
transformation de programme peut permettre de sortir les effets de bord des
expressions.
Dans \framac, cette transformation est faite par la bibliothèque
\cil~\cite{\citecil}.

Le terme \lstinline'\abs' fait référence à la valeur absolue d'un terme.
Le terme \lstinline'\sum(t1, t2, \lambda integer k; t3)' désigne la somme
généralisée
$\sum_{\mbox{\lstinline'k = t1'}}^{\mbox{\lstinline't2'}}\mbox{\lstinline't3'}$.
La notation \lstinline'\product(t1, t2, \lambda integer k; t3)' désigne le
produit généralisé
$\Pi_{\mbox{\lstinline'k = t1'}}^{\mbox{\lstinline't2'}}\mbox{\lstinline't3'}$.
Enfin, \lstinline'\numof(t1, t2, \lambda integer k; t3)' fait référence au
nombre de valeurs de \lstinline'k' telles que
$\mbox{\lstinline't1'} \le \mbox{\lstinline'k'} \le \mbox{\lstinline't2'}$ et
$\mbox{\lstinline't3'} \neq 0$.
\lstinline'\null' est le terme correspondant au pointeur \lstinline'NULL'.
L'opérateur ternaire \lstinline'? :' signifie que le terme \lstinline'x ? y : z'
a la valeur de \lstinline'y' si le terme \lstinline'x' est non nul, ou la valeur
de \lstinline'z' si \lstinline'x' est nul.


\subsubsection*{Entiers mathématiques}


En plus des entiers machine, les termes \eacsl peuvent être des entiers
mathématiques de type \lstinline'integer' dans l'ensemble $\mathbb{Z}$.
Dans un terme \eacsl, les constantes entières, les opérations sur les entiers,
ainsi que les variables logiques sont de type \lstinline'integer'.
L'arithmétique entière est non bornée et donc les débordements d'entier
(\textit{overflows}) sont impossibles.
\eacsl dispose d'un système de sous-typage pour convertir automatiquement les
types entiers bornés du C vers les entiers mathématiques \cite{\citeacsl}.
Par exemple, si \lstinline{x} est une variable C de type \lstinline{int}, les
termes \eacsl \lstinline{1} et \lstinline{x+1} sont de type \lstinline{integer}
et une conversion implicite d'\lstinline{int} vers \lstinline{integer} est
introduite dans ce contexte quand la variable \lstinline{x} est typée.

Ce choix a été fait pour plusieurs raisons.
Premièrement, un des buts principaux de \framac est la preuve de programmes par
appel à des prouveurs externes.
Or, la plupart de ces prouveurs fonctionnent mieux
avec l'arithmétique entière qu'avec l'arithmétique bornée (ou modulaire).
Deuxièmement, les spécifications sont habituellement écrites sans se préoccuper
des détails d'implémentation, et les débordements d'entiers sont des détails
d'implémentation.
Troisièmement, il reste possible d'utiliser l'arithmétique bornée, si besoin
est, en utilisant des conversions explicites (\textit{casts}).
Par exemple, \lstinline{(int)(INT_MAX + 1)} est égal à \lstinline{INT_MIN}, la
plus petite valeur représentable de type \lstinline{int}.
Quatrièmement, ce choix facilite l'expression des débordements potentiels dans
les spécifications : par exemple, grâce à l'arithmétique entière,
\lstinline{/*@ assert INT_MIN <= x+y <= INT_MAX;*/} est le moyen le plus simple
de spécifier que \lstinline{x+y} ne déborde pas.
Dans la suite, sauf mention contraire, ``entier'' fait référence à
``entier non borné''.


\subsection{Syntaxe des prédicats \eacsl}
\label{sec:gram-pred}

\begin{figure}[h!]
  \begin{tabular}{lrl}
    \textit{relation} & ::= & \underline{\lstinline'>'}
    $\mid$ \underline{\lstinline'>='}
    $\mid$ \underline{\lstinline'<'}
    $\mid$ \underline{\lstinline'<='}
    $\mid$ \underline{\lstinline'=='}
    $\mid$ \underline{\lstinline'\!='} \\
    \textit{pred} & ::= & \underline{\lstinline'\\true'} \\
    & $\mid$ & \underline{\lstinline'\\false'} \\
    & $\mid$ & \underline{\lstinline'\\valid'} \underline{\lstinline'('} \textit{id}
    \underline{\lstinline')'} \\
    & $\mid$ & \underline{\lstinline'\\valid'} \underline{\lstinline'('} \textit{id}
    \underline{\lstinline'+'} \underline{\lstinline'('}
    \textit{term} \underline{\lstinline'..'} \textit{term}
    \underline{\lstinline')'} \underline{\lstinline')'} \\
    & $\mid$ & \underline{\lstinline'\\forall integer'} \textit{id}
    \underline{\lstinline'\;'} \textit{term} \underline{\lstinline'<='}
    \textit{id} \underline{\lstinline'<='} \textit{term}
    \underline{\lstinline'==>'} \textit{pred} \\
    & $\mid$ & \underline{\lstinline'\\exists integer'} \textit{id}
    \underline{\lstinline'\;'} \textit{term} \underline{\lstinline'<='}
    \textit{id} \underline{\lstinline'<='} \textit{term}
    \underline{\lstinline'&&'} \textit{pred} \\
    & $\mid$ & \underline{\lstinline'\!'} \textit{pred} \\
    & $\mid$ & \textit{pred} \underline{\lstinline'&&'} \textit{pred} \\
    & $\mid$ & \textit{pred} \underline{\lstinline'||'} \textit{pred} \\
    & $\mid$ & \textit{pred} \underline{\lstinline'==>'} \textit{pred} \\
    & $\mid$ & \textit{pred} \underline{\lstinline'<==>'} \textit{pred} \\
    & $\mid$ & \textit{term} \underline{\texttt{?}} \textit{pred}
    \underline{\lstinline'\:'} \textit{pred} \\
    & $\mid$ & \textit{term} \textit{relation} \textit{term} \\
  \end{tabular}
  \caption{Grammaire des prédicats \eacsl}
  \label{fig:gram-pred}
\end{figure}


La figure~\ref{fig:gram-pred} définit la grammaire des prédicats du langage
\eacsl.
Un prédicat a une valeur de vérité vraie ou fausse.
Le prédicat \lstinline'\valid' renvoie vrai si son argument est un pointeur
valide : c'est-à-dire \lstinline'\valid(ptr)' est vrai si et seulement si
on peut déréférencer \lstinline'ptr' et donc écrire \lstinline'*ptr'.
Dans le cas où son argument est de la forme \lstinline'ptr+(x..y)', le prédicat
est valide si et seulement si \lstinline'ptr+x', \lstinline'ptr+(x+1)', \ldots,
\lstinline'ptr+y' sont valides.
Dans \eacsl, les quantifications existentielles (\lstinline'\exists') et
universelles (\lstinline'\forall') doivent porter sur des domaines finis
d'entiers.

Le symbole \lstinline'!' est l'opérateur de négation logique, le symbole
\lstinline'&&' est l'opérateur de conjonction logique et le symbole
\lstinline'||' est l'opérateur de disjonction logique.
Le symbole \lstinline'==>' encode l'implication logique et le symbole
\lstinline'<==>' encode l'équivalence logique de deux prédicats.
L'opérateur ternaire \lstinline'? :' signifie que le prédicat
\lstinline'x ? y : z' a la valeur de \lstinline'y' si le terme \lstinline'x' est
non nul, ou la valeur de \lstinline'z' si \lstinline'x' est nul.

\subsection{Fonction sous vérification et fonctions appelées}

\todo{distinction}
Dans ce document nous distinguons la ``fonction sous vérification'' et les
``fonctions appelées''.
La fonction sous vérification est unique, et n'est appelée par aucune fonction :
nous excluons toute récursivité (directe ou indirecte) dans le langage
considéré.
La précondition de la fonction sous vérification est supposée vraie.

Les fonctions appelées sont les fonctions atteignables à partir de la fonction
sous vérification.
Leurs préconditions doivent être vérifiées.


\section{Sémantique des programmes annotés}
\label{sec:lang-semantics}


Nous définissons dans les parties~\ref{sec:env} et~\ref{sec:store} les notions
d'environnement et de {\em store}, qui nous permettent de définir la notion de
mémoire en partie~\ref{sec:mem}.
Un environnement associe à chaque désignation (appelée \textit{left-value} dans
la grammaire de la figure~\ref{fig:gram-c}) une adresse en mémoire.
La notion de {\em store} associe aux adresses en mémoire des valeurs.
Le tout constitue une mémoire.
Cette notion de mémoire nous permet de définir la sémantique dénotationnelle des
instructions et fonctions d'un programme annoté comme des modifications de
mémoires.
Dans la suite du document, nous utilisons les notations suivantes :
\begin{itemize}
\item $\mathit{lv}$, $\mathit{lv_1}$ et $\mathit{lv_2}$ sont des left-values;
\item $x$, $x_1$, ..., $x_N$ sont des identificateurs de variable;
\item $f$ est un identificateur de fonction;
\item $v$ et $v_i$ sont des valeurs;
\item $\mathit{cst}$ est une constante;
\item $G$ et $X$ sont des listes de left-values \eacsl;
\item $i$, $i_1$, $i_2$ sont des instructions C;
\item $A$, $A_1$, $A_2$ et $B$ sont des séquences d'instructions C;
\item $e$, $e_1$, ..., $e_N$ sont des expressions C;
\item $t$, $t_1$, $t_2$ et $t_3$ sont des termes \eacsl;
\item $h$, $h_1$, $h_2$, $h_3$ sont des expressions C ou des termes \eacsl;
\item $p$, $p_1$, $p_2$ sont des prédicats \eacsl;
\item $\env$, $\env_1$, $\env_2$ sont des environnements
  (voir définition en partie~\ref{sec:env});
\item $\store$, $\store_1$, $\store_2$ sont des stores
  (voir définition en partie~\ref{sec:store});
\item $\mem$ est une mémoire (voir définition en partie~\ref{sec:mem});
\item $\loc$, $\loc_1$, ..., $\loc_N$ \todo{nouveau symbole} sont des
  adresses en mémoire (voir définition en partie~\ref{sec:mem});
\item $T$ est un type C;
\item $\mathit{unop}$ est un opérateur unaire;
\item $\mathit{binop}$ est un opérateur binaire;
\item $\mathit{rel}$ est un opérateur binaire relationnel.
\end{itemize}

Pour des raisons de commodité, nous supposons que toutes les variables logiques
liées dans les annotations et toutes les variables du programme sont
différentes, ceci nous permet en particulier de traduire les variables logiques
en C sans les renommer.\todo{hypothèse}


\subsection{Environnements}
\label{sec:env}

Le domaine $LV$ correspond à l'entité syntaxique \textit{left-value}.
$LOC$ est le domaine des adresses (positions) en mémoire, incluant l'adresse
nulle, notée \lstinline'NULL' $\in LOC$.
$\bot$ dénote une adresse indéfinie et donc invalide.
Un environnement est une fonction $\env \in LV \rightarrow LOC \cup \{\bot\}$
qui associe une adresse ou $\bot$ (adresse indéfinie) à chaque left-value
définie dans le fragment de programme considéré.
Nous notons $ENV = LV \rightarrow LOC \cup \{\bot\}$ le domaine des
environnements.
La notation $\env$[$\mathit{lv} \mapsto \loc$] désigne un nouvel environnement,
correspondant à l'environnement $\env$ dans lequel l'adresse $\loc$ est associée
à la left-value $\mathit{lv}$.
L'environnement $\env$[$\mathit{lv} \mapsto \loc$] est défini ainsi :

\begin{center}
\begin{tabular}{rclr}
  ($\env$[$\mathit{lv} \mapsto \loc$])($\mathit{lv}$) &=& $\loc$
  & \eqlabel{env-get-1}\\
  ($\env$[$\mathit{lv_1} \mapsto \loc$])($\mathit{lv_2}$) & =
  & $\env$($\mathit{lv_2}$) si $\mathit{lv_1} \neq \mathit{lv_2}$
  & \eqlabel{env-get-2}\\
  %$\env$($\mathit{lv}$) &=& $\bot$ si le domaine de $\env$ est vide
  %& \eqlabel{env-get-3}\\
\end{tabular}
\end{center}

L'adresse d'une left-value ne peut pas être \lstinline'NULL'.
Dans la suite, nous utiliserons la notation
$\env$[$\mathit{lv_1} \mapsto \loc_1, \mathit{lv_2} \mapsto \loc_2$]
(en supposant que $\mathit{lv_1} \neq \mathit{lv_2}$)
comme un raccourci pour
$\env$[$\mathit{lv_1} \mapsto \loc_1$][$\mathit{lv_2} \mapsto \loc_2$].
La notation $\env - \{\mathit{lv}\}$ signifie que la left-value
$\mathit{lv}$ et l'adresse associée sont enlevées de l'environnement $\env$ :

\begin{center}
\begin{tabular}{rclr}
  $(\env - \{\mathit{lv}\})(\mathit{lv})$ &=& $\bot$ & \eqlabel{env-del-1} \\
  $(\env - \{\mathit{lv_1}\})(\mathit{lv_2})$ &=&
  $\env(\mathit{lv_2})$ si $\mathit{lv_1} \neq \mathit{lv_2}$ &
  \eqlabel{env-del-2} \\
\end{tabular}
\end{center}

Nous définissons également une relation d'inclusion sur les environnements :
$\env_1$ est inclus dans $\env_2$ si et seulement si, pour chaque paire
$(\mathit{lv}, \loc)$, si $\loc \neq \bot$ est l'adresse de $\mathit{lv}$
dans $\env_1$, alors $\loc$ est aussi l'adresse de $\mathit{lv}$ dans $\env_2$.

\[
\env_1~\subenv~\env_2~\equiv~
\forall \mathit{lv} \in \mathit{LV}.~\forall \loc \in \mathit{LOC}.
~(\env_1(\mathit{lv})=\loc \Rightarrow \env_2(\mathit{lv})=\loc)
\]
\todo{domaines}


\subsection{Stores}
\label{sec:store}

Le domaine des valeurs $VAL$ est défini comme l'union des valeurs des termes et
prédicats \eacsl et des expressions C :
$VAL = \mathbb{Z}~\cup~LOC~\cup~\{\bot\}$.
Le symbole $\bot$ dénote une valeur indéfinie qui peut être soit une adresse
indéfinie soit une valeur entière indéfinie.
$LOC$ est inclus dans $VAL$ car la valeur d'un pointeur est une adresse.
Un {\em store} est une fonction $\store \in LOC \rightarrow VAL$ qui associe une
valeur à chaque adresse.
Nous notons $STORE = LOC \rightarrow VAL$ le domaine des {\em stores}.
La notation $\store$[$\loc \mapsto v$] désigne un nouveau {\em store},
correspondant au {\em store} $\store$ dans lequel la valeur $v$ est stockée à
l'adresse $\loc$.
Le {\em store} $\store$[$\loc \mapsto v$] est défini ainsi :

\begin{center}
\begin{tabular}{rclr}
  ($\store$[$\loc \mapsto v$])($\loc$) &=& $v$ & \eqlabel{store-get-1}\\
  ($\store$[$\loc_1 \mapsto v$])($\loc_2$) & =
  & $\store$($\loc_2$) si $\loc_1 \neq \loc_2$
  & \eqlabel{store-get-2}\\
  %$\store$($\loc$) &=& $\bot$ si le domaine de $\store$ est vide
  %& \eqlabel{store-get-3}\\
  $\store(\mbox{\lstinline'NULL'})$ &=& $\bot$ & \eqlabel{store-get-3} \\
\end{tabular}
\end{center}
\todo{dernière règle}

%Notons que $\bot$ n'est pas dans le domaine des stores.
%Par conséquent, $\store(\bot)$ n'est pas défini.

Dans la suite, nous utiliserons la notation
$\store$[$\loc_1 \mapsto v_1, \loc_2 \mapsto v_2$] (en supposant
que $\loc_1 \neq \loc_2$)
comme un raccourci pour
$\store$[$\loc_1 \mapsto v_1$][$\loc_2 \mapsto v_2$].
La notation $\store - \{\loc\}$ signifie que l'adresse
$\loc$ et la valeur associée sont enlevées du store $\store$ :

\begin{center}
\begin{tabular}{rclr}
  $(\store - \{\loc\})(\loc)$ &=& $\bot$ &
  \eqlabel{store-del-1} \\
  $(\store - \{\loc_1\})(\loc_2)$ &=& $\store(\loc_2)$ si $\loc_1 \neq \loc_2$ &
  \eqlabel{store-del-2} \\
\end{tabular}
\end{center}

Nous définissons également une relation d'inclusion sur les stores :
$\store_1$ est inclus dans $\store_2$ si et seulement si, pour chaque paire
$(\loc, v)$, si $v \neq \bot$ est la valeur à l'adresse $\loc$ dans $\store_1$,
alors $v$ est également la valeur à l'adresse $\loc$ dans $\store_2$.

\[
\store_1~\subenv~\store_2~\equiv~
\forall \loc \in \mathit{LOC}.~\forall v \in \mathit{VAL}-\{\bot\}.
~(\store_1(\loc)=v \Rightarrow \store_2(\loc)=v)
\]
\todo{domaines}


\subsection{Mémoires}
\label{sec:mem}

Nous définissons la notion de mémoire, correspondant à un couple environnement,
store : $MEM = ENV \times STORE~\cup~\{\errormem\}$.
Une mémoire $\mem \in MEM$ est un paramètre et un résultat des fonctions
sémantiques définies dans la suite du chapitre.
Cette mémoire sera notée :
\begin{itemize}
\item $(\env, \store)$ si nous devons accéder à l'environnement $\env$ ou au
  store $\store$;
\item $\mem$ sinon.
\end{itemize}

Nous définissons une mémoire spécifique d'erreur, notée $\errormem \in MEM$.
Une telle mémoire est le résultat de l'évaluation soit d'une instruction
provoquant une erreur soit d'une annotation invalide (par exemple,
\lstinline'assert' ou \lstinline'loop invariant' faux).
%L'évaluation des instructions s'arrête dès qu'une mémoire d'erreur est produite.

Une variable simple $x$ de type C scalaire (\lstinline'int', ...) est
représentée dans une mémoire $\mem = (\env, \store)$ par :

\begin{itemize}
\item une association d'une adresse $\loc$ à $x$ dans l'environnement $\env$
  notée $x \mapsto \loc$
\item une association de sa valeur $v$ à l'adresse mémoire $\loc$ dans le
  {\em store} $\store$ notée $\loc \mapsto v$.
\end{itemize}

Donc la sémantique d'une variable simple $x$ est sa valeur définie par
$\store(\env(x))$.

Un tableau $x$ de taille $n$ est représenté dans une mémoire $(\env, \store)$
par $n$ adresses auxquelles sont associées respectivement ses $n$ valeurs
désignées syntaxiquement par $x\mbox{\lstinline'[0]'}$,
$x\mbox{\lstinline'[1]'}$, ..., $x\mbox{\lstinline'['}n\mbox{\lstinline'-1]'}$.
À la notation syntaxique $x\mbox{\lstinline'['}i\mbox{\lstinline']'}$ est
associée une {\em left-value} ($\in LV$) également dénotée
$x\mbox{\lstinline'['}i\mbox{\lstinline']'}$.
À cette {\em left-value} est associée une adresse mémoire $\loc_i$ dans
l'environnement $\env$.
À cette adresse mémoire est associée une valeur $v_i$ dans le {\em store}.
Donc la sémantique d'un accès à un élément d'un tableau
$x\mbox{\lstinline'['}i\mbox{\lstinline']'}$ est sa valeur définie par
$\store(\env(x$\lstinline'['\eval{$i$}{$(\env, \store)$}\lstinline']'$))$ où
\eval{$i$}{$(\env, \store)$} est la sémantique de $i$ dans la mémoire,
c'est-à-dire la valeur de $i$ dans la mémoire $(\env, \store)$.
Définissons maintenant la fonction sémantique $\mathcal{E}$.


\subsection{Sémantique dénotationnelle des termes, prédicats et expressions}

\begin{figure}[h!]\centering
\begin{spacing}{1.6}
  \begin{tabular}{rclr}
    \eval{$\mathit{cst}$}{$\mem$} &=& $\mathit{cst}$ & \eqlabel{E-cst} \\

    \eval{$\mathit{x}$}{$(\env, \store)$} &=& $\store(\env(\mathit{x}))$
    & \eqlabel{E-lval} \\

    \eval{$\mathit{x}$\lstinline'['$h$\lstinline']'}{$(\env, \store)$}
    &=&
    $\store(\env(\mathit{x}$
    \lstinline'['\eval{$h$}{$(\env, \store)$}\lstinline']'$))$
    & \eqlabel{E-deref} \\

    \eval{$\mathit{unop}~h$}{$\mem$}
    &=& $\mathit{unop}$ (\eval{$h$}{$\mem$}) & \eqlabel{E-unop} \\

    \eval{$h_1~\mathit{binop}~h_2$}{$\mem$}
    &=& (\eval{$h_1$}{$\mem$}) $\mathit{binop}$
    (\eval{$h_2$}{$\mem$}) & \eqlabel{E-binop} \\

    \eval{\lstinline'\\abs('$t$\lstinline')'}{$\mem$} &=&
    $\lvert$ \eval{$t$}{$\mem$} $\rvert$ & \eqlabel{E-abs} \\

    \multicolumn{4}{l}{
      \eval{\lstinline'\\sum('$t_1$, $t_2$,
        \lstinline'\\lambda integer' $k \semicolon t_3$ \lstinline')'}{$\mem$}
    } \\
    &=&
    $\Sigma'(\mbox{\eval{$t_1$}{$\mem$}},\mbox{\eval{$t_2$}{$\mem$}},(\lambda~k.\mbox{\eval{$t_3$}{$\mem$}}))$
    & \eqlabel{E-sum} \\
    \multicolumn{4}{c}{
      où $\Sigma'(v_1,v_2,f) = \sum_{i = v_1}^{v_2} f(i)$
    } \\

    \multicolumn{4}{l}{
      \eval{\lstinline'\\product('$t_1$, $t_2$,
        \lstinline'\\lambda integer' $k \semicolon t_3$ \lstinline')'}{$\mem$}
    } \\
    &=&
    $\Pi'(\mbox{\eval{$t_1$}{$\mem$}},\mbox{\eval{$t_2$}{$\mem$}},(\lambda~k.\mbox{\eval{$t_3$}{$\mem$}}))$
    & \eqlabel{E-prod} \\
    \multicolumn{4}{c}{
      où $\Pi'(v_1,v_2,f) = \prod_{i = v_1}^{v_2} f(i)$
    } \\

    \multicolumn{4}{l}{
      \eval{\lstinline'\\numof('$t_1$, $t_2$,
        \lstinline'\\lambda integer' $k \semicolon t_3$ \lstinline')'}{$\mem$}
    } \\
    &=&
    $N'(\mbox{\eval{$t_1$}{$\mem$}},\mbox{\eval{$t_2$}{$\mem$}},(\lambda~k.\mbox{\eval{$t_3$}{$\mem$}}))$
    & \eqlabel{E-num} \\
    \multicolumn{4}{c}{
      où $N'(v_1,v_2,f) = \sum_{i = v_1}^{v_2} (f(i) ? 1 : 0)$
    } \\

    \eval{$h_1$ \texttt{?} $h_2$ \texttt{:} $h_3$}{$\mem$}
    &=& \eval{$h_2$}{$\mem$} si (\eval{$h_1$}{$\mem$}) $\neq 0$,
    & \eqlabel{E-tif} \\
    & & \eval{$h_3$}{$\mem$} sinon & \\

    \eval{\lstinline'\\old('$t$\lstinline')'}{$\mem$}
    &=& \eval{$t$}{$\mem_{Beg_{\mathtt{f}}}$}
    & \eqlabel{E-old} \\

    \eval{\lstinline'\\null'}{$\mem$} &=& \lstinline'NULL'
    & \eqlabel{E-null} \\

    \eval{\lstinline'\\result'}{$\mem$}
    &=& \eval{\lstinline'res'}{$\mem$} & \eqlabel{E-res} \\
  \end{tabular}
  \caption{Sémantique dénotationnelle des termes et expressions}
  \label{fig:sem-exp-term}
\end{spacing}
\end{figure}


La sémantique dénotationnelle des expressions C, des termes et des prédicats
\eacsl est définie par la fonction
$\mathcal{E} : (EXP \cup TERM \cup PRED) \rightarrow MEM \rightarrow VAL$.
Les domaines de syntaxe abstraite $EXP$, $TERM$ et $PRED$ correspondent
respectivement aux entités syntaxiques \textit{exp}, \textit{term} et
\textit{pred} définies dans les figures~\ref{fig:gram-exp-term}
et~\ref{fig:gram-pred}.
Cette fonction associe une valeur à une expression C en fonction d'une mémoire
$\mem = (\env, \store)$.
Ici on utilise l'hypothèse selon laquelle les expressions du langage considéré
ne génèrent pas d'effets de bord.
L'absence d'effet de bord permet à la fonction $\mathcal{E}$ de ne pas devoir
retourner une nouvelle mémoire.

La figure~\ref{fig:sem-exp-term} donne la sémantique des expressions et des
termes.
Ces deux éléments du langage se retrouvent sur la même figure car les termes
peuvent être vus comme une extension des expressions C.
Les cinq premières règles de la figure~\ref{fig:sem-exp-term} sont communes aux
termes et aux expressions, les autres sont spécifiques aux termes.
La règle \eqlabel{E-lval} traite le cas où une left-value est un identificateur
de variable $x$.
$\env(x)$ est l'adresse de $x$, la valeur à cette adresse est $\store(\env(x))$.
La règle \eqlabel{E-deref} traite le cas où une left-value est un identificateur
de variable $x$, accédé à l'offset de valeur $h$.
L'unique différence avec la règle \eqlabel{E-lval} est l'évaluation de $h$.
Dans les règles \eqlabel{E-unop} et \eqlabel{E-binop}, $\mathit{unop}$ et
$\mathit{binop}$ correspondent à l'application de l'opérateur unaire ou binaire
correspondant, parmi ceux autorisés par la grammaire de la
figure~\ref{fig:gram-exp-term}.
Dans la règle \eqlabel{E-abs}, $\lvert~.~\rvert : VAL \rightarrow VAL$ est la
fonction valeur absolue.

Dans les règles \eqlabel{E-sum}, \eqlabel{E-prod} et \eqlabel{E-num}, la
variable logique $k$ liée dans le terme
$\mbox{\lstinline'\\lambda integer'}~k\semicolon~t_3$ est supposée différente de
toute autre variable du programme (logique ou C).
Pour la règle \eqlabel{E-sum} nous définissons une fonction $\Sigma'(v_1,v_2,f)$
pour tous $v_1$ et $v_2$ de type $VAL$ et pour toute fonction
$f : VAL \rightarrow VAL$.
La fonction $\Sigma'$ est définie en fonction de la somme mathématique sur les
entiers en figure~\ref{fig:gram-exp-term}.
Lors de l'application de la fonction $\Sigma'$, les deux premiers paramètres
($t_1$ et $t_2$) sont évalués avant le troisième.
Ainsi, la variable $k$ a toujours une valeur définie lors l'évaluation du terme
$t_3$.
Les règles \eqlabel{E-prod} et \eqlabel{E-num} sont définies de manière
similaire.
\todo{lambda k}

Dans la règle \eqlabel{E-old}, $\mem_{Beg_{\mathtt{f}}}$ désigne la mémoire au
début de l'exécution de la fonction courante, nous supposons
qu'une telle mémoire est conservée à chaque appel.
Notons que nous n'avons pas formalisé la mémorisation des états mémoires au
début de la fonction en cours d'évaluation car cela alourdirait considérablement
la définition de la sémantique.

La figure~\ref{fig:sem-pred} donne la sémantique des prédicats.
Les valeurs de vérité des prédicats sont encodées par les entiers 0 et 1 comme
en C.
La règle \eqlabel{P-valid} énonce qu'un identificateur $x$ est valide
si l'adresse associée à $x$ dans l'environnement est définie (différente de
$\bot$).
Un pointeur est valide en mémoire si on peut le déréférencer afin de pouvoir
accéder ou modifier l'emplacement mémoire qu'il référence (c'est-à-dire vers
lequel il pointe).
Le pointeur égal à l'adresse d'une variable locale est valide dans la portée de
cette variable.
De la même manière, \lstinline'\valid(x+(t1..t2))' est valide si les pointeurs
\lstinline'(x+t1)', ..., \lstinline'(x+t2)' peuvent être déréférencés.

\begin{figure}[bt]
  \begin{center}
    \begin{tabular}{c|c|c|c|c|c|c}
      $p_1$ & $p_2$ & \lstinline'!' $p_1$ & $p_1$ \lstinline'&&' $p_2$
      & $p_1$ \lstinline'||' $p_2$ & $p_1$ \lstinline'==>' $p_2$
      & $p_1$ \lstinline'<==>' $p_2$ \\ \hline
      0 & 0 & 1 & 0 & 0 & 1 & 1 \\ \hline
      0 & 1 & 1 & 0 & 1 & 1 & 0 \\ \hline
      1 & 0 & 0 & 0 & 1 & 0 & 0 \\ \hline
      1 & 1 & 0 & 1 & 1 & 1 & 1 \\
    \end{tabular}
    \caption{Table de vérité des prédicats\label{fig:truth-table}}
  \end{center}
\end{figure}

La figure~\ref{fig:truth-table} donne la table de vérité des opérations de
négation, de conjonction, de disjonction, d'implication et d'équivalence sur les
prédicats \eacsl.


\begin{figure}[h!]\centering
\begin{spacing}{1.6}
  \begin{tabular}{rclr}
    \eval{\lstinline'\\true'}{$\mem$} &=& $1$ & \eqlabel{P-true} \\
    \eval{\lstinline'\\false'}{$\mem$} &=& $0$ & \eqlabel{P-false} \\
    \eval{\lstinline'\\valid('$x$\lstinline')'}{$(\env, \store)$}
    &=& $0$ si $\env(x) = \bot$,
    $1$ sinon & \eqlabel{P-valid} \\
    \eval{\lstinline'\\valid('$x$\lstinline'+('$\mathit{t_1}$
      \lstinline'..'$\mathit{t_2}$\lstinline'))'}{$\mem$} &=&
    \eval{
      \lstinline'\\valid('$x$\lstinline'+'$\mathit{t_1}$\lstinline')'
      \lstinline'\&\&' $\ldots$ \lstinline'\&\&'
      \lstinline'\\valid('$x$\lstinline'+'$\mathit{t_2}$\lstinline')'
    }{$\mem$}
    & \eqlabel{P-validr} \\

    \multicolumn{4}{l}{
      \eval{\lstinline'\\forall integer' $k \semicolon t_1$ \lstinline'<=' $k$
        \lstinline'<=' $t_2$ \lstinline'==>' $p$}{$\mem$}
    }\\
    &=&
    $F'(\mbox{\eval{$t_1$}{$\mem$}}, \mbox{\eval{$t_2$}{$\mem$}},(\lambda~k.\mbox{\eval{$p$}{$\mem$}}))$
    & \eqlabel{P-forall} \\
    \multicolumn{4}{c}{
      où $F'(v_1,v_2,f) = \forall i. (v_1 \le i \le v_2 \Rightarrow f(i))$
    } \\

    \multicolumn{4}{l}{
      \eval{\lstinline'\\exists integer' $k \semicolon t_1$ \lstinline'<=' $k$
        \lstinline'<=' $t_2$ \lstinline'\&\&' $p$}{$\mem$}
    }\\
    &=&
    $E'(\mbox{\eval{$t_1$}{$\mem$}}, \mbox{\eval{$t_2$}{$\mem$}},(\lambda~k.\mbox{\eval{$p$}{$\mem$}}))$
    & \eqlabel{P-exists} \\
    \multicolumn{4}{c}{
      où $E'(v_1,v_2,f) = \exists i. (v_1 \le i \le v_2 \land f(i))$
    } \\

    \eval{\lstinline'\!'$p$}{$\mem$} &=& $\lnot$ (\eval{$p$}{$\mem$})
    & \eqlabel{P-not} \\
    \eval{$p_1$ \lstinline'\&\&' $p_2$}{$\mem$} &=&
    (\eval{$p_1$}{$\mem$}) $\land$ (\eval{$p_2$}{$\mem$})
    & \eqlabel{P-and} \\
    \eval{$p_1$ \lstinline'||' $p_2$}{$\mem$} &=&
    (\eval{$p_1$}{$\mem$}) $\lor$ (\eval{$p_2$}{$\mem$})
    & \eqlabel{P-or} \\
    \eval{$p_1$ \lstinline'==>' $p_2$}{$\mem$} &=&
    (\eval{$p_1$}{$\mem$}) $\Rightarrow$ (\eval{$p_2$}{$\mem$})
    & \eqlabel{P-impl} \\
    \eval{$p_1$ \lstinline'<==>' $p_2$}{$\mem$} &=&
    (\eval{$p_1$}{$\mem$}) $\Leftrightarrow$(\eval{$p_2$}{$\mem$})
    & \eqlabel{P-eq} \\
    \eval{$t$ \texttt{?} $p_1$ \texttt{:} $p_2$}{$\mem$} &=&
    \eval{$p_1$}{$\mem$} si (\eval{$t$}{$\mem$}) $\neq 0$,
    \eval{$p_2$}{$\mem$} sinon
    & \eqlabel{P-pif} \\
    \eval{$t_1~\mathit{rel}~t_2$}{$\mem$}
    &=& (\eval{$t_1$}{$\mem$}) $\mathit{rel}$
    (\eval{$t_2$}{$\mem$}) & \eqlabel{P-rel} \\
  \end{tabular}
  \caption{Sémantique dénotationnelle des prédicats}
  \label{fig:sem-pred}
\end{spacing}
\end{figure}



\subsection{Sémantique dénotationnelle des instructions}

La sémantique dénotationnelle des instructions C est exprimée avec la fonction
$\mathcal{C} : INSTR \rightarrow MEM \rightarrow MEM$.
Cette fonction associe à une instruction $i$, dans un contexte mémoire $\mem$,
une nouvelle mémoire $\mem'$ qui tient compte des modifications effectuées par
$i$ dans $\mem$.
La sémantique d'une séquence d'instructions est exprimée avec la fonction
$\mathcal{C}^{*} : INSTR^{*} \rightarrow MEM \rightarrow MEM$.
Cette fonction définie en figure~\ref{fig:c*} revient à appeler successivement
$\mathcal{C}$ sur chaque instruction dans l'ordre de la séquence.

\begin{figure}
\begin{center}
\begin{tabular}{rclr}
  \comps{$A_1~A_2$}{$\mem$}
  &=& \comps{$A_2$}{(\comps{$A_1$}{$\mem$})} & \eqlabel{C-seq-1} \\
  \comps{$i~A$}{$\mem$}
  &=& \comps{$A$}{(\comp{$i$}{$\mem$})} & \eqlabel{C-seq-2} \\
  \comps{}{$\mem$} &=& $\mem$ & \eqlabel{C-seq-3} \\
\end{tabular}
\caption{Sémantique dénotationnelle des séquences d'instructions\label{fig:c*}}
\end{center}
\end{figure}


\begin{figure}[h!]
\begin{spacing}{1.1}
  \begin{tabular}{rcll}
    \comp{$i$}{$\errormem$} &=& $\errormem$ & \eqlabel{C-err} \\
    \comp{$T~x\semicolon$}{$(\env, \store)$}
    &=& ($\env$[$x \mapsto \loc$], $\store$[$\loc \mapsto \bot$])
    où $\loc$ est une adresse fraîche & \eqlabel{C-decl} \\
    \comp{$\mathit{x}$ \lstinline'=' $e\semicolon$}{$(\env, \store)$}
    &=&
    ($\env, \store$[$\env(\mathit{x}) \mapsto$ \eval{$e$}{$(\env, \store)$}])
    & \eqlabel{C-set} \\

    \comp{$\mathit{x}$\lstinline'['$\mathit{e}$\lstinline']'
      \lstinline'=' $e_2\semicolon$}{$(\env, \store)$}
    &=& ($\env, \store[\env(\mathit{x}$
      \lstinline'['\eval{$\mathit{e}$}{$(\env, \store)$}\lstinline']') $\mapsto$
      \eval{$e_2$}{$(\env, \store)$}])
    & \eqlabel{C-set-2} \\

    \comp{$\Zinit$ $\mathit{x}$ \lstinline'=' $e$ $\semicolon$}{
      $(\env, \store)$}
    &=& ($\env[x \mapsto \loc], \store[\loc \mapsto$
      \eval{$e$}{$(\env, \store)$}])
    & \eqlabel{C-Z-set} \\
    && où $\loc$ est une adresse fraîche &\\

    \comp{$x \Zclear \semicolon$}{$(\env, \store)$}
    &=& ($\env-\{x\}, \store-\{\env(x)\}$)
    & \eqlabel{C-Z-unset} \\

    \comp{\lstinline'fassert('$e$\lstinline');'}{$\mem$}
    &=& $\mem$ si (\eval{$e$}{$\mem$}) $\neq 0$, $\errormem$ sinon
    & \eqlabel{C-fassert} \\

    \comp{\lstinline'fassume('$e$\lstinline');'}{$\mem$}
    &=& $\mem$ (et on suppose dans la suite (\eval{$e$}{$\mem$}) $\neq 0$)
    & \eqlabel{C-fassume} \\

    \comp{$\mathit{lv}$ \lstinline'= fvalid('$x$\lstinline');'}{$(\env, \store)$}
    &=&
    $(\env, \store[\env(\mathit{lv}) \mapsto 0]$)
    si $\env(x) = \bot$,
    & \eqlabel{C-fvalid} \\
    && $(\env, \store[\env(\mathit{lv}) \mapsto 1])$ sinon & \\

    \comp{$\mathit{lv}$ \lstinline'= fvalidr('$e_1$,$e_2$,$e_3$
      \lstinline');'}{$(\env, \store)$}
    &=& $(\env, \store[\env(\mathit{lv}) \mapsto 0])$ si
    ($\env(e_1+$\eval{$e_2$}{$(\env, \store)$}$) = \bot$)
    & \eqlabel{C-fvalidr} \\
    && $\lor \cdots \lor$ ($\env(e_1+$\eval{$e_3$}{$(\env, \store)$}$) = \bot$),
    &\\
    & & $(\env, \store[\env(\mathit{lv}) \mapsto 1])$ sinon & \\

    \comp{$x$ \lstinline' = malloc('$e$\lstinline','$s$\lstinline');'}{$(\env, \store)$}
    &=&
    $(\env[x$\lstinline'[0]' $\mapsto \loc_1$, ...,
      $x$\lstinline'['$N$\lstinline'-1]'
      $\mapsto \loc_N]$,
    & \eqlabel{C-malloc} \\
    && $\store[\env(x) \mapsto \loc_1, \loc_1 \mapsto \bot, ..., \loc_N \mapsto \bot])$
    &\\
    && où $N =$ \eval{$e$}{$(\env, \store)$} &\\
    && et $\loc_1, ..., \loc_N$ sont des adresses fraîches &\\

    \comp{\lstinline'free('$x$\lstinline')'}{$(\env, \store)$}
    &=&
    $(\env-\{x\mbox{\lstinline'[0]'}\}-\ldots-\{x$\lstinline'['$N$\lstinline'-1]'$\},$
    & \eqlabel{C-free} \\
    && $\store-\{\env(x)\}-\{\env(x$\lstinline'[0]'$)\}-...-\{\env(x$\lstinline'['$N$\lstinline'-1]'$)\})$ &\\
    \multicolumn{3}{r}{
    où $N$ est le nombre d'éléments alloués par le \lstinline'malloc' correspondant
    } & \\
    && si $x$ est un pointeur alloué et non libéré, $\errormem$ sinon &\\

    \comp{$f$\lstinline'('$e_1$,...,$e_N$\lstinline');'}{$(\env, \store)$} &=&
    $\mathcal{F} \llbracket$ \lstinline'/*@ ... */'
    $f$\lstinline'('$x_1$,...,$x_N$
    \lstinline')' $\bopen ... \bclose \rrbracket$ & \eqlabel{C-fct1} \\
    && $(\env[x_1 \mapsto \loc_1$, ...,
      $x_N \mapsto \loc_N],$ &\\
    && $\store[\loc_1 \mapsto$ \eval{$e_1$}{$(\env, \store)$}, ...,
       $\loc_N \mapsto$ \eval{$e_N$}{$(\env, \store)$} $])$ &\\
    && où $\loc_1, ..., \loc_N$ sont des adresses fraîches &\\

    \comp{$\mathit{lv}$ \lstinline'=' $f$\lstinline'('$e_1$,...,$e_N$
      \lstinline');'}{$(\env, \store)$} &=&
    $(\env, \store[\env(\mathit{lv}) \mapsto$
      \eval{\lstinline'res'$_f$}{$(\env', \store')$}$])$
    & \eqlabel{C-fct2} \\
    && où $(\env'$, $\store')$ =
    $\mathcal{F} \llbracket$ \lstinline'/*@ ... */'
    $f$\lstinline'('$x_1$,...,$x_N$
    \lstinline')' $\bopen ... \bclose \rrbracket$ &\\
    && $(\env[x_1 \mapsto \loc_1$, ...,
      $x_N \mapsto \loc_N],$ &\\
    && $\store[\loc_1 \mapsto$ \eval{$e_1$}{$(\env, \store)$}, ...,
      $\loc_N \mapsto$ \eval{$e_N$}{$(\env, \store)$}
    $])$ &\\
    && où $\loc_1, ..., \loc_N$ sont des adresses fraîches &\\

    \comp{\lstinline'return res'$_f\semicolon$}{$\mem$}
    &=& $\mem$ & \eqlabel{C-return} \\

    \comp{\lstinline'if('$e$\lstinline')' $\bopen A \bclose$
      \lstinline'else' $\bopen B \bclose$}{$\mem$}
    &=& \comps{$A$}{$\mem$} si (\eval{$e$}{$\mem$})
    $\neq 0$, \comps{$B$}{$\mem$} sinon & \eqlabel{C-if} \\

    \comp{\lstinline'/*@ assert' $p\semicolon$ \lstinline' */'}{$\mem$}
    &=& $\mem$ si (\eval{$p$}{$\mem$}) $\neq 0$, $\errormem$ sinon
    & \eqlabel{C-assert} \\
    \multicolumn{3}{l}{
      \comp{
        \lstinline'/*@ loop invariant' $p\semicolon$
        \lstinline'loop assigns' $X\semicolon$
        \lstinline'loop variant' $t\semicolon$
        \lstinline'*/ while(' $e$ \lstinline')'
        $\bopen A \bclose$}{$\mem$}
    } & \eqlabel{C-while} \\
    & = & $\errormem$ si \eval{$p$}{$\mem$} $= 0$ & \eqlabel{C-while-1} \\
    &  & $\mem$ si (\eval{$e$}{$\mem$}) $= 0$ & \eqlabel{C-while-2} \\
    &  & $\errormem$ si (\eval{$t$}{$\mem$}) $< 0$ & \eqlabel{C-while-3} \\
    &  & $\errormem$ si \eval{$p$}{(\comps{$A$}{$\mem$})} $= 0$
    & \eqlabel{C-while-4} \\
    &  & $\errormem$
    si (\eval{$t$}{(\comps{$A$}{$\mem$})}) $\ge$ (\eval{$t$}{$\mem$})
    & \eqlabel{C-while-5} \\
    &  & $\errormem$ si
    $\exists x. (x \in G-X.$
    (\eval{$x$}{(\comps{$A$}{$\mem$})}) $\ne$ (\eval{$x$}{$\mem$}) $)$
    & \eqlabel{C-while-6} \\
    &  & \comp{\lstinline'/*@ ... */ while('$e$\lstinline')'
      $\bopen A \bclose$}{(\comps{$A$}{$\mem$})} sinon
    & \eqlabel{C-while-7} \\
  \end{tabular}\vspace{-.5cm}
  \caption{Sémantique dénotationnelle des instructions}
  \label{fig:sem-instr}
\end{spacing}
\end{figure}


La sémantique des instructions C est donnée en figure~\ref{fig:sem-instr}.
La règle \eqlabel{C-err} indique que toute évaluation d'instruction à partir de
la mémoire d'erreur $\errormem$ produit $\errormem$.
Pour toutes les règles suivantes, nous supposons que l'évaluation se fait à
partir d'une mémoire différente de $\errormem$. \todo{règle \eqlabel{C-err}}
La règle \eqlabel{C-decl} alloue l'adresse $\loc$ à la variable $x$
dans l'environnement $\env$, où $\loc$ est une adresse fraîche (non utilisée).
Les règles \eqlabel{C-set} et \eqlabel{C-set-2} associent la valeur de $e$
à l'adresse de la left-value correspondante dans le store.

Dans les règles \eqlabel{C-Z-set} et \eqlabel{C-Z-unset} nous utilisons la
notation abrégée ${}^{\square}$\lstinline'var' pour indiquer que la variable
entière \lstinline'var' doit être déclarée et allouée au début de la portion de
code insérée, et la notation \lstinline'var'${}^{\boxtimes}$ pour indiquer que la
variable \lstinline'var' doit être désallouée à la fin du code inséré.
Cette notation est utilisée dans les instructions manipulant des données
entières non bornées.
La partie~\ref{sec:integers} illustre la détection de débordements d'entiers par
GMP~\cite{GMP}.
Dans la règle \eqlabel{C-Z-set}, la variable entière (non bornée) $x$ est
allouée et on lui affecte la valeur de $e$ dans $\store$.
Dans la règle \eqlabel{C-Z-unset}, la variable entière (non bornée) $x$ et son
adresse sont enlevées de la mémoire.
Les règles \eqlabel{C-Z-set} et \eqlabel{C-Z-unset} traitent uniquement le cas
où une left-value est un identificateur et pas un accès à un tableau car le
langage \eacsl ne supporte pas les tableaux d'entiers non bornés.

La sémantique de la fonction \lstinline'fassert' donnée par la règle
\eqlabel{C-fassert} énonce que si l'expression $e$ s'évalue en une expression
vraie (non nulle), alors la mémoire $\mem$ est inchangée, sinon on obtient
la mémoire d'erreur $\errormem$.
La sémantique de la fonction \lstinline'fassume' donnée par la règle
\eqlabel{C-fassume} énonce que l'expression $e$ est supposée vraie et
doit donc s'évaluer en une expression non nulle.
La mémoire est inchangée après cet appel de fonction.
$e$ ne peut pas s'évaluer en faux ($0$) par définition de la fonction
\lstinline'fassume', qui peut être vue comme un moyen de rajouter une
précondition à l'intérieur d'une fonction.

La sémantique des fonctions \lstinline'fvalid' et \lstinline'fvalidr' est donnée
par les règles \eqlabel{C-fvalid} et \eqlabel{C-fvalidr}.
Elles énoncent que dans la nouvelle mémoire, la left-value $\mathit{lv}$ a
la valeur vrai ($1$) si les pointeurs en argument de la fonction sont valides,
et faux ($0$) sinon.

La sémantique de la fonction \lstinline'malloc' donnée par la règle
\eqlabel{C-malloc} énonce que de nouvelles adresses (fraîches) sont associées
aux left-values allouées pour le tableau $x$, et que la valeur
associée à l'adresse de $x$ dans le store est l'adresse du premier
élément alloué ($\env(x$\lstinline'[0]'$)$), comme c'est le cas dans le langage
C.
Nous utilisons une fonction \lstinline'malloc' prenant deux paramètres :
$e$, le nombre d'éléments à allouer, et $s$, la taille (en octets) des éléments.
\todo{$e,s$}
Un pointeur alloué dynamiquement à l'aide de la fonction \lstinline'malloc' est
valide tant qu'il n'est pas libéré par la fonction \lstinline'free'.
La sémantique de la fonction \lstinline'free' donnée par la règle
\eqlabel{C-free} est l'inverse de la sémantique de \lstinline'malloc' : les
left-values allouées et l'adresse de la variable $x$ sont enlevées de la
mémoire.
Dans ces deux règles, $x$ est une variable de type pointeur, la
valeur associée à son adresse dans la règle \eqlabel{C-malloc} est donc une
adresse.
La variable $N$ dans la règle \eqlabel{C-free} est le nombre d'éléments alloués
dynamiquement dans le tableau $x$ par un appel à la fonction \lstinline'malloc'.
\todo{$N$}

La sémantique des appels de fonction est donnée par les règles \eqlabel{C-fct1}
et \eqlabel{C-fct2}.
Dans ces règles, la sémantique de la fonction $f$ est calculée à
l'aide de la fonction $\mathcal{F}$ donnée en figure~\ref{fig:sem-fct}.
$(\env[x_1 \mapsto \loc_1$, ...,
  $x_N \mapsto \loc_N], \store[\loc_1 \mapsto$
  \eval{$e_1$}{$(\env, \store)$}, ...,
  $\loc_N \mapsto$ \eval{$e_N$}{$(\env, \store)$} $])$
est la mémoire dans laquelle les paramètres effectifs (les expressions
$e_1$, ..., $e_N$) sont affectés aux paramètres formels (les identificateurs
$x_1$, ..., $x_N$).

%Les règles \eqlabel{C-fassert}, \eqlabel{C-fassume} et \eqlabel{C-free} sont
%des cas particuliers de la règle \eqlabel{C-fct1}.
%Les règles \eqlabel{C-fvalid}, \eqlabel{C-fvalidr} et \eqlabel{C-malloc} sont
%des cas particuliers de la règle \eqlabel{C-fct2}.

La règle \eqlabel{C-return} énonce qu'une instruction \lstinline'return' ne
modifie pas la mémoire.
En effet, le programme est dans une forme normalisée où la valeur de retour
d'une fonction est stockée dans une variable \lstinline'res'$_f$, où
$f$ est le nom de la fonction courante.
Cette variable est lue après l'appel de la fonction comme l'énonce la règle
\eqlabel{C-fct2}.

La sémantique des assertions \eacsl donnée par la règle \eqlabel{C-assert} est
similaire à la sémantique de la fonction \lstinline'fassert'.
Si le prédicat $p$ s'évalue en une expression non nulle,
la mémoire $\mem$ est inchangée, sinon la mémoire d'erreur $\errormem$ est
obtenue.

La sémantique des boucles annotées par un contrat contenant un invariant
$p$, une liste de termes $X$ et un variant $t$ est
donnée par la règle \eqlabel{C-while}.
Nous appelons $G$ l'ensemble des left-values dans la portée courante.
$G$ contient donc les left-values des paramètres formels de la fonction
courante et des variables locales définies avant la boucle.
Les différents cas possibles sont notés \eqlabel{C-while-1} à
\eqlabel{C-while-7} et sont ordonnés de telle sorte que la sémantique est
définie par le premier cas qui s'applique dans l'ordre 1 à 7.
Tout d'abord, si l'invariant de boucle n'est pas établi avant l'exécution du
corps de la boucle, alors la mémoire d'erreur $\errormem$ est obtenue
(\eqlabel{C-while-1}).
Si la condition de boucle $e$ s'évalue à $0$ (la boucle n'est pas exécutée),
alors la mémoire reste inchangée (\eqlabel{C-while-2}).
Si le variant de boucle n'est pas positif ou nul avant l'exécution de la boucle,
alors $\errormem$ est obtenue (\eqlabel{C-while-3}).
Si l'invariant de boucle n'est pas maintenu après l'exécution du corps de
boucle, alors $\errormem$ est obtenue (\eqlabel{C-while-4}).
Si le variant de boucle ne décroît pas strictement entre le début et la fin de
l'exécution du corps de boucle, alors $\errormem$ est obtenue
(\eqlabel{C-while-5}).
Si l'évaluation d'un élément appartenant à $G-X$ (ensemble des left-values
n'appartenant pas à la clause \lstinline'loop assigns' de la boucle) est
différente entre le début et la fin de l'exécution du corps de la boucle, alors
$\errormem$ est obtenue (\eqlabel{C-while-6}).
Si aucun des six cas précédents ne s'applique, alors il faut ré-évaluer la
sémantique de la boucle dans la mémoire obtenue après exécution du corps de la
boucle dans $\mem$ (\eqlabel{C-while-7}).
Nous supposons pour simplifier que la non-terminaison éventuelle d'une boucle
est toujours évitée grâce au variant.


\begin{figure}[h!]
  \begin{tabular}{p{.5cm} p{12cm} p{2cm}}
    \multicolumn{2}{l}{
      \compf{
        \lstinline'/*@ requires' $p_1\semicolon$
        \lstinline'assigns' $X\semicolon$
        \lstinline'ensures' $p_2\semicolon$
        \lstinline'*/' $f$\lstinline'('$\mathit{id_1}$, ..., $\mathit{id_N}$
        \lstinline')'
        $\bopen A \bclose$}{$\mem$}
    } & \eqlabel{F} \\
    =& $\errormem$ si \eval{$p_1$}{$\mem$} $= 0$ et $f$ est
    une fonction appelée & \eqlabel{F-1} \\
    & $\errormem$ si
    $\exists x. (x \in G-X.$
    (\eval{$x$}{(\comps{$A$}{$\mem$})}) $\ne$
    (\eval{$x$}{$\mem$}) $)$ & \eqlabel{F-2} \\
    & $\errormem$ si
    \eval{$p_2$}{(\comps{$A$}{$\mem$})} $= 0$
    & \eqlabel{F-3} \\
    & \comps{$A$}{$\mem$} sinon & \eqlabel{F-4} \\
  \end{tabular}
  \caption{Sémantique dénotationnelle des fonctions}
  \label{fig:sem-fct}
\end{figure}



La sémantique dénotationnelle des fonctions est exprimée avec la fonction
$\mathcal{F} : FCT \rightarrow MEM \rightarrow MEM$ où $FCT$ correspond à
l'entité syntaxique \textit{function} définie par la figure~\ref{fig:gram-c}.
$FCT$ est l'ensemble des déclarations de fonctions du programme.
Cette fonction calcule une nouvelle mémoire à partir d'une fonction annotée et
d'une mémoire $\mem$, ce qui permet de traiter les appels de fonction (règles
\eqlabel{C-fct1} et \eqlabel{C-fct2} de la figure~\ref{fig:sem-instr}) et la
fonction sous vérification.

La sémantique des fonctions est donnée en figure~\ref{fig:sem-fct} par la règle
\eqlabel{F}.
Les différents cas possibles sont notés \eqlabel{F-1} à \eqlabel{F-4} et sont
ordonnés comme pour le \eqlabel{while}.
Si $f$ est la fonction sous vérification, alors sa précondition $p_1$ est
supposée vraie.
Si $f$ n'est pas la fonction sous vérification, alors la mémoire d'erreur
$\errormem$ est obtenue si le prédicat $p_1$ s'évalue à faux (\eqlabel{F-1}).
Rappelons que nous appelons $G$ l'ensemble des left-values dans la portée
courante.
$G$ contient donc les left-values des paramètres formels de la fonction
courante.
Si l'évaluation d'un élément appartenant à $G-X$ (ensemble des left-values
n'appartenant pas à la clause \lstinline'assigns' de la fonction) est différente
entre le début et la fin de l'exécution du corps de la fonction, alors
$\errormem$ est obtenue (\eqlabel{F-2}).
Si le prédicat de la postcondition $p_2$ est faux à la fin de la fonction, alors
$\errormem$ est obtenue (\eqlabel{F-3}).
Enfin, si aucun des trois cas précédents ne s'est produit, le résultat est la
mémoire obtenue après exécution des instructions du corps de la fonction
(\eqlabel{F-4}).


\section*{Conclusion du chapitre}

Dans ce chapitre nous avons présenté un sous-ensemble du langage C et du
langage \eacsl normalisé que nous traitons.
Nous avons défini la sémantique dénotationnelle de chaque élément du langage
pour pouvoir justifier de la correction de la traduction des annotations en C
dans le but de générer des tests.
Le chapitre~\ref{sec:traduction} présente la traduction de ces annotations
\eacsl en code C et justifie qu'ils sont sémantiquement équivalents, en
utilisant les fonctions définissant la sémantique dénotationnelle.
