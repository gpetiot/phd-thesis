
\chapter{Découverte de Faiblesses de Sous-contrats}
\label{sec:swd}

\chapterintro

Dans ce chapitre, ...



\section{Illustrating Example}
\label{subsec:example}

We illustrate the issues arising in deductive verification of programs and the
solutions we propose on the example of C program of
Fig.~\ref{fig:rgf1}. It comes from an ongoing work on formal specification and
deductive verification
~\cite{Genestier/TAP15}
and implements an
algorithm proposed in \cite[page 235]{Arndt/10}.
The example of Fig.~\ref{fig:rgf1} 
concerns the generation of
\emph{Restricted Growth Functions} (RGF),
defined by the property expressed by the \acsl
predicate \lstinline{is_rgf} on lines 1--2 of Fig.~\ref{fig:rgf1}, where the RGF
$a$ is represented by the C array
of its values. 
For convenience of the reader, 
some \acsl notations are  replaced by mathematical symbols
(e.g. keywords \lstinline[style=c]{\exists}, \lstinline[style=c]{\forall} and
\lstinline[style=c]{integer} are respectively denoted by $\exists$, $\forall$ and $\mathbb{Z}$).


\begin{figure}[tb]
  \centering
  \lstinputlisting{listings/rgf_0.c}
  \caption{Successor function for restricted growth functions (RGF)\label{fig:rgf1}}
  \vspace{-5mm}
\end{figure}


Fig.~\ref{fig:rgf1} shows a main function \lstinline{f} 
and an auxiliary function \lstinline{g}.
The precondition of \lstinline{f} states that \lstinline{a} is a valid array 
of size  \lstinline{n>0} (lines 22--23) and must be an RGF (line 24).
The postcondition states 
that the function is only allowed to modify the values of array \lstinline{a} except
the first one \lstinline{a[0]} (line 25), 
and that the generated array \lstinline{a} is still an RGF (line 26).
Moreover, if the function returns 1 then the generated RGF \lstinline{a} must
respect an additional property
(lines 27--30).
Here \lstinline{\at(a[j],Pre)} denotes the value of \lstinline{a[j]} in the 
\lstinline{Pre} state, i.e. before the function starts execution.


We focus now on the body of the function \lstinline{f} in
Fig.~\ref{fig:rgf1}. 
The loop on lines 36--37 goes through the array from right
to left to find the rightmost non-increasing element, that is,
the maximal  array index \lstinline{i} such that \lstinline{a[i] <= a[i-1]}.
If such an index \lstinline{i} is found,
the function increments \lstinline{a[i]} (line 40) and fills out the rest
of the array with \lstinline{0}'s (call to \lstinline{g}, line 41).
The loop contract (lines 33--35) specifies the interval of values of the loop variable,
the variable that the loop can modify as well as a loop variant 
that can be used to ensure the termination of the loop.
The loop variant expression must be non-negative whenever an iteration starts,
and strictly decrease after it.

The function \lstinline{g} is used to fill the array with zeros to the right of index \lstinline{i}.
In addition to size and validity constraints (lines 7--8),
its precondition requires that the elements of \lstinline{a} up to index  \lstinline{i}
form an RGF (lines 9--10). 
The function is allowed to modify the elements of \lstinline{a} starting from the index \lstinline{i+1} (line 11)
and generates an RGF (line 12).
The loop invariants 
indicate the value interval of the loop variable \lstinline{k} (line 15), 
and state that the property \lstinline{is_rgf} is satisfied
up to \lstinline{k} (line 16).
This invariant allows a deductive verification tool to deduce the postcondition.
The annotation \lstinline{loop assigns} (line 17) says that the only values the loop can
change are \lstinline{k} and the elements of \lstinline{a} starting from the index \lstinline{i+1}.
The term \lstinline{n-k} is a variant of the loop (line 18).

The \acsl lemma \lstinline'max_rgf' on lines 4--5
states
that if an array is an RGF, then each of its elements is at most equal to its
index.
This lemma is not proved as such by \Wp but can be used to ensure the absence of
overflow at line 40.

The functions of Fig.~\ref{fig:rgf1} can be fully proved using \Wp.
Suppose now this example contains one of the  following four mistakes: 
the verification engineer \emph{either} forgets the precondition on line 24, 
\emph{or} writes the wrong assignment \lstinline[style=c]'a[i]=a[i]+2;' on line 40,
\emph{or} puts a too general clause \lstinline[style=c]'loop assigns i,a[1..n-1];' on line 34,
\emph{or} forgets to provide the lemma  on lines 4--5.
In each of these four cases, the proof  fails (for the precondition of \lstinline{g} on line 41 
and/or the assertion on line 39) for different reasons.
In fact, only in the first two cases the code and specification are not compliant, while
the third failure is due to a too weak subcontract, and the last one comes from a prover incapacity.
To the best of our knowledge, 
none of the existing  techniques allows to automatically
distinguish the three reasons and suggest suitable actions.
This work proposes a complete methodology to provide such assistance.



\begin{figure}[tb]
\begin{minipage}{0.4\columnwidth}
\begin{lstlisting}[mathescape]
/*@ assigns k1,...,kN;
  @ ensures P; */
$Type_g$ g(...){ code1; }




$Type_f$ f(...){ code2;
  g($Args_g$);
  code3; }
\end{lstlisting}
\end{minipage}
\hspace{-6mm}
\begin{minipage}{0.07\columnwidth}$\to$\end{minipage}
\begin{minipage}{0.6\columnwidth}
\begin{lstlisting}[mathescape]
$Type_g$ g_swd(...){
  k1=Nondet(); ... kN=Nondet();
  $Type_g$ ret = Nondet();
  int post; $Spec2Code$(P, post);
  fassume(post); return ret;
} //respects contract of g
$Type_g$ g(...){ code1; }
$Type_f$ f(...){ code2;
  g_swd($Args_g$);
  code3; }
\end{lstlisting}
\end{minipage}
\vspace{-5mm}
\caption{(a) A contract $c\in\C$ of callee $g$ in $f$, vs.
(b) its translation %in $P_\all^{\CW}$ 
for \CWD}
\vspace{-3mm}
\label{fig:CW-transf-functions}
\end{figure}



\begin{figure*}[tb]
\begin{center}
\begin{minipage}{0.8\columnwidth}
\begin{lstlisting}[mathescape]
$Type_f$ f(...){ code1;
  /*@ loop assigns x1,...,xN;
    @ loop invariant I; */
  while(b){ code2; }
  code3; }
\end{lstlisting}
\end{minipage}
\hspace{-6mm}
\begin{minipage}{0.07\columnwidth}$\to$\end{minipage}
\begin{minipage}{0.8\columnwidth}
\begin{lstlisting}[mathescape]
$Type_f$ f(...){ code1;
  x1=Nondet(); ... xN=Nondet();
  int inv1; $Spec2Code$(I, inv1);
  fassume(inv1 &$ $& !b); //respects l$ $oop contract
  code3; }
\end{lstlisting}
\end{minipage}
\vspace{-3mm}
\caption{(a) A contract $c\in\C$ of a loop in $f$, vs. 
(b) its translation %in $P_\all^{\CW}$ 
for \CWD}
\vspace{-3mm}
\label{fig:CW-transf-loops}
\end{center}
\end{figure*}




\subsection{Subcontract Weakness and Prover Incapacity}
\label{subsec:SW}

To introduce the new categories of proof failures, we follow
the modular verification approach and need a few definitions.
A \emph{non-imbricated} loop (resp. function, assertion) in $f$ is a loop
 (resp. function called, assertion) in $f$ outside any loop in
$f$. A \emph{subcontract for $f$} is the contract of some non-imbricated loop or function
 in $f$. A \emph{non-imbricated annotation} in $f$ is either a
non-imbricated assertion or an annotation in a subcontract for $f$.
For instance, the function $f$ of Fig.~\ref{fig:rgf1} has two subcontracts:
the contract of the called function $g$ and the contract of the loop on lines 33--37.
The contract of the loop in $g$ on lines 15--19 is not a subcontract for $f$, but is a subcontract for $g$.

We focus on non-imbricated annotations in $f$ and assume  that all subcontracts for $f$ are respected:
the called functions in $f$ respect their contracts, and the loops in $f$  preserve their loop invariants
and respect all imbricated annotations.
Let $c_f$ denote the contract of $f$,
$\C$ the set of non-imbricated subcontracts for $f$,
and $\A$ the set of all non-imbricated annotations in $f$ and the annotations of $c_f$.
In other words, $\A$ contains  
the annotations included in the contracts $\C\cup\{c_f\}$ as well as non-imbricated  assertions in $f$.
We also assume that any subcontract of $f$ contains a (loop) assigns clause.
This assumption is not restrictive since such a clause is anyway necessary 
for the proof of any nontrivial code.



\textbf{Subcontract weakness.}
To apply testing for  the contracts of called functions and loops in $\C$
instead of their code,
we use a program transformation of $P$ producing a new program $P^{\GSW}$.
The code of all non-imbricated function calls and loops in $f$
is replaced by a new one as follows.

For the contract $c\in\C$ of a called function $g$ in $f$, 
the program transformation  (illustrated by Fig.~\ref{fig:CW-transf-functions})
generates a new function \lstinline{g_swd} with the same signature 
whose code simulates any possible behavior respecting the postcondition in $c$, 
and replaces all calls to $g$ by  a call to \lstinline{g_swd}.
First, \lstinline{g_swd} allows any of the variables
(or, more generally, left-values) present in
the \lstinline{assigns} clause of $c$ to change its value 
(line 2 in Fig.\ref{fig:CW-transf-functions}(b)).
It can be realized by assigning a non-deterministic 
value of the appropriate type
using a dedicated function, denoted here by \lstinline{Nondet()}
(or simply by adding an array of fresh input variables and reading
a different value for each use and each function invocation).
If the return type of $g$ is not \lstinline{void},
another non-deterministic value is read for the returned value
\lstinline{ret} (line 3 in Fig.\ref{fig:CW-transf-functions}(b)).
Finally, the validity of the postcondition is evaluated (taking into
account these new non-deterministic values) 
and assumed
in order to consider only executions that respect 
the postcondition, and
the function returns (lines 4--5 in Fig.\ref{fig:CW-transf-functions}(b)).

Similarly, for the contract $c\in\C$ of a loop in $f$, 
the program transformation  
replaces the code of the loop by 
another code that simulates any possible behavior respecting $c$, that is, 
ensuring the ``loop postcondition'' $I\wedge \neg b$ after the loop as shown in Fig.~\ref{fig:CW-transf-loops}.
In addition, the transformation treats in the same way as in $P^{\NC}$
all other annotations in $\A$:
preconditions of called functions, initial loop invariant verifications 
and the  pre- and postcondition of $f$
(they are not shown 
in Fig. \ref{fig:CW-transf-functions}(b) and \ref{fig:CW-transf-loops}(b)).

\begin{tabular}{cc}
%\begin{figure}[tb]
  %\begin{minipage}{0.5\columnwidth}
  %\begin{figure}
    \lstinputlisting[label=fig:contracts-ex1,caption={Absence of \CWCE{}s for any single subcontract does not imply absence of global \CWCE{}s}]{listings/ex1.c}
    %\vspace{-2mm}
    %\caption{Absence of \CWCE{}s for any single subcontract does not imply absence of global \CWCE{}s}
    %\label{fig:contracts-ex1}
  %\end{figure}
  %\hspace{4mm}
  %\end{minipage}
  %\begin{minipage}{0.5\columnwidth}
  %\begin{figure}
&
    \lstinputlisting[label=fig:contracts-ex2,caption={Global \CWCE{}s do not help to find precisely a too weak subcontract}]{listings/ex2.c}
    %\vspace{-2mm}
    %\caption{Global \CWCE{}s do not help to find precisely a too weak subcontract}
    %\label{fig:contracts-ex2}
  %\end{figure}
  %\end{minipage}
  %\vspace{-2mm}
  %\caption{Two examples with several subcontracts}
  %\vspace{-5mm}
  %\label{fig:many-contracts}
%\end{figure}
\end{tabular}

\begin{definition}[Global subcontract weakness] 
\label{def:GSW}
We say that $P$ has a \emph{global subcontract weakness} for $f$ 
if there exists a test datum $V$ for $f$ respecting its precondition,
such that $P^{\NC}$ does not report any annotation failure on $V$,
while $P^{\GSW}$ reports an annotation failure on $V$.
In this case, we say that $V$ is a 
\emph{global subcontract weakness counter-example} (\GSWCE)
for the set  of subcontracts $\C$.
\end{definition}

Notice that we do not consider the same counter-example as an \NCCE  and 
an \SWCE. Indeed, even if some counter-examples may illustrate both a
subcontract weakness and a non-compliance, we consider that
non-compliances usually come
from a direct conflict between the code and the specification
and should be addressed first, while 
contract weaknesses are often more subtle and will be
easier to address when non-compliances are eliminated.

Again, test generation can be applied on  $P^{\GSW}$ to generate
\GSWCE candidates. When it finds a test datum $V$ such that 
$P^{\GSW}$ fails on $V$, we use runtime assertion checking:
if $P^{\NC}$ fails on $V$, then
$V$ is classified as an \NCCE, otherwise  $V$ is a \GSWCE.
We call this technique \emph{Global Subcontract Weakness Detection} for the set of 
all subcontracts, denoted $\GSWD$. 
The $\GSWD$ step may have four outcomes.
It returns (\nc,\,$V$,\,$a$) if an \NCCE $V$ has been found for the failing
annotation $a$,
and  (\cw,\,$V$,\,$a$,\,$C$) if $V$ has been finally classified as an \SWCE
indicating the failing annotation $a$ and the set of subcontracts $C$.
The program path $\pi_V$ activated by $V$ and leading to the failure 
(on $P^{\NC}$ or $P^{\GSW}$) is recorded as well.
If the $\GSWD$ has managed to perform a complete exploration 
of all program paths without finding an \GSWCE, it returns \no.
Otherwise, if only a partial exploration of program paths
has been performed it returns \textsf{?} (unknown).




A \GSWCE indicates a global subcontract weakness but
does not explicitly identify which single subcontract $c\in\C$ is too 
weak.
To do that, we propose another program transformation of $P$ into
an instrumented program  $P_c^{\SSW}$.
It is realized by replacing only one non-imbricated function call or loop 
by the code respecting the postcondition of corresponding subcontract $c$ 
(as indicated in Fig. \ref{fig:CW-transf-functions} and \ref{fig:CW-transf-loops})
and transforming other annotations in $\A$ as in $P^{\NC}$.


\begin{figure*}[bt]\centering
\begin{tikzpicture}
  \node(p) [data] {$P$};
  \node(ncd) [right of=p,test,node distance=1.7cm] {$\NCD(P)$};
  \node(ncce1) [below of=ncd,node distance=1.2cm] {\circled{1} $V$ is \NCCE};
  \path[darrow] (ncd) -- node[left] {(\nc, $V$, $a$)} (ncce1);
  \path[arrow] (p) -- (ncd);
  \node(cwd) [right of=ncd,test,node distance=4cm] {$\CWD(P)$};
  \path[darrow] (ncd.east) -- node[above] {\no{} / \textsf{?}} (cwd);
  %\path[darrow] (ncd.east)+(0,.2) -- node[above] {\no} (cwd);
  %\path[darrow] (ncd.east)+(-.1,-.2) -- node[below] {?} (cwd);

  % CWD did not found any CE
  \node(cov-2) [right of=cwd,test,node distance=4cm]{
    $\NCD(P)=~$\no$\land$\\ $\CWD(P)=~$\no};
  \path[darrow] (cwd.east) -- node[above] {\no{} / \textsf{?}} (cov-2);
  %\path[darrow] (cwd.east)+(0,.2) -- node[above] {\no} (cov-2);
  %\path[darrow] (cwd.east)+(-.1,-.2) -- node[below] {?} (cov-2);
  \node(pw) [below of=cov-2,node distance=1.2cm]{\circled{3} Prover incapacity};
  \path[darrow] (cov-2) -- node[right] {true} (pw);
  \node(qm) [right of=pw,node distance=4cm] {\circled{4} Unknown};
  \path[darrow] (cov-2) -| node[below left] {false} (qm);

  % CWD found a CE
  \node(cwce-ncce) [below of=cwd,node distance=1.2cm]{\circled{2} $V$ is \CWCE};
  \path[darrow] (cwd) -- node[right] {(\cw, $V$, $a$, $S$)} (cwce-ncce);
  \path[darrow] (cwd) -- node[above,sloped,xshift=-3mm] {~~~(\nc, $V$, $a$)} (ncce1);
\end{tikzpicture}
\vspace{-2mm}
\caption{Combined verification methodology in case of a proof failure on $P$}
\vspace{-5mm}
\label{fig:method-short}
\end{figure*}


\begin{definition}[Single subcontract weakness] 
\label{def:SSW}
Let $c$ be a subcontract for $f$.
We say that $c$ is a \emph{too weak subcontract}
(or has a \emph{single subcontract weakness}) for $f$ 
if there exists a test datum $V$ for $f$ respecting its precondition,
such that $P^{\NC}$ does not report any annotation failure on $V$,
while $P_c^{\SSW}$ reports an annotation failure on $V$.
In this case, we say that $V$ is a \emph{single subcontract weakness counter-example} (\SSWCE)
for the subcontract  $c$ in $f$.
\end{definition}

For any subcontract $c\in\C$, test generation can be separately applied on  $P_{c}^{\SSW}$ to generate
\SSWCE candidates. If such a test datum $V$ is generated, it is checked on $P^{\NC}$ to 
classify it as an \NCCE or an \SSWCE.
We call this technique,
applied for all subcontracts one after another until a first counter-example  $V$ is found,  
\emph{Single Contract Weakness Detection,}
denoted $\SSWD$. 
The $\SSWD$ step may have three outcomes.
It returns (\nc,\,$V$,\,$a$) if an \NCCE $V$ has been found for a failing annotation $a$,
and  (\cw,\,$V$,\,$a$,\,$\{c\}$) if $V$ has been finally classified as an \SSWCE
indicating the failing annotation $a$ and the single too weak subcontract $c$.
The program path $\pi_V$ activated by $V$ and leading to the failure 
(on $P^{\NC}$ or $P_c^{\SSW}$) is recorded as well.
Otherwise, it returns \textsf{?} (unknown), since even after an exhaustive path testing
the absence of \SSWCE for any individual subcontract $c$ does not imply
the absence of a \GSWCE.

Indeed, sometimes $\SSWD$ cannot exhibit a subcontract weakness for a single subcontract
while there is a global subcontract weakness for all of them at once.
For example in Fig.~\ref{fig:contracts-ex1},
if we apply $\SSWD$ to any of the subcontracts, we always have
\lstinline'x >=  \old(x)+5' at the end of $f$
(we add $1$ to $x$ by executing the translated subcontract,
and add $2$ twice by executing the other two functions' code),
so the postcondition of \lstinline'f' holds and no weakness is detected.
If we run  $\GSWD$  to consider all subcontracts at once, we only get
\lstinline'x>= \old(x)+3' after executing the three subcontracts, and can exhibit a counter-example.

On the other hand, running  $\GSWD$
produces a \GSWCE that does not indicate which one of the subcontracts is too weak,
while $\SSWD$ can sometimes be more precise.
For Fig.~\ref{fig:contracts-ex2},
since the three callees are replaced by their subcontracts for  $\GSWD$,  it is impossible
to find out which one is too weak.
Counter-examples generated by a prover suffer from the same precision issue:
taking into account all subcontracts instead of the corresponding code 
prevents from a precise identification of a single too week subcontract.
In this example we can be more precise with $\SSWD$,
since only the replacement of the  subcontract of  \lstinline'g3' also leads to an \SSWCE:
we can have \lstinline'x >= \old(x)+3' by executing
\lstinline'g1',  \lstinline'g2' and the subcontract of \lstinline'g3',
exhibiting the contract weakness of \lstinline'g3'.
Thus, the proposed $\SSWD$  technique can provide the verification engineer
with a more precise diagnostic than counter-examples extracted from a prover.

We define a combined subcontract weakness detection technique, denoted \SWD, applying
first $\SSWD$ followed by  $\GSWD$ until the first \SWCE is found. 
$\SWD$ may have the same four outcomes as  $\SSWD$.
It allows us
to be both precise (and indicate when possible a single subcontract being too weak),
and complete (capable to find \GSWCE{}s even when there are no single subcontract weaknesses).



\textbf{Prover incapacity.}
When neither a non-compliance nor a global subcontract weakness
exist, we cannot demonstrate that it is impossible to prove the property.

\begin{definition}[Prover incapacity] 
\label{def:prov-incap}
We say that a proof failure in $P$ is due to a \emph{prover incapacity} 
if for any test datum $V$ for $f$ respecting its precondition,
neither $P^{NC}$ nor $P^{\GSW}$ report any annotation failure on $V$.
In other words, there is no \NCCE and no \GSWCE for $P$.
\end{definition}




%%%%%%%%%%%%%%%%%%%%%%%%%%%%%%%%%%%%%%%%%%%%%%% TODO

\begin{figure}[bt]
  \scriptsize{
    {\myinference{$\alpha$-assume-assigns}
      {}
      {(End_f, \mbox{\lstinline'assigns x;'})
        \rulearrow
        I \concat (End_f, \mbox{\lstinline'x = nondet();'})}{}
    }

    {\myinference{$\alpha$-assume-post}
      {(End_f, \mbox{\lstinline'p'}) \rulearrow (I, \mbox{\lstinline'e'})}
      {(End_f, \mbox{\lstinline'ensures p;'})
        \rulearrow
        I \concat (End_f, \mbox{\lstinline'fassume(e);'})}{}
    }
  }
  \caption{Règles de traduction pour les faiblesses de contrats :
    postconditions et assigns}
  \label{fig:assume-annot}
\end{figure}

La règle \textsc{$\alpha$-assume-assigns} suppose que toutes les l-values
présentes dans la clause \lstinline'assigns' ont changé de valeur : elle affecte
une nouvelle valeur non déterministe à chacune d'elle à la fin de la fonction.
La règle \textsc{$\alpha$-assume-post} suppose sa validité à la fin de la
fonction instrumentée.

\begin{figure}[bt]
  \scriptsize{
    {\myinference{$\alpha$-assume-loop-assigns}
      {}
      {(l, \mbox{\lstinline'loop assigns x;'})
        \rulearrow
        I \concat (l, \mbox{\lstinline'x = nondet();'})}{}
    }

    {\myinference{$\alpha$-assume-invariant-1}
      {(l, \mbox{\lstinline'p'}) \rulearrow (I, \mbox{\lstinline'e'})}
      {
        (l, \mbox{\lstinline'loop invariant p;'}) \rulearrow
        I \concat (l, \mbox{\lstinline'fassume(e && !loopcond);'})
      }{}
    }

    {\myinference{$\alpha$-assume-invariant-2}
      {(l, \mbox{\lstinline'p'}) \rulearrow (I_1, \mbox{\lstinline'e1'}) \\
        (EndIter_l, \mbox{\lstinline'p'}) \rulearrow
        (I_2, \mbox{\lstinline'e2'})}
      {
        (l, \mbox{\lstinline'loop invariant p;'}) \rulearrow
        I_1 \concat (l, \mbox{\lstinline'fassume(e1);'})
        \concat I_2 \concat (EndIter_l, \mbox{\lstinline'fassert(e2);'})
      }{}
    }
  }
  \caption{Règles de traduction pour les faiblesses de contrats :
    invariants et assigns de boucle}
  \label{fig:assume-loop-annot}
\end{figure}

La règle \textsc{$\alpha$-assume-loop-assigns} suppose que toutes les l-values
présentes dans la clause \lstinline'loop assigns' ont changé de valeur : une
nouvelle valeur non déterministe est affectée à chacune d'elle.
La règle \textsc{$\alpha$-assume-invariant-1} a vocation à remplacer toute la
boucle, elle suppose que le prédicat de l'invariant est vrai, ainsi que la
négation de la condition de boucle (appelée ici \lstinline'loopcond').
La règle \textsc{$\alpha$-assume-invariant-2} permet de supposer que la boucle
a déjà itéré un certain nombre de fois, elle suppose que l'invariant est vrai
avant la boucle, et vérifie l'invariant à la fin de chaque itération.


\section*{Conclusion du chapitre}

\commentGP{TODO}
