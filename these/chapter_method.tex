
\chapter{Méthode d'Aide à la Preuve combinant NCD et SWD}
\label{sec:method}

\chapterintro

TODO



In this section, we present an overview of our method for 
diagnosis of proof failures using the detection techniques of Sec.~\ref{sec:definitions},
and illustrate it on several examples.
We also provide a comprehensive list of  suggestions of actions for 
each category of proof failures.



\begin{figure*}[bt]\centering
\begin{tikzpicture}
  \node(p) [data] {$P$};
  \node(ncd) [right of=p,test,node distance=1.7cm] {$\NCD(P)$};
  \node(ncce1) [below of=ncd,node distance=1.2cm] {\circled{1} $V$ is \NCCE};
  \path[darrow] (ncd) -- node[left] {(\nc, $V$, $a$)} (ncce1);
  \path[arrow] (p) -- (ncd);
  \node(cwd) [right of=ncd,test,node distance=4cm] {$\CWD(P)$};
  \path[darrow] (ncd.east) -- node[above] {\no{} / \textsf{?}} (cwd);
  %\path[darrow] (ncd.east)+(0,.2) -- node[above] {\no} (cwd);
  %\path[darrow] (ncd.east)+(-.1,-.2) -- node[below] {?} (cwd);

  % CWD did not found any CE
  \node(cov-2) [right of=cwd,test,node distance=4cm]{
    $\NCD(P)=~$\no$\land$\\ $\CWD(P)=~$\no};
  \path[darrow] (cwd.east) -- node[above] {\no{} / \textsf{?}} (cov-2);
  %\path[darrow] (cwd.east)+(0,.2) -- node[above] {\no} (cov-2);
  %\path[darrow] (cwd.east)+(-.1,-.2) -- node[below] {?} (cov-2);
  \node(pw) [below of=cov-2,node distance=1.2cm]{\circled{3} Prover incapacity};
  \path[darrow] (cov-2) -- node[right] {true} (pw);
  \node(qm) [right of=pw,node distance=4cm] {\circled{4} Unknown};
  \path[darrow] (cov-2) -| node[below left] {false} (qm);

  % CWD found a CE
  \node(cwce-ncce) [below of=cwd,node distance=1.2cm]{\circled{2} $V$ is \CWCE};
  \path[darrow] (cwd) -- node[right] {(\cw, $V$, $a$, $S$)} (cwce-ncce);
  \path[darrow] (cwd) -- node[above,sloped,xshift=-3mm] {~~~(\nc, $V$, $a$)} (ncce1);
\end{tikzpicture}
\vspace{-2mm}
\caption{Combined verification methodology in case of a proof failure on $P$}
\vspace{-5mm}
\label{fig:method-short}
\end{figure*}




\textbf{The method.}
The proposed method is illustrated by Fig.~\ref{fig:method-short}.
Suppose that the proof of the annotated program $P$ fails for some non-imbricated annotation $a\in\A$.
The first step tries to find a non-compliance using \NCD. 
If such a non-compliance is found, it generates an \NCCE (marked by \circled{1}
in Fig.~\ref{fig:method-short})
and classifies the proof failure as a non-compliance.
If the first step cannot generate a counter-example,
the \SWD step combines $\SSWD$ and $\GSWD$ 
and tries to generate single \SWCE{}s, then global \SWCE{}s, 
until the first counter-example is generated and classified 
(either as an \NCCE \circled{1} or an \SWCE \circled{2}).
If no counter-example has been found, the last step checks the outcomes.
If both \NCD and \SWD have returned \textsf{no}, that is, 
both  $\NCD$  and $\GSWD$ have performed a complete path exploration 
without finding a counter-example,
the proof failure is classified as a prover incapacity \circled{3} (cf. Def. \ref{def:prov-incap}).
Otherwise, it  remains unclassified \circled{4}.
Fig.~\ref{tab:versions-rgf} associates a variant of the illustrating example to
each case.
For each case, we detail the lines we modified in the program of
Fig.~\ref{fig:rgf1} to obtain a new program, the intermediate results of
deductive verification, \NCD and \SWD and the final verdict (including the
generated counter-example if any).


\begin{figure*}[bt]\centering
  \includegraphics[scale=.85]{table_rgf.pdf}
  \caption{Method results for different versions of the illustrating example.}
  \label{tab:versions-rgf}
\end{figure*}


The proof failure category and the counter-example $V$, along with  
the recorded path $\pi_V$,
the reported failing annotation $a$ and set of too weak subcontracts $S$,
can be extremely helpful for the verification engineer.
Suppose we try to prove in \Wp a modified version of the function $f$ of
Fig.~\ref{fig:rgf1}
where the precondition at line 24 is missing.
The proof of the precondition of $g$ on line 10 for the call on line 41
fails without indicating a precise reason.
The \NCD step of \stady  generates an \NCCE (case \circled{1},
\#1 in Fig.~\ref{tab:versions-rgf}) where \lstinline'is_rgf(a,n)'
is clearly false due to \lstinline'a[0]' being non-zero, and indicates the
failing annotation (coming from line 10).  
That helps the verification engineer to understand and fix the issue. 



Let us suppose now that the clause on line 34 has been erroneously
written as follows: \lstinline'loop assigns i, a[1..n-1];'.
The loop on lines 36--37 still preserves its invariant.
The \NCD step does not find any \NCCE, as this modification did not introduce
any non-compliance between the code and its specification.
Thanks to the replacement shown in Fig.~\ref{fig:CW-transf-loops},
$\SSWD$ for the contract of this loop will detect a single
subcontract weakness for the loop contract (case \circled{2},
\#2 in Fig.~\ref{tab:versions-rgf}),
and report a fail to establish the 
precondition of $g$ (on line 10) for the call on line 41.
With the indication of the single subcontract weakness for the loop, 
the verification engineer will try to strengthen the loop contract
and find the issue. 



Suppose now we want to prove the absence of overflow at line 40
of Fig.~\ref{fig:rgf1}, but the lemma on lines 4--5 
(that allows the prover to deduce this property) is missing.
The proof fails  without giving a precise reason since
the prover does not perform the induction needed to deduce the right bounds on
\lstinline'a[i]'.
Neither \NCD nor \CWD can produce a counter-example, and
as the initial program has too many paths, their outcomes are \textsf{?}
(unknown) (case \circled{4}, \#4 in Fig.~\ref{tab:versions-rgf}).
For such situations, \stady offers the possibility to reduce the input domain.
The verification engineer can add the \acsl clause 
\lstinline'typically n<5;' to reduce the array size 
for testing (this clause is ignored by the proof). 
Running \stady now allows the tool to complete the exploration of all
program paths (for \lstinline'n<5') both for \NCD and \CWD without finding a counter-example.
\stady classifies the proof failure for the program with 
the reduced domain as a prover incapacity (case \circled{3},
\#3 in Fig.~\ref{tab:versions-rgf}).
That gives the verification engineer more confidence that the proof failure
has the same reason on the initial program for bigger sizes \lstinline{n}.

The verification engineer prefers to  try interactive proof or adding additional
lemmas or assertions,
and does not waste time looking for a bug or a too week subcontract.





\textbf{Prover incapacity.}
When neither a non-compliance nor a global subcontract weakness
exist, we cannot demonstrate that it is impossible to prove the property.

\begin{definition}[Prover incapacity] 
\label{def:prov-incap}
We say that a proof failure in $P$ is due to a \emph{prover incapacity} 
if for any test datum $V$ for $f$ respecting its precondition,
neither $P^{NC}$ nor $P^{\GSW}$ report any annotation failure on $V$.
In other words, there is no \NCCE and no \GSWCE for $P$.
\end{definition}



\textbf{Suggestions of actions.}
From the possible outcomes of the method illustrated in
Fig.~\ref{fig:method-short} we are able to suggest to the verification engineer 
the most suitable actions (displayed in Fig.~\ref{fig:suggestions})
to help her with the verification task.
A \emph{non-compliance} of the code w.r.t. annotation $a$  means that 
there is an inconsistency between the precondition, the annotation $a$ and the code 
of the path $\pi_V$  leading to $a$.
Thanks to the counter-example, 
the values of variables at different program points along $\pi_V$ 
can be either traced or explored in a debugger \cite{Muller/FM11}. 
In \framac, the execution on $V$ can be 
conveniently explored using \Value or \pathcrawler.
This helps the verification engineer to understand the issue.
Indeed, if an \NCCE is generated, there is no need to
try automatic proof or look for a too weak subcontract --- it will not help.
The reason of the proof failure is necessarily related 
to a non-compliance between 
the code and annotations
traversed by the path $\pi_V$.

A \emph{weakness} of a set of subcontracts $S$ means that at least one of the contracts of $C$
has to be strengthened. By Definitions \ref{def:GSW} and \ref{def:SSW}, the non-compliance is excluded here, 
that is, the execution of $P^{\NC}$ on  $V$  respects the annotation $a$, thus
the suggested action is to strengthen the subcontract(s).
In the case of single subcontract weakness, $S$ is a singleton so the suggestion
is very precise and helpful to the user.
Again, trying interactive proof or additional assertions or lemmas 
will be useless here since the property can obviously not be proved 
because of the counter-example.
For a \emph{prover incapacity,} the verification engineer
may write lemmas or assertions, add hypotheses that may help the theorem prover to
succeed or try another theorem prover.
She also may want to use a proof assistant like \textsc{Coq}, so that she does
not suffer from the limitations of the theorem provers, but 
this task can be more complex and time-consuming.
Finally, when the verdict is \emph{unknown,} test generation for \NCD and/or \SWD times out, 
so the verification engineer may strengthen the
precondition for testing to reduce the input domain, or extend the timeout to
give \stady more time to conclude.


\begin{figure*}[bt]\centering
  \begin{tabular}{p{.7cm}|>{\centering\arraybackslash}p{5.8cm}|>{\centering\arraybackslash}p{8cm}}
    \textbf{Case} & {\centering\textbf{Verdict}} & \textbf{Suggestions} \\
    \hline
    \circled{1} & Non-compliance w.r.t. the annotation $a$:
    (\nc, $V$, $a$)
    &
    check the violated annotation $a$
    or the code leading to $a$ in the path $\pi_V$,
    or strengthen the precondition of the function under verification
    \\
    \hline
    \circled{2} & Weakness of subcontracts in $S$ w.r.t. the annotation $a$:
    \ (\cw,~$V$,~$a$,~$S$)
    & strengthen one or several subcontracts in $S$ to exclude the subcontract weakness\\
    \hline
    \circled{3} & Prover incapacity
    & add lemmas or assertions to help the theorem prover,
    or use another prover,
    or an interactive  proof assistant \\
    \hline
    \circled{4} & Unknown
    & strengthen the $\mbox{\lstinline'typically'}$ clause or coverage criterion (e.g. $k$-path),
    or increase the timeout limit for testing \\
  \end{tabular}
  \caption{Suggestions of actions for different categories of proof failures}
  \label{fig:suggestions}
\end{figure*}


\section*{Conclusion du chapitre}

TODO
