
\documentclass[french]{spimufcphdthesis}

\usepackage[utf8]{inputenc}
\usepackage[T1]{fontenc}
\usepackage[frenchb]{babel}
\usepackage[nounderscore]{syntax}
\usepackage{listings,courier,multirow,colortbl,hyperref,url,etex,pstricks-add,
  lscape,tikz,pgfplots}
\usetikzlibrary{matrix,shapes,backgrounds,fit}

\tikzstyle{pnode}=[rounded corners,rectangle,inner sep=1mm,draw]
\tikzstyle{pleaf}=[pnode,fill=gray!30]
\tikzstyle{ppcommon}=[inner sep=1mm,font=\scriptsize]
\newcommand\ppnode[1]{\tikz{\node[pnode,ppcommon]{#1};}}
\newcommand\ppleaf[1]{\tikz{\node[pleaf,ppcommon]{#1};}}

\tikzstyle{memcell}=[minimum width=1cm,minimum height=1.5em,outer sep=0pt,
  draw=black,semithick]
\tikzstyle{empty}=[]
\tikzstyle{full}=[fill=blue!20]
\tikzstyle{init}=[fill=green!20]
\tikzstyle{uninit}=[fill=white]




\def\danger{\begingroup
\psset{unit=1ex}%
\begin{pspicture}(0,0)(3,3)
\pspolygon[linearc=0.2,linewidth=0.12,linecolor=red](0,0)(1.5,2.6)(3,0)
\psellipse*(1.5,1.33)(0.14,0.75)\pscircle*(1.5,0.3){0.15}\end{pspicture}
\endgroup}


\newcommand{\tool}[1]{\textsc{\small #1}\xspace}

\newcommand{\stady}{\tool{StaDy}}
\newcommand{\framac}{\tool{Frama-C}}
\newcommand{\pathcrawler}{\tool{PathCrawler}}
\newcommand{\acsl}{\tool{acsl}}
\newcommand{\eacsl}{\tool{e-acsl}}
\newcommand{\Wp}{\tool{Wp}}
\newcommand{\Value}{\tool{Value}}
\newcommand{\sante}{\tool{sante}}
\newcommand{\eacsltoc}{\tool{e-acsl2c}}
\newcommand{\cil}{\tool{CIL}}
\newcommand{\jml}{\tool{JML}}
\newcommand{\colibri}{\tool{Colibri}}


\newcommand{\citeframac}{\cite{Frama-C}\xspace}
\newcommand{\citepathcrawler}{\cite{PathCrawler}\xspace}
\newcommand{\citevalue}{\cite{DBLP:conf/scam/CanetCM09}\xspace}
\newcommand{\citeeacsl}{\cite{DBLP:conf/sac/DelahayeKS13}\xspace}
\newcommand{\citesante}{\cite{DBLP:conf/sac/ChebaroKGJ12}\xspace}
\newcommand{\citeeacsltoc}{\cite{DBLP:conf/rv/KosmatovPS13}\xspace}

\newcommand{\SliceTest}{\ensuremath{\mbox{Slice\,\&\,Test}}}
\newcommand{\dep}{\leadsto}
\newcommand{\alarms}{\mathop{\textit{alarms}}\nolimits}


\newtheorem{theorem}{Theorem}
\newtheorem{proposition}[theorem]{Proposition}
\newtheorem{lemma}[theorem]{Lemma}

\newcommand{\commentAG}[1]{\colorbox{red}{{\color{white}((AG: {#1}))}}}
\newcommand{\commentGP}[1]{\colorbox{violet}{{\color{white}((GP: {#1}))}}}
\newcommand{\commentNK}[1]{\colorbox{blue}{{\color{white}((NK: {#1}))}}}
\newcommand{\commentJJ}[1]{\colorbox{green!60!black}{{\color{white}((JJ: {#1}))}}}

\lstloadlanguages{C}

% Warnings: 
% - Do not use other quantified variables than i, j and k
% - Do not add 
\lstdefinelanguage{pretty-ACSL}{%
  escapechar={},
  literate=
   {==}{{$==$}}2
   {==>}{{$\Rightarrow\ $}}1
   %{integer\ i}{{i$\,\in \mathbb{Z}\,$}}4
   %{integer\ j}{{j$\,\in \mathbb{Z}\,$}}4
   %{integer\ k}{{k$\,\in \mathbb{Z}\,$}}4
   %{integer\ m}{{m$\,\in \mathbb{Z}\,$}}4
   %{integer\ l}{{l$\,\in \mathbb{Z}\,$}}4
   %{\\sum(x,\ y,\ \\lambda\ integer\ k;\ t(k))}{{$\sum_{k=x}^y t_k$}}2
   %{\\product(x,\ y,\ \\lambda\ integer\ k;\ t(k))}{{$\prod_{k=x}^y t_k$}}2
   %{\\numof(x,\ y,\ \\lambda\ integer\ k;\ t(k))}{{$\#_{k=x}^y t_k$}}2
   %{\\min(x,\ y,\ \\lambda\ integer\ k;\ t(k))}{{$MIN_{k=x}^y t_k$}}2
   %{\\max(x,\ y,\ \\lambda\ integer\ k;\ t(k))}{{$MAX_{k=x}^y t_k$}}2
   %{\\lambda\ integer\ i;}{{$\lambda i . $}}2
   %{\\forall}{{$\forall$}}1
   %{\\exists}{{$\exists$}}1
   %{integer}{{$\mathbb{Z}$}}1
   {real}{{$\mathbb{R}$}}1
   {&&}{{$\wedge\ $}}1
   {||}{{$\vee\ $}}1
   {!=}{{$\neq$}}1
   {<}{{$<$}}1
   {<=}{{$\le$}}1
   {>}{{$>$}}1
   {>=}{{$\ge$}}1
   {<==>}{{$\Leftrightarrow\ $}}1,
  morekeywords={assert,assigns,assumes,axiom,axiomatic,behavior,behaviors,
    boolean,breaks,complete,continues,data,decreases,disjoint,ensures,
    exit_behavior,ghost,global,inductive,invariant,lemma,logic,loop,
    model,predicate,reads,requires,sizeof,strong,struct,terminates,
    type,union,variant,uchar,byte,typically,\\offset,\\result,\\old,\\at,\\valid,
    \\separated,\\nothing,Pre,Old,\\true,\\false,\\fresh,\\min,\\max,\\numof,
    \\sum,\\product,\\lambda,real,integer,postcondition,\\exists,\\forall},
  alsoletter={\\},
  keywordsprefix={\\},
  morecomment=[l]{//}
}
\lstnewenvironment{pretty-codeACSL}{\lstset{language=pretty-ACSL,stepnumber=0}}{\smallskip}

\lstdefinelanguage{ACSL}{%
  escapechar={},
  literate=,
  morekeywords={assert,assigns,assumes,axiom,axiomatic,behavior,behaviors,
    boolean,breaks,complete,continues,data,decreases,disjoint,ensures,
    exit_behavior,ghost,global,inductive,invariant,lemma,logic,loop,
    model,predicate,reads,requires,sizeof,strong,struct,terminates,postcond,precond,
    type,union,variant,uchar,byte,typically,\\offset,\\result,\\old,\\at,\\valid,
    \\separated,\\nothing,Pre,Old,\\true,\\false,\\fresh,\\min,\\max,\\numof,
    \\sum,\\product,\\lambda,real,integer,postcondition,\\exists,\\forall},
  alsoletter={\\},
  keywordsprefix={\\},
  morecomment=[l]{//}
}
\lstnewenvironment{codeACSL}{\lstset{language=ACSL,stepnumber=0}}{\smallskip}



\lstdefinestyle{pretty-c}{language={[ANSI]C},%
  alsolanguage=pretty-ACSL,%
  moredelim={*[l]{//}},%
  moredelim={*[s]{/*}{*/}},%
  moredelim={**[s]{/*@}{*/}},%
  deletecomment={[s]{/*}{*/}},
  moredelim={*[l]{//@}},%
}

\lstdefinestyle{c}{language={[ANSI]C},%
  alsolanguage=ACSL,%
  moredelim={*[l]{//}},%
  moredelim={*[s]{/*}{*/}},%
  moredelim={**[s]{/*@}{*/}},%
  deletecomment={[s]{/*}{*/}},
  moredelim={*[l]{//@}},%
}

\lstset{language=C,
  escapechar=§,
  style=c,
  basicstyle=\ttfamily\scriptsize,
  numberstyle=\tiny,
  numbers=left,
  stepnumber=1,
  numbersep=5pt,
  tab=\rightarrowfill,
  commentstyle=\color{green!40!black},
  keywordstyle={\bfseries\color{blue}},
  showstringspaces=false,
  stringstyle={\rmfamily\color{gray}},
  breaklines,
  breakatwhitespace,
  captionpos=b,
  float=tpb
}

%%%%%%%%%%%%%%%%% from Frama-C

\definecolor{lstnum}{gray}{0.4}
\definecolor{lstmodule}{rgb}{0.3,0.6,0.2}

\lstdefinestyle{framac-code-style}{%
basicstyle=\ifmmode\normalfont\mathtt\scriptstyle\else\normalfont\ttfamily\mdseries\small\fi,%
numberstyle=\rmfamily\tiny\color{lstnum},%
keywordstyle=[1]\sffamily\color{lstmodule},%
keywordstyle=[2]\sffamily\color{lstspecial},%
keywordstyle=[3]\sffamily\bfseries,%
identifierstyle=\rmfamily,%
stringstyle=\ttfamily\color{lstfg},%
commentstyle=\rmfamily\bfseries\color{lsttxt},%
}

\lstdefinelanguage{Ocaml}[Objective]{Caml}{%
style=framac-code-style,%
deletekeywords={when,module,struct,sig,begin,end},%
morekeywords=[2]{failwith,raise,when},%
morekeywords=[3]{module,struct,sig,begin,end},%
literate=%
{~}{${\scriptstyle\thicksim}$}1%
{<}{$<$}1%
{>}{$>$}1%
{->}{$\rightarrow$ }1%
{<-}{$\leftarrow$}1%
{:=}{$\leftarrow$}1%
{<=}{$\leq$}1%
{>=}{$\geq$}1%
{==}{$\equiv$}1%
{!=}{$\not\equiv$}1%
{<>}{$\neq$}1%
{'a}{$\alpha$}1%
{'b}{$\beta$}1%
{'c}{$\gamma$}1%
{µ}{`{}}1%
}
\lstdefinestyle{ocaml-basic}%
{language=Ocaml,basicstyle=\ifmmode\normalfont\mathtt\scriptstyle\else\normalfont\ttfamily\mdseries\small\fi}
\newcommand{\ocamlinput}[2][]{\lstinputlisting[style=ocaml-basic,#1]{#2}}
\lstnewenvironment{ocamlcode}[1][]{\lstset{style=ocaml-basic,#1}}{}
\newcommand{\ocamlinline}[1]{\lstinline[style=ocaml-basic]'#1'}



\declarethesis{Contribution à la Vérification de Programmes C par Combinaison
  Tests et Preuves}{\today}{reference number}
\addauthor{Guillaume}{Petiot}
\addjury{Firstname}{Lastname}{Role}{Job}

\thesisabstract[english]{Abstract}
\thesiskeywords[english]{keyword1, keyword2}
\thesisabstract[french]{Résumé}
\thesiskeywords[french]{keyword1, keyword2}


\begin{document}


\chapter*{Remerciements}

TODO


\tableofcontents
\mainmatter


\chapter*{Introduction}

TODO


\part{Contexte et objectifs}


\chapter{Vérification et validation de programmes}


Le domaine de la vérification et de la validation regroupe un ensemble de
techniques du cycle de développement des logiciels qui ont pour objectif de
s'assurer de leur correction et de leur sûreté. Ces deux notions sont apparues
dans les années soixante-dix avec les travaux de \bsc{Dijkstra} \cite{Dijkstra},
\bsc{Floyd} \cite{Floyd} et \bsc{Hoare} \cite{Hoare}.

La correction d'un logiciel représente le respect de l'implémentation par
rapport aux spécifications. La sûreté d'un logiciel est liée à son absence
d'erreurs à l'exécution.


\section{Model-checking}
\label{sec:model-checking}

Le model-checking \cite{model-checking} permet de vérifier algorithmiquement si
un modèle donné (le système ou une abstraction de ce système) satisfait une
spécification, formulée en termes de logique temporelle \cite{LTL}. Un modèle
est un ensemble d'états, de propriétés que vérifie chaque état, et de
transitions entre ces états qui décrivent l'évolution du système.

Le model-checking couvre l'ensemble des états du système et des transitions afin
d'analyser toutes les exécutions possibles du système. Sur de grands systèmes,
cette méthode est pénalisée par l'explosion combinatoire du nombre des états (et
la complexité en temps ou en espace qui en résulte). Il est néanmoins possible
de modéliser des algorithmes asynchrones répartis et des automates non
déterministes, comme le fait notamment l'outil \textsc{Spin} \cite{SPIN}.

La plateforme \textsc{Frama-C} du LSL intègre \textsc{Aoraï}, un greffon
permettant d'annoter automatiquement le code source d'un programme C d'après
une formule de logique temporelle linéaire, de sorte que les annotations sont
vérifiées si le programme repecte la formule.


\section{Analyse statique}
\label{sec:AS}


L'analyse statique \cite{static-analysis} examine le code source du programme
sans l'exécuter. Elle raisonne sur tous les comportements qui pourraient
survenir lors de l'exécution et permet donc de déduire des propriétés devant
être vérifiées pour toutes ces exécutions, dans le but de prouver la correction
du programme.

En revanche, la vérification de programme étant en général indécidable
\cite{undecidability}, il est souvent nécessaire d'utiliser des
sur-approximations, ce qui implique que les résultats peuvent être moins précis
que ce que l'on souhaite mais ils sont garantis pour toutes les exécutions.
Ainsi, on peut établir des propriétés de sûreté ({\em safety}), où l’on cherche
des invariants sur les valeurs des variables du programme (une plage de valeurs
par exemple), afin d'exclure certains risques d'erreurs à l'exécution.

Parmi les méthodes statiques sont distinguées : l'interprétation abstraite
(section~\ref{sec:interpretation-abstraite}), l'abstraction à partir de
prédicats (section~\ref{sec:abstraction-predicats}) et la preuve de théorèmes
(section~\ref{sec:preuve}).


\subsection{Interprétation abstraite}
\label{sec:interpretation-abstraite}

L'interprétation abstraite \cite{abstract-interpretation} s'appuie sur les
théories du point-fixe et des domaines pour introduire des sur-approximations
des comportements d'un programme. Elle consiste à abstraire les domaines des
variables par des domaines finis et beaucoup plus petits. Par exemple, le
domaine des entiers pourrait être abstrait par un domaine de trois valeurs :
$(-, 0, +)$. Appliquée à l’analyse de valeurs, elle consiste à calculer à
chaque ligne du code une sur-approximation de l’ensemble des valeurs prises par
chaque variable en cette ligne lors de toutes les exécutions du programme,
permettant ainsi de détecter certaines erreurs comme les divisions par zéro ou
les accès en dehors des bornes des tableaux.

Pour contourner le problème d’indécidabilité, la théorie de l’interprétation
abstraite construit une méthode qui, à la même question, répondra ``oui'',
``non'' ou ``peut-être''. Si la méthode répond ``peut-être'', c’est qu’on n’a pu
prouver ni l’un ni l’autre des deux premiers cas. C’est ce qu’on appelle une
alarme : il est possible qu’une des exécutions du programme produise une erreur
donnée, mais nous n’avons été capable ni de le confirmer ni de l’infirmer.
L’erreur signalée par une alarme peut ne jamais apparaître à l’exécution, dans
ce cas on l’appelle fausse alarme. On ne calcule donc pas la propriété exacte
mais une abstraction de cette propriété, en imposant la contrainte de sûreté
suivante : ``la propriété abstraite calculée ne doit oublier aucune exécution
concrète''. L'abstraction est effectuée à partir de prédicats atomiques
définissant des abstractions des domaines des variables.

\textsc{PolySpace C Verifier} \cite{PolySpace} a été le premier outil commercial
utilisant l'interprétation abstraite pour détecter les erreurs à l'exécution
dans les programmes en C, C++ et Ada mais signale beaucoup de fausses alarmes.
L'ENS a développé \textsc{Astrée} \cite{ASTREE}, spécifique au langage C et aux
logiciels critiques. Le CEA LIST a développé \textsc{Fluctuat} \cite{Fluctuat},
qui mesure précisément les approximations faites à l'exécution d'un programme C.
\textsc{Frama-C} intègre un greffon d'interprétation abstraite : \textsc{Value}
\cite{Value}.



\subsection{Abstraction à partir de prédicats}
\label{sec:abstraction-predicats}

L'abstraction à partir de prédicats \cite{predicate-abstraction} est une
technique permettant de générer automatiquement des abstractions de systèmes au
nombre d'états infini. Pour un programme $P$ au nombre d'états infini, un
ensemble fini de prédicats $E = \{f_1, ..., f_n\}$ est défini, ces prédicats
sont des expressions booléennes sur les variables de $P$ et les constantes du
langage.

Chaque état concret de $P$ est mis en correspondance avec un état abstrait de
l'abstraction de $P$, après évaluation par les prédicats de $E$. Un état
abstrait est un $n$-upplet de valeurs booléennes correspondant à la
satisfaisabilité des $n$ prédicats (au moyen d'un solveur SMT).

Par exemple, si 3 prédicats $f_1, f_2, f_3$ sont définis et que la
satisfaisabilité de ces prédicats à l'état concret $e$ est évaluée
respectivement à $false, true, true$, alors dans l'abstraction générée, l'état
$e$ correspond à l'état abstrait $(\lnot f_1, f_2, f_3)$.

L'abstraction générée comporte un nombre fini d'états (au plus $2^n$) car il n'y
a qu'un nombre fini de prédicats, le model-checking peut donc être appliqué à
cette abstraction. Si une propriété de sûreté est vérifiée dans l'abstraction,
elle l'est également dans le système concret.

Cette technique est notamment utilisée par l'outil \textsc{Slam} \cite{SLAM} de
Microsoft.


\subsection{Preuve de théorèmes}
\label{sec:preuve}

La preuve de théorèmes utilise des fondements mathématiques et logiques
\cite{Hoare} pour prouver des propriétés de programmes. Tout d'abord, le système
est décrit par un ensemble d'axiomes et de règles d'inférence. Puis, le calcul
de la plus faible précondition \cite{Dijkstra} est utilisé pour générer des
formules appelées obligations de preuve, qui sont finalement soumises à un
prouveur de théorèmes, qui applique différentes techniques de résolution.

Contrairement au model-checking, la preuve de théorèmes a l'avantage d'être
indépendante de la taille de l'espace des états, et peut donc s'appliquer sur
des systèmes de grande taille. En contre-partie, cette technique requiert une
expertise de l'utilisateur pour adapter le programme à la preuve (en l'annotant
par exemple) et guider le prouveur si nécessaire.

Il existe des prouveurs automatiques tels que \textsc{Simplify}
\cite{Simplify}, \textsc{Ergo} \cite{Ergo} et \textsc{Z3} \cite{Z3}; et des
prouveurs interactifs, où la preuve est guidée par l'utilisateur, tels que
\textsc{Coq} \cite{Coq}, \textsc{Isabelle} \cite{Isabelle}, et \textsc{Hol}
\cite{HOL}. Certains de ces prouveurs ou assistants de preuve sont intégrés à
d'autres outils tels Boogie \cite{Boogie} ou ESC/Java \cite{ESC/Java}.
\textsc{Frama-C} intègre deux greffons de preuve, \textsc{Jessie} et
\textsc{Wp}, qui traitent des programmes dont le code contient des annotations
\textsc{Acsl} \cite{ACSL}.


\section{Analyse dynamique}
\label{sec:AD}


L’analyse dynamique est basée sur des techniques d’exécution du programme, de
simulation \cite{simulation} d’un modèle ou d'exécution symbolique
\cite{symbolic-execution}, regroupées sous le terme générique ``test''.

Les tests peuvent s’appliquer tout au long du cycle de développement d’un
logiciel. Les tests unitaires vérifient le bon fonctionnement des différentes
entités d’un système, indépendamment les unes des autres. Les tests
d'intégration vérifient la bonne communication entre ces entités. Les tests de
validation s'assurent que les fonctionnalités correspondent au besoin de
l’utilisateur final. Enfin, les tests de non-régression vérifient que l'ajout de
nouvelles fonctionnalités ne détériore pas les anciennes fonctionnalités.

En général, les techniques de test ne sont pas exhaustives et n'explorent qu'un
sous-ensemble des chemins d'exécutions du programme, en conséquence, l’absence
d’échecs lors du passage des tests n’est pas une garantie de bon fonctionnement
du système. Néanmoins, selon les critères utilisés pour la génération des
tests, et selon la couverture des chemins d'exécution fournie par les tests, un
système ainsi validé peut acquérir un certain niveau de confiance.

Les méthodes de test peuvent être classées en trois catégories : le test
aléatoire, le test structurel (section~\ref{sec:test-structurel}) et le test
fonctionnel (section~\ref{sec:test-fonctionnel}). Comme son nom l'indique, le
test aléatoire consiste à générer des valeurs d'entrée du programme au hasard et
ne sera pas détaillé dans cette thèse.



\subsection{Exécution symbolique pour le test structurel}
\label{sec:exec-sym}



\subsubsection{Exécution symbolique augmentée}



\subsection{Test structurel}
\label{sec:test-structurel}

Le test structurel, ou test ``boîte blanche'', est une technique de test qui
fonde la détermination des différents cas de test sur une analyse de la
structure du code source du programme étudié. On distingue deux types de tests
structurels : le test orienté flot de contrôle et le test orienté flot de
données.

Le test orienté flot de données cherche à couvrir certaines relations entre la
définition d’une variable et son utilisation, par exemple, on peut souhaiter
couvrir toutes les lectures d'une variable suivant une écriture.

Le test orienté flot de contrôle s’intéresse quant à lui à la structure du
programme : l'ordre dans lequel les instructions sont exécutées. Il se base sur
le graphe de flot de contrôle du programme : un graphe connexe orienté avec un
unique n\oe{}ud d’entrée et un unique n\oe{}ud de sortie, dont les n\oe{}uds
sont les blocs de base du programme et les arcs représentent les branchements
(conditions). Une couverture structurelle de ce graphe est recherchée, selon un
critère qui peut être par exemple ``toutes les instructions'',
``toutes les branches'' (toutes les décisions), ``tous les chemins'' ou
``tous les $k$-chemins''.

L’exécution symbolique dynamique, ou exécution ``concolique'', associe
l’exécution concrète du programme et l’exécution symbolique afin d’explorer les
chemins du programme. L’exécution concrète sert à confirmer que le chemin
parcouru est bien celui pour lequel le cas de test exécuté a été généré.

Plusieurs outils se basent sur l'exécution concolique pour explorer un programme
sous test, dont \textsc{Smart} \cite{SMART}, \textsc{Pex} \cite{PEX} et
\textsc{Sage} \cite{SAGE} développés par Microsoft, \textsc{Cute} \cite{CUTE},
\textsc{Klee} \cite{KLEE}, \textsc{Exe} \cite{EXE} et enfin
\textsc{PathCrawler}, \cite{PathCrawler} développé au CEA LIST. Ces outils
utilisent des solveurs de contraintes pour générer des cas de test permettant
d'aboutir à une couverture souhaitée des exécutions du programme par les tests.


\subsection{Test fonctionnel}
\label{sec:test-fonctionnel}

Le test fonctionnel, ou test ``boîte noire'', génère des jeux de test en
fonction du comportement attendu du programme : un cas de test sera choisi pour
chaque comportement particulier. Le test fonctionnel est utilisé pour vérifier
la conformité des réactions du logiciel avec les attentes de l'utilisateur, sans
connaissance du code source. Il existe de nombreuses techniques qui se
différencient par la manière de choisir les données de test, parmi lesquelles :

\begin{description}
\item[le test de partition] \hfill \\
les valeurs d’entrées du logiciel sont regroupées en classes d’équivalence, sur
lesquelles le logiciel doit avoir le même comportement ({\em domain splitting}),
une seule valeur aléatoire est choisie dans chaque classe de la partition;
\item[le test aux limites] \hfill \\
les données de test sont choisies aux bornes des domaines de définition des
variables.
\end{description}

Le CEA LIST développe \textsc{GATeL} \cite{GATEL}, un outil de génération de
tests fonctionnels qui se base sur une représentation symbolique des états du
système : le programme, les invariants et les contraintes décrivant l'objectif
de test sont exprimés dans le langage \textsc{Lustre} \cite{Lustre}. Cet outil
offre la possibilité de réaliser des partitions de domaines.



\subsection{Monitoring}
\label{sec:monitoring}


TODO


\section{Combinaison d'analyses statiques et dynamiques}
\label{sec:combinaison}

Les méthodes statiques et les méthodes dynamiques ont des avantages et des
inconvénients complémentaires : l'analyse statique étant complète mais
imprécise, l'analyse dynamique étant précise mais incomplète. L’idée de les
combiner pour associer leurs avantages et combattre leurs inconvénients
\cite{duality} est une voie de recherche active et fructueuse dans le domaine de
la vérification de programmes.


\subsection{\textsc{Static ANalysis and TEst}}

La méthode \textsc{Sante} \cite{TheseOmar, SANTE}, mise en \oe{}uvre au sein de
\textsc{Frama-C}, combine l'interprétation abstraite, le {\em slicing} et la
génération de tests structurels avec \textsc{PathCrawler}. L’analyse statique
signale les instructions risquant de provoquer des erreurs à l’exécution par des
alarmes, dont certaines peuvent être de fausses alarmes, puis l’analyse
dynamique génère des tests confirmant ou infirmant ces alarmes. Sur des
programmes de grande taille, l'analyse dynamique peut manquer de temps pour
classer toutes ces alarmes (à cause de l'explosion combinatoire des exécutions
possibles). Le {\em slicing} est utilisé pour réduire la taille des programmes
testés et donc le temps nécessaire à leur analyse.

Le fonctionnement général de la méthode est illustré par la
Fig.~\ref{figSlicing} ~\textbf{a}. Cette méthode utilise l'analyse de valeurs
(interprétation abstraite) afin de sélectionner les instructions pour lesquelles
le risque d'une erreur à l'exécution n'est pas écarté (nous les appellerons
``alarmes'' par la suite), par exemple une division par zéro ou un accès
invalide à la mémoire.

Ces alarmes vont être validées par des tests. Nous utilisons
\textsc{PathCrawler} \cite{PathCrawler}, un outil de génération de tests
structurels, pour tenter de mettre en évidence un cas de test pour lequel le
chemin d'exécution passe par cette alarme, et provoque une erreur.

Pour diminuer le coût de la génération de tests, nous simplifions
syntaxiquement les programmes qui lui seront soumis pour ne garder que les
instructions dont dépend l'alarme considérée, on appelle cette opération
{\em slicing} \cite{slicing}. Nous obtenons ainsi des programmes plus simples,
tels que si une alarme est présente dans le programme d'origine, elle l'est
également dans les programmes simplifiés si le {\em slicing} a été paramétré
pour conserver les instructions dont dépend cette alarme. Et si elle provoque
une erreur dans le programme d'origine, alors il en sera de même dans les
programmes simplifiés, avec les mêmes entrées. L'utilisation du {\em slicing}
peut être paramétrée par différentes options, qui seront détaillées plus bas.

La première phase de la méthode (l'interprétation abstraite) peut donc assurer
que certaines instructions ne provoqueront pas d'erreur à l'exécution, tandis
que la seconde phase permet de confirmer que certaines alarmes produisent
effectivement une erreur à l'exécution. Et la mise en commun des résultats des
deux phases permet de classifier davantage d'alarmes dans l'une ou l'autre de
ces deux catégories que chacune des deux méthodes prise séparément. La preuve de
la correction de cette méthode et le résultat des analyses comparées sont
détaillés dans \cite{TheseOmar} et \cite{SANTE} et ne seront pas répétés ici.


\begin{figure}[h]
  \begin{center}
    
    %\hfill{}
    \begin{tabular}{cccc}

      \begin{pspicture}(0,0)(6,6.7)
        \centering

        \rput[tl](0,6.6){\ovalnode{P}{\scriptsize{$\lefteqn{\mbox{Program}\,p}\phantom{Preond p pre}$}}}
        \rput[tr](5.3,6.6){\ovalnode{Context}{\scriptsize{$\lefteqn{\mbox{Precondition}}\phantom{Preond p pre}$}}}

        \rput(2.5,5.5){\rnode{VA}{\psshadowbox[shadowsize=0,fillcolor=red!10,fillstyle=solid]{
              \centering \begin{scriptsize}\begin{tabular}{c}Value Analysis
                  %\\(Frama-C plugin)
        \end{tabular}\end{scriptsize}}}}
        \rput(2.5,4.6){\ovalnode{VA-SL}{\scriptsize{$\lefteqn{\!\!\!\!p,\, A = \alarms(p)}\phantom{aaaaaaaaaaa}$}}}   
        \rput(2.5,3.7){\rnode{SL}{\psshadowbox[shadowsize=0,fillcolor=red!10,fillstyle=solid]{
              \begin{scriptsize}\begin{tabular}{c}$\SliceTest$\end{tabular}\end{scriptsize}}}}
        
        \rput[tr](0.8,4.1){\rnode{SLOP}{\begin{scriptsize}\begin{tabular}{l} Option:
                \textit{none},\\
                \textit{all}, \textit{each}, \\
                \textit{min}, \textit{smart}\end{tabular}\end{scriptsize}}}

        \rput(2.5,2.9){\ovalnode{Diagnostic}{\scriptsize{Diagnostic}}}
        \rput(0.05,6.7){\textbf{a)}}
        
        \ncline[]{->}{P}{VA}
        \ncline[]{->}{Context}{VA}

        \ncline[nodesep=0pt]{->}{VA}{VA-SL}
        \ncline[nodesep=0pt]{->}{VA-SL}{SL}
        \ncline[nodesep=0pt]{->}{SLOP}{SL}
        
        \ncline[]{->}{SL}{Diagnostic}
      \end{pspicture}

     &
      
      \begin{pspicture}(0,0)(3.5,7)
        \centering

        \rput(1.75,6.9){\ovalnode{allInput}{\scriptsize{\raisebox{-0.5mm}[0.9mm][1mm]{$p,\,A$}}}}
        \rput(1.75,4.37){\rnode{SL}{\psshadowbox[shadowsize=0,fillcolor=red!10,fillstyle=solid]{
              \parbox{3.5cm}{\begin{tabular}{c} \\ \\ \\ \\ \\ \\ \\ \end{tabular}}
        }}}
        \rput(1.75,5.7){\rnode{SEL}{\psshadowbox[shadowsize=0,fillcolor=blue!10,fillstyle=solid]{\parbox{3cm}{\centering\scriptsize{Select \textit{all}}}}}}
        \rput(1.75,4.8){\rnode{SLICE}{\psshadowbox[shadowsize=0,fillcolor=blue!10,fillstyle=solid]{\parbox{3cm}{\centering\scriptsize{Slice}}}}}
       
        \rput(1.75,4.0){\ovalnode{allOutput}{\scriptsize{\raisebox{-0.5mm}[0.2mm][1mm]{$p_{A}$}}}}
        \rput(1.95,5.32){\scriptsize{$A$}}

        \rput(1.75,3.1){\rnode{DA}{\psshadowbox[shadowsize=0,fillcolor=blue!10,fillstyle=solid]{\parbox{3cm}{\centering\scriptsize{Dynamic Analysis}}}}}
        \rput(0.05,6.7){\textbf{b)}}
        \rput(1.75,1.8){\ovalnode{Diagnostic}{\scriptsize{\raisebox{-0.5mm}[1mm][1mm]{$Diagnostic$}}}}
        \ncline[nodesep=0pt]{->}{allInput}{SL}
        \ncline[nodesep=0pt]{->}{SEL}{SLICE}
        \ncline[nodesep=0pt]{->}{SLICE}{allOutput}
        \ncline[nodesep=0pt]{->}{allOutput}{DA}
        \ncline[]{->}{DA}{Diagnostic}
      \end{pspicture}
      
      
      & &
      
      %each
      \begin{pspicture}(0,0)(3.5,7)
        \centering
        \rput(1.75,6.9){\ovalnode{allInput}{\scriptsize{\raisebox{-0.5mm}[0.9mm][1mm]{$p,\,A$}}}}
        \rput(1.75,4.37){\rnode{SL}{\psshadowbox[shadowsize=0,fillcolor=red!10,fillstyle=solid]{
              \parbox{3.5cm}{\begin{tabular}{c} \\ \\ \\ \\ \\ \\ \\ \end{tabular}}
        }}}
        \rput(1.75,5.7){\rnode{SEL}{\psshadowbox[shadowsize=0,fillcolor=blue!10,fillstyle=solid]{\parbox{3cm}{\centering\scriptsize{Select \textit{each}}}}}}
        
        \rput(0.6,4.8){\rnode{SLICE1}{\psshadowbox[shadowsize=0,fillcolor=blue!10,fillstyle=solid]{\parbox{0.8cm}{\centering\scriptsize{Slice}}}}}

        \rput(2.9,4.8){\rnode{SLICEn}{\psshadowbox[shadowsize=0,fillcolor=blue!10,fillstyle=solid]{\parbox{0.8cm}{\centering \scriptsize{Slice}}}}}
        
        \rput(0.6,4.0){\ovalnode{p1}{\scriptsize{\raisebox{-0.5mm}[0.2mm][1mm]{$p_{a_1}$}}}}
        \rput(2.9,4.0){\ovalnode{pn}{\scriptsize{\raisebox{-0.5mm}[0.2mm][1mm]{$p_{a_n}$}}}}
        
        \rput(1,5.3){\scriptsize{$\{a_1\}$}}
        \rput(3.3,5.3){\scriptsize{$\{a_n\}$}}
        \rput(1.8,4.8){. . .}

        \rput(1.75,3.1){\rnode{DA}{\psshadowbox[shadowsize=0,fillcolor=blue!10,fillstyle=solid]{\parbox{3cm}{\centering\scriptsize{Dynamic Analysis}}}}}
        \rput(0.05,6.7){\textbf{c)}}
        \rput(1.75,1.8){\ovalnode{Diagnostic}{\scriptsize{\raisebox{-0.5mm}[1mm][1mm]{$Diagnostic$}}}}

        \ncline[nodesep=0pt]{->}{allInput}{SL}



        \ncangle[angleA=-90,angleB=90,offsetA=-1.15]{->}{SEL}{SLICE1}
        \ncangle[angleA=-90,angleB=90,offsetA=1.15,offsetB=0]{->}{SEL}{SLICEn}
        \ncline[nodesep=0pt]{->}{SLICE1}{p1}    
        \ncline[nodesep=0pt]{->}{SLICEn}{pn}    
%        \ncangle[angleA=-90,angleB=90]{->}{SLICE1}{p1}
%        \ncangle[angleA=-90,angleB=90]{->}{SLICEn}{pn}
        \ncangle[angleA=-90,angleB=90,offsetB=-1.15]{->}{p1}{DA}
        \ncangle[angleA=-90,angleB=90,offsetB=1.15]{->}{pn}{DA}
        \ncline[]{->}{DA}{Diagnostic}
      \end{pspicture}
    
    \end{tabular}
\vspace{-14mm}
\caption{\textbf{a)} General overview, Basic $\SliceTest$  options:
\textbf{b)} \textit{all},  \textbf{c)} \textit{each}} 
\label{figSlicing}\vspace{-5mm}
\end{center}
\end{figure}







L'option {\em none} signifie l'absence de {\em slicing} : le programme n'est pas
simplifié et est soumis en l'état à \textsc{PathCrawler}.

L'option {\em all} (Fig.~\ref{figSlicing} ~\textbf{b}) génère un seul programme
simplifié contenant toutes les alarmes signalées par l'analyse de valeur. Cette
option ne tire pas profit du fait que certaines alarmes peuvent être
indépendantes (et peuvent donc être soumises séparément), la complexité du
programme soumis à \textsc{PathCrawler} peut l'empêcher de générer des cas de
test pour toutes les alarmes dans le temps qui lui est imparti.

L'option {\em each} (Fig.~\ref{figSlicing} ~\textbf{c}) tente de corriger ce
défaut, elle génère un programme simplifié par alarme et invoque
\textsc{PathCrawler} autant de fois. Ainsi, les alarmes les plus simples ne sont
pas pénalisées par les plus complexes. Néanmoins, en cas de dépendances
mutuelles entre alarmes, plusieurs programmes identiques seront soumis à
\textsc{PathCrawler}, ce qui est une perte de temps.

\begin{figure}[h]
\begin{center}
  \mbox{}\hspace{3mm}
	\begin{tabular}{cc}
      %min
      \begin{pspicture}(0,0)(3,12)
        \centering
        \rput(1,11.6){\ovalnode{allInput}{\centering  \scriptsize{\raisebox{-0.5mm}[0.9mm][1mm]{$p,\,A$}}}}

        \rput(1,10.8){\rnode{DEPS}{\psshadowbox[shadowsize=0,fillcolor=blue!10,fillstyle=solid]{
              \parbox{3cm}{\centering \scriptsize{Dependency Analysis}}}}}
        \rput(1,9.9){\ovalnode{DEPS-SL}{\centering  \scriptsize{\raisebox{-1mm}[0.2mm][1mm]{$p,\,A,\,\dep$}}}}

        \rput(1,7.7){\rnode{SL}{\psshadowbox[shadowsize=0,fillcolor=red!10,fillstyle=solid]{
              \parbox{2.9cm}{\begin{tabular}{c}\\ \\ \\ \\ \\ \\ \\ \end{tabular}}
        }}}
       
        \rput(1,9.05){\rnode{SEL}{\psshadowbox[shadowsize=0,fillcolor=blue!10,fillstyle=solid]{\parbox{2.6cm}{\centering\scriptsize{Select \textit{min}}}}}}

        \rput(0.05,8.2){\rnode{SLICE1}{\psshadowbox[shadowsize=0,fillcolor=blue!10,fillstyle=solid]{\parbox{0.7cm}{\centering\scriptsize{Slice}}}}}
        
        \rput(1.95,8.2){\rnode{SLICEn}{\psshadowbox[shadowsize=0,fillcolor=blue!10,fillstyle=solid]{\parbox{0.7cm}{\centering\scriptsize{Slice}}}}}
        
        \rput(0.05,7.3){\ovalnode{p1}{\scriptsize{\raisebox{0mm}[1mm][1mm]{$\,p_{A_1}$}}}}
        \rput(1.95,7.3){\ovalnode{pn}{\scriptsize{\raisebox{0mm}[1mm][1mm]{$\,p_{A_k}$}}}}
        
        
        \rput(0.3,8.65){\scriptsize{$A_1$}}
        \rput(2.2,8.65){\scriptsize{$A_k$}}
        \rput(1,8.3){. . .}

        \rput(1,6.4){\rnode{DA}{\psshadowbox[shadowsize=0,fillcolor=blue!10,fillstyle=solid]{\parbox{2.6cm}{\centering\scriptsize{Dynamic Analysis}}}}}

        \rput(-0.5,11.6){\textbf{a)}}        
        \rput(1,5.4){\ovalnode{Diagnostic}{\scriptsize{\raisebox{-0.5mm}[1mm][1mm]{$Diagnostic$}}}}

        \ncline[nodesep=0pt]{->}{allInput}{DEPS}
        \ncline[nodesep=0pt]{->}{DEPS}{DEPS-SL}

        \ncline[nodesep=0pt]{->}{DEPS-SL}{SL}
        
        \ncangle[angleA=-90,angleB=90,offsetA=-0.95]{->}{SEL}{SLICE1}
        \ncangle[angleA=-90,angleB=90,offsetA=0.95,offsetB=0]{->}{SEL}{SLICEn}
        
        
        \ncangle[angleA=-90,angleB=90]{->}{SLICE1}{p1}
        \ncangle[angleA=-90,angleB=90]{->}{SLICEn}{pn}

        \ncangle[angleA=-90,angleB=90,offsetB=-0.95]{->}{p1}{DA}
        \ncangle[angleA=-90,angleB=90,offsetB=0.95]{->}{pn}{DA}
        \ncline[]{->}{DA}{Diagnostic}
      \end{pspicture}
      
      
      &
      
      \begin{pspicture}(0,0)(4,12)
        \centering
        \rput(1.3,11.6){\ovalnode{allInput}{\centering \scriptsize{\raisebox{-0.5mm}[0.9mm][1mm]{$p,\,A$}}}}
        \rput(2.4,10.35){\begin{scriptsize}$A^0:=A;\,i:=0$\end{scriptsize}}
        
        \rput(1.3,10.8){\rnode{DEPS}{\psshadowbox[shadowsize=0,fillcolor=blue!10,fillstyle=solid]{
              \parbox{3cm}{\centering \scriptsize{Dependency Analysis} }}}}
        \rput(1.3,9.9){\ovalnode{DEPS-SL}{\scriptsize{\raisebox{-1mm}[0.2mm][1mm]{{$p,\,A^i,\,\dep$}}}}}
        
        
        \rput(1.3,7.7){\rnode{SL}{\psshadowbox[shadowsize=0,fillcolor=red!10,fillstyle=solid]{
              \parbox{2.9cm}{\begin{tabular}{c}\\ \\ \\ \\ \\ \\ \\ \end{tabular}}
        }}}
        

        \rput(1.3,9.05){\rnode{SEL}{\psshadowbox[shadowsize=0,fillcolor=blue!10,fillstyle=solid]{\parbox{2.6cm}{\centering\scriptsize{Select \textit{min}}}}}}
        
        \rput(0.3,8.2){\rnode{SLICE1}{
            \psshadowbox[shadowsize=0,fillcolor=blue!10,fillstyle=solid]{
              \parbox{0.7cm}{\centering\scriptsize{Slice}}}}}
        
        \rput(2.2,8.2){\rnode{SLICEn}{
            \psshadowbox[shadowsize=0,fillcolor=blue!10,fillstyle=solid]{
              \parbox{0.7cm}{\centering\scriptsize{Slice}}}}}
        
        \rput(1.3,6.4){\rnode{DA}{\psshadowbox[shadowsize=0,fillcolor=blue!10,fillstyle=solid]{\parbox{2.6cm}{\centering\scriptsize{Dynamic Analysis}}}}}
        
        \rput(0.3,7.3){\ovalnode{p1}{\scriptsize{\raisebox{0.4mm}[1mm][1mm]{$\,\,p_{A^i_{1}}$}}}}
        \rput(2.2,7.3){\ovalnode{pn}{\scriptsize{\raisebox{0.8mm}[1mm][1mm]{$\,p_{A^i_{k_i}}$}}}}
        \rput(3.8,7.7){\ovalnode{Ai}{\scriptsize{\raisebox{-0.4mm}[1mm][0.1mm]{$A^{i+1}$}}}}


        \rput(1.3,5.4){\ovalnode{DIAGi}{\scriptsize{\raisebox{-0.5mm}[1mm][1mm]{$Diagnostic^i$}}}}

        \rput(1.3,4.7){\rnode{FILTER}{\psshadowbox[shadowsize=0,fillcolor=blue!10,fillstyle=solid]{\parbox{3.12cm}{\centering\scriptsize{Refine}}}}}
        
        
        \rput(0.55,8.65){\scriptsize{$A^i_{1}$}}
        \rput(2.5,8.645){\scriptsize{$A^i_{k_i}$}}
        \rput(1.3,8.2){. . .}
        \rput(-0.3,11.6){\textbf{b)}}
        \rput(1.3,3.7){\ovalnode{Diagnostic}{\scriptsize{\raisebox{-0.5mm}[1mm][1mm]{$Diagnostic$}}}}

        \rput(3.8,9.2){\begin{scriptsize}$i:=i+1$\end{scriptsize}}
        \rput(0.6,4.3){\begin{scriptsize}$A^{i+1}=\emptyset$\end{scriptsize}}
        \rput(3.6,4.5){\begin{scriptsize}$A^{i+1}\ne\emptyset$\end{scriptsize}}

        \ncline[nodesep=0pt]{->}{allInput}{DEPS}
        \ncline[nodesep=0pt]{->}{DEPS}{DEPS-SL}

        \ncline[nodesep=0pt]{->}{DEPS-SL}{SL}
        

        \ncangle[angleA=-90,angleB=90,offsetA=-1]{->}{SEL}{SLICE1}
        \ncangle[angleA=-90,angleB=90,offsetA=0.9,offsetB=0]{->}{SEL}{SLICEn}

        \ncline[]{->}{DA}{DIAGi}
        \ncline[]{->}{DIAGi}{FILTER}
        
        
        \ncangle[angleA=-90,angleB=90]{->}{SLICE1}{p1}
        \ncangle[angleA=-90,angleB=90]{->}{SLICEn}{pn}

%        \ncangles[angleA=-90,angleB=-90,arm=0.5,linearc=.15,offsetA=0.5]{->}{FILTER}{Ai}
        \ncangle[angleA=0,angleB=-90, linearc=.15,offsetA=0]{->}{FILTER}{Ai}
        \ncangle[angleA=90,arm=0.6cm, linearc=.15]{->}{Ai}{SEL}
        
        \ncangle[angleA=-90,angleB=90,offsetB=-1]{->}{p1}{DA}
        \ncangle[angleA=-90,angleB=90,offsetB=0.9]{->}{pn}{DA}
%        \ncangle[angleA=-90,angleB=90,offsetA=-0.5]{->}{FILTER}{Diagnostic}
        \ncline[nodesep=0pt]{->}{FILTER}{Diagnostic}
      \end{pspicture}
      
    
    \\
    \\
%    \textbf{a)} Option \textit{min} &
%    \textbf{b)} Option \textit{smart}
    
    \end{tabular}
\vspace{-40mm}
\caption{Advanced $\SliceTest$  options:
\textbf{a)} \textit{min},  \textbf{b)} \textit{smart}} 
\label{figSlicingAdvanced}
\vspace{-5mm}
\end{center}
\end{figure}



L'option {\em min} (Fig.~\ref{figSlicingAdvanced} ~\textbf{a}) tente de corriger
les inconvénients de {\em all} et de {\em each} et propose de générer une
couverture minimale des alarmes par {\em slicing}, de manière à regrouper dans
un même ensemble les alarmes ayant une relation de dépendance \cite{SANTE}.
Considérons par exemple les alarmes $\{a_1; a_2; a_3; a_4; a_5\}$, dont les
dépendances (Fig.~\ref{fig:deps}) sont : $a_2$ dépend de $a_1$; $a_5$ dépend de
$a_4$ qui dépend de $a_3$, qui dépend de $a_1$.
Dans cet exemple, la couverture minimale est constituée des deux ensembles
couvrants $\{a_1; a_2\}$ et $\{a_1; a_3; a_4; a_5\}$.


\begin{figure}
  \centering
  \begin{tikzpicture}
    \node(s) at (0,1) {$a_1$};
    \node(t) at (1,0) {$a_2$};
    \node(q) at (1,1) {$a_3$};
    \node(d) at (2,1) {$a_4$};
    \node(v) at (3,1) {$a_5$};
    \path[->,thick] (s) edge (q);
    \path[->,thick] (q) edge (d);
    \path[->,thick] (d) edge (v);
    \path[->,thick] (s) edge (t);
  \end{tikzpicture}
  \caption{Graphe de dépendances des alarmes}
  \label{fig:deps}
\end{figure}


Chacun de ces ensembles sera utilisé pour générer un programme simplifié ne
contenant que des alarmes ayant une relation de dépendance, évitant ainsi
d'avoir plusieurs programmes identiques en cas de dépendances mutuelles (et
réduisant le nombre d'appels à \textsc{PathCrawler}).
On remarque que si toutes les alarmes sont dépendantes, on se retrouve dans le
cas de {\em all}, et si elles sont toutes indépendantes, on se retrouve dans le
cas de {\em each}.

Dans le cas où les ensembles couvrants contiennent de nombreuses alarmes,
{\em min} hérite de l'inconvénient de l'option {\em all} : les alarmes les plus
complexes à classifier pénalisent les autres, qui peuvent ne pas être
diagnostiquées dans le temps imparti.

Pour corriger ce défaut, l'option {\em smart} (Fig.~\ref{figSlicingAdvanced}
~\textbf{b}) est définie. Elle consiste à appliquer {\em min} itérativement en
diminuant l'ensemble des alarmes considérées à chaque itération. Les alarmes
supprimées seront des alarmes classifiées ou des alarmes finales.

\begin{definition}[Alarme finale]
Une alarme finale est une alarme dont aucune autre alarme de dépend mais
pouvant avoir des dépendances mutuelles avec d'autres alarmes finales.
Sur notre exemple, $a_2$ et $a_5$ sont des alarmes finales (sans dépendances
mutuelles).
\end{definition}

De cette manière, le {\em slicing} appliqué à l'itération suivante (sur
l'ensemble des alarmes privé des alarmes finales) produira un programme
simplifié plus petit. Ainsi, les alarmes qui n'ont pas été diagnostiquées à
l'itération précédente auront une chance supplémentaire de l'être, et ainsi de
suite jusqu'à ce que toutes les alarmes soient diagnostiquées ou que l'ensemble
considéré soit vide.

Sur l'exemple considéré plus haut, la première itération génère la
couverture minimale $\{\{a_1; a_2\}; \{a_1; a_3; a_4; a_5\}\}$. S'il reste des
alarmes non diagnostiquées, les alarmes finales ($a_2$ et $a_5$) et les alarmes
diagnostiquées ne seront plus considérées. L'itération suivante génère la
couverture $\{\{a_1; a_3; a_4\}\}$. Et ainsi de suite.


\subsection{\textsc{DyTa}}

\textsc{DyTa} \cite{DyTa} est un outil combinant une phase d'analyse statique et
une phase d'analyse dynamique. \textsc{Code Contracts} \cite{CodeContracts} est
utilisé pour spécifier des pré/post-conditions et des invariants de programmes
C\#, il identifie également les bugs potentiels (violations de contrat) par
interprétation abstraite.

\textsc{DyTa} instrumente ensuite le programme pour rajouter des instructions
assurant le rôle de pré-conditions, pour ne pas générer de cas de test qui n'ont
aucune chance de produire d'erreur à l'exécution. L'instrumentation rajoute
également des points de contrôle à l'endroit des instructions signalées par
l'analyse statique, afin de guider l'exécution symbolique dynamique du programme
par \textsc{Pex} \cite{PEX}. Cette étape est semblable à l'instrumentation
opérée par \textsc{Sante} \cite{SANTE} pour rajouter des points de contrôle afin
de guider l'exécution concolique du programme par \textsc{PathCrawler}
\cite{PathCrawler}.

L'analyse statique du graphe du flot de contrôle du programme permet à
\textsc{DyTa} de calculer les points de contrôle à partir desquels les alarmes
(instructions potentiellement dangereuses) sont inatteignables. La génération
dynamique de tests réduit le nombre de faux positifs de l'analyse statique, et
cette dernière guide l'exploration pour la génération dynamique de tests.

\subsection{Vérification collaborative et test}

Cette méthode \cite{collaborative-verification} part du constat que la plupart
des outils de vérifications statiques (outils utilisant simplement des
heuristiques, outils d'interprétation abstraite, model-checkers, ou outils de
preuve) font des compromis afin d'améliorer les performances, de réduire le
nombre de faux positifs ou de limiter l'effort à fournir pour annoter le
programme. Cela se traduit par la supposition d'une propriété tout au long de
l'analyse dy programme (par exemple qu'un certains type d'erreurs ne peut pas
se produire), cette propriété n'est pas vérifiée par l'analyseur. Ceci
implique que de tels analyseurs ne peuvent garantir l'absence d'erreurs dans un
programme. Si un analyseur fait un compromis en supposant une propriété, il
faut utiliser un autre analyseur capable de valider cette propriété.

Cette méthode propose d'exprimer les compromis dans un langage de contrats,
afin de faciliter la collaboration entre plusieurs outils d'analyse statique
et permettant de décrire le fait qu'un assertion ait été complètement vérifiée
par un analyseur ou partiellement vérifiée sous certaines hypothèses.

Pour faciliter la vérification statique du programme, l'utilisateur doit au
préalable ajouter des annotations (invariants de boucle par exemple). Les
propriétés qui n'ont pas été vérifiées statiquement sont validées par des tests
unitaires. Le programme est instrumenté afin de rajouter des vérifications à
l'exécution pour guider la génération de contre-exemples. \textsc{Pex}
\cite{PEX} est utilisé pour générer des cas de test par exécution concolique,
mettant en évidence des contre-exemples. L'utilisateur peut décider de
privilégier l'analyse statique ou le test selon qu'il spécifie ou non son
programme.

Cette combinaison d'analyse statique, d'une instrumentation et du test est
comparable à \textsc{Sante} \cite{SANTE}, bien qu'il manque l'étape de slicing
qui permet à \textsc{Sante} d'économiser du temps sur l'analyse dynamique. En
revanche, cette méthode présente l'avantage de combiner plusieurs analyseurs
statiques et de vérifier partiellement des propriétés, ce qui n'est pas encore
possible avec \textsc{Sante}.

\subsection{Vérification de propriétés décrites par des automates finis}

Cette méthode \cite{checking-prop-state-machines} combine une instrumentation du
code source, une étape de {\em slicing} et une exécution symbolique.
L'instrumentation rajoute des instructions simulant le comportement de
l'automate fini correspondant au programme, dont les états correspondent aux
propriétés du programme à vérifier.
Le {\em slicing} est appliqué pour réduire la taille du programme, on ne
conserve que le code relatif aux états d'erreur de l'automate. Le programme
simplifié doit être équivalent au programme instrumenté en ce qui concerne
l'atteignabilité des états d'erreur de l'automate.
L'exécution symbolique du programme par \textsc{Klee} \cite{KLEE} va permettre
de mettre en évidence des contre-exemples pour ces propriétés.

L'utilisation du {\em slicing} afin de réduire la portion de code analysée
(et donc le nombre de chemins d'exécution) par l'exécution symbolique est
similaire à \textsc{Sante} \cite{SANTE}. En revanche, cette méthode ne semble
pas applicable quel que soit le type d'erreur.
\cite{checking-prop-state-machines} prend l'exemple de verrous à poser sur des
variables du programme. L'instrumentation repère des motifs bien précis dans le
programme et le code généré est spécifique à ce type d'erreur. Contrairement à
\textsc{Sante}, cette méthode n'applique pas d'analyse statique plus générale
(pouvant lever différents types d'alarmes) avant l'instrumentation.

\subsection{Localisation d'erreurs par slicing guidé par la trace d'exécution}

Cette méthode \cite{fault-localization} utilise un {\em slicing} arrière à
partir d'une instruction de déréférencement produisant une {\em Null Pointer
Exception} (en Java). Le {\em slicing} est guidé par la trace du programme
fournissant la pile des appels de méthode n'ayant pas terminé. Une analyse
statique des pointeurs est ensuite opérée sur le programme slicé afin de
déterminer si chacun des pointeurs peut être \lstinline[language=java]{null}.
Puis une analyse d'alias est opérée afin d'augmenter la précision de l'analyse
statique.

Contrairement à \textsc{Sante} \cite{SANTE} qui commence par utiliser une
analyse statique, dont le résultat paramètre le {\em slicing}, puis termine par
une analyse dynamique; cette méthode commence par une analyse dynamique, dont
le résultat paramètre le {\em slicing}, puis termine par une analyse statique.

\subsection{\textsc{Blast}}

Cet outil \cite{BLAST} vérifie les propriétés temporelles de sûreté d'un
programme C, ou met en évidence un chemin d'exécution violant une propriété. Le
raffinement des abstractions du programme est basé sur une abstraction par
prédicat et la découverte des prédicats se fait par interpolation. C'est une
implémentation de la méthode \textsc{Cegar} \cite{CEGAR} tout comme
\textsc{Slam} \cite{SLAM} et \textsc{Magic} \cite{MAGIC}. La génération des cas
de test se fait par exécution symbolique.

\textsc{Blast} ne traite ni les débordements arithmétiques ni les opérations
bit-à-bit, et considère que toutes les opérations arithmétiques sur les
pointeurs sont sûres. Le langage d'invariants utilisé pour décrire les
propriétés ne contient pas de quantificateurs, contrairement à \textsc{Acsl}
\cite{ACSL}.

\subsection{\textsc{Smart}}

Cet outil \cite{SMART} (basé sur son prédécesseur \textsc{Dart} \cite{DART})
génère des tests par exécution concolique. Pour résoudre le problème de
l'explosion du nombre de chemins, il va calculer à la demande des résumés de
fonction qui sont des pré-conditions et post-conditions pour chaque fonction
(contraintes sur les variables en entrée et en sortie). Ces résumés vont être
réutilisés si possible afin d'éviter de ré-exécuter la fonction correspondante.
La génération automatique de ces résumés est présentée comme se faisant par
``analyse statique interprocédurale'' mais se rattacherait plutôt à une analyse
dynamique car utilisant une exécution symbolique (et résolution de contraintes).

\subsection{\textsc{Synergy}, \textsc{Dash} et \textsc{Yogi}}

\textsc{Synergy} \cite{SYNERGY} est un algorithme combinant du test (essayer
d'atteindre un état d'erreur) et une abstraction (trouver une abstraction
suffisamment précise montrant qu'aucun chemin ne peut atteindre un état
d'erreur). La sous-approximation du test et la sur-approximation de
l'abstraction sont raffinées de manière itérative. L'abstraction est utilisée
pour guider la génération de tests. Les tests sont utilisés pour décider
{\em où} raffiner l'abstraction.

Les états de l'abstraction, les régions, sont des classes d'équivalence des
états du programme concret. S'il n'existe aucun chemin de la région initiale
vers une région d'erreur, alors il n'existe aucune suite de transitions
concrètes menant d'un état initial concret à un état d'erreur concret.

\textsc{Dash} \cite{DASH} est une évolution de \textsc{Synergy}, prenant en
compte les appels de procédure et les pointeurs (contrairement à
\textsc{Synergy}). \textsc{Dash} raffine l'abstraction en utilisant uniquement
les relations d'alias mises en évidence par les tests. La génération de tests
guide non seulement {\em où} raffiner l'abstraction, mais aussi {\em comment}
la raffiner. Cet algorithme est implémenté dans l'outil \textsc{Yogi}
\cite{YOGI}.

\subsection{\textsc{Sage}}

Cet outil \cite{SAGE} utilise une analyse dynamique uniquement afin de générer
des tests pour des programmes au format binaire x86. Il combine une exécution
symbolique à du {\em fuzz testing}. Une première exécution avec des entrées
valides permet de récupérer une trace d'exécution du programme. L'exécution
symbolique de cette trace permet de collecter les contraintes du chemin
d'exécution. De nouveaux chemins sont générés par négation des contraintes,
à la manière de \textsc{PathCrawler} \cite{PathCrawler} et autres outils
similaires.

\subsection{\textsc{Smash}}

Cet outil \cite{SMASH} combine une abstraction par prédicats et une génération
dynamique de tests (par exécution concolique). Pour chaque fonction, un résumé
est calculé par analyse statique (vrai pour toutes les exécutions, permettant
de prouver l'absence d'erreurs, c'est une sur-approximation) et un autre résumé
est calculé par analyse dynamique (vrai pour quelques exécutions uniquement,
permettant de montrer l'existence d'erreurs, c'est une sous-approximation). Ces
résumés sont calculés à la demande et seront utilisés aussi bien par l'analyse
statique que par l'analyse dynamique (les deux analyses s'exécutent
simultanément).

Ces résumés sont progressivement raffinés pour chaque fonction, afin de prouver
qu'une propriété n'est jamais violée (si un résumé statique est applicable), ou
de mettre en évidence une exécution violant une propriété (si un résumé
dynamique est applicable). Par construction, il n'est pas
possible que les deux résumés soient applicables.


\begin{definition}[Analyse compositionnelle]
Mémoization des résultats intermédiaires sous la forme de résumés réutilisables.
\end{definition}


\subsubsection{Résumé des outils Microsoft}


\textsc{Slam} \cite{SLAM} utilise une abstraction par prédicat et un raffinement
de partition (implémentation de l'algorithme \textsc{Cegar} \cite{CEGAR}) pour
effectuer une analyse statique compositionnelle mais n'effectue pas d'analyse
dynamique. \textsc{Smart} \cite{SMART} exécute une analyse dynamique
compositionnelle. C'est une extension de l'algorithme d'analyse dynamique non
compositionnelle de \textsc{Dart} \cite{DART}, mais n'effectue pas d'analyse
statique. \textsc{Synergy} \cite{SYNERGY} combine \textsc{Slam} \cite{SLAM} et
\textsc{Dart} \cite{DART} (\textsc{Cegar} compositionnel + test non
compositionnel). L'algorithme est intra-procédural. \textsc{Dash} \cite{DASH}
étend l'algorithme \textsc{Synergy}, il est inter-procédural mais non
compositionnel. \textsc{Smash} \cite{SMASH} est la version compositionelle de
\textsc{Dash}.



La figure~\ref{fig:microsoft-summary} résume graphiquement ces outils.


\begin{figure}
  \centering
  \begin{scriptsize}
    \begin{tabular}{|c|c|c|c|c|c|c|}
      \hline
      \multirow{3}{*}{Outils} & \multicolumn{3}{c|}{Analyse statique}
      & \multicolumn{3}{c|}{Analyse dynamique} \\
      & \multirow{2}{*}{intra-proc} & \multicolumn{2}{c|}{inter-proc}
      & \multirow{2}{*}{intra-proc} & \multicolumn{2}{c|}{inter-proc} \\
      & & non-compo & compo & & non-compo & compo \\
      \hline
      \textsc{Slam} \cite{SLAM} & & & $\surd$ & & & \\
      \hline
      \textsc{Dart} \cite{DART} & & & & & $\surd$ & \\
      \hline
    \textsc{Smart} \cite{SMART} & & & & & & $\surd$ \\
    \hline
    \textsc{Synergy} \cite{SYNERGY} & $\surd$ & & & $\surd$ & & \\
    \hline
    \textsc{Dash} \cite{DASH} & & $\surd$ & & & $\surd$ & \\
    \hline
    \textsc{Smash} \cite{SMASH} & & & $\surd$ & & & $\surd$ \\
    \hline
    \end{tabular}
  \end{scriptsize}
  \caption{Type d'analyse et caractéristiques des outils de Microsoft}
  \label{fig:microsoft-summary}
\end{figure}




\subsection{Analyses statique et dynamique pour la génération d'invariants}

Cette méthode \cite{fromTestsToProofs} propose une solution au problème du
passage à l'échelle de la génération d'invariants (de programmes impératifs) par
résolution de contraintes. Les invariants arithmétiques linéaires sont générés
d'après les informations obtenues par interprétation abstraite du programme, par
exécution concrète et par exécution symbolique du programme. Ces informations
permettent de générer des contraintes qui vont permettre au solveur de
contraintes de simplifier le système de contraintes et de réduire l'espace de
recherche.

\subsection{\textsc{Cegar} : Raffinement d'abstraction guidé par des
  contre-exemples}

Le raffinement d'abstraction guidé par des contre-exemples \cite{CEGAR} associe
l’abstraction par prédicats et le model-checking : une abstraction du programme
est générée à partir d’un ensemble de prédicats et invariants. Si le
model-checking prouve la non-accessibilité des états d'erreurs de l'automate,
alors le modèle concret est correct (pas d'erreurs de sûreté ou de liveness).
Si le model-checking a trouvé un contre-exemple pour le modèle abstrait il faut
déterminer s'il correspond à un contre-exemple réel dans le système concret.
Pour cela, l'algorithme détermine s'il existe une trace d'exécution concrète
correspondant à la trace d'exécution abstraite aboutissant au contre-exemple.
Si une telle trace existe, un bug a été trouvé. Sinon, de nouveaux prédicats
seront créés pour raffiner l'abstraction afin que ce contre-exemple en soit
absent à la prochaine itération. Et ainsi de suite. Ce processus peut ne pas
terminer. Plusieurs outils de vérification se basent sur cette méthode, parmi
lesquels \textsc{Blast} \cite{BLAST}, \textsc{Magic} \cite{MAGIC} et
\textsc{Slam} \cite{SLAM}.

Cette approche élimine les contre-exemples un par un, l'abstraction ainsi
générée peut donc mettre un certain temps avant de converger vers une forme
acceptable. Une amélioration a été proposée afin d'accélérer la convergence du
raffinement de l'abstraction : \textsc{Cegaar} \cite{CEGAAR}. Cette technique
élimine une infinité de contre-exemples (traces infaisables) de la forme
$\alpha . \lambda^* . \beta$ en une seule étape, où $\lambda$ correspond à une
ou plusieurs exécutions d'une boucle.

\subsection{\textsc{DSD-Crasher} et \textsc{Check'n'Crash}}

L'outil \textsc{DSD-Crasher} \cite{DSD-Crasher} combine une première analyse
dynamique, une analyse statique et une seconde analyse dynamique. La première
analyse dynamique utilise une génération de tests et des techniques
d'apprentissage pour générer des invariants probables (inférence de la
spécification par \textsc{Daikon} \cite{discover-invariants}).

Les deux dernières étapes sont celles de l'outil \textsc{Check'n'Crash}
\cite{ChecknCrash}. L'analyse statique (\textsc{ESC/Java} \cite{ESC/Java}) va
émettre des alarmes concernant le non-respect des invariants, et la seconde
analyse dynamique va tenter de confirmer ces alarmes par résolution de
contraintes et génération de tests (\textsc{JCrasher} \cite{JCrasher}). La
classification des alarmes et la génération de tests sont très dépendantes de
la qualité des invariants générés par \textsc{Daikon}.

Cette méthode se distingue de \textsc{Sante} \cite{SANTE} par son analyse
dynamique préliminaire qui détecte des invariants afin de guider l'analyse
statique. En revanche, elle n'utilise pas le {\em slicing} qui permet
d'augmenter le taux de classification des alarmes par l'analyse dynamique.

\subsection{Génération de données de test par algorithme génétique}

\cite{search-based-testing-of-NPE} propose une méthode de génération de données
de test mettant en évidence des {\em Null Pointer Exceptions} de Java. Tout
d'abord une analyse inter-procédurale du flot de contrôle et du flôt de données
collecte les chemins menant aux exceptions. Cette analyse se fait en arrière, en
partant des exceptions, propageant les contraintes sur les entrées dans le CFG.
Les entrées de test sont ensuite générées par un algorithme génétique, dans le
but de couvrir ces chemins. Les individus (des entrées potentielles), dont le
type de donnée peut être complexe, sont encodés sous forme XML.

Cette méthode a été comparée avec d'autres façons de générer des données de
test \cite{DbZ-exception-raising-via-branch-coverage}, leurs expérimentations
ont montré qu'elle était plus efficace que d'autres stratégies de recherche
optimale comme le {\em hill climbing} et le recuit simulé, mais est moins
efficace que la programmation par contraintes (en terme de temps d'exécution). 

\subsection{Génération de tests unitaires à partir de preuves formelles}

\cite{generating-unit-tests-from-formal-proofs} présente une méthode de
génération automatique de tests unitaires pour \textsc{Java Card} à partir d'une
tentative de preuve formelle du système, qui doit être au préalable annoté en
utilisant le langage de spécification \textsc{JML} \cite{JML}.

L'information contenue dans la preuve (même partielle) est utilisée
pour extraire des données de test à partir des conditions de chemins. Les
oracles sont générés à partir des postconditions. En revanche, il n'y a aucune
garantie sur le critère de couverture des chemins par les tests générés,
celle-ci dépend de la qualité de la spécification du code (présence des
invariants de boucle, etc.).


\section{Langages de spécification}


TODO


\section{Frama-C}


\textsc{Frama-C} \cite{Frama-C} est une plate-forme dédiée à l'analyse statique
des programmes C, conjointement dévelopée par INRIA et le CEA LIST. Son
architecture (Fig.~\ref{fig:archi}) comporte un noyau et un écosystème de
greffons, rendant l’outil extensible. Les greffons peuvent échanger des
informations et utiliser les  services fournis par le noyau, permettant ainsi
une collaboration entre différentes analyses.




\begin{figure}[h]
  \begin{center}
    \includegraphics{frama_c_architecture.mps}
  \end{center}
  \caption{Architecture de \textsc{Frama-C}}\label{fig:archi}
\end{figure}


\textsc{Frama-C} est basé sur CIL \cite{CIL}, une bibliothèque qui normalise des
programmes C (ISO C99) en opérant des modifications syntaxiques : normalisation
des boucles en utilisant la structure \lstinline{while}, unique
\lstinline{return} pour
chaque fonction, etc. \textsc{Frama-C} étend CIL pour supporter des annotations
dédiées portant sur le code source, exprimées dans le langage \textsc{Acsl}.
\textsc{Acsl} \cite{ACSL} est un langage formel de spécification
comportementale \cite{BISL}, inspiré de JML \cite{JML}, pouvant exprimer des
propriétés fonctionnelles de programmes C : pré-conditions, post-conditions,
invariants, etc.

En effet, la spécification d'une fonction comprend les pré-conditions requises
(exprimées par une clause \lstinline{requires}) lors de l'appel et les
post-conditions assurées (\lstinline{ensures}) lors du retour. Parmi ces
post-conditions, une clause indique quels sont les emplacements mémoire qui
peuvent être affectés (\lstinline{assigns}) par la fonction.


\begin{figure}[h]
\begin{lstlisting}
/*@ requires \valid(a) && \valid(b);
    requires \separated(a,b);
    assigns *a, *b;
    ensures *a == \at(*b,Pre);
    ensures *b == \at(*a,Pre); */
void swap(int* a, int* b);
\end{lstlisting}
\caption{Exemple de spécification \textsc{Acsl}}\label{fig:acsl-spec}
\end{figure}


Considérons par exemple une spécification fournie pour une fonction
\lstinline{swap}
(Fig.~\ref{fig:acsl-spec}). La première pré-condition établit que les deux
arguments doivent être des pointeurs valides, autrement dit, le déréférencement
de $a$ ou de $b$ ne produira pas d'erreur à l'exécution. La seconde
pré-condition impose que les emplacements mémoire occupés par chacune de ces
variables soient disjoints. En plus de \lstinline{\valid} et
\lstinline{\separated}, \textsc{Acsl} fournit de nombreux prédicats et
fonctions afin de décrire les états de la mémoire. \lstinline{\at(e,l)}
fait référence à la valeur de l'expression \lstinline{e} à l'état de la mémoire
au label \lstinline{l}. \lstinline{Pre} est un label prédéfini qui fait
référence à l'état de la mémoire avant l'exécution de la fonction. Ainsi, les
post-conditions (\lstinline{ensures}) signifient qu'à la fin de la fonction,
\lstinline{*a} aura la valeur que \lstinline{*b} avait au
début de la fonction, et réciproquement.


\textsc{Acsl} offre aussi la possibilité d'écrire des annotations dans le code
source, permettant d'exprimer des propriétés devant être vraies à un point donné
du programme : les assertions (\lstinline{assert}).
Il est également possible d'exprimer des propriétés devant être vraies avant une
boucle et après chaque itération de cette boucle : les invariants de boucle
(\lstinline{loop invariant}).


Les annotations du langage \textsc{Acsl} sont écrites en utilisant la logique
du premier ordre, et il est possible de définir ses propres fonctions et
prédicats.
Les greffons peuvent valider ou invalider les propriétés \textsc{Acsl} et
générer des annotations \textsc{Acsl}, les annotations sont donc un moyen
d'échanger des informations entre les différentes analyses opérées par les
greffons.


\section{PathCrawler}

\textsc{PathCrawler} \cite{PathCrawler} est un outil de génération de tests
structurels pour les programmes C, accessible sous la forme d'un service web :
\textsc{PathCrawler Online} \cite{PathCrawlerOnline}.

Étant donné un programme C sous test $p$ et une pré-condition sur ses entrées,
il génère des cas de test respectant un critère de couverture de test. Le
critère \emph{tous les chemins} impose une couverture de tous les chemins
faisables de $p$. L'exploration exhaustive de tous les chemins étant en pratique
irréalisable sur des programmes réels, le critère \emph{tous les k-chemins} a
été défini, il limite l'exploration aux chemins qui ont au plus $k$ itérations
consécutives de chaque boucle.

\textsc{PathCrawler} commence par construire une version instrumentée de $p$
permettant de tracer l'exécution de chaque cas de test, puis il génère les
contraintes représentant la sémantique de chaque instruction de $p$. La
prochaine étape est la génération et la résolution de contraintes pour produire
les cas de test pour un ensemble de chemins $\Pi$ satisfaisant le critère de
couverture. La résolution de contraintes s'effectue à l'aide
d'\textsc{ECLiPSe Prolog} \cite{ECLiPSe}, un environnement de programmation en
logique par contraintes basé sur Prolog.

Étant donné un préfixe de chemin $\pi$, c'est-à-dire un chemin partiel de $p$,
l'idée est de résoudre les contraintes correspondant à l'exécution symbolique
de $p$ en suivant le chemin $\pi$.
 
La méthode de génération de test est composée des étapes suivantes :

\begin{itemize}
\item[$(\mathcal{G}_1)$]
Création d'une variable logique pour chaque entrée.
Prise en compte des contraintes de la pré-condition.
Le préfixe de chemin initial $\pi$ est vide.
Aller à $(\mathcal{G}_2)$.

\item[$(\mathcal{G}_2)$]
Exécuter symboliquement le chemin $\pi$ : ajout des contraintes et
mise à jour de la mémoire en fonction des instructions de $\pi$.
Si certaines contraintes sont insatisfiables, aller à $(\mathcal{G}_5)$.
Sinon, aller à $(\mathcal{G}_3)$.

\item[$(\mathcal{G}_3)$]
Appeler le solveur de contraintes pour générer un cas de test $t$ satisfaisant
les contraintes du chemin courant. Si les contraintes sont insatisfiables, aller
à $(\mathcal{G}_5)$.
Sinon, aller à $(\mathcal{G}_4)$.

\item[$(\mathcal{G}_4)$]
Exécuter le programme avec trace sur le cas de test $t$ généré pour obtenir
le chemin d'exécution, qui doit commencer par $\pi$.
Aller à $(\mathcal{G}_5)$.

\item[$(\mathcal{G}_5)$]
Calculer le prochain chemin partiel $\pi$ à couvrir. Un parcours en profondeur
détermine la dernière décision $d$ pour laquelle il reste une branche à
explorer. S'il n'existe pas une telle décision, l'algorithme s'arrête. Sinon,
$\pi$ est recalculé et contient maintenant le chemin partiel précédent dans
lequel les contraintes correspondant à $d$ ont été niées, et retour à l'étape
$(\mathcal{G}_2)$. Cela nous assure que tous les chemins faisables sont couverts
(en considérant que le solveur de contraintes peut trouver une solution dans un
temps raisonnable) et que seulement le plus court des préfixes infaisables de
chaque chemin infaisable est exploré.
\end{itemize}


\chapter{Motivations et contributions}


Liste des contributions :
\begin{itemize}
\item extension d'une méthode de génération de tests structurels de programmes C
  à des programmes C annotés par le sous-ensemble du langage ACSL appelé E-ACSL
\item développement d'une méthode de combinaison Test et Preuve
\item proposition de scénarios types d'utilisation de cette méthode
\item implémentation de la combinaison Test et preuve dans STADY
\item expérimentations évaluant l'impact de la méthode
\item stage E-ACSL
\end{itemize}


\part{Combinaisons statique-dynamique}


\chapter{Existant en Preuve et Test}


Problématique ciblée : Preuves, échecs de
   preuve, raisons de ces échecs état de l'art sur la preuve de programmes
         
         test concolique, couverture


\chapter{Scénarios de vérification}


\section{Détection de non-conformité}
\section{Détection de faiblesse de contrat}
\section{Méthode globale: génération de contre-exemples pour la preuve}


\chapter{Annotations ACSL testables et exécutables}


\chapter{Preuve de correction}


\chapter{Modèle mémoire pour la validation d'assertions à l'exécution}


\section{Introduction}


Dans ce chapitre nous présentons une solution pour le monitoring mémoire des
programmes C, dévelopée pour la vérification d'assertions à l'exécution
({\em runtime assertion checking} \cite{RAC}) dans \textsc{Frama-C}
\cite{Frama-C}.
Cette solution inclut un langage de spécification exécutable, \textsc{e-acsl},
et un traducteur, \textsc{e-acsl2c}\footnote{http://frama-c.com/eacsl.html}.
Notre objectif est de pouvoir exécuter des annotations écrites avec le langage
de spécification \textsc{e-acsl}.

\subsection{\textsc{Executable-acsl}}

\textsc{e-acsl} est un sous-ensemble ``exécutable'' du langage \textsc{acsl}
implémenté dans \textsc{Frama-C}. Contrairement à \textsc{acsl}, chaque
spécification \textsc{e-acsl} est exécutable : elle peut être évaluée à
l'exécution.

\subsection{Bibliothèque de monitoring de la mémoire}

Afin de supporter les annotations relatives au modèle mémoire (validité d'un
pointeur, initialisation d'un bloc, appartenance à un bloc, etc.) nous devons
monitorer les opérations effectuées en mémoire par le programme. Pour ce faire,
nous avons développé une bibliothèque C permettant d'enregistrer et de récupérer
les informations de validité et d'initialisation d'un programme. Les appels aux
fonctions de cette bibiliothèque C sont automatiquement ajoutés aux emplacements
adéquats dans le code source du programme par \textsc{e-acsl2c} durant la
traduction de la spécification \textsc{e-acsl} en C.

\textsc{e-acsl2c} est un greffon de \textsc{Frama-C} qui traduit automatiquement
un programme C annoté en un autre programme dont l'exécution échouera si une
annotation n'est pas valide. Si aucune annotation n'est violée, le comportement
du nouveau programme est exactement le même que celui du programme d'origine.




\section{Informations relatives à la validité et à l'initialisation}


Détaillons les annotations \textsc{acsl} que nous souhaitons prendre en compte.
Elles sont au nombre de 6 : \lstinline{\base_addr}, \lstinline{\block\_length},
\lstinline{\offset}, \lstinline{\valid}, \lstinline{\valid_read} et
\lstinline{\initialized}.

\subsection{Adresse de base d'un bloc}

\lstinline'\base_addr{L}(p)' retourne l'adresse de base du bloc alloué
qui contient, au label \lstinline{L}, le pointeur \lstinline{p}.

\begin{figure}[h]
  \begin{center}
    \begin{tikzpicture}[>=latex,font=\sffamily]
      \node [memcell] at (0,0) (A) {};
      \node [anchor=west,memcell,full] at (A.east) (B) {$bloc_1$};
      \node [anchor=west,memcell,full,minimum width=4cm] at (B.east) (C)
            {$bloc_2$};
      \node [anchor=west,memcell] at (C.east) (D) {};
      \node [anchor=west,memcell,full] at (D.east) (E) {$bloc_3$};
      \node [anchor=west,memcell] at (E.east) (F) {};
      \path (C.south)+(0,-1) node (p) {\lstinline{p}};
      \path (C.south)+(-2,-1) node (basep) {\lstinline{\base_addr(p)}};
      
      \draw [->,line width=1.2pt] (basep) -- (C.south west);
      \draw [->,line width=1.2pt] (p) -- (C);
    \end{tikzpicture}
  \end{center}
  \caption{Adresse de base d'un bloc}
  \label{fig:base-addr}
\end{figure}

La Fig.~\ref{fig:base-addr} illustre \lstinline{\base_addr(p)}.

\begin{figure}[h]
\begin{lstlisting}
int main() {
  int *p = malloc(3*sizeof(int));
  //@ assert \base_addr(p+2) == p;
  free(p);
  return 0;
}
\end{lstlisting}
\caption{Exemple d'utilisation de ${\tt \backslash base\_addr}$}
\label{fig:base-addr-example}
\end{figure}

La Fig.~\ref{fig:base-addr-example}
illustre l'utilisation de \lstinline{\base_addr}, l'annotation de ce
programme est vraie.


\subsection{Longueur d'un bloc}

\lstinline'\block_length{L}(p)' retourne la longueur (en octets) du
bloc alloué qui contient, au label \lstinline{L}, le porinteur \lstinline{p}. 


\begin{figure}[h]
  \begin{center}
    \begin{tikzpicture}[>=latex,font=\sffamily]
      \node [memcell] at (0,0) (A) {};
      \node [anchor=west,memcell,full] at (A.east) (B) {$bloc_1$};
      \node [anchor=west,memcell,full,minimum width=4cm]at(B.east)(C){$bloc_2$};
      \node [anchor=west,memcell] at (C.east) (D) {};
      \node [anchor=west,memcell,full] at (D.east) (E) {$bloc_3$};
      \node [anchor=west,memcell] at (E.east) (F) {};
      \path (C.south west)+(0,-1) node (p) {\lstinline{p}};
      \draw [->,line width=1.2pt] (p) -- (C.south west);
      \draw [|<->|,line width=1.2pt]
      (C.south west) -- (C.south east)
      node [midway,below] {\lstinline{\block_length(p)}};
    \end{tikzpicture}
  \end{center}
  \caption{Longueur d'un bloc}
  \label{fig:block-length}
\end{figure}

La Fig.~\ref{fig:block-length} illustre \lstinline{\block_length(p)}.

\begin{figure}[h]
\begin{lstlisting}
int main() {
  int *p = malloc(3*sizeof(int));
  //@ assert \block_length(p) == 3*sizeof(int);
  free(p);
  return 0;
}
\end{lstlisting}
\caption{Exemple d'utilisation de ${\tt \backslash block\_length}$}
\label{fig:block-length-example}
\end{figure}

La Fig.~\ref{fig:block-length-example} illustre l'utilisation de
\lstinline{\block_length}, l'annotation de ce programme est vraie.


\subsection{Offset d'un pointeur dans un bloc}

\lstinline'\offset{L}(p)' retourne le décalage (en octets), au label
\lstinline{L}, entre \lstinline{p} et son adresse de base.


\begin{figure}[h]
  \begin{center}
    \begin{tikzpicture}[>=latex,font=\sffamily]
      \node [memcell] at (0,0) (A) {};
      \node [anchor=west,memcell,full] at (A.east) (B) {$bloc_1$};
      \node [anchor=west,memcell,full,minimum width=4cm] at (B.east) (C)
            {$bloc_2$};
      \node [anchor=west,memcell] at (C.east) (D) {};
      \node [anchor=west,memcell,full] at (D.east) (E) {$bloc_3$};
      \node [anchor=west,memcell] at (E.east) (F) {};
      \path (C.south)+(0,-1) node (p) {\lstinline{p}};
      
      \draw [->,line width=1.2pt] (p) -- (C);
      \draw [|<->|,line width=1.2pt]
      (C.south west) -- (C.south)
      node [midway,below] {\lstinline{\offset(p)}};
    \end{tikzpicture}
  \end{center}
  \caption{Offset d'un pointeur dans un bloc}
  \label{fig:offset}
\end{figure}

La Fig.~\ref{fig:offset} illustre \lstinline{\offset(p)}.


\begin{figure}[h]
\begin{lstlisting}
int main() {
  int *p = maloc(3*sizeof(int));
  //@ assert \offset(p+2) == 2;
  free(p);
  return 0;
}
\end{lstlisting}
\caption{Exemple d'utilisation de ${\tt \backslash offset}$}
\label{fig:offset-example}
\end{figure}

 La
Fig.~\ref{fig:offset-example} illustre l'utilisation de
\lstinline{\offset}, l'annotation de ce programme est vraie.


\subsection{Validité d'un bloc}

\lstinline'\valid{L}(p)' (respectivement \lstinline'\valid_read{L}(p)') est vrai
si le déréférencement de \lstinline{p} au label \lstinline{L} est autorisé en
lecture et en écriture (resp. au moins en lecture).

\lstinline'\valid{L}(p)' implique \lstinline'\valid_read{L}(p)'
mais l'inverse n'est pas vrai.




\begin{figure}[h]
  \begin{center}
    \begin{tikzpicture}[>=latex,font=\sffamily]
      \node [memcell] at (0,0) (A) {};
      \node [anchor=west,memcell,full] at (A.east) (B) {$bloc_1$};
      \node[anchor=west,memcell,full,minimum width=4cm] at(B.east)(C){$bloc_2$};
      \node [anchor=west,memcell] at (C.east) (D) {};
      \node [anchor=west,memcell,full] at (D.east) (E) {$bloc_3$};
      \node [anchor=west,memcell] at (E.east) (F) {};
      \path (C.south)+(0,-1) node (p) {\lstinline{p}};
      \path (C.south)+(0,-1.5) node {\lstinline{\valid(p)}};
      \path (D.south)+(0,-1) node (q) {\lstinline{q}};
      \path (D.south)+(0,-1.5) node {\lstinline{!\valid(q)}};
      \draw [->,line width=1.2pt] (p) -- (C);
      \draw [->,line width=1.2pt] (q) -- (D);
    \end{tikzpicture}
  \end{center}
  \caption{Validité d'un bloc}
  \label{fig:valid}
\end{figure}

La Fig.~\ref{fig:valid} illustre le prédicat \lstinline{\valid}.

\begin{figure}[h]
\begin{lstlisting}
int main(void) {
  int *a, *b;
  //@ assert ! \valid(a);
  //@ assert ! \valid(b);
  a = malloc(sizeof(int));
  b = a;
  //@ assert \valid(a);
  //@ assert \valid(b);
  free(b);
  //@ assert ! \valid(a);
  //@ assert ! \valid(b);
  return 0;
}
\end{lstlisting}
\caption{Exemple d'utilisation du prédicat ${\tt \backslash valid}$}
\label{fig:valid-example}
\end{figure}

La Fig.~\ref{fig:valid-example} illustre l'utilisations du prédicat
\lstinline{\valid}. Dans ce programme toutes les assertions de validité
sont vraies.


\subsection{Initialisation des octets d'un bloc}

\lstinline'\initialized{L}(p)' est un prédicat prenant un pointeur \lstinline{p}
sur une l-value en argument. Ce prédicat est vrai si la l-value en question est
initialisée au label \lstinline{L}.



\begin{figure}[h]
  \begin{center}
    \begin{tikzpicture}[>=latex,font=\sffamily]
      \node [memcell] at (0,0) (A) {};
      \node [anchor=west,memcell,full] at (A.east) (B) {$bloc_1$};
      \node [anchor=west,memcell,full,minimum width=4cm] at (B.east) (C) {
        \begin{tikzpicture}[>=latex,font=\sffamily]
          \node [memcell,init] at (0,0) (c1) {c1};
          \node [anchor=west,memcell,init] at (c1.east) (c2) {c2};
          \node [anchor=west,memcell,uninit] at (c2.east) (c3) {c3};
        \end{tikzpicture}
      };
      \node [anchor=west,memcell] at (C.east) (D) {};
      \node [anchor=west,memcell,full] at (D.east) (E) {$bloc_3$};
      \node [anchor=west,memcell] at (E.east) (F) {};
      \path (C.south)+(-1,-1) node (p) {\lstinline{p}};
      \path (C.south)+(-2,-1.5) node {\lstinline{\initialized(p)}};
      \path (C.south)+(1,-1) node (q) {\lstinline{q}};
      \path (C.south)+(2,-1.5) node {\lstinline{!\initialized(q)}};
      \draw [->,line width=1.2pt] (p) -- (p)+(0,.5);
      \draw [->,line width=1.2pt] (q) -- (q)+(0,.5);
    \end{tikzpicture}
  \end{center}
  \caption{Initialisation des octets d'un bloc}
  \label{fig:initialized}
\end{figure}

La Fig.~\ref{fig:initialized} illustre le prédicat \lstinline{\initialized}.


\begin{figure}[h]
\begin{lstlisting}
int main(void) {
  int *p = malloc(3*sizeof(int));
  p[0] = 0;
  //@ assert \initialized(p+0);
  //@ assert ! \initialized(p+1);
  free(p);
  return 0;
}
\end{lstlisting}
\caption{Exemple d'utilisation du prédicat ${\tt \backslash initialized}$}
\label{fig:initialized-example}
\end{figure}

La Fig.~\ref{fig:initialized-example} illustre
l'utilisation du prédicat \lstinline{\initialized}. Dans ce programme,
toutes les assertions sont vraies.


Pour pouvoir traiter ces annotations \textsc{acsl}, nous devons donc conserver
pour chaque bloc les informations suivantes :
\begin{itemize}
\item l'adresse de base
\item le nombre d'octets occupés
\item le nombre d'octets initialisés
\item l'initialisation de chaque octet (un bit par octet, sauf si aucun ou tous
  les octets sont initialisés)
\item un booléen indiquant si le bloc est en lecture seule (par exemple si c'est
  une chaîne littérale)
\item un booléen indiquant s'il y a eu un accès au bloc hors bornes
\end{itemize}




\section{Enregistrements et requêtes efficaces}


\subsection{Patricia tries}

Le code instrumenté pouvant accéder et modifier fréquemment les données du
$store$, une implémentation efficace du requiert une structure de données
offrant une bonne complexité en temps et en espace. Cette structure doit être
triée : on peut avoir besoin d'accéder directement à un bloc à partir de son
adresse de base, mais aussi à partir de n'importe quelle adresse contenue dans
le bloc (donc accéder au bloc précédent). Par exemple, la fonction
\lstinline{__base_addr(p)} utilisée pour le traitement de la construction
\textsc{acsl} \lstinline{\base_addr(p)} cherche l'adresse de base la plus
proche et inférieure à \lstinline{p} (et enfin vérifie les bornes du bloc).
Cette contrainte ne nous permet pas d'utiliser une table de hachage. Les listes
chaînées ne sont pas assez efficaces à cause de la complexité linéaire au pire
cas. Les arbres binaires de recherche non équilibrés ont aussi une complexité
linéaire au pire cas quand les données sont insérées dans un ordre strictement
monotone, ce qui est souvent le cas. Enfin, le coût du rééquilibrage de l'arbre
(pour un arbre binaire de recherche équilibré) serait amorti dans le cas où les
modifications de la structure de l'arbre sont moins nombreuses que les accès
simples; ce qui n'est pas nécessairement vrai sur les exemples de code que nous
avons instrumentés avec \textsc{e-acsl2c}.

Notre implémentation du $store$ se base sur un {\em Patricia trie}
\cite{Patricia-tries} (appelé aussi {\em radix tree} ou ``arbre à préfixe
compact'', cette structure est efficace même si l'abre n'est pas équilibré.
Les clés sont les adresses de base des blocs (c'est-à-dire des mots de 32 ou 64
bits) ou des préfixes d'adresses. Chaque feuille contient les données relatives
à un bloc en mémoire (voir section précédente pour le détail des informations
stockées). Le routage de la racine jusqu'à une feuille particulière se fait
grâce aux n\oe{}uds internes, chacun d'eux contient le plus grand préfixe commun
de l'adresse de base de ses deux fils. La Fig.~\ref{fig:insertion-Patricia-trie}
a illustre un Patricia trie (sur des adresses 8-bits pour des raisons de
simplicité).

Il contient 3 blocs dans ses feuilles (seules les adresses de base apparaissent
sur le schéma), et 2 préfixes stockés dans les n\oe{}uds internes. Le symbole
``{\tt *}'' signifie que la valeur du bit à cette position n'a pas d'importance.

La complexité théorique au pire cas d'un accès dans un Patricia trie dans notre
cas est en {\em O(k)} où $k$ est la longueur d'un mot (c'est-à-dire 32 ou 64
bits). En pratique, un programme ne pouvant allouer des blocs que dans un
espace mémoire limité, la profondeur de l'arbre est inférieure à cette borne.
De plus, contrairement aux chaînes de caractères (la première application des
Patricia tries), la comparaison des mots peut être implémentée très
efficacement par des opérations bit-à-bit.

Les données de chaque bloc n'occupent que quelques octets en mémoire, exception
faite des données d'initialisation du bloc. Le statut d'initialisation de chaque
octet est monitoré séparément
(les champs de bits ne sont pas encore supportés). Dans le pire cas (bloc
partiellement initialisé), chaque octet utilise un bit supplémentaire portant
l'information sur son initialisation. Dans le cas où tous les octets (ou aucun)
sont initialisés (on utilise un compteur d'octets initialisés), le tableau
censé contenir les bits portant l'information d'initialisation est libéré, et
cette information est donc portée uniquement par le compteur. De plus, nous
utilisons une fonction spécifique dans le cas où tous les octets d'un bloc sont
initialisés d'un coup, au lieu d'invoquer une fonction d'initialisation sur
chaque octet du bloc.


\subsection{Calcul du plus grand préfixe commun}


\begin{figure}[h]
\begin{lstlisting}
typedef unsigned char byte;
// index            0    1    2    3    4    5    6    7    8
byte  masks[] = {0x00,0x80,0xC0,0xE0,0xF0,0xF8,0xFC,0xFE,0xFF};
int longer [] = {   0,  -1,   3,  -3,   6,  -5,   7,   8,  -8};
int shorter[] = {   0,   0,   1,  -2,   2,  -4,   5,  -6,  -7};
byte gtCommonPrefixMask(byte a, byte b) {
  byte nxor = ~(a ^ b);   // a bit = 1 iff this bit is equal in a and b
  int i = 4;              // search starts in the middle of the word
  while(i > 0)            // we stop when i<=0
    if (nxor >= masks[i]) // first i bits equal,
      i = longer[i];      // try a longer prefix 
    else i = shorter[i];  // otherwise, try a shorter prefix 
  return masks[-i];       // when i<=0, masks[-i] is the answer
}
\end{lstlisting}
\caption{Calcul du plus grand préfixe commun}
\label{fig:prefix}
\end{figure}


Appelons ``masque du plus grand préfixe commun'' M de A et B. M est composé
d'une suite de $n$ ``1'', suivie d'une suite de ``0'', où $n$ est le nombre de
bits communs entre A et B. Par exemple, le plus grand préfixe commun de
A = \ppleaf{\texttt{0110\,0111}} et B = \ppleaf{\texttt{0111\,1111}} est P = 
\ppnode{\texttt{011*\,****}}
et le masque du plus grand préfixe commun est M = \texttt{1110\,0000}.

Les calculs et comparaisons des préfixes ont été optimisés par un usage
intensif des opérations bit-à-bit. Le calcul du plus grand préfixe commun a lui
aussi été reconçu pour de meilleures performances. L'implémentation naïve
initiale consistait en un parcours linéaire des mots mémoire de gauche à droite
jusqu'à trouver des bits différents, de la même manière qu'on pourrait le faire 
sur des chaînes de caractères.

La version optimisée de ce calcul consiste
maintenant en une recherche dichotomique dans un tableau pré-calculé qui
contient tous les préfixes possibles. Les transitions entre les étapes de la
recherche se font en utilisant des indices pré-calculés, de manière à obtenir
le prochain masque à essayer. La Fig.~\ref{fig:prefix} illustre cette
implémentation du calcul sur des mots de 8 bits. Les masques sont
stockés dans le tableau de la ligne 3. Les indices à utiliser pour tester un
masque plus long (resp. plus court) sont stockés dans le tableau ligne 4 (resp.
ligne 5). Par exemple, pour A et B définis plus haut, nxor = \texttt{1110\,0111}
et la fonction essaie i = 4, puis i = shorter[4] = 2, puis i = longer[2] = 3,
puis i = longer[3] = -3, pour finalement retourner mask[3] = \texttt{0xE0}, qui
est précisément \texttt{1110\,0000}.



\subsection{Recherche}



\subsubsection{Recherche exacte}

Cet algorithme retourne le block $B$ tel que l'adresse de base de $B$ soit égale
au pointeur passé en paramètre. On suppose que l'algorithme n'est utilisé que
lorsqu'un tel bloc existe.

Tant qu'on n'est pas sur une feuille -- on est donc sur un n\oe{}ud ayant deux
fils
-- on se dirige vers le fils ayant le plus grand préfixe commun avac l'adresse
passée en paramètre de l'algorithme. Quand on arrive sur une feuille, cette
dernière contient donc forcément l'adresse que l'on cherchait.

\subsubsection{Recheche du ``contenant''}

Cet algorithme retourne le block $B$ contenant l'adresse $ptr$ passée en
paramètre, tel que : $beginAddr_B \le ptr < beginAddr_B + size_B$. Si un tel
bloc n'existe pas, \lstinline{NULL} est renvoyé. Cet algorithme est utilisé pour
des requêtes du type \lstinline{\valid} ou \lstinline{\initialized} où
l'adresse passée en paramètre ne correspond pas forcément à l'adresse de base
d'un bloc.

Cet algorithme est similaire à celui de la recherche exacte, au détail près
qu'il faut vérifier la formule de l'encadrement de $ptr$ par $B$ quand on arrive
sur une feuille. On utilise une pile contenant les n\oe{}uds à partir desquels
il
faut reprendre le parcours si la recherche n'aboutit pas. Quand on explore le
fils droit, on empile le fils gauche. Quand on arrive sur une feuille, on
vérifie l'encadrement, s'il est vérifié on a trouvé le bloc qu'on cherchait,
sinon on utilise le dernier n\oe{}ud empilé (et on le dépile) s'il existe, sinon
on retourne \lstinline{NULL}.

\subsection{Ajout}


\begin{figure}[h]
  \begin{center}
    \begin{tabular}{ccc}
      \begin{tikzpicture}[grow=down,sibling distance=18mm,level distance=6mm,
          style={font=\scriptsize}]
        \node[pnode] {\texttt{0010\,****}}
        child { node[pleaf] {\texttt{0010\,0110}} }
        child { node[pnode] {\texttt{0010\,1***}}
          child { node[pleaf] {\texttt{0010\,1001}} }
          child { node[pleaf] {\texttt{0010\,1101}} }
        };
        \node at (-1.7,0) {\textbf{a)}};
      \end{tikzpicture}
& 
      \hspace{1mm} 
&
      \begin{tikzpicture}[grow=down,level 2/.style={sibling distance=17mm},
          sibling distance=35mm,level distance=6mm,style={font=\scriptsize}]
        \node[pnode] {\texttt{0010\,****}}
        child { node[pnode] {\texttt{0010\,011*}}
          child { node[pleaf] {\texttt{0010\,0110}} }
          child { node[pleaf] {\texttt{0010\,0111}} }
        }
        child { node[pnode] {\texttt{0010\,1***}}
          child { node[pleaf] {\texttt{0010\,1001}} }
          child { node[pleaf] {\texttt{0010\,1101}} }
        };
        \node at (-3.4,0) {\textbf{b)}};
     \end{tikzpicture}   
    \end{tabular}
  \end{center}
  \vspace{-3mm}
  \caption{Exemple de Patricia trie \textbf{a)} avant,
    et \textbf{b)} après insertion de \texttt{0010\,0111}}
  \vspace{-3mm}
  \label{fig:insertion-Patricia-trie}
\end{figure}


L'ajout d'un nouvel élément à l'arbre se déroule comme suit. Si l'arbre est
vide, l'élément est ajouté à la racine. Sinon, on recherche le n\oe{}ud $x$ le
plus
similaire au n\oe{}ud à insérer. Celui-ci sera son frère dans la nouvelle
configuration de l'arbre. Nous créons ensuite le n\oe{}ud correspondant au père.
Le
plus grand préfixe commun des deux fils est calculé pour le père, et les fils
sont ordonnés en fonction de leur adresse (le plus petit à gauche). Puis le
père est inséré dans l'arbre à l'ancien emplacement de $x$. Les champs de $x$ et
de l'ancien père de $x$ (s'il existe) sont également mis à jour pour maintenir
la cohérence de l'arbre. La Fig.~\ref{fig:insertion-Patricia-trie} illustre
l'insertion de l'adresse \ppleaf{\texttt{0010\,0111}} dans un arbre.
L'algorithme a déterminé que le n\oe{}ud le plus similaire serait
\ppleaf{\texttt{0010\,0110}}, et crée
le père correspondant : \ppnode{\texttt{0010\,011*}}.


\subsubsection{Recherche du n\oe{}ud le plus similaire}

Dès qu'on arrive sur une feuille on la renvoie. Sinon on calcule le plus grand
préfixe commun de l'adresse passée en paramètre avec le fils gauche, de même
avec le fils droit. Si l'un des préfixes est strictement supérieur à l'autre
on se dirige vers la branche correspondante, sinon on renvoie le n\oe{}ud
courant.



\subsection{Suppression}


\begin{figure}[h]
  \begin{center}
    \begin{tabular}{ccc}
      \begin{tikzpicture}[grow=down,level 2/.style={sibling distance=17mm},
          sibling distance=35mm,level distance=6mm,style={font=\scriptsize}]
        \node[pnode] {\texttt{0010\,****}}
        child { node[pnode] {\texttt{0010\,011*}}
          child { node[pleaf] {\texttt{0010\,0110}} }
          child { node[pleaf] {\texttt{0010\,0111}} }
        }
        child { node[pnode] {\texttt{0010\,1***}}
          child { node[pleaf] {\texttt{0010\,1001}} }
          child { node[pleaf] {\texttt{0010\,1101}} }
        };
        \node at (-3.4,0) {\textbf{a)}};
     \end{tikzpicture}   
& 
      \hspace{1mm} 
&
       \begin{tikzpicture}[grow=down,sibling distance=18mm,level distance=6mm,
          style={font=\scriptsize}]
        \node[pnode] {\texttt{0010\,****}}
        child { node[pleaf] {\texttt{0010\,0110}} }
        child { node[pnode] {\texttt{0010\,1***}}
          child { node[pleaf] {\texttt{0010\,1001}} }
          child { node[pleaf] {\texttt{0010\,1101}} }
        };
        \node at (-1.7,0) {\textbf{b)}};
      \end{tikzpicture}
    \end{tabular}
  \end{center}
  \vspace{-3mm}
  \caption{Exemple de Patricia trie \textbf{a)} avant,
    et \textbf{b)} après suppression de \texttt{0010\,0111}}
  \vspace{-3mm}
  \label{fig:suppression-Patricia-trie}
\end{figure}

La suppression d'une feuille $x$ du Patricia trie se déroule comme suit. Si cet
élément est la racine, l'arbre devient vide. Sinon, Cette feuille a un frère $y$
et un père. $x$ et son père sont supprimés, et $y$ remonte d'un niveau (et prend
donc la place du père). Enfin, les champs du nouveau père de $y$ sont mis à jour
pour maintenir la cohérence de l'arbre (le plus grand préfixe commun est
recalculé). La Fig.~\ref{fig:suppression-Patricia-trie} illustre la suppression
de l'adresse \ppleaf{\texttt{0010\,0111}} dans un Patricia trie. Ce n\oe{}ud
ainsi que son père \ppnode{\texttt{0010\,011*}} sont supprimés, et
\ppleaf{\texttt{0010\,0110}} est remonté d'un niveau.

\section{Conclusion}
conclusion + future work + application dans le greffon PathCrawler (transition
avec le chapitre suivant)


\part{Outils et expérimentations}


\begin{itemize}
\item évaluer la capacité à trouver des erreurs dans le code par détection de
  contre-exemples. On prend des codes faux engendrés par mutation, une
  spécification juste et on mesure pour combien de codes faux on trouve des
  contr-exemples.
\item même chose en mutant les spécifications
\item détection de sous-spécification (par mutation)
\end{itemize}

Mais comment évaluer ce qu'on annonce en conclusion ? :
\begin{itemize}
\item gain de temps de vérification (idée d'expérimenter en TP). 
\item capacité à détecter des erreurs plus tôt dans le processus ?
\end{itemize}


\chapter{Greffon Statique-Dynamique (StaDy)}


\section{Implémentation}


\section{Expérimentations}


\chapter{Modèle mémoire E-ACSL}


\section{Implémentation}


\section{Expérimentations}


Pour évaluer notre solution, nous avons effectué plus de 300 exécutions sur
plus de 30 programmes, obtenus à partir d'une dizaine d'exemples. Nous avons
volontairement gardés des exemples plutôt courts (moins de 200 lignes de code)
car ils ont dû être annotés en \textsc{acsl} manuellement.

Nous avons mesuré le temps d'exécution du programme d'origine et du code
instrumenté par \textsc{e-acsl2c} avec différentes options, afin d'évaluer les
performances des différentes implémentations et optimisations. Des indicateurs
comme le nombre de variables, d'allocations mémoires, d'enregistrements et de
requêtes a également été enregistré.



\begin{description}

\item[Implémentation du store] \hfill \\
Pour déterminer quelle implémentation du $store$ est la plus appropriée, nous
avons comparé des implémentations utilisant : des Patricia tries, des listes
chaînées, des arbres binaires de recherche non équilibrés et des Splay trees.

Notre implémentation utilisant les Patricia tries est en moyenne 2500 fois plus
rapide que l'implémentation à base de listes chaînées, 200 fois plus rapide que
celle utilisant les arbres binaires de recherche, et 27 fois plus rapide que
celle se basant sur les Splay trees.

La version utilisant les Splay trees offre
des performances comparables (ou légèrement meilleures, jusqu'à 3 fois) sur les
exemples contenant de fréquents accès mémoire consécutifs au même bloc dans le
$store$. En revanche, sur des examples où les accès méoire consécutifs ne se
font pas sur le même bloc (une multiplication de matrices dans notre exemple),
les performances sont beaucoup moins bonnes (jusqu'à 500 fois). Ceci est dû à
la nature des Splay treees : le dernier élément accédé est remonté à la racine
de l'arbre.

\item[Calcul du plus grand préfixe commun] \hfill \\
Nous avons comparé deux implémentations de ce calcul. La première utilise un
parcours linéaire de l'adresse (bit-à-bit, de gauche à droite). La seconde
est une recherche dichotomique du meilleur préfixe dans un tableau dont le
contenu et les indices (indiquant le prochain élément à tester) sont
pré-calculés. Cette seconde implémentation s'est révélée en moyenne 2.7 fois
plus rapide que la première sur nos exemples.

\item[Capacité de détection d'erreurs] \hfill \\
Nous avons utilisé le ``test mutationnel'' pour évaluer la capacité de détection
d'erreurs en utilisant la vérification d'assertion à l'exécution avec
\textsc{Frama-C}. Nous avons considéré 5 exemples annotés et généré leurs
{\em mutants} (en appliquant une {\em mutation} sur leur code source) et leur
avons appliqué la vérification à l'exécution. Les mutations incluent :
modifications d'opérateur arithmétique numérique, modifications d'opérateur
arithmétique sur les pointeurs, modifications d'opérateur de comparaison et
modifications d'opérateur logique ($land$ et $lor$).

L'outil de génération de test \textsc{PathCrawler} \cite{PathCrawler} a été
utilisé pour produire les cas de test. Chaque mutant a été instrumenté par
\textsc{e-acsl2c} et exécuté sur chaque cas de test pour vérifier que la
spécification était satisfaite à l'exécution. Les programmes d'origine passent
toutes les vérifications à l'exécution. Lorsqu'une violation d'une annotation a
été reportée pour au moins un cas de testn le mutant est considéré comme étant
{\em tué}. La Fig.~\ref{fig:mutation-exp} illustre les résultats. Exception
faite des mutants équivalents (lorsque la mutation produit un programme
équivalent au programme d'origine), tous les mutants erronés ont été tués.

\end{description}

La Fig.~\ref{fig:mmodel-exp} contient les résultats des expérimentations
comparant les différentes implémentations du $store$ et du calcul du plus grand
préfixe commun. bS$_{10000}$ est une recherche binaire dans un tableaux de 10000
éléments. iS$_{10000}$ est un tri par insertion d'un tableau de 10000 éléments.
mM$_{n^2}$ est une multiplication de matrices $n \times n$. mI$_{n^2}$ contient
des calculs matriciels (dont inversion et multiplication) sur des  matrices
$n \times n$. qS$_n$ est un tri rapide sur un tableau de $n$ éléments.
bbS$_{10000}$ est un tri à bulles sur un tableau à 10000 éléments. m$_{30000}$ est
une fusion de deux listes chaînées de 10000 et 20000 éléments. Rbt$_{10000}$ est
une insertion/suppression de 10000 éléments dans un arbre rouge et noir. mS$_n$
est un tri fusion d'une liste chaînée de $n$ éléments. La ligne supplémentaire
``+ RTE'' de chaque exemple correspond à une application préalable du greffon
\textsc{Rte} qui génère des assertions qui sont vraies si le programme ne
contient pas d'erreur à l'exécution.

Les colonnes ont la signification suivante : \danger contient le nombre
d'alarmes du programme,  $\emptyset$ contient le temps d'exécution du programme
original, bst correspond à l'implémentation par arbres binaires de recherche,
mask est le nombre de fois qu'est effectué le calcul du plus grand préfixe
commun, sb est le nombre d'insertion dans le $store$, Pt correspond à
l'implémentation par Patricia tries, St correspond à l'implémentation par Splay
trees. L'exposant $^1$ (respectivement $^2$) correspond aux expérimentations
sans (resp. avec) application d'une analyse statique $Dataflow$ permettant de
n'instrumenter que ce qui est nécessaire (section 6 de \cite{RV2013}). L'indice
$_1$ (resp. $_2$)
correspond à l'implémentation non optimisée (resp. optimisée) du calcul du
plus grand préfixe commun pour l'implémentation utilisant les Patricia tries.
Le temps d'analyse du programme avec \textsc{Valgrind} \cite{Valgrind} est
indiqué dans la dernière colonne.

Nous remarquons que le temps d'exécution de \textsc{Valgrind} n'est pas
comparable avec celui de notre solution, cela s'explique simplement par le fait
que celui-ci ne prend pas en compte la spécification \textsc{acsl}, et se
contente de vérifier des propriétés comme l'absence d'erreur de segmentation ou
l'absence de fuite de mémoire. En revanche, notre démarche vise à supporter au
maximum les annotations \textsc{acsl}, ce qui nécessite un monitoring plus
lourd.

Nos expérimentations, présentées dans la Fig.~\ref{fig:mmodel-exp}, confirment
nos hypothèses, à savoir :
\begin{itemize}
\item le Patricia trie est la structure de données la plus appropriée pour
  l'implémentation du $store$;
\item notre optimisation du calcul du plus grand préfixe commun par recherche
  dichotomique et utilisation d'indices pré-calculés entraîne un vrai gain de
  performance;
\item l'utilisation d'une analyse statique visant à réduire l'instrumentation
  du programme permet de réduire le temps d'exécution de manière efficace.
\end{itemize}


\begin{landscape}
  \begin{table}[h]
    \centering
    \begin{tabular}{l|c|c|c|c|c|c|c|HHc|c|HHc|c|c|c}
  & $\danger$ & $\emptyset$ & list & list-AS & bst & bst-AS & Pt & mask$^1$ & sb$^1$ & Pt-dicho & Pt-AS & mask$^2$ & sb$^2$ & Pt-dicho-AS & St & St-AS & valgrind \\
  \hline
  bS$_{10000}$ &22 &\multirow{2}{*}{.01} &1.10 &0.50 &1.14 &0.64 &1.55 &99 &16 &1.55 &0.57 &0 &1 &0.55 &1.39 &0.61 &\multirow{2}{*}{0.27}\\
  + RTE &41 &&1.10 &0.51 &1.14 &0.62 &1.59 &109 &16 &1.59 &0.53 &0 &1 &0.53 &1.39 &0.64 &\\
  \hline
  iS$_{10000}$ &5 &\multirow{2}{*}{.12} &2.29 &0.12 &1.83 &0.12 &2.89 &170k &20k &2.91 &0.12 &0 &0 &0.12 &2.46 &0.12 &\multirow{2}{*}{2.81}\\
  + RTE &24 &&3.52 &1.27 &2.89 &1.26 &3.99 &170k &20k &3.86 &1.25 &0 &0 &1.25 &3.46 &1.30 &\\
  \hline
  mM$_{100^2}$ &0 &\multirow{2}{*}{.01} &2.29 &0.73 &2.94 &1.15 &0.14 &17k &1k &0.14 &0.10 &5k &612 &0.09 &1.07 &0.98 &\multirow{2}{*}{0.34}\\
  + RTE &82 &&13.17 &10.66 &21.92 &17.39 &2.78 &18k &1k &3.00 &2.64 &5k &612 &2.82 &75.97 &73.62 &\\
  \hline
  mM$_{150^2}$ &0 &\multirow{2}{*}{.01} &13.23 &3.96 &15.65 &6.20 &0.54 &21k &2k &0.51 &0.36 &9k &912 &0.35 &5.86 &5.64 &\multirow{2}{*}{0.48}\\
  + RTE &82 &&72.36 &58.48 &110.70 &90.43 &10.77 &24k &2k &9.01 &8.57 &9k &912 &8.75 &403.50 &398.60 &\\
  \hline
  mI$_{100^2}$ &2 &\multirow{2}{*}{.01} &22.51 &0.10 &7.74 &0.13 &0.09 &68k &5k &0.08 &0.01 &7k &609 &0.01 &0.19 &0.10 &\multirow{2}{*}{0.35}\\
  + RTE &155 &&28.96 &4.22 &13.67 &5.48 &0.54 &73k &5k &0.55 &0.53 &7k &611 &0.47 &26.37 &26.16 &\\
  \hline
  mI$_{150^2}$ &2 &\multirow{2}{*}{.02} &130.04 &0.34 &40.35 &0.45 &0.28 &99k &8k &0.27 &0.02 &12k &909 &0.02 &0.68 &0.34 &\multirow{2}{*}{0.47}\\
  + RTE &155 &&153.30 &21.54 &73.55 &29.94 &2.00 &105k &8k &1.90 &1.42 &12k &911 &1.53 &146.15 &145.80 &\\
  \hline
  qS$_{1000}$ &15 &\multirow{2}{*}{.01} &12.70 &2.08 &1.76 &0.59 &0.33 &1M &92k &0.06 &0.13 &683k &39k &0.02 &0.02 &0.01 &\multirow{2}{*}{0.27}\\
  + RTE &32 &&12.38 &2.13 &1.64 &0.56 &0.38 &1M &92k &0.12 &0.14 &727k &39k &0.04 &0.03 &0.02 &\\
  \hline
  qS$_{2000}$ &15 &\multirow{2}{*}{.01} &85.99 &11.31 &8.39 &2.78 &0.71 &3M &198k &0.14 &0.28 &1M &84k &0.05 &0.03 &0.02 &\multirow{2}{*}{0.27}\\
  + RTE &32 &&81.65 &11.15 &7.72 &2.67 &1.13 &4M &198k &0.48 &0.36 &1M &84k &0.13 &0.05 &0.02 &\\
  \hline
  bbS$_{10000}$ &4 &\multirow{2}{*}{.22} &13.78 &1.02 &16.84 &1.67 &117.47 &499M &49M &22.36 &1.57 &30 &7 &1.54 &8.80 &1.67 &\multirow{2}{*}{3.36}\\
  + RTE &16 &&23.08 &4.64 &30.69 &7.16 &107.05 &599M &49M &32.58 &7.26 &29 &7 &6.90 &17.29 &7.21 &\\
  \hline
  m$_{30000}$ &2 &\multirow{2}{*}{.01} &412.10 &11.38 &176.35 &11.01 &1.11 &5M &420k &0.26 &0.30 &1M &60k &0.06 &0.08 &0.01 &\multirow{2}{*}{0.45}\\
  + RTE &49 &&451.58 &101.33 &219.12 &94.80 &1.15 &5M &420k &0.29 &0.47 &2M &130k &0.14 &0.10 &0.05 &\\
  \hline
  Rbt$_{10000}$ &0 &\multirow{2}{*}{.01} &47.39 &0.28 &48.44 &0.27 &0.32 &1M &159k &0.09 &0.03 &151k &10k &0.01 &0.59 &0.01 &\multirow{2}{*}{0.51}\\
  + RTE &270 &&120.02 &101.69 &165.77 &145.20 &0.47 &1M &159k &0.30 &0.39 &979k &119k &0.27 &18.82 &19.59 &\\
  \hline
  mS$_{1000}$ &7 &\multirow{2}{*}{.01} &6.45 &0.34 &6.32 &0.11 &0.32 &1M &95k &0.07 &0.06 &331k &18k &0.01 &0.02 &0.01 &\multirow{2}{*}{0.27}\\
  + RTE &45 &&6.82 &1.35 &7.98 &0.38 &0.34 &1M &95k &0.10 &0.13 &701k &38k &0.04 &0.02 &0.01 &\\
  \hline
  mS$_{5000}$ &7 &\multirow{2}{*}{.01} &362.87 &11.00 &218.01 &3.43 &2.28 &11M &562k &0.76 &0.43 &2M &106k &0.09 &0.14 &0.03 &\multirow{2}{*}{0.27}\\
  + RTE &45 &&371.40 &50.94 &290.88 &10.34 &2.46 &11M &562k &0.80 &0.83 &4M &218k &0.22 &0.16 &0.08 &\\
  \hline
  mS$_{10000}$ &7 &\multirow{2}{*}{.01} &3624.01 &47.94 &1673.00 &16.10 &6.46 &23M &1M &2.75 &1.00 &5M &223k &0.21 &0.31 &0.08 &\multirow{2}{*}{0.27}\\
  + RTE &45 &&3406.43 &257.18 &2086.32 &46.22 &6.30 &23M &1M &2.66 &1.83 &9M &457k &0.51 &0.35 &0.18 &\\
  \hline
  mS$_{50000}$ &7 &\multirow{2}{*}{.01} &$\infty$ &3847.72 &$\infty$ &1100.93 &135.54 &146M &6M &111.22 &6.90 &33M &1M &1.65 &2.08 &0.58 &\multirow{2}{*}{0.63}\\
  + RTE &45 &&$\infty$ &25554.08 &$\infty$ &2781.90 &118.86 &145M &6M &95.74 &11.64 &54M &2M &3.37 &2.18 &1.15 &\\
  \hline
  mS$_{100000}$ &7 &\multirow{2}{*}{.01} &$\infty$ &$\infty$ &$\infty$ &$\infty$ &631.41 &296M &14M &559.93 &13.55 &70M &2M &3.35 &4.03 &1.15 &\multirow{2}{*}{0.27}\\
  + RTE &45 &&$\infty$ &$\infty$ &$\infty$ &$\infty$ &573.47 &308M &14M &513.85 &25.02 &116M &5M &7.63 &4.68 &2.50 &\\
\end{tabular}

    \label{fig:mmodel-exp}
    \caption{Comparaison des différentes implémentations du $store$}
  \end{table}
\end{landscape}



\begin{tikzpicture}
  \begin{axis}[axis y line=left,width=\textwidth,height=\textwidth,ymode=log]
    \pgfplotstableread{data/table_eacsl_experiments_merge_sort.dat}\loadedtable;
    \foreach \i in {
      list,list-DFA,bst,bst-DFA,Pt,Pt-opti,Pt-DFA,Pt-opti-DFA,St,St-DFA} {
      \addplot table [x=N, y=\i] {\loadedtable};
    }
    \legend{list,list-DFA,bst,bst-DFA,Pt,Pt-opti,Pt-DFA,Pt-opti-DFA,St,St-DFA}
  \end{axis}
\end{tikzpicture}



\begin{table}[t]
  \centering
  \begin{tabular}{|c|c|c|c|c|c|c|}
    \hline
    & alarmes & mutants & équivalents & tués & \% erronés tués \\
    \hline
    fibonacci & 19  & 27 & 2 & 25 & 100\% \\
    \hline
    bubbleSort & 15  & 44 & 2 & 42 & 100\% \\
    \hline
    insertionSort & 10  & 39 & 3 & 36 & 100\% \\
    \hline
    binarySearch & 7 & 38 & 1 & 37 & 100\% \\
    \hline
    merge & 5 & 92 & 5 & 87 & 100\% \\
    \hline
  \end{tabular}
  \label{fig:mutation-exp}
  \caption{Capacité de détection d'erreurs}
\end{table}


\part{Conclusion}


\chapter{Bilan}


TODO


\chapter{Perspectives}


TODO



\backmatter

\bibliographystyle{phdthesisapa}
\bibliography{biblio}

\listoffigures
\listoftables
\listofdefinitions

\appendix
\part{Annexes}

\chapter{Premier chapitre des annexes}
\chapter{Second chapitre des annexes}

\end{document}

