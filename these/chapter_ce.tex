
\chapter{Génération de contre-exemples pour la preuve}
\label{sec:ce}

%\chapterintro

TODO

\section{Détection de non-conformité}
\section{Détection de faiblesse de contrat}

\begin{figure}[bt]
  \scriptsize{
    {\myinference[$\alpha$-assume-assigns]
      {(End_f, \mbox{\lstinline'x'}) \rulearrow (I, \mbox{\lstinline'e'})}
      {(End_f, \mbox{\lstinline'assigns x;'})
        \rulearrow
        I \concat (End_f, \mbox{\lstinline'e = nondet();'})}
    }

    {\myinference[$\alpha$-assume-post]
      {(End_f, \mbox{\lstinline'p'}) \rulearrow (I, \mbox{\lstinline'e'})}
      {(End_f, \mbox{\lstinline'ensures p;'})
        \rulearrow
        I \concat (End_f, \mbox{\lstinline'fassume(e);'})}
    }
  }
  \caption{Règles de traduction pour les faiblesses de contrats :
    postconditions et assigns}
  \label{fig:assume-annot}
\end{figure}

La règle \textsc{$\alpha$-assume-assigns} suppose que toutes les l-values
présentes dans la clause \lstinline'assigns' ont changé de valeur : elle affecte
une nouvelle valeur non déterministe à chacune d'elle à la fin de la fonction.
La règle \textsc{$\alpha$-assume-post} suppose sa validité à la fin de la
fonction instrumentée.

\begin{figure}[bt]
  \scriptsize{
    {\myinference[$\alpha$-assume-loop-assigns]
      {(l, \mbox{\lstinline'x'}) \rulearrow (I, \mbox{\lstinline'e'})}
      {(l, \mbox{\lstinline'loop assigns x;'})
        \rulearrow
        I \concat (l, \mbox{\lstinline'e = nondet();'})}
    }

    {\myinference[$\alpha$-assume-invariant-1]
      {(l, \mbox{\lstinline'p'}) \rulearrow (I, \mbox{\lstinline'e'})}
      {
        (l, \mbox{\lstinline'loop invariant p;'}) \rulearrow
        I \concat (l, \mbox{\lstinline'fassume(e && !loopcond);'})
      }
    }

    {\myinference[$\alpha$-assume-invariant-2]
      {(l, \mbox{\lstinline'p'}) \rulearrow (I_1, \mbox{\lstinline'e1'}) \\
        (EndIter_l, \mbox{\lstinline'p'}) \rulearrow
        (I_2, \mbox{\lstinline'e2'})}
      {
        (l, \mbox{\lstinline'loop invariant p;'}) \rulearrow
        I_1 \concat (l, \mbox{\lstinline'fassume(e1);'})
        \concat I_2 \concat (EndIter_l, \mbox{\lstinline'fassert(e2);'})
      }
    }
  }
  \caption{Règles de traduction pour les faiblesses de contrats :
    invariants et assigns de boucle}
  \label{fig:assume-loop-annot}
\end{figure}

La règle \textsc{$\alpha$-assume-loop-assigns} suppose que toutes les l-values
présentes dans la clause \lstinline'loop assigns' ont changé de valeur : une
nouvelle valeur non déterministe est affectée à chacune d'elle.
La règle \textsc{$\alpha$-assume-invariant-1} a vocation à remplacer toute la
boucle, elle suppose que le prédicat de l'invariant est vrai, ainsi que la
négation de la condition de boucle (appelée ici \lstinline'loopcond').
La règle \textsc{$\alpha$-assume-invariant-2} permet de supposer que la boucle
a déjà itéré un certain nombre de fois, elle suppose que l'invariant est vrai
avant la boucle, et vérifie l'invariant à la fin de chaque itération.

\section{Méthode globale}
