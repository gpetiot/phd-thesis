\documentclass{article}

\usepackage{helvet,hyperref}
\usepackage[T1]{fontenc}
\usepackage[utf8]{inputenc}
\usepackage[french]{babel}

\begin{document}

\begin{abstract}
  La vérification de logiciels repose le plus souvent sur une spécification
  formelle encodant les propriétés du programme à vérifier.
  La tâche de spécification et de vérification déductive des programmes est
  longue et difficile et nécessite une connaissance des outils de preuve de
  programmes.
  En effet, un échec de preuve de programme peut être dû \textsf{a)} à une
  non-conformité du code par rapport à sa spécification, \textsf{b)} à un
  contrat de boucle ou de fonction appelée trop faible pour prouver une autre
  propriété, ou \textsf{c)} à une incapacité du prouveur.
\end{abstract}

\end{document}
