\documentclass{article}

\usepackage{helvet,hyperref}
\usepackage[T1]{fontenc}
\usepackage[utf8]{inputenc}
\usepackage[french]{babel}

\begin{document}

\begin{abstract}
  La vérification de logiciels repose le plus souvent sur une spécification
  formelle encodant les propriétés du programme à vérifier.
  La tâche de spécification et de vérification déductive des programmes est
  longue et difficile et nécessite une connaissance des outils de preuve de
  programmes.
  En effet, un échec de preuve de programme peut être dû \textsf{a)} à une
  non-conformité du code par rapport à sa spécification, \textsf{b)} à un
  contrat de boucle ou de fonction appelée trop faible pour prouver une autre
  propriété, ou \textsf{c)} à une incapacité du prouveur.
  Il est souvent difficile pour l'utilisateur de décider laquelle de ces trois
  raisons est la cause de l'échec de la preuve car cette information n'est pas
  (ou rarement) donnée par le prouveur et requiert donc une revue approfondie
  du code et de la spécification.
  L'objectif de cette thèse est de fournir une méthode de diagnostic automatique
  des échecs de preuve afin d'améliorer le processus de spécification et de
  preuve des programmes C.
  Nous nous plaçons dans le cadre de la plate-forme d'analyse des programmes C
  Frama-C, qui fournit un langage de spécification unique ACSL, un greffon de
  vérification déductive WP et un générateur de tests structurels PathCrawler.

  La méthode que nous proposons consiste à diagnostiquer les échecs de preuve en
  utilisant la génération de tests structurels sur une version instrumentée du
  programme d'origine.
  Cette méthode consiste en deux étapes.
  Premièrement, nous essayons de détecter les non-conformités du code par
  rapport à la spécification.
  Cette phase nécessite une traduction en C des annotations ACSL afin
  d'expliciter les violations potentielles d'annotations par la création de
  nouvelles branches dans le programme.
  La préservation de la sémantique par cette traduction nous assure de trouver
  les erreurs dans le code et dans la spécification par génération de tests
  structurels à condition de couvrir tous les chemins d'exécution faisables du
  programme.
  Deuxièmement, si le programme ne contient pas de non-conformité, nous essayons
  de détecter les faiblesses de sous-contrats (contrats de boucle ou de fonction
  appelée).
  Cette seconde phase nécessite une traduction en C des annotations ACSL
  différente de la première, car nous voulons pouvoir ``exécuter'' certains
  contrats au lieu du code qu'ils spécifient.
  Si aucune de ces deux phases n'a fourni de raison quant à l'échec de la preuve
  et si la génération de tests a couvert tous les chemins d'exécution faisables
  du programme, alors l'échec de la preuve est dû à une incapacité du prouveur,
  sinon le problème reste indécidable.

  Nous présentons les règles de traduction des deux étapes de notre méthode
  ainsi qu'une justification rigoureuse de leur correction.
  Nous présentons également l'implémentation de notre méthode au sein d'un
  greffon de Frama-C, StaDy, ainsi que nos résultats expérimentaux mettant en
  évidence sa capacité et son efficacité de diagnostic des échecs de preuve.
\end{abstract}

\end{document}
