
\chapter{Traduction pour le test}
\label{sec:traduction}

\chapterintro

TODO

\section{Transformation de programme \acsl vers C}



\begin{figure}[h!]
  \begin{tabular}{lrl}
    \textit{instr} & ::= & \underline{\semicolon} \\
    & $\mid$ & \textit{type} \textit{id} \underline{\semicolon} \\
    & $\mid$ & \textit{left-value} \underline{\lstinline'='} \textit{exp}
    \underline{\semicolon} \\
    & $\mid$ & \textit{id} \underline{\lstinline'('} \textit{exp}$^{*}$
    \underline{\lstinline');'} \\
    & $\mid$ & \textit{left-value} \underline{\lstinline'='} \textit{id}
    \underline{\lstinline'('} \textit{exp}$^{*}$ \underline{\lstinline');'} \\
    & $\mid$ & \underline{\lstinline'return'} \textit{exp}
    \underline{\semicolon} \\
    & $\mid$ & \underline{$\bopen$} \textit{instr}$^{*}$ \underline{$\bclose$} \\
    & $\mid$ & \underline{\lstinline'if('} \textit{exp}
    \underline{\lstinline')'}
    \underline{$\bopen$} \textit{instr}$^{*}$ \underline{$\bclose$}
    \underline{\lstinline'else'} \underline{$\bopen$} \textit{instr}$^{*}$
    \underline{$\bclose$} \\
    & $\mid$ & \textit{label} \underline{\lstinline'\:'}
    \underline{\lstinline'/*@ assert'} \textit{pred}
    \underline{\lstinline'\; */'} \\
    & $\mid$ & \textit{label}$^{\textit{as}\;l}$ \underline{\lstinline'\:'}
    \underline{\lstinline'/*@ loop invariant'} \textit{pred}
    \underline{\semicolon} \\
    &        & \underline{\lstinline'    loop assigns'} \textit{term}$^{+}$
    \underline{\semicolon} \\
    &        & \underline{\lstinline'    loop variant'} \textit{term}
    \underline{\lstinline'\; */'} \\
    &        & \underline{\lstinline'while('} \textit{exp}
    \underline{\lstinline')'}
    \underline{$\bopen$}
    \textit{BegIter}$_l$ \underline{\lstinline'\:'} \underline{\semicolon}
    \textit{instr}$^{*}$
    \textit{EndIter}$_l$ \underline{\lstinline'\:'} \underline{\semicolon}
    \underline{$\bclose$} \\
    \textit{type} & ::= & \textit{type} \underline{\lstinline'*'}
    $\mid$ \underline{\lstinline'int'}
    $\mid$ \underline{\lstinline'char'}
    $\mid$ \underline{\lstinline'long'}
    $\mid$ \ldots \\
    \textit{function} & ::= & \underline{\lstinline'/*@ requires'}
    \textit{pred} \underline{\semicolon} \\
    &     & \underline{\lstinline'    typically'} \textit{pred}
    \underline{\semicolon} \\
    &     & \underline{\lstinline'    assigns'}
    \textit{term}$^{+}$ \underline{\semicolon} \\
    &     & \underline{\lstinline'    ensures'} \textit{pred}
    \underline{\lstinline'\; */'} \\
    &     & \textit{type}$^{\textit{as}\;T}$ \textit{id}$^{\textit{as}\;f}$
    \underline{\lstinline'('} (\textit{type} \textit{left-value})$^{*}$
    \underline{\lstinline')'} \underline{$\bopen$}
    \textit{Beg}$_f$ \underline{\lstinline'\:'}
    $T$ \underline{\lstinline'res;'}
    \textit{instr}$^{*}$
    \textit{End}$_f$ \underline{\lstinline'\:'}
    \underline{\lstinline'return res;'}
    \underline{$\bclose$} \\
  \end{tabular}
  \caption{Grammaire des instructions et fonctions C}
  \label{fig:gram-c}
\end{figure}


This section presents the instrumentation process for C functions annotated in
\textsc{ACSL} that we support during test generation for
the annotated code.
%either to check the annotations at runtime or to test them. 
It will be convenient to consider that specific labels 
are introduced for each statement and annotation inside a function body, as well
as for  the beginning and the end of each function body and loop body. 
(Strictly speaking, labels cannot be put just before declarations or ``\lstinline|}|'' in C, but this
can be easily fixed by adding a skip instruction ``\lstinline{;}'' after such
labels.)
We assume that functions respect the normal form defined by the syntactic
entity \textit{function} from the grammar in Fig.~\ref{fig:grammar}. 
In this
figure, the superscript ``${\textit{as}\;X}$'' means that any occurrence of $X$
in the analysis of the current rule should be replaced with the string referred
by the superscripted syntactic entity.
For example, if \texttt{foo} is the name $id$ of the analyzed function in the
rule $function$, labels $\texttt{Beg}_f$ and $\texttt{End}_f$ in this function's
body are replaced by $\texttt{Beg}_{\texttt{foo}}$ and $\texttt{End}_{\texttt{foo}}$
providing unique labels for the beginning and the end of the function body.
Array and pointer accesses are supposed to be written as \lstinline|*(p+i)|.
The terminal symbols are presented in a \texttt{typewriter font}.
\underline{Underlined} non-terminal symbols are not detailed because they are
part of the C or ACSL languages.
$Terms$ and $predicates$ are ACSL expressions, most of them are described in
Sec.~\ref{sec:term} and Sec.~\ref{sec:pred}.



\subsection{Sous-langage considéré}

The specification language \eacsl is a strict executable subset of \acsl, which
is a behavioral specification language implemented in \framac. On the one hand,
designed as a subset of \acsl{}, \eacsl{} preserves \acsl{} semantics.
Therefore, existing \framac analyzers supporting \acsl{} continue to be used
with \eacsl{} without any change. On the other hand, the \eacsl{} language is
\textit{executable}, that is, all its annotations can be translated into C and
executed at runtime.  Thus it can be used by dynamic analyses and monitors.  Due
to these two specific features %(preserving \acsl{} semantics and being
%executable), 
\eacsl{} facilitates combinations of static and dynamic analyses.

\eacsl is based on a typed first-order logic in which terms may contain
\textit{pure} (\emph{i.e.} side-effect free) C expressions and special keywords.
For instance, the \lstinline{\result} keyword allows the user to talk about the
result of a function, while \lstinline{\valid} is a builtin predicate stating
that its argument is a valid pointer. Quantifications are bounded by constraints
to finite intervals of integers
in order to remain executable. An \textsc{Eiffel}-like contract \cite{eiffel}
may be associated to each function in order to specify its pre- and
postconditions. These contracts may be split into several named guarded
behaviors for which the users may require completeness and/or
disjointness. Assertions, loop invariants and loop variants may also be
associated to statements. We now focus on two of the most important design
choices of the language: integers and undefinedness.

\subsubsection{Integers}\label{sec:eacsl-integers}

In addition to all machine types, \eacsl terms also include mathematical
integers of type  \lstinline{integer}: integer constants and operators, as well as logic variables are of
this type. Integer arithmetics is unbounded and never
overflows. \eacsl holds a small subtyping system to automatically coerce C
integral types into mathematical integers. For instance, if \lstinline{x} is a C
variable of type \lstinline{int}, \lstinline{x+1} and \lstinline{1} are of type
\lstinline{integer} and a coercion from \lstinline{int} to \lstinline{integer} is
introduced when typing \lstinline{x} in this context. This design was chosen for
several reasons. First, one of the main goals of \framac is program proving by
discharging proof obligations to automatic theorem provers. Such provers usually
work much better with mathematical arithmetics than with \textit{modular
  arithmetics}, that is, bounded arithmetics with overflows.  Second,
specifications are usually written without any implementation detail in mind,
and potential overflows are implementation details. Third, it is still possible
to use bounded modular arithmetics when required by using explicit casts: for
instance, \lstinline{(int)(INT_MAX + 1)} is equal to \lstinline{INT_MIN}, the
smallest representable value of \lstinline{int}. Fourth, this choice makes it
much easier to talk about potential overflows in specifications: for example,
thanks to mathematical arithmetics, 
\lstinline{/*@ assert INT_MIN <= x+y <= INT_MAX;*/} specifies in the easiest way
that \lstinline{x+y} must not overflow.
Unless otherwise stated, ``integer'' will refer below to ``unbounded integer''.

\subsubsection{Undefinedness}\label{sec:undefined}

\eacsl is executable. However, evaluation of undefined terms like \lstinline{1/0} is not
possible. To solve this issue, \eacsl follows Chalin's Runtime Assertion
Checking semantics~\cite{chalin10} by stating that semantics of such terms is
``undefined'': \eacsl uses a 3-valued logic~\cite{konikowska91} like
\textsc{Spark2014}~\cite{sparkERTS2014} or \textsc{JML}~\cite{assert}. It is
then the responsibility of the tools interpreting \eacsl{} to ensure that an
undefined term is never evaluated. An indirect consequence of this design is
that operators \lstinline{&&}\,, \ \lstinline{||}\,, \ \lstinline{_?_:_} \ and
\ \lstinline{==>} \ in \eacsl are lazy (like the C counterparts for the first
three of them).


\subsection{Support des entiers mathématiques}\label{sec:integers}


\lstinputlisting[style=c,escapechar=§,caption={Properties over integers: naive
    instrumentation gives false negative},label=lst:Z:1]
                {listings/naive_false_negative.c}

\lstinputlisting[style=c,caption={Properties over integers: naive
    instrumentation gives false positive},label=lst:Z:2]
                {listings/naive_false_positive.c}


Listing~\ref{lst:Z:1} and Listing~\ref{lst:Z:2} exhibit two examples where 
naive translation of annotations with mathematical integers leads to unsoundness.
 Let  \lstinline'x' be of type \lstinline'int'.
The assertion at line 1 of Listing~\ref{lst:Z:1}
is obviously false when \lstinline'x = INT_MAX'. 
The naive translation of this assertion
(line 3) uses machine integers with modular arithmetics
(we assume a 32-bit architecture), so 
\lstinline'x+1' remains less or equal to \lstinline'INT_MAX' for any value of
\lstinline'x', making it impossible to find the assertion violation.
The correct translation for this annotation
(sketched at lines 6--8) maintains the semantics of unbounded integer arithmetics
using an external unbounded integer library (we use GMP, the GNU Multi-Precision library)
to represent values that may overflow otherwise (like
\lstinline'INT_MAX+1' here).
It creates and initializes necessary variables for unbounded integers,
then computes and compares the values as unbounded integers.
The second example (Listing~\ref{lst:Z:2}) defines an ACSL assertion
(line 2) that is always correct here: for any positive integer \lstinline'x',
its successor also is positive. The naive instrumentation
(line 4) will exhibit an error for \lstinline'x = INT_MAX':
due to modular arithmetics, \lstinline'x+1' overflows and becomes negative,
violating the assertion. A correct translation using unbounded integers 
(not detailed here)
%(lines 7--9) 
maintains the expected behavior, so that the assertion remains valid for any positive integer
\lstinline'x'.
Using the naive translation for these two examples would result in a
false negative in the first case, and in a false positive in the second case,
undermining the tool's correctness and precision.
The translation rules for E-ACSL constructs %terms and predicates 
presented below %in Sec. \ref{sec:pred}, \ref{sec:term} and \ref{sec:annot} 
respect unbounded integer semantics of E-ACSL mathematical integers and
assume the usage of an external library as illustrated in Listing~\ref{lst:Z:1}.
To simplify the notation of code insertions, we will use the abbreviated notation
${}^{\square}$\lstinline{var} 
to indicate that the variable \lstinline{var} must be declared and
allocated (with \lstinline{Z_t var; Z_init(var);})
at the beginning of the inserted code, and the notation
\lstinline{var}${}^{\boxtimes}$ 
to indicate that the variable \lstinline{var} must be de-allocated
(with \lstinline{Z_clear(var);})
at the end of the inserted code.
We will also use underlined code fragments to indicate that the corresponding
operation (assignment, comparison, \dots) should be translated using
corresponding functions from the unbounded integer library.
So line 12 of Listing~\ref{lst:Z:1} illustrates abbreviated notation of 
instrumentation for lines 6--9.
As mentioned in Sec. \ref{sec:eacsl-integers}, %we assume that all 
type coersions  
are automatically made explicit in the annotations in \framac, 
so \lstinline'x+1' in line 1 of Listing~\ref{lst:Z:1}  becomes \lstinline'(integer)x+1'.


\subsection{Principe général}



\lstinputlisting[style=c,escapechar=§,caption={Instrumented version of program
    of Listing~\ref{lst:normalized}},label=lst:abbrv]
                {listings/is_present_instrumented.c}


Let us describe the principles of instrumentation for an annotated function
$f$ respecting the grammar of Fig.~\ref{fig:grammar}.
First, each input value $x$ (a formal parameter $x$ in $params$ or a global variable $x$) of type
$T$ is stored as \lstinline|T old_x = x;|
at the beginning of the instrumented FUT, i.e. at label $Beg_f$, in
the $decl^*$ section.
For an input array  (or pointer) \lstinline'x', the values are stored in the dynamically
allocated array \lstinline'old_val_x' whose size is inferred from the
We also generate an additional function named \lstinline'f_precond'
that is used to check the precondition of the FUT. For the  FUT  we ensure
that the precondition is assumed by inserting 
\lstinline|fassume(f_precond(x1, ..., xn));| 
at label $Beg_{f}$
(cf line 11 in Listing~\ref{lst:abbrv}).

Second, any \textsc{ACSL} annotation of the form \lstinline|kwd w| (where
keyword \lstinline|kwd| belongs to \{\lstinline|assert|, \lstinline|requires|,
\lstinline|typically|, \lstinline|ensures|, \lstinline|loop variant|,
\lstinline|loop invariant|\} and $w$ is a predicate or a term) is translated.
Some other \textsc{ACSL} constructs are not detailed here because they can be
obtained from the described \textsc{ACSL} fragment. For example, behaviors can
be rewritten as implications in \lstinline|ensures| annotations,
statement contracts can be rewritten as implicative \lstinline|assert|'s and
global (resp., loop or statement)
\lstinline|assigns| clauses can be rewritten as postconditions
(resp., loop invariants or assertions)
checking the non-modification of some variables. 

Each pair
$(label, annotation)$ is translated into a sequence of \emph{code insertions} 
$(l_1, c_1) \concat (l_2, c_2) \concat \ldots \concat (l_n, c_n)$, that 
represents a list of fragments of C programs $c_1, c_2, \ldots, c_n$ where
the fragment $c_i$ will be inserted into the instrumented program at label
$l_i$. Program fragments $c_i$ are parts of a correct program that might be
incomplete if taken separately
since a syntactically complete statement can be split into several insertions.
%for example \lstinline|while(1) {| and
%\lstinline|}| are two fragments of a larger C program containing a loop.
When there are several fragments to insert in the same location $l$, they
are inserted according  to their order in the list. \textsc{ACSL} annotations
are translated separately, and the  resulting sequences of insertions are treated 
respecting the order of annotations in the source program.

Translation of annotations is defined by the rules in Sec.~\ref{sec:annot}. It
requires translating \textsc{ACSL} terms and predicates. These
transformations are described by the rules in Sec.~\ref{sec:term} and
Sec.~\ref{sec:pred}. In these rules we use the following notation:
\begin{itemize}
\item $p$, $p_1$, $p_2$ and $p_3$ are \textsc{E-ACSL} predicates;
\item $t$, $t_1$, $t_2$ and $t_3$ are \textsc{E-ACSL} terms and $w$ is either a
  predicate or a term;
\item $c$, $c_1$, $c_2$ and $c_3$ are fragments of C programs;
%% TODO: renommer c_i en s_i ?
\item $e$, $e_1$, $e_2$ and $e_3$ are C expressions;
\item $l$, $l_1$, $l_2$, $l_3$, $Beg_f$, $End_f$, $BegIter_l$ and $EndIter_l$ are
  program labels;
\item $i$ is an identifier of a bounded variable in an \textsc{ACSL} predicate
  and an iterating counter in a C program, $x$ is an identifier of a C variable;
\item $I$, $I_1$, $I_2$ and $I_3$ are lists of code insertions $(l_i, c_i)$.
\end{itemize}



We denote by \lstinline'res', \lstinline'var_n', \lstinline'i_n',
\lstinline|old_x|, \lstinline'old_val_x' and \lstinline|old_variant|, fresh
variables, that is, identifiers different from all other identifiers of the
instrumented program.
When we use several times a rule introducing a fresh variable, all occurrences
must be different. It can be easily implemented incrementing the number
\lstinline'n' in \lstinline|var_n| or \lstinline'i_n' and replacing
\lstinline|x| in \lstinline|old_x| or \lstinline'old_val_x' by the
names of parameters of $f$ or global variables.

For convenience, we assume that all bounded variables in annotations and all program variables are different
from one another.
In particular, we can translate bounded variables into C without renaming.


\subsubsection{Exemple}
\label{sec:running-example}
%\vspace{-2mm}

\lstinputlisting[style=c,escapechar=§,caption={Annotated C function deciding if
    \lstinline{v} is present in array \lstinline{t} of size \lstinline{n}},
  label=lst:normalized]{listings/is_present_normalized.c}

We present in Listing~\ref{lst:normalized} an example of FUT  
normalized according to the grammar of Fig.~\ref{fig:grammar}. It returns $1$
when a given value is present in a given array, or $0$ otherwise. The
instrumented program obtained after translation of annotations of this function
is presented  (using abbreviated notation) in Listing~\ref{lst:abbrv}. The generated  precondition function is
defined at lines 1--6 (Listing~\ref{lst:abbrv}), it returns a nonzero value when the
precondition holds. The \lstinline|requires| clause in line 1
(Listing~\ref{lst:normalized}) stating that the array size \lstinline|n| is
positive  is
translated as the condition in line 3 (Listing~\ref{lst:abbrv}).
The \lstinline|requires| clause in line 2 (Listing~\ref{lst:normalized}) states
that \lstinline|(t+0)|$, ..., $\lstinline|(t+(n-1))| are valid pointers, and leads
to the condition in line 4 (Listing~\ref{lst:abbrv}). The
\lstinline|typically| clause 
in line 3 of Listing~\ref{lst:normalized} (translated
as line 5  of Listing~\ref{lst:abbrv}), 
is an extension of 
\acsl defining a precondition
considered only for testing. It strengthens the precondition to restrict
the (potentially too big) number of paths to be explored by test generation to
user-controlled partial coverage. Here it bounds the state space of
\lstinline|n| (and thus the size of \lstinline|t|) to $[0, 6]$ (it can be seen
as a domain {\em finitization} \cite{Korat}).
The loop invariant in line 9 of Listing~\ref{lst:normalized} 
%is bounding \lstinline|i|
%to values between $[0, \lstinline|n|]$. It 
is translated as lines 13--14 of
Listing~\ref{lst:abbrv} to check that the invariant holds before the first loop
iteration, and lines 21--22 of Listing~\ref{lst:abbrv} to check the preservation of
the invariant by any iteration. (An additional loop invariant 
\lstinline|\forall integer k; 0<=k<i ==> \old(*(t+k))!=v;|
necessary to formally prove the postcondition using deductive verification
was not included in this simplified example).
The loop variant line 10 in Listing~\ref{lst:normalized} is translated as
line 15 of Listing~\ref{lst:abbrv} to check that the variant is positive or
zero before the loop, line 18 of Listing~\ref{lst:abbrv} to back up the
value of the variant at the beginning of the loop, and lines 23--25 of
Listing~\ref{lst:abbrv} to check that it strictly decreases but remains positive
or zero, thus ensuring the termination of the loop.
The postcondition at lines 4-5 of Listing~\ref{lst:normalized} is translated as
lines 28--32 of Listing~\ref{lst:abbrv}, it states that \lstinline|\result| is
non zero if and only if there exists an element of \lstinline|t| equal
to \lstinline|v|.
The values of formal parameters \lstinline|t|, \lstinline|n| and
\lstinline|v|
are saved in lines 10--11
(Listing~\ref{lst:abbrv}), a new array \lstinline|old_val_t| saves the old values
contained in \lstinline|t|, it is allocated line 10, filled line 11 and
deallocated line 33.



\section{Traduction des termes \acsl}
\label{sec:term}

TODO


\section{Traduction des prédicats \acsl}
\label{sec:pred}

TODO


\section{Traduction des annotations \acsl}
\label{sec:annot}

TODO


\section{Différences entre le Test et la Validation d'Assertions à l'Exécution}

TODO
