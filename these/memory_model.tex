
\chapter{Modèle mémoire pour la validation d'assertion à l'exécution}

\section{Introduction}
introduction + related work (Valgrind, AddressSanitizer, ...)

\begin{description}
\item[Validation à l'exécution] \hfill \\
  {\em Runtime Assertion Checking}\\
\item[\textsc{Executable-acsl}] \hfill \\
\item[Gestion de la mémoire en C] \hfill \\
  {\em Memory safety}\\
\end{description}


\section{Informations relatives à la validité et à l'initialisation}
sous-section 3.2 de l'article RV2013

détail des prédicats E-ACSL gérés

{\color{red}schéma pour expliquer chaque prédicat}

présentation formelle des fonctions de la bibliothèque

\section{Stockage efficace des informations}
section 4 de l'article RV2013

utilisation des Patricia tries, possibilité d'utiliser autre chose

{\color{red}mettre quelques schémas}

\section{Enregistrements et requêtes efficaces}
section 5 de l'article RV2013

calcul du plus grand préfixe commun

\section{Analyse {\em Dataflow}}
{\color{red}développé par Julien, ai-je le droit d'en parler ici ?}


\section{Expérimentations}
comparaison des temps d'exécution suivant les implémentations du store

evaluation de l'efficacité sur les mutants générés

\section{Conclusion}
conclusion + future work + application dans le greffon PathCrawler (transition
avec le chapitre suivant)
