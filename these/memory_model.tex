
\chapter{Modèle mémoire pour la validation d'assertion à l'exécution}


%%%%%%%%%%%%%%%%%%%%%%%%%%%%%%%%%%%%%%%%%%%%%%%%%%%%%%%%%%%%%%%%


\section{Introduction}
introduction + related work (Valgrind, AddressSanitizer, ...)

\subsection{Validation à l'exécution}

{\em Runtime Assertion Checking}

\subsection{Gestion de la mémoire en C}

{\em Memory safety}

\subsection{\textsc{Executable-acsl} (\textsc{e-acsl})}

{\textsc e-acsl} est un sous-ensemble ``exécutable'' du langage \textsc{acsl}
implémenté dans \textsc{Frama-C}. Contrairement à \textsc{e-acsl}, chaque
spécification \textsc{e-acsl} est exécutable : elle peut être évaluée à
l'exécution.\\

Le travail présenté dans ce chapitre s'appuie sur et étend \textsc{e-acsl2c}
\footnote{http://frama-c.com/eacsl.html}, un greffon de \textsc{Frama-C}. Ce
dernier traduit automatiquement un programme C annoté en un autre programme dont
l'exécution échouera si une annotation n'est pas valide. Si aucune annotation
n'est violée, le comportement du nouveau programme est exactement le même que
celui du programme d'origine.





%%%%%%%%%%%%%%%%%%%%%%%%%%%%%%%%%%%%%%%%%%%%%%%%%%%%%%%%%%%%%%%



\section{Informations relatives à la validité et à l'initialisation}


Détaillons les annotations \textsc{acsl} que nous souhaitons prendre en compte.
Elles sont au nombre de 6 : ${\tt \backslash base\_addr}$,
${\tt \backslash block\_length}$, ${\tt \backslash offset}$,
${\tt \backslash valid}$, ${\tt \backslash valid\_read}$ et
${\tt \backslash initialized}$.

\subsection{Adresse de base d'un bloc}

${\tt \backslash base\_addr\{L\}(p)}$ retourne l'adresse de base du bloc alloué
qui contient, au label $L$, le pointeur $p$.

\begin{figure}[h]
  \begin{center}
    \begin{tikzpicture}[>=latex,font=\sffamily]
      \node [memcell] at (0,0) (A) {};
      \node [anchor=west,memcell,full] at (A.east) (B) {$bloc_1$};
      \node [anchor=west,memcell,full,minimum width=4cm] at (B.east) (C)
            {$bloc_2$};
      \node [anchor=west,memcell] at (C.east) (D) {};
      \node [anchor=west,memcell,full] at (D.east) (E) {$bloc_3$};
      \node [anchor=west,memcell] at (E.east) (F) {};
      \path (C.south)+(0,-1) node (p) {${\tt p}$};
      \path (C.south)+(-2,-1) node (basep) {${\tt \backslash base\_addr(p)}$};
      
      \draw [->,line width=1.2pt] (basep) -- (C.south west);
      \draw [->,line width=1.2pt] (p) -- (C);
    \end{tikzpicture}
  \end{center}
  \caption{Adresse de base d'un bloc}
  \label{fig:base-addr}
\end{figure}

La Fig.~\ref{fig:base-addr}
illustre ${\tt \backslash base\_addr(p)}$.

\begin{figure}[h]
\begin{lstlisting}
int main() {
  int *p = malloc(3*sizeof(int));
  //@ assert \base_addr(p+2) == p;
  free(p);
  return 0;
}
\end{lstlisting}
\caption{Exemple d'utilisation de ${\tt \backslash base\_addr}$}
\label{fig:base-addr-example}
\end{figure}

La Fig.~\ref{fig:base-addr-example}
illustre l'utilisation de ${\tt \backslash base\_addr}$, l'annotation de ce
programme est vraie.


\subsection{Longueur d'un bloc}

${\tt \backslash block\_length\{L\}(p)}$ retourne la longueur (en octets) du
bloc alloué qui contient, au label $L$, le porinteur $p$. 


\begin{figure}[h]
  \begin{center}
    \begin{tikzpicture}[>=latex,font=\sffamily]
      \node [memcell] at (0,0) (A) {};
      \node [anchor=west,memcell,full] at (A.east) (B) {$bloc_1$};
      \node [anchor=west,memcell,full,minimum width=4cm] at (B.east) (C)
            {$bloc_2$};
      \node [anchor=west,memcell] at (C.east) (D) {};
      \node [anchor=west,memcell,full] at (D.east) (E) {$bloc_3$};
      \node [anchor=west,memcell] at (E.east) (F) {};
      \path (C.south west)+(0,-1) node (p) {${\tt p}$};
      
      \draw [->,line width=1.2pt] (p) -- (C.south west);
      \draw [|<->|,line width=1.2pt]
      (C.south west) -- (C.south east)
      node [midway,below] {${\tt \backslash block\_length(p)}$};
    \end{tikzpicture}
  \end{center}
  \caption{Longueur d'un bloc}
  \label{fig:block-length}
\end{figure}

La Fig.~\ref{fig:block-length} illustre ${\tt \backslash block\_length(p)}$.

\begin{figure}[h]
\begin{lstlisting}
int main() {
  int *p = malloc(3*sizeof(int));
  //@ assert \block_length(p) == 3*sizeof(int);
  free(p);
  return 0;
}
\end{lstlisting}
\caption{Exemple d'utilisation de ${\tt \backslash block\_length}$}
\label{fig:block-length-example}
\end{figure}

La Fig.~\ref{fig:block-length-example} illustre l'utilisation de
${\tt \backslash block\_length}$, l'annotation de ce programme est vraie.


\subsection{Offset d'un pointeur dans un bloc}

${\tt \backslash offset\{L\}(p)}$ retourne le décalage entre $p$ et son adresse
de base.


\begin{figure}[h]
  \begin{center}
    \begin{tikzpicture}[>=latex,font=\sffamily]
      \node [memcell] at (0,0) (A) {};
      \node [anchor=west,memcell,full] at (A.east) (B) {$bloc_1$};
      \node [anchor=west,memcell,full,minimum width=4cm] at (B.east) (C)
            {$bloc_2$};
      \node [anchor=west,memcell] at (C.east) (D) {};
      \node [anchor=west,memcell,full] at (D.east) (E) {$bloc_3$};
      \node [anchor=west,memcell] at (E.east) (F) {};
      \path (C.south)+(0,-1) node (p) {${\tt p}$};
      
      \draw [->,line width=1.2pt] (p) -- (C);
      \draw [|<->|,line width=1.2pt]
      (C.south west) -- (C.south)
      node [midway,below] {${\tt \backslash offset(p)}$};
    \end{tikzpicture}
  \end{center}
  \caption{Offset d'un pointeur dans un bloc}
  \label{fig:offset}
\end{figure}

La Fig.~\ref{fig:offset} illustre ${\tt \backslash offset(p)}$.


\begin{figure}[h]
\begin{lstlisting}
int main() {
  int *p = maloc(3*sizeof(int));
  //@ assert \offset(p+2) == 2;
  free(p);
  return 0;
}
\end{lstlisting}
\caption{Exemple d'utilisation de ${\tt \backslash offset}$}
\label{fig:offset-example}
\end{figure}

 La
Fig.~\ref{fig:offset-example} illustre l'utilisation de
${\tt \backslash offset}$, l'annotation de ce programme est vraie.


\subsection{Validité d'un bloc}

${\tt \backslash valid\{L\}(p)}$ (respectivement
${\tt \backslash valid\_read\{L\}(p)}$) est vrai si le déréférencement de $p$
au label $L$ est autorisé en lecture et en écriture (resp. au moins en lecture).

${\tt \backslash valid\{L\}(p)}$ implique ${\tt \backslash valid\_read\{L\}(p)}$
mais l'inverse n'est pas vrai.




\begin{figure}[h]
  \begin{center}
    \begin{tikzpicture}[>=latex,font=\sffamily]
      \node [memcell] at (0,0) (A) {};
      \node [anchor=west,memcell,full] at (A.east) (B) {$bloc_1$};
      \node [anchor=west,memcell,full,minimum width=4cm] at (B.east) (C)
            {$bloc_2$};
      \node [anchor=west,memcell] at (C.east) (D) {};
      \node [anchor=west,memcell,full] at (D.east) (E) {$bloc_3$};
      \node [anchor=west,memcell] at (E.east) (F) {};
      \path (C.south)+(0,-1) node (p) {${\tt p}$};
      \path (C.south)+(0,-1.5) node {${\tt \backslash valid(p)}$};
      \path (D.south)+(0,-1) node (q) {${\tt q}$};
      \path (D.south)+(0,-1.5) node {${\tt \lnot \backslash valid(q)}$};
      
      \draw [->,line width=1.2pt] (p) -- (C);
      \draw [->,line width=1.2pt] (q) -- (D);
      
    \end{tikzpicture}
  \end{center}
  \caption{Validité d'un bloc}
  \label{fig:valid}
\end{figure}

La Fig.~\ref{fig:valid} illustre le prédicat ${\tt \backslash valid}$.

\begin{figure}[h]
\begin{lstlisting}
int main(void) {
  int *a, *b;
  //@ assert ! \valid(a) && ! \valid(b);
  a = malloc(sizeof(int));
  b = a;
  //@ assert \valid(a) && \valid(b);
  free(b);
  //@ assert ! \valid(a) && ! \valid(b);
  return 0;
}
\end{lstlisting}
\caption{Exemple d'utilisation du prédicat ${\tt \backslash valid}$}
\label{fig:valid-example}
\end{figure}

La Fig.~\ref{fig:valid-example} illustre l'utilisations du prédicat
${\tt \backslash valid}$. Dans ce programme toutes les assertions de validité
sont vraies.


\subsection{Initialisation des octets d'un bloc}

${\tt \backslash initialized\{L\}(p)}$ est un prédicat prenant un pointeur $p$
sur une l-value en argument. Ce prédicat est vrai si la l-value en question est
initialisée au label $L$.



\begin{figure}[h]
  \begin{center}
    \begin{tikzpicture}[>=latex,font=\sffamily]
      \node [memcell] at (0,0) (A) {};
      \node [anchor=west,memcell,full] at (A.east) (B) {$bloc_1$};
      \node [anchor=west,memcell,full,minimum width=4cm] at (B.east) (C)
            {
              \begin{tikzpicture}[>=latex,font=\sffamily]
                \node [memcell,init] at (0,0) (c1) {c1};
                \node [anchor=west,memcell,init] at (c1.east) (c2) {c2};
                \node [anchor=west,memcell,uninit] at (c2.east) (c3) {c3};
              \end{tikzpicture}
            };
      \node [anchor=west,memcell] at (C.east) (D) {};
      \node [anchor=west,memcell,full] at (D.east) (E) {$bloc_3$};
      \node [anchor=west,memcell] at (E.east) (F) {};

      \path (C.south)+(-1,-1) node (p) {${\tt p}$};
      \path (C.south)+(-2,-1.5) node {${\tt \backslash initialized(p)}$};
      \path (C.south)+(1,-1) node (q) {${\tt q}$};
      \path (C.south)+(2,-1.5) node {${\tt \lnot \backslash initialized(q)}$};
      
      \draw [->,line width=1.2pt] (p) -- (p)+(0,.5);
      \draw [->,line width=1.2pt] (q) -- (q)+(0,.5);
      
    \end{tikzpicture}
  \end{center}
  \caption{Initialisation des octets d'un bloc}
  \label{fig:initialized}
\end{figure}

La Fig.~\ref{fig:initialized} illustre le prédicat
${\tt \backslash initialized}$.


\begin{figure}[h]
\begin{lstlisting}
int main(void) {
  int *p = malloc(3*sizeof(int));
  p[0] = 0;
  //@ assert \initialized(p+0);
  //@ assert ! \initialized(p+1);
  free(p);
  return 0;
}
\end{lstlisting}
\caption{Exemple d'utilisation du prédicat ${\tt \backslash initialized}$}
\label{fig:initialized-example}
\end{figure}

La Fig.~\ref{fig:initialized-example} illustre
l'utilisation du prédicat ${\tt \backslash initialized}$. Dans ce programme,
toutes les assertions sont vraies.


~\\
Pour pouvoir traiter ces annotations \textsc{acsl}, nous devons donc conserver
pour chaque bloc les informations suivantes :
\begin{itemize}
\item l'adresse de base
\item le nombre d'octets occupés
\item le nombre d'octets initialisés
\item l'initialisation de chaque octet (un bit par octet, sauf si aucun ou tous
  les octets sont initialisés)
\item un booléen indiquant si le bloc est en lecture seule (par exemple si c'est
  une chaîne littérale)
\item un booléen indiquant s'il y a eu un accès au bloc hors bornes
\end{itemize}




%%%%%%%%%%%%%%%%%%%%%%%%%%%%%%%%%%%%%%%%%%%%%%%%%%%%%%%%%%%%%%%%%%



\section{Enregistrements et requêtes efficaces}

utilisation des Patricia tries, possibilité d'utiliser autre chose

calcul du plus grand préfixe commun




\subsection{Recherche}

schéma + algo

\subsection{Ajout}


\begin{figure}[h]
  \begin{center}
    \begin{tabular}{ccc}
      \begin{tikzpicture}[grow=down,sibling distance=18mm,level distance=6mm,
          style={font=\scriptsize}]
        \node {\texttt{0010\,****}}
        child { node[leaf] {\texttt{0010\,0110}} }
        child { node {\texttt{0010\,1***}}
          child { node[leaf] {\texttt{0010\,1001}} }
          child { node[leaf] {\texttt{0010\,1101}} }
        };
        \node at (-1.7,0) {\textbf{a)}};
      \end{tikzpicture}
& 
      \hspace{1mm} 
&
      \begin{tikzpicture}[grow=down,level 2/.style={sibling distance=17mm},
          sibling distance=35mm,level distance=6mm,style={font=\scriptsize}]
        \node {\texttt{0010\,****}}
        child { node {\texttt{0010\,011*}}
          child { node[leaf] {\texttt{0010\,0110}} }
          child { node[leaf] {\texttt{0010\,0111}} }
        }
        child { node {\texttt{0010\,1***}}
          child { node[leaf] {\texttt{0010\,1001}} }
          child { node[leaf] {\texttt{0010\,1101}} }
        };
        \node at (-3.4,0) {\textbf{b)}};
     \end{tikzpicture}   
    \end{tabular}
  \end{center}
  \vspace{-3mm}
  \caption{Exemple de Patricia trie \textbf{a)} avant,
    et \textbf{b)} après insertion de \texttt{0010\,0111}}
  \vspace{-3mm}
  \label{fig:insertion-Patricia-trie}
\end{figure}


schéma + algo

\subsection{Suppression}

schéma + algo




\section{Expérimentations}


Pour évaluer notre solution, nous avons effectué plus de 300 exécutions sur
plus de 30 programmes, obtenus à partir d'une dizaine d'exemples. Nous avons
volontairement gardés des exemples plutôt courts (moins de 200 lignes de code)
car ils ont dû être annotés en \textsc{acsl} manuellement.

Nous avons mesuré le temps d'exécution du programme d'origine et du code
instrumenté par \textsc{e-acsl2c} avec différentes options, afin d'évaluer les
performances des différentes implémentations et optimisations. Des indicateurs
comme le nombre de variables, d'allocations mémoires, d'enregistrements et de
requêtes a également été enregistré.



\begin{description}

\item[Implémentation du store] \hfill \\
Pour déterminer quelle implémentation du $store$ est la plus appropriée, nous
avons comparé des implémentations utilisant : des Patricia tries, des listes
chaînées, des arbres binaires de recherche non équilibrés et des Splay trees.

Notre implémentation utilisant les Patricia tries est en moyenne 2500 fois plus
rapide que l'implémentation à base de listes chaînées, 200 fois plus rapide que
celle utilisant les arbres binaires de recherche, et 27 fois plus rapide que
celle se basant sur les Splay trees.

La version utilisant les Splay trees offre
des performances comparables (ou légèrement meilleures, jusqu'à 3 fois) sur les
exemples contenant de fréquents accès mémoire consécutifs au même bloc dans le
$store$. En revanche, sur des examples où les accès méoire consécutifs ne se
font pas sur le même bloc (une multiplication de matrices dans notre exemple),
les performances sont beaucoup moins bonnes (jusqu'à 500 fois). Ceci est dû à
la nature des Splay treees : le dernier élément accédé est remonté à la racine
de l'arbre.

\item[Calcul du plus grand préfixe commun] \hfill \\
Nous avons comparé deux implémentations de ce calcul. La première utilise un
parcours linéaire de l'adresse (bit-à-bit, de gauche à droite). La seconde
est une recherche dichotomique du meilleur préfixe dans un tableau dont le
contenu et les indices (indiquant le prochain élément à tester) sont
pré-calculés. Cette seconde implémentation s'est révélée en moyenne 2.7 fois
plus rapide que la première sur nos exemples.

\item[Capacité de détection d'erreurs] \hfill \\
Nous avons utilisé le ``test mutationnel'' pour évaluer la capacité de détection
d'erreurs en utilisant la vérification d'assertion à l'exécution avec
\textsc{Frama-C}. Nous avons considéré 5 exemples annotés et généré leurs
{\em mutants} (en appliquant une {\em mutation} sur leur code source) et leur
avons appliqué la vérification à l'exécution. Les mutations incluent :
modifications d'opérateur arithmétique numérique, modifications d'opérateur
arithmétique sur les pointeurs, modifications d'opérateur de comparaison et
modifications d'opérateur logique ($land$ et $lor$).

L'outil de génération de test \textsc{PathCrawler} \cite{PathCrawler} a été
utilisé pour produire les cas de test. Chaque mutant a été instrumenté par
\textsc{e-acsl2c} et exécuté sur chaque cas de test pour vérifier que la
spécification était satisfaite à l'exécution. Les programmes d'origine passent
toutes les vérifications à l'exécution. Lorsqu'une violation d'une annotation a
été reportée pour au moins un cas de testn le mutant est considéré comme étant
{\em tué}. La Fig.~\ref{fig:mutation-exp} illustre les résultats. Exception
faite des mutants équivalents (lorsque la mutation produit un programme
équivalent au programme d'origine), tous les mutants erronés ont été tués.

\end{description}


\begin{landscape}
\begin{figure}[h]
  \centering

%\begin{tiny}
  \begin{tabular}{|l|c|c|c|c|c|c|c|c|c|c|c|c|c|c|c|c|c|}
    \hline
     & \danger & $\emptyset$ & list$^1$ & list$^2$ & bst$^1$ & bst$^2$ & Pt$_1^1$ & mask$^1$ & sb$^1$ & Pt$_2^1$ & Pt$_1^2$ & mask$^2$ & sb$^2$ & Pt$_2^2$ & St$^1$ & St$^2$ & valgrind \\
    \hline
    bS$_{10000}$ &22 &\multirow{2}{*}{.01} &1.10 &0.50 &1.14 &0.64 &1.55 &99 &16 &1.55 &0.57 &0 &1 &0.55 &1.39 &0.61 &\multirow{2}{*}{0.27}\\
+ RTE &41 &&1.10 &0.51 &1.14 &0.62 &1.59 &109 &16 &1.59 &0.53 &0 &1 &0.53 &1.39 &0.64 &\\
    \hline
    iS$_{10000}$ &5 &\multirow{2}{*}{.12} &2.29 &0.12 &1.83 &0.12 &2.89 &170k &20k &2.91 &0.12 &0 &0 &0.12 &2.46 &0.12 &\multirow{2}{*}{2.81}\\
+ RTE &24 &&3.52 &1.27 &2.89 &1.26 &3.99 &170k &20k &3.86 &1.25 &0 &0 &1.25 &3.46 &1.30 &\\
    \hline
    mM$_{100^2}$ &0 &\multirow{2}{*}{.01} &2.29 &0.73 &2.94 &1.15 &0.14 &17k &1k &0.14 &0.10 &5k &612 &0.09 &1.07 &0.98 &\multirow{2}{*}{0.34}\\
+ RTE &82 &&13.17 &10.66 &21.92 &17.39 &2.78 &18k &1k &3.00 &2.64 &5k &612 &2.82 &75.97 &73.62 &\\

    \hline
    mM$_{150^2}$ &0 &\multirow{2}{*}{.01} &13.23 &3.96 &15.65 &6.20 &0.54 &21k &2k &0.51 &0.36 &9k &912 &0.35 &5.86 &5.64 &\multirow{2}{*}{0.48}\\
+ RTE &82 &&72.36 &58.48 &110.70 &90.43 &10.77 &24k &2k &9.01 &8.57 &9k &912 &8.75 &403.50 &398.60 &\\
    \hline
    mI$_{100^2}$ &2 &\multirow{2}{*}{.01} &22.51 &0.10 &7.74 &0.13 &0.09 &68k &5k &0.08 &0.01 &7k &609 &0.01 &0.19 &0.10 &\multirow{2}{*}{0.35}\\
+ RTE &155 &&28.96 &4.22 &13.67 &5.48 &0.54 &73k &5k &0.55 &0.53 &7k &611 &0.47 &26.37 &26.16 &\\

    \hline
    mI$_{150^2}$ &2 &\multirow{2}{*}{.02} &130.04 &0.34 &40.35 &0.45 &0.28 &99k &8k &0.27 &0.02 &12k &909 &0.02 &0.68 &0.34 &\multirow{2}{*}{0.47}\\
+ RTE &155 &&153.30 &21.54 &73.55 &29.94 &2.00 &105k &8k &1.90 &1.42 &12k &911 &1.53 &146.15 &145.80 &\\

    \hline
    qS$_{1000}$ &15 &\multirow{2}{*}{.01} &12.70 &2.08 &1.76 &0.59 &0.33 &1M &92k &0.06 &0.13 &683k &39k &0.02 &0.02 &0.01 &\multirow{2}{*}{0.27}\\
+ RTE &32 &&12.38 &2.13 &1.64 &0.56 &0.38 &1M &92k &0.12 &0.14 &727k &39k &0.04 &0.03 &0.02 &\\

    \hline
    qS$_{2000}$ &15 &\multirow{2}{*}{.01} &85.99 &11.31 &8.39 &2.78 &0.71 &3M &198k &0.14 &0.28 &1M &84k &0.05 &0.03 &0.02 &\multirow{2}{*}{0.27}\\
+ RTE &32 &&81.65 &11.15 &7.72 &2.67 &1.13 &4M &198k &0.48 &0.36 &1M &84k &0.13 &0.05 &0.02 &\\
    \hline
    bbS$_{10000}$ &4 &\multirow{2}{*}{.22} &13.78 &1.02 &16.84 &1.67 &117.47 &499M &49M &22.36 &1.57 &30 &7 &1.54 &8.80 &1.67 &\multirow{2}{*}{3.36}\\
+ RTE &16 &&23.08 &4.64 &30.69 &7.16 &107.05 &599M &49M &32.58 &7.26 &29 &7 &6.90 &17.29 &7.21 &\\
    \hline
    m$_{30000}$ &2 &\multirow{2}{*}{.01} &412.10 &11.38 &176.35 &11.01 &1.11 &5M &420k &0.26 &0.30 &1M &60k &0.06 &0.08 &0.01 &\multirow{2}{*}{0.45}\\
+ RTE &49 &&451.58 &101.33 &219.12 &94.80 &1.15 &5M &420k &0.29 &0.47 &2M &130k &0.14 &0.10 &0.05 &\\

    \hline
    Rbt$_{10000}$ &0 &\multirow{2}{*}{.01} &47.39 &0.28 &48.44 &0.27 &0.32 &1M &159k &0.09 &0.03 &151k &10k &0.01 &0.59 &0.01 &\multirow{2}{*}{0.51}\\
+ RTE &270 &&120.02 &101.69 &165.77 &145.20 &0.47 &1M &159k &0.30 &0.39 &979k &119k &0.27 &18.82 &19.59 &\\
    \hline
mS$_{1000}$ &7 &\multirow{2}{*}{.01} &6.45 &0.34 &6.32 &0.11 &0.32 &1M &95k &0.07 &0.06 &331k &18k &0.01 &0.02 &0.01 &\multirow{2}{*}{0.27}\\
+ RTE &45 &&6.82 &1.35 &7.98 &0.38 &0.34 &1M &95k &0.10 &0.13 &701k &38k &0.04 &0.02 &0.01 &\\
        \hline
mS$_{5000}$ &7 &\multirow{2}{*}{.01} &362.87 &11.00 &218.01 &3.43 &2.28 &11M &562k &0.76 &0.43 &2M &106k &0.09 &0.14 &0.03 &\multirow{2}{*}{0.27}\\
+ RTE &45 &&371.40 &50.94 &290.88 &10.34 &2.46 &11M &562k &0.80 &0.83 &4M &218k &0.22 &0.16 &0.08 &\\

    \hline
mS$_{10000}$ &7 &\multirow{2}{*}{.01} &3624.01 &47.94 &1673.00 &16.10 &6.46 &23M &1M &2.75 &1.00 &5M &223k &0.21 &0.31 &0.08 &\multirow{2}{*}{0.27}\\
+ RTE &45 &&3406.43 &257.18 &2086.32 &46.22 &6.30 &23M &1M &2.66 &1.83 &9M &457k &0.51 &0.35 &0.18 &\\
\hline
mS$_{50000}$ &7 &\multirow{2}{*}{.01} &$\infty$ &3847.72 &$\infty$ &1100.93 &135.54 &146M &6M &111.22 &6.90 &33M &1M &1.65 &2.08 &0.58 &\multirow{2}{*}{0.63}\\
+ RTE &45 &&$\infty$ &25554.08 &$\infty$ &2781.90 &118.86 &145M &6M &95.74 &11.64 &54M &2M &3.37 &2.18 &1.15 &\\
\hline
mS$_{100000}$ &7 &\multirow{2}{*}{.01} &$\infty$ &$\infty$ &$\infty$ &$\infty$ &631.41 &296M &14M &559.93 &13.55 &70M &2M &3.35 &4.03 &1.15 &\multirow{2}{*}{0.27}\\
+ RTE &45 &&$\infty$ &$\infty$ &$\infty$ &$\infty$ &573.47 &308M &14M &513.85 &25.02 &116M &5M &7.63 &4.68 &2.50 &\\
\hline
  \end{tabular}

%\end{tiny}
  \caption{Comparaison des différentes implémentations du $store$}
\end{figure}
\label{fig:mmodel-exp}
\end{landscape}
%\restoregeometry






\begin{figure}[t]
  \centering
  \begin{tabular}{|c|c|c|c|c|c|c|}
    \hline
    & alarmes & mutants & équivalents & tués & \% erronés tués \\
    \hline
    fibonacci & 19  & 27 & 2 & 25 & 100\% \\
    \hline
    bubbleSort & 15  & 44 & 2 & 42 & 100\% \\
    \hline
    insertionSort & 10  & 39 & 3 & 36 & 100\% \\
    \hline
    binarySearch & 7 & 38 & 1 & 37 & 100\% \\
    \hline
    merge & 5 & 92 & 5 & 87 & 100\% \\
    \hline
  \end{tabular}
  \vspace{-2mm}
  \caption{Capacité de détection d'erreurs}
  \vspace{-2mm}
  \label{fig:mutation-exp}
\end{figure}


%%%%%%%%%%%%%%%%%%%%%%%%%%%%%%%%%%%%%%%%%%%%%%%%%%%%%%%%%%%%%%ù


\section{Conclusion}
conclusion + future work + application dans le greffon PathCrawler (transition
avec le chapitre suivant)
